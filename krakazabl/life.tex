\custompart{Колыбель жизни}{Краказябл}{http://klanz.ru}

Создатель трудился долго, корпел над своим рабочим столом, ярко освещенным 
светом трех лун, просачивавшемся через многочисленные иллюминаторы 
био-лаборатории. Работа шла медленно, но верно.\\
--- Привет, Шеф~--- Поздоровался подошедший главный помощник,~--- Все 
трудишься?!\\
--- Привет, Михаил! Да, вот, задержался малехо.~--- Устало улыбнулся 
Создатель.~--- Или, даже правильнее будет сказать, увлекся!\\
--- Ух ты,~--- усмехнулся Михаил,~--- Твое очередное Чудо?!\\
--- Ага, и довольно таки интересное~--- Задумчиво заметил Создатель.~--- Я 
придумал его вчера во сне и с утра сразу же взялся за работу. Но ты знаешь, это 
оказалось гораздо тяжелее чем я думал. Пришлось многое изменить в планах и 
чертежах в процессе изготовления. Я чуть было не запутался сам в своей задумке 
и 
даже почти пожалел, что затеял все это. Но потом, знаешь, махнул на все рукою и 
решил~--- сейчас доделаю, а там те, кто будет пользоваться пускай сами 
разбираются!\\
--- Оригинальное решение, Шеф~--- Улыбнулся Михаил,~--- Может быть 
расскажешь, чем же это ты тут все-таки занимаешься?!\\
--- Нет--нет, ни за что!~--- сухо отрезал Создатель, задумчиво ковыряясь в 
чем-то, накрытом белым покрывалом.~--- Впрочем, тебе я могу показать Это!\\
--- Гляди~--- гордо воскликнул Создатель, скидывая покрывало!\\
--- Хм, и что это?! Новое звено пищевой цепочки!?~---  Пошутил помощник, 
скептически оглядывая новое творение~--- И что же в нем такого интересного? 
А-а-а понял, у него, наверное, просто необыкновенно чудесный вкус!\\
--- А ты все шутишь, Михаил~--- Строго отрезал Создатель.~--- Нет, это 
кое-что 
поинтереснее. То, что ты видишь сейчас на моем лабораторном столе, есть 
не что иное, как новый виток в эволюции! И, возможно, даже он 
когда-нибудь заменит и нас с тобою в спирали жизни...\\
--- Ого, Шеф, ты меня пугаешь,~--- Удивился помощник, невольно делая шаг 
назад,~--- Так зачем же ты тогда создал это отвратительное чудовище?!\\
--- О нет, мой юный друг,~--- Улыбнулся Создатель,~--- это далеко не 
чудовище, как ты соизволил выразиться! И даже наоборот~--- это величайшее 
творение, которое когда--либо мне удавалось создать!\\
--- Ну не знаю, твои слова на счет нового витка эволюции меня совсем не 
вдохновили.~--- Задумчиво заметил Михаил.~--- И как же ты назвал это свое 
чудо творения?!\\
--- А назвал я его~--- ЧЕЛОВЕК!~--- Гордо воскликнул Создатель.~--- А точнее 
Женщина! Впрочем это базовая модель, в ней еще есть кое-какие недоработки.\\
--- Жен--щи--на\ldots~--- Медленно произнес Михаил.~--- Звучит красиво! А 
какова ее функциональность?\\
--- Думаю, в основном для красоты~--- Задумался 
Создатель.~--- Нет, изначально я, конечно, хотел сделать ее более 
полезной, ты ведь знаешь, как нам сейчас не хватает рабочих рук.\\
--- Ага, знаю~--- вздохнул помощник~--- от этих твои семилапов, которых ты 
сотворил на прошлой недели только одни проблемы. Как говориться, силы 
много~--- мозгов мало!\\
--- Вот-вот, и я о том же!~--- Воскликнул Создатель~--- Сегодня я
хотел исправить положение, но слегка ошибся в расчетах да и исходного 
материала немного не рассчитал. Впрочем, и устал я очень, ты даже не 
представляешь себе, на сколько тяжелый механизм я в нее вложил!\\
--- Мне так кажется или она довольно таки хрупка на вид?~--- Спросил Михаил, 
дотрагиваясь рукою до женщины.~--- Ее кожа, она настолько мягкая, 
бархатистая и нежная, что я весьма сомневаюсь в ее способностях к 
выживанию.\\
--- Верно, тебе это действительно только кажется.~--- Хитро 
прищурился Создатель.~--- При всем своем невинном и безобидном виде, 
Женщина весьма сильное и коварное создание. А внешний вид~--- это всего 
лишь приманка и мой тонкий юмор. Как бы Женщина не выглядела слабой, но в
этом и есть ее сила! Внутри нее скрыты безграничные возможности. Если 
надо, она будет трудиться с утра до вечера, обеспечивая свою семью. Она 
может отдавать последний кусок пищи своим детям, долго оставаясь при 
этом на ходу. Она может сама не есть, не пить, не спать, раздеться 
полностью, дабы укрыть своих малышей от холода, она отдаст ВСЕ ради 
продолжения своего рода. И в этом ее наибольшая сила!\\
--- Ты меня снова пугаешь, Шеф.~--- Тихо сказал помощник.~--- Дети, 
потомство, отдаст все для продолжения своего рода\ldots\ Я, надеюсь, ты не 
забыл о тех кроколюпах из галактики Темного Родона, которых нам потом с 
таким трудом пришлось уничтожить вместе с несколькими солнечными 
системами, на которых они расплодились за весьма короткий час?!\\
--- Нет--нет, помню конечно же~--- Успокоил его Создатель,~--- и поэтому на 
этот раз я все предусмотрел! Женщина будет приносить потомство не чаще 
одного раза в год и, в основном, не более одного экземпляра за раз. Да и
то не каждая и при весьма определенных обстоятельствах. Первую декаду 
своей жизни она вообще не сможет репродуктироваться, а потом, для 
процесса воспроизводства, ей понадобится самец. И не какой попало, а 
самый лучший. Ну, по крайней мере по ее мнению\ldots\ И еще, ты знаешь, я 
решил весьма запутать процесс брачных игр Человека, ну что б хоть как-то
регулировать их размножение.\\
--- Погоди--погоди, мне снова показалось или 
ты сказал самец?!~--- Изумился помощник,~--- Ты что сделал Женщину в двух 
экземплярах?!\\
--- Вот и в очередной раз ты все путаешь, Михаил,~--- обиделся 
Создатель, бросая в ванночку с физраствором острый скальпель,~--- даже не 
знаю, как я вообще взял тебя к себе в помощники. Слушай, может мне тебя 
аннигилировать и взять на твое место Андрея?!\\
--- Так, Шеф, спокойно, не 
кипятись!~--- Взволновано залепетал Михаил, медленно отходя назад.~--- Нет, 
ну ты же знаешь, что я просто шучу, как всегда. И к тому же, Андрей 
гораздо глупее меня. Ну ты ведь помнишь ту историю, когда пытался 
научить его хождению по воде? Так что, может, повременим с этой 
аннигиляцией, ну хотя бы недельки на две?\\
--- Помню-помню этого 
индивидуума, вот поэтому здесь со мной сейчас стоишь ты, а не он.~--- 
Усмехнулся Создатель, вытирая руки о белое полотенце.~---  А ты я вижу 
здорово струхнул? Ну вот видишь, не только ты умеешь шутить!\\ 
--- Фуух, так
это была шутка~--- облегченно вздохнул Михаил,~--- а я уж было начал думать
куда бы так запрятаться, чтоб ты меня не достал своим аннигиляторм. 
Так, чем мы там окончили наш разговор о этом восхитительнейшем из всех  
твоих созданий?!\\
--- Я тебе говорил, балбес, что Женщине для продолжения рода нужен 
будет самец.~--- Продолжил Создатель свою лекцию, устало усаживаясь в 
мягкое кресло.~--- Причем знаешь, я думаю сделать так, что б он ей был 
нужен не только для размножения но и\ldots\ Но и не знаю для чего. Ну просто
так вот, нужен он ей да и все\ldots\ Может быть, потом придумаю. К тому же,
за самца я еще пока не брался. На сегодня мне хватит и этой дамочки.\\
--- Интригующая задумка, Шеф, браво!~--- Явно подхалимно воскликнул Михаил, 
украдкой поглядывая на шкатулку, в которой хранился тот самый 
аннигилятор, с которым он только что чуть не познакомился.~--- Но я 
чувствую, вы мне еще не все рассказали о Женщине!\\
--- О да, я вложил в нее столь многое, что и сам уже запутался, зачем 
ей все это надо.~--- Усмехнулся Создатель, наливая себе бокал лучшего вина
из галактики Диониса,~--- Как мы уже говорили, она весьма нежна как 
внешне, так и внутренне. В ней будет постоянно кипеть вулкан эмоций, 
которым она все время будет пытаться заразить окружающих. К тому же, я 
весьма расширил ее словарный запас и скорость речи. Думаю, это 
пригодится ей для общения с противоположным полом. Да, еще я весьма 
увеличил мозг человека, что бы он мог не просто существовать, но и вести
себя весьма обдуманно и логически. Впрочем, я уже говорил тебе, что это
лишь пробная модель, на ней я пока тренируюсь\ldots\\
--- Шеф, по-моему у нее что-то с головою, а точнее с глазами.~--- 
Испугано заметил помощник.~--- У нее сейчас проступает какая-то 
межглазовая жидкость!\\ --- У нее все в порядке и с глазами и с головою, в 
отличие от тебя.~--- Усмехнулся Создатель, слегка позевывая.~--- Ты зачем ей
так сильно голову сдавил своими ручищами? Это же не черепаха, которую я
сделал позавчера и которую, ты так забавно вертел в руках, пытаясь 
разглядеть что у нее где, когда она спряталась в свой панцирь. Отпусти 
ее немедленно, а та жидкость, которую ты у нее заметил на лице~--- это 
слезы, ты сделал ей сейчас больно.\\
--- Слезы,~--- Удивился Михаил, отходя 
от лабораторного стола,~--- а зачем они нужны?\\
--- Вот этого я и сам еще 
точно не понял.~--- Задумчиво ответил Создатель.~--- Я хотел сделать 
Человека более эмоциональным, чем остальные. Для этого я и придумал 
слезы и одно из движений его лицевых мышц, которое назвал улыбкой. Не 
такой как у нас с тобой, а другой, более выразительной и более 
многогранной. Женщина будет радоваться обычным растениям, которые ей 
подарят. Любоваться маленькой снежинкой, упавшей в ее прекрасные 
ладошки. Она будет петь под шум дождя и танцевать босоногой на осенней 
поляне, поднимая вверх стаи опавших листьев. И она будет ЛЮБИТЬ! Это 
новое чувство, которое я давно хотел вложить в одно из своих созданий, 
но все никак не получалось. И любовь Женщины будет безгранична, 
бескорыстна и всеобъемлюща. \\
Впрочем, ты знаешь, в ходе экспериментов я немного запутался. Эта 
Женщина, она\ldots\ Она ведет себя крайне не предсказуемо и как-то весьма 
неадекватно. Она плачет когда ей хорошо, хотя казалось бы должна 
смеяться. И наоборот, улыбается сквозь слезы, когда ей совсем невмоготу.
В общем ты знаешь, Михаил, на сегодня я уже весьма устал. Накрой ее 
новой простыней с электроподогревом и выключи свет в лаборатории, а я 
пойду подремлю до утра, ведь завтра еще столько работы."  "Хорошо, Шеф, 
идите отдыхайте, а я тут все сделаю как надо.~--- Тихо сказал Михаил 
помогая полусонному Создателю встать с кресла и провожая его за руку до 
выхода из лаборатории.~--- Это ж надо было и придумать такое~--- Человек, 
Жен--щи--на\ldots\\
--- Михаил, зайди ко мне в лабораторию~--- Прозвучал хриплый голос 
селектора, заполняя все пространство огромного межгалактического 
корабля,~--- ты мне очень нужен!\\
--- И почему это меня всегда раздражает мой
собственный голос, услышанный со стороны~--- Задумчиво произнес 
Создатель, раскладывая на место свой рабочий инструмент,~--- и этот 
ужасный звук из селектора\ldots\ Ну неужели нельзя было придумать более 
приятный звуковой фильтр для общения?\\ 
--- Привет, Шеф, вызывали?~--- Весело 
поздоровался старший помощник.~--- Что-то случилось или вы просто решили 
похвастаться доработкой вчерашнего экземпляра?\\ 
--- Скорее всего второе, 
Михаил, но не совсем так.~--- Хитро улыбнулся Создатель.~--- Ты знаешь, 
вчерашний наш с тобою вечерний разговор так меня завел, что я не смог 
долго спать и, встав, сразу же принялся за работу. И вот я готов 
представить свое новое творение. Вуаля,~--- гордо воскликнул Создатель, 
срывая с рабочего стола белоснежное покрывало,~--- смотри и преклоняйся 
предо моим гением!\\
--- Хм, ну что ж, весьма интригующе,~--- Удивленно произнес помощник, 
рассматривая новое творение,~--- Ты что, слегка переделал Женщину?! Как по
мне, то предыдущий вариант был несколько поэстетичнее. Особенно 
интересна та штукенция, которую ты прилепил ей в околонижней части 
туловища. И этот волосяной покров по всему телу\ldots\ Она что, у тебя такая
мерзлячка или ночью в лаборатории были сбои в системе 
кондиционирования?\\
--- А--а--а, смеешься надо мною гад?~--- Саркастически 
заметил Создатель,~--- Где там моя шкатулка с моим любимым 
аннигилятором?"  "Не-не, Шеф, все нормально, все в порядке!~--- 
Успокаивающе заерзал помощник.~--- Ну зачем же сразу столь крайние меры? 
Может, обойдемся просто подзатыльником, как всегда?\\ 
--- Ну уж нет, одним 
подзатыльником теперь ты у меня не отделаешься!~--- Радостно ухмыльнулся 
Создатель, явно довольно потирая руки.~--- И сегодня, в наказание, тебе 
придется целый вечер выслушивать бредни старого маразматика в моем 
лице!\\
--- О нет, Шеф,~--- Наиграно расстроено махнул рукой Михаил,~--- может, 
лучше все-таки на этот раз аннигилирование?!\\ --- Молчи, мерзавец, и налей 
мне бокал вина, того что нам с тобой подарили в Галактике Диониса!~--- 
Улыбнулся Создатель, устало падая в свое любимое кресло.~--- Это вовсе не 
видоизмененный вид Женщины, как по своей природной глупости успел 
заметил ты. Нет, ну я, конечно, сперва так и хотел сделать, поменять там
кое-что, дополнить, но мне это показалось несколько скучновато. И 
поэтому, пред тобой сейчас лежит новый вид Человека~--- Мужчина!\\
--- Странно, но у меня такое ощущение, что ты особо не стал 
заморачиваться с НОВЫМ видом Человека~--- Тихо заметил Михаил.~--- Так как 
он через чур уж похож на предыдущую модель\\
--- Да, ты прав,~--- улыбнулся 
Создатель,~--- да и к чему было фантазировать, если тело человека и так 
столь идеально? К тому же, Мужчина просто таки обязан быть похожим на 
Женщину, иначе, они не уживутся вместе. Поэтому, я почти все оставил на 
прежнем уровне, несколько видоизменив экземпляр и исправив те 
недостатки, которые были присущи предыдущей модели. Я сделал тело 
Мужчины более крепким, сильным, выносливым, более приспособленным к 
физическим нагрузкам и даже поставил ему новый функциональный 
мыслительный центр, крайне отличающийся от подобного женского!\\ 
--- Это~--- восхищает,~--- Удивленно заметил Михаил,~--- значит, ты успел уже и 
Женщину доработать?\\ 
--- Лучше б я тебя доработал,~--- Рассердился Создатель, 
запустив в помощника бумажной моделью какого-то летательного аппарата, 
созданного им в процессе беседы из подручных средств,~--- Ну сам подумай, 
если б я дорабатывал Женщину, то когда бы я успел создать Мужчину?!\\
--- И то верно, прости , Шеф~--- Извинился Михаил,~--- так значит, ты забросил 
Женщину в долгий ящик?\\
--- Ну почему сразу забросил,~--- Задумчиво парировал Создатель,~--- я 
отложил еще модифицирование на более благополучный момент\ldots\ К тому же 
ты только взгляни на Мужчину! Это же просто таки невероятно, сколько 
силы хранится в этом скрытом механизме. Я даже сам испугался, когда 
оценил весь его потенциал и поэтому пошел на маленькую хитрость!\\ 
--- Интересно на какую же,~--- Интригующе потер руки помощник,~--- зная ваш 
извращенный юмор и безудержную фантазию, можно ожидать на многое.\\
--- Ну ты мне льстишь, Михаил, хоть и приятно.~--- Довольно откинулся на спинку 
кресла Создатель.~--- Я одарил мужчину новым своим изобретением, которое я
назвал~--- Лень! Лень~--- это такая штука, знаешь ли, которая способна 
остановить самые великие помыслы и возможности Мужчины. Она будет 
ограничивать его мощь, дабы он не натворил беды. Впрочем это будет 
несколько мешать его совместной жизни с Женщиной и доставлять ей 
определенные неудобства. Но ты же меня знаешь, я не смог удержатся, что 
бы и тут немного не нагадить. И поэтому я все-таки сегодня несколько 
поковырялся и в Женщине, что бы и она в свою очередь на определенные 
приставания Мужчины смогла ответить что ей лень, ну или там еще 
чего-нибудь придумать\ldots\\
--- Вы меня вновь и вновь поражаете, Шеф,~--- Восхищенно зааплодировал 
Михаил,~--- Слушайте, а может вам когда-нибудь, ну допустим в ближайшее 
время, придет блестящая и гениальная мысль уничтожить свой аннигилятор. 
Ну тот, которым вы все время меня пугаете?\\ 
--- Да, ты прав, Михаил, как это я сам не додумался,~--- По заговорщицки 
подмигнул помощнику Создатель,~--- точно, я его уничтожу! Сразу же, как 
только сдержу свое обещание и 
аннигилирую тебя вместе с твоими исходниками!\\
--- Нет, ну тогда особо 
торопиться не стоит,~--- Забеспокоился Михаил,~--- мало ли, вдруг 
аннигилятор еще не раз нам пригодиться. Кто знает с какими трудностями 
мы еще столкнемся в нашем далеком будущем!\\ 
--- Вот, шельма, выкрутился 
таки,~--- Улыбнулся Создатель,~--- посмешил ты меня, посмешил, молодец! И 
так, вернемся как говориться к нашим Альдебаранам! Мужчина будет груб на
вид, иметь надменный взгляд и скверный характер. Он будет задирист и 
вспыльчив. Его взгляда будут бояться даже самые дикие звери. А его 
улыбка будет напоминать оскал льва. И лишь для избранных будет 
открываться его сердце и тот внутренний мир, что остался у него от 
Женщины! Да-да, и не делай такие округлые глаза, ты так становишься 
похож совух из Галактики Лупоглазых. Хочешь верь, хочешь не верь, но под
этой грубой и черствой оболочкой скрывается вулкан чувств! В нем есть 
сила, которая изумляет врагов и ласка, которая покорит даже Женщину. Он 
может справиться с любой бедой и перенести любые горечи на своем 
жизненном пути. Более того, он  не только сам справиться с этим, но и 
семью свою вытянет из любого водоворота событий! За его широкой спиной, 
Женщина будет чувствовать себя как в самой непреступной крепости. А его 
крепкие, но в тоже время ласковые руки смогут укачать младенца не хуже 
Женщины. Он будет улыбаться, даже глядя смерти в глаза. И при этом он 
способен, хоть и иногда, но все же пустить слезу – тихо, где-нибудь в 
укромном уголку, что бы никто не заметил, что у него тоже может быть 
мягкое сердце."   "Но это ведь будет означать, что Мужчина будет 
настолько силен, что никто не сможет покорить его!?~--- Воскликнул 
Михаил.\\ 
--- Никто\ldots\ Никто кроме Женщины.~--- Хитро подмигнул помощнику 
Создатель,~--- Ибо только ей будет подвластно сделать из этого грубого 
зверя ласкового и нежного пушистика.\\
--- Да, Шеф, иногда вы даже меня ставите в тупик своей гениальностью. -
Растрогался Михаил,~--- Но все что вы творите безусловно прекрасно, хоть и
не всегда полезно. Так и когда же вы собираетесь порадовать нас 
окончательными вариантами данных моделей?\\ 
--- Скоро, мой юный друг, скоро!~--- Мечтательно улыбнулся Создатель.~--- Я 
думаю еще дня два-три и ты уже сможешь лицезреть самые прекрасные и идеальные 
создания, которые только рождались в этой лаборатории. И имя им~--- ЛЮДИ\ldots\\
--- Опасность! Опасность!~--- Резко завопил хриплый голос селектора 
голосом бортового компьютера.~--- Опасность! Неисправность левого 
двигателя. Пожар в усилителе потока. Корабль несанкционированно меняет 
свой курс и через несколько минут неизбежно пройдет траекторией, 
пересекающей метеоритное облако!\\
--- Что это, Михаил? Не может быть!~--- Взволновано вскрикнул Создателей.~--- 
Бегом в командирскую рубку!\\
--- Все напрасно, Шеф,~--- обреченно заметил старший помощник, пытаясь 
вручную выровнять курс корабля,~--- Времени осталось слишком мало и  мы не
успеем существенно изменить свой курс так, что бы избежать столкновения
с метеоритам. Бортовой компьютер, вирус на твои микропроцессоры, почему
не доложил об опасности заранее?!\\ 
--- Никакой опасности не было, Старший 
Помощник!~--- Хрипло пищал селектор бортового навигатора.~--- Курс был 
идеально просчитан мною и проходил мимо траектории метеоритного облака!\\
--- Так почему же он сейчас проходит как раз по самой его середине?~--- 
Бушевал Михаил, в сердцах громя сенсорную клавиатуру кулаком.\\
--- Неисправность левого двигателя. Пожар в усилителе потока.~--- Повторял 
свою старую песню селектор.~--- Я лучший бортовой компьютер галактики~--- 
<<Навигатор Х13>>, последняя модель в своем роде. Я не могу ошибаться, мои
программы идеальны. И я не отвечаю за фарс--мажорные обстоятельства!\\
--- Да в Черную Дыру тебя и все твои программы!~--- Не унимался Михаил,~--- 
Теперь у тебя есть все шансы действительно стать самой последней моделью
в своем роде!\\ 
--- Тише, Михаил,~--- Успокоил Создатель своего помощника,~--- 
Х13 действительно ни в чем не виноват. Он действительно был сделан 
чересчур идеально и, поэтому, даже не просчитывал варианты своих ошибок.
Это не его вина, а скорее всего моя. Вот что случается, когда пытаешься
создать нечто совершенное\ldots\\
--- Так что же нам теперь делать, Шеф?~--- Огорченно спросил Михаил.\\
--- Приготавливай запуск всех спасательных капсул.~--- Скомандовал 
Создатель.~--- Мы катапультируем весь био-отсек, зверинец и склад с 
био-исходниками!
Нам еще повезло, что мы проходим курсом возле одной из немногих 
пригодных для жизни солнечных систем этой галактики. Быстро просчитай, 
какая из этих девяти планет наиболее пригодна для обитания.\\
--- Скорей всего, это третья планета, Шеф~--- Задумчиво ответил Михаил.~--- Она 
наиболее подходит по большинству необходимых нам параметров. Но многое 
факторы на ней крайне отрицательно повлияют на некоторые био-экземпляры,
которые мы собираемся на нее отправить.\\
--- Это уже неважно, Михаил, 
запускай капсулы!~--- Скомандовал Создатель.~--- Впрочем, и особого выбора 
то у нас нет. Будем надеяться, что они приспособятся к местным условиям и
эта планета сможет стать хорошим домом для них.\\ 
--- Готово, Шеф, пошел 
обратный отсчет!~--- Доложил помощник,~--- Скоро произойдет 
катапультирование. Я оставил отдельный двойной крио-отсек и для нас с 
вами.\\ 
--- О нет, Михаил, я остаюсь.~--- Задумчиво прошептал Создатель.~--- На 
этом корабле слишком много оборудования, так необходимого мне для 
творчества. Если я потеряю его, то не смогу дальше работать и тогда 
исчезнет весь смысл моей жизни. Катапультируйся сам, Михаил, и 
позаботься о том материале, которые мы так бережно собирали по крупицам 
по всей вселенной.\\ 
--- О нет, Шеф, так дело не пойдет. Пусть это и звучит 
несколько банально, прям как в этих нудных мыльных операх, которые 
показывали у нас на родине, но я вас не покину!~--- Воскликнул Михаил, 
одевая спасательный костюм.~--- Мы вместе начали все это приключение, 
вместе его и продолжим. Вот, одевайте свой скафандр, скоро возможна 
аварийная разгерметизация корабля.\\
--- Спасибо тебе, Михаил! Я 
действительно хотел, что б ты со мной остался, но был не вправе сам тебе
это предложить.~--- Растрогался Создатель, крепко обнимая своего старшего
помощника.~--- Всю жизнь я жил в гордом одиночестве и избегал шумных 
компаний и только теперь осознаю, как неприятно и даже, иногда, страшно 
оставаться одному.\\ 
--- Но вы не один, Шеф, с вами я и наш бортовой 
идеалист Х13.~--- Улыбнулся помощник.~--- Думаю вместе мы еще поборемся за 
свое существование. Кстати, Шеф, а что делать с Людьми?!\\
--- Как что 
делать?~--- Встревожено изумился Создатель.~--- Разве ты не катапультировал 
их вместе со всеми?!\\ 
--- Конечно нет, Шеф,~--- Грустно оправдывался Михаил, -
Они же еще недоработаны и поэтому хранятся в вашем рабочем кабинете. А у
него, как вы знаете, нет системы катапультирования. Вы же сами не 
захотели ее туда устанавливать\ldots\\
--- Да-да, я помню, Михаил, и что же теперь делать?~--- Сокрушенно спросил 
Создатель, надевая на голову шлем 
спасательного скафандра.~--- Идея! Ты, вроде, сказал что приготовил для 
нас двойную кабинку катапульты?\\
--- Да, Шеф, я понял ход вашей мысли!~--- 
Обрадовался старший помощник.~--- Поспешим перенести их туда. Вот 
только\ldots\ Жаль, что вы их так и не успели довести до совершенства, ведь в
них осталось еще столько недостатков.\\ 
--- Ты знаешь, Михаил, я уже и 
несколько об этом не жалею.~--- Задумчиво улыбнулся Создатель.~--- Сегодня я
понял, что даже идеальные создания, все равно не могут быть совершенно 
идеальными. И пусть Люди мое первое творение, которое я так и не довел 
до конца, думаю, в этом и будет их основная особенность и изюминка. 
Единственное о чем я жалею, так это о том, что я так и не успел дать им 
хорошие Имена\ldots

\newpage 

     Большой межгалактический корабль <<Элизиум>> боролся за свое 
существование с силами природы, пронзая собою насквозь густое 
метеоритное облако и оставляя позади себя блестящий шлейф из вырванных 
кусков метала и частей оборудования. Он дрожал, скрипел, разваливался, 
но все равно боролся за свою жизнь. Так же, как и боролись за свою жизнь
двое друзей в капитанской рубке, отчаянно пытаясь вырвать свой корабль 
из цепких лап смерти. А в это время сотни тысяч спасательных капсул 
спокойно обходили стороною метеоритное облако и медленно, но уверенно 
приближались к третей планете пока неизвестной им еще солнечной системы.
И внутри каждой из этих капсул теплилась жизнь. Она пока еще спала, 
готовясь вот--вот проснутся и подарить новых детей этой гостеприимной 
планете. А внутри одной из спасательных капсул, выпущенной самой 
последней и поэтому отстающей от основной массы катапульт, летели самые 
интересные творения Гениального Создателя.  Создания, которые так и не 
успели получить свои имена. И лишь на маленьких картонных бирочках, 
привязанных к руке каждого из них, едва можно было разглядеть надписи, 
сделанные наспех мелким почерком~--- <<Модели Х и Y. Люди>>.  Они спали, 
нежно обнимая друг дружку и даже не догадывались, какую роль им еще 
предстоит сыграть в истории этой планеты\ldots