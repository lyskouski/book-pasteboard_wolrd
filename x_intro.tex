\thispagestyle{empty}
~\\
\section*{Предисловие}

Рождение человека, формирование души, зарождение мыслей, становление личности. 
Весь этот процесс незрим, где-то неосознан, а порой осуществляется и вовсе не 
нами. Так выкристаллизовывается наша уникальность\dots\ И пусть для Вселенной 
мы не ценнее снежинок, они ведь тоже уникальны по своей структуре, но для 
кого-то значимы, и лишь это имеет значение.

Движение по временным циклам, от рождения до самой смерти, никогда не стихает, 
никогда не меняется, для каждого оставаясь незыблемым, как само существование. 
Время стирает кости в пыль, воскрешая в чужих сердцах лишь некоторые зачатки 
наших идей\dots\ но бесконечность циклов перерождения неизбежно стирает любое 
свершение души. И если всё так, как расписывают зыбучие пески времени, то для 
чего все усилия, терзания, страдания и превозмогания над собственной бренностью?.. 
проще остаться пылинкой, чем обрести очертания скал.

Мир внутри мира, таких множество на нашей планете. Даже внутри человека можно 
встретить порой бесконечную череду миров… и его где-то там, затерявшегося и 
навсегда покинувшего действительное настоящее. Закрывшись от смутных реалий 
современности, вооружившись игрушечными мечами и волшебными посохами, повергая 
чудовищ и врагов в дикий ужас, тешется виртуальным величием. Куда проще 
вообразить себя кем-то и играть эту роль, чем в действительности бросить вызов и 
стать этим Кем-то. Боимся быть осмеянными и движемся от того вдоль чужих теней и 
следов, в страхе нанести собственную метку на историческую шкалу.

Череда проблем и сумбурного быта обгладывают крылья детства, так величественно 
раскинувшиеся некогда. Возможности преображения в поэта, 
композитора, писателя, архитектора, в кого угодно, обращаются в обезьяньи 
происки пищи и совокупления. Отсутствие преград, граничащих со смертью\dots\ 
зацикленность на сущном\dots\ созидание человека из закалки души превратилось в 
навозное месиво. 
\newpage
\thispagestyle{empty}
\noindent И, выползая из неё желеобразной массой, уже не в состоянии 
водрузить на себя Мир и задать ему вектор развития. И уже не толчок вперед, а 
пинок под зад позволяет опомниться, что слово <<человек>> может писаться и с 
большой буквы. И в злобном оскале вспоминаем о чести и достоинстве, не всегда, 
правда, находя при этом в себе человечность. 

Человеческие добродетели граничат с душевным убожеством по прихоти безумной 
толпы. Низменные позывы преобладают над созиданием, да и к чему оно?.. вот тебе 
загроможденные разнообразными яствами столы… там нет места письменам и 
рукописям, и трепещущей свече в сквозняке окон\dots\ лишь животноподобное 
поедание пряностей. Вот тебе зрелища! Сотни окон, тысячи лиц, миллионы 
историй. И воображение сходит на нет: маг незрим без волшебной трости, а 
дракон без крыльев. Всё укладывается в стандарты сотворенных кем-то оков. 
Правителям не нужны разумные лица, гораздо проще управлять бездумным стадом 
овец. И даже демиурги виртуальных миров в большинстве своём давят на первобытные 
принципы жизнеуклада. Хлеба и зрелищ, а что же еще?.. Легкой наживы, 
халявы, вульгары\dots

\vspace{8mm}
Мир катится в бездну лености и чревоугодия, лишь проблески творчества обращают 
всё в спять. И пусть среди сотен исписанных страниц найдется лишь одна, 
озаряющая, её появлению будет способствовать именно эта исписанная никчемная 
кипа бумаг. Алмаз выкристаллизовывается временем и давлением, творческие навыки 
вышлифовываются практикой и усердием. \texttt{Мысли не сразу обрамляются в слова 
и уж гораздо реже в трепещущие мотивы, не бойтесь творить, не стесняйтесь 
писать. Пусть будет смешно даже Вам через годы, но ведь и птицы начинают свой 
полет с падения.}

\vspace{24mm}
\noindent Из предисловия к рассказу <<Цена альтруизма>>\\
\noindent \textline{%
\textnormal{\noindent Лысковский В. В.}
}{\hfill}{%
\raggedleft 2012-08-21\par\smallskip
}
