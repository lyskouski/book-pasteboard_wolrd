\custompart{Огонь приносящий}{fanatizm}{http://creativity.by}

Сквозь промозглое туманное утро начинали прорываться солнечные лучи. Белёсую 
пелену, укутавшую лес, медленно заполняло золотистое сияние солнца. Рассвет пока 
не спешил в этот мир. Восток окрасился багровым, под стать крови, перемешавшейся 
с росой на траве. Горящее красным солнце нехотя поднималось над горизонтом, 
роняя свет на картину давно угасшего сражения. На холодной земле в нелепых позах 
застыли распростёртые тела. Свет, словно избегая мертвецов, несущих на себе 
отпечаток боли и агонии, выхватывал из сумрака кроны и стволы деревьев. Их 
листья, проснувшись, потянулись к солнцу. Вслед за деревьями настал черёд травы. 
Последними лучи выхватили из сумрака три чёрных силуэта, стоявших над 
поверженными врагами. Они так и не сменили своего цвета, и когда всё вокруг было 
залито светом, силуэты стали казаться ещё чернее. Тени не двигались. Время для 
них прекратило свой ход, пряча бремя жизни. Бремя, выкованное кузне природы её 
законами.

Наступала минута молчания.

Огромная закованная в матово--чёрный металл махина нависала над трупами подобно 
надгробию. Наголо выбритая голова была низко опущена. Грубая линия губ беззвучно 
шептала последние слова. Веки смежены. Правая рука, сжата в кулак и прижата к 
груди. Расс прощался со своими братьями. Для тех двоих, что стояли за его 
спиной, это было пустой тратой времени. Но он. Он был другим. Вернее стал, 
когда-то.

Всё было сказано, и Расс дал им знак. Воздух наполнился скрежетом металла и 
стонами потрёпанных механизмов, наполнявших их чёрную броню.

\noindent --- Связь потеряна, командир. Считаю, необходимо двигаться к лагерю 
немедленно, чтобы предупредить остальных, --- прогудело из-под забрала шлема, 
где-то за спиной.

Расс скривился. Жуткие шрамы, пересекавшие лицо названного командира, 
сморщились, окончательно превращая его в злобную гримасу. Ох уж эти 
новоиспечённые солдаты с конвейера. В последнее время Рассу стали вверять всё 
больше зелёных новичков, так и не прошедших должных испытаний, но уже получивших 
доступ к <<взрослым игрушкам>>, таким например, как экзоброня в которую был 
облачён весь его отряд. Это последнее слово техники в индивидуальной защите 
обычно вверялось лишь опытным, прошедшим не одно сражение бойцам.

\noindent --- Эти парни хуже детей, --- лениво подумал Расс, доставая из 
потайного отсека брони продолговатую скрутку из высушенных листьев. Он знал, что 
раньше это называлось сигарой. В столице такие можно было купить за огромные 
деньги, но Рассу они достались в качестве трофея. Пять лет назад ему довелось 
поучаствовать в чистке одного городка на севере, и в древнем сейфе он обнаружил 
небольшую коробку, на которой было написано крупными буквами на древнем языке 
<<ВКУС ПЕРЕМЕН>>. Расс улыбнулся. Тогда ему было не до этого куска картона и 
сушёных листьев в нём, но что-то внутри настаивало захватить её с собой. Коробку 
здорово потрепало, и он выкинул её в тот же день, но вот содержимое надёжно 
спрятал. Потом, правда, долго выяснял, зачем же нужны эти странно пахнущие 
штуки. Он даже едва не продал их, какому-то Советнику в городе за баснословную 
цену, но в последний момент всё-таки не удержался и попробовал. Прямо перед 
сделкой. 

Свой первый вдох этой отравы он запомнил на всю жизнь. Бархатный дым мягко 
заполнил его лёгкие, но неблагодарные лёгкие попытались вытолкнуть его назад. 
Улыбка Расса стала шире. Глупый организм. Он настолько привык быть боевой 
машиной, что уже совсем забыл о прелестях жизни, которые были столь дороги людям 
прошлого. Сейчас везде кричат, о сгинувшем во тьме веков человечестве. О его 
наследии. О том, что всем им нужно стремиться возродить, то утраченное много лет 
назад. Стать прежними. Встать на путь истинный, вернувшись к истокам. Воздух 
настолько пропитался этой истеричной фанатичной идеологией, что от одного вдоха 
выворачивало на изнанку и хотелось харкать кровью. Но тот дым. Тот дым хоть и 
был настоящей отравой, пробудившей в Расе тогда жесточайший кашель, но его он 
мог вдыхать свободно. Этот дым. Он всё расставил по своим местам. Именно тогда, 
пять лет назад Расс впервые почувствовал себя человеком. Настоящим, хоть и с 
изрядной примесью этой жуткой штуки в крови, называвшейся эссенцией, и с изрядно 
извращённым исковерканным прогрессом сознанием, но все, же человеком. Пять лет 
назад. А ведь с момента его инициации, тогда уже прошло лет двадцать. Расс 
напрягся, заставляя себя вновь вспомнить испытанное.

Он давился кашлем, из глаз его текли слёзы, но всё равно делал новый вдох. 
Кашель звучал как гром. Всё после него стало таким громким, словно из ушей 
выбило пробки. А слезы, текшие из глаз, смыли, наконец застилавшую их пелену. 
Расс к своему величайшему изумлению плакал. Но он был счастлив. Первый раз в 
жизни.

На самом деле Расс внезапно понял, что курение единственная вещь в мире, которая 
делает его счастливым. Ни лживые идеалы его страны,  ни война, ради которой он 
был создан, ни друзья--ветераны, с которыми он бывал во многих сражениях. Даже 
собственноручно собранная им броня и оружие не давали ему такой уверенности в 
себе.

\noindent --- А должны ли? --- подумал Расс, --- мы все этим с рождения 
занимаемся. <<Броня для гвардейца вторая кожа, оружие же смертоносное 
продолжение его руки и разума...>> \\
--- Кажется, это проповедают клонам, едва они 
только начинают свой путь защитника истинного человечества, а по 
совместительству пушечного мяса? --- ветеран затянулся, пытаясь проникнуть в 
вырезанные участки памяти, --- нет, ни черта не помню. А жаль, хорошее, 
наверное, было время. Все, поди, казалось таким простым. Верь в идеалы, 
подчиняйся программе и будь счастлив.

И всё же. Его оружие. Его броня. И то и другое не раз спасало ему жизнь, но 
воспринималось им как данность. Это было частью его тела ни больше, ни меньше. 
Человек может жить без руки или ноги, и даже возможно при этом быть счастливым. 
Наверное. Возможно, стоит попробовать. В гвардии приветствуется протезирование. 
Бывали случаи, и ветераны сами отсекали себе руку, чтобы заменить её 
смертоносной игрушкой. Говорят, в бою вещь незаменимая.

По-хорошему Расса ничего не держало здесь. Он мог уйти, мог скинуть вторую кожу 
и бросить оружие, но тогда бы он, наверное, умер. Рано или поздно. Анклав - 
место, которое он должен называть своей родиной, поставил себя в ранг врага 
всего мира. Даже ожившие мертвецы с севера, готовы идти на компромисс и мирные 
переговоры, но только не упёртые фанатики из Совета. Если Расс уйдёт, то он 
умрёт. Всё было просто и очевидно, но от этого становилось только хуже. Анклав 
давал ему жизнь. Здесь он был как нищий калека, каждый раз вынужденный 
попрошайничать за ломоть хлеба, добыть который он сам не в состоянии. Его 
бросали из сражения в сражение. Вся его жизнь --- череда кровавых картин, 
переходящих из леденящего естество ужаса в кровавое безумие.

С кем они сейчас здесь воюют? Зачем? Потому что враг не такой как мы? Расс вроде 
бы знал, зачем и почему. Но эти ответы. Он порождали лишь новые вопросы. Он 
осознавал, что всё не так как кажется. Что мир хранит от него пугающее число 
загадок и тайн, о которых человеку его уровня известно быть не может. Но всё же 
он знал многое. А додумывал ещё больше. Всё-таки он не расходный материал, как 
те двое у него за спиной.

\noindent --- Вот вы ребята --- настоящее пушечное мясо, --- подумал Расс, --- 
хоть ваше снаряжение и выходит в копеечку. В конечном счёте вы все здесь 
сдохнете, но столь дорогие для страны оружие и броню вернёте даже с того света. 
Их вырвут из лап кого угодно, лишь бы отправить в утиль, спасти ресурс. А вы для 
них марионетки. Всего-то. Радует только, что сейчас вы по большей части мои 
марионетки, а значит, я могу вертеть вами как захочу. Ну, или почти. Всё-таки 
здесь есть кое-кто по званию повыше меня. Но с нынешним раскладом это 
ненадолго. 

Расс ухмыльнулся своим мыслям, откусывая кончик сигары. Эти трупы. Трупы у него 
под ногами. Девять штук. Разорванные тяжёлыми снарядами магнитных винтовок тела 
семерых дикарей. Двое убитых патрульных. Интересно\ldots

Под приятный для ветерана гул механизмов, заплечный огнемёт занял боевое 
положение. Небольшое мысленное усилие, и тонкая струйка алого пламени вырвалась 
из сопла. Чёрную броню окрасили багровые блики. Расс аккуратно подставил сигару 
под огонь. Серая струйка дыма потянулась к небу, но внезапный порыв ветра, 
нахлынувший с леса, унёс её прочь.

\noindent --- Сержант, это запрещено --- Снова этот глухой голос, почти ничем 
не отличающийся от предыдущего. Но звук исходил с немного другой стороны, 
благодаря чему Расс и распознал в говорившем своего второго подчинённого. Хотя 
разницы между ними не было почти никакой. Ветеран готов был поспорить, что 
думают эти два клона сейчас об одном и том же.\\
--- Знаю. Считайте это продолжением ритуала, --- сказал он, затягиваясь, ---- 
память усопших и всё такое\ldots\\
--- Командующий на связи, --- продолжил гвардеец, --- Транслировать?\\
--- Транслируй, --- проворчал Расс, разворачиваясь лицом к подчинённым. Из 
узкой прорези шлема одного из гвардейцев пробился синеватый луч, и перед 
отрядом появилось лицо генерала. \\
--- Чёртов старикан, похоже, до конца миссии, будет ставить меня в один ряд с 
этими болванками! Связывается через общий канал, да ещё и заставляет любоваться 
на его голографическую рожу\ldots\ --- Мысли об очередных проблемах с 
командованием несколько раздражали Расса, но он как обычно не подавал вида. 
Ветеран хмуро дымил сигарой и давил взглядом окружающий его лес.\\
--- В чём дело сержант? Почему прекратили движение? --- морщинистое лицо 
командующего слегка подрагивало. Усмехаясь про себя, Расс читал в глазах 
генерала волнение и тревогу.\\
--- Подтвердилось, --- коротко ответил ветеран, лениво выпуская облако дыма.\\
--- Уверен? --- дрожь пропала, теперь морщинистое лицо генерала осунулось, - 
что ты там откопал?\\
--- У меня два трупа под ногами. Патрульные. Срок два дня, --- ветеран 
прервался, чтобы сделать ещё одну затяжку, --- Сходится.\\
Генерал помрачнел\\

\noindent --- Не нравится тебе, --- думал Расс, вглядываясь в изрезанное 
морщинами лицо, --- вы, ребята из командования, постепенно выходите из моды. И 
это понятно, учитывая происходящее в мире. Вы теперь годны только чтобы, 
двигать солдатиков по карте, но в реальном бою, будь он на севере или юге, вы 
обгадите свою фамильную броню в два счёта\ldots \\
--- Картинку мне, --- потребовал генерал.\\
--- Простите? --- Расс сделал вид, что не расслышал. За последние пять лет, он 
стал мастером по игре на нервах, что было довольно редким явлением среди 
ветеранов гвардии. \\
--- Покажи трупы, --- повторил командующий.\\ 
--- Похоже, мне не доверяют?  \\
--- Во имя Мудрейшей! --- брови генерала сошлись на переносице, --- меня 
предупреждали, что с тобой будут проблемы, но я и представить себе не мог, что 
ты не в состоянии выполнять даже простейшие приказы! \\
--- Когда--то это называлось быть занозой в заднице, --- заметил Расс 
усмехаясь, --- жаль, что меня тогда не было. Уверен, та жизнь мне пришлась бы 
по вкусу\ldots\ Ладно, --- ветеран указал тлеющей сигарой в одного из 
подчинённых, --- покажи ему.
    
Когда изображение начало передаваться, всё что смог сделать генерал, это 
воскликнуть:\\
--- Ты что ещё и куришь, на задании!?

Расс лишь махнул рукой. Приятный скрип доспехов. Гвардеец, включивший камеру, 
перевёл взгляд на трупы, передавая на базу картину бойни.

\noindent --- Я повышу приоритет этого сектора у Чистильщика. Прибудет примерно 
через семь часов, --- генерал стал ещё мрачнее. Увиденное заставило его забыть 
о ненадлежащем поведении ветерана, --- Запись будет передана аналитикам, --- 
добавил командующий. В разговоре наступила пауза, которая могла означать только 
то, что генерал обдумывает ситуацию. Его морщинистое лицо приняло отстранённый 
вид. Расс молча, курил. Прищурившись, он вглядывался в тихий утренний лес, 
выпуская облака дыма. Решив что-то для себя, командующий посмотрел на угрюмого 
ветерана. Тот сделал последнюю затяжку и отбросил окурок в траву. Генерал устало 
вздохнул. \\
--- Думаешь, причина именно в этом? --- спросил он Расса.\\
--- Сомнений нет.\\
--- Но на их телах следы от оружия дикарей. \\
--- Верно. Они даже умерли, от этих ран. Но\ldots\ я не вижу в трупах эссенции. 
Это юг, и я могу отчётливо чувствовать её. Прошло всего два дня. Эссенция не 
могла рассеяться так быстро. Её что-то впитало. Этих парней выжали досуха. А мы 
оба знаем, кто охотится за тем, что течёт внутри нас. За тем, что отличает нас 
от людей прошлого, от этих варваров.

Генерал молчал. Расс посмотрел на его лицо, стареющее на глазах. В нём уже не 
было ни присущей командующим важности, ни пафоса. Лицо старика, с отпечатком 
страха.\\
--- Мне его почти жаль. Похоже, он сильно мешал кому-то в столице, раз попал 
сюда. С его-то заслугами и оказаться здесь, с нами\ldots\ --- думал ветеран, 
отделяя взором морщины от шрамов, на старческом лице генерала, --- Семь часов 
до чистильщика. Хорошо бы. Нет желания прибирать трупы одному\ldots

Генерал отдал короткий приказ, о продолжении выполнения текущей цели и 
отключился. Прежде чем изображение растворилось в воздухе, Расс успел разглядеть 
отчаяние на лице старика. Что ж, вполне адекватная реакция на происходящее. Мир 
продолжает медленно, но верно катиться ко всем чертям. Расс уже внутренне 
подготавливал себя к тому, что скоро и его приятели по ремеслу, примерят на себе 
маску отчаяния. Хотя вряд ли. Такое выражение лица не для них.

Ветеран постоял ещё немного. В его голове начали всплывать воспоминания, всех 
тех ужасов, через которые он прошёл когда-то. Наступало время для очередного 
кошмара. Кошмара имеющего все шансы стать последним.

Расс сбросил оцепенение. Его взгляд остановился на тлеющем окурке в траве.

--- Одной сигары, всегда, было мало, --- заключил он.

Под металлический скрежет брони, ветеран забросил на плечо свой тяжёлый Мэгг и 
зашагал в сторону леса.

\newpage

\begin{mssg}{Запись 306\\
\\
Датировано 256 годом со дня падения Человечества\\
Отправитель: Сарн Ветеран Восточного Фронта\\
Получатель: Расс Ветеран Восточного Фронта Выжигатель\\
Тема: Результаты исследования\\
\\
Текст сообщения:
}%
Прибыл в столицу. Количество новостей поражает. Большая часть очень похожа на 
обычные слухи и выдумку, которые тебе наверняка известны. Но кое-что я просто не 
могу не удостоить твоего внимания.\\
\\
Один из Карателей поделился со мной сведениями, что племена дикарей 
активизируются как на юге, так и на севере. Самые крупные из них начинают 
объединяться в военные союзы. Это означает что ситуация обостряется, и те 
угрозы, которых мы опасаемся, становятся всё более реальными.\\
\\
Теперь о деле. Как и договаривались, я подверг себя эссенциальному 
воздействию. Жрицы провели операцию выверено и хладнокровно, что показывает их 
большой опыт.\\
Итак, я забыл всё через что прошёл в Каньоне Смерти. Это напоминает мне одну 
фразу из книг, которые ты мне давал. «Вертится на языке». Вокруг моих 
воспоминаний словно выстроена стена, и я могу сколь угодно долго ходить вокруг 
неё, но дверь мне найти не под силу. Я брожу около кромки этого кошмара, но 
никак не могу погрузиться в его тёмные воды. Похоже, Жрицы блокируют потоки 
эссенции у меня в голове, делая некоторые участки моей памяти недоступными. 
Уверен эти барьеры можно сломать, но для этого нужен достаточный уровень силы 
воли и управления эссенцией в своём теле. Кроме того пока остаётся 
неопределённым, почему эту методику используют так широко и открыто. Ведь 
эссенциальное воздействие такого порядка, более опасно чем, внушение или 
извлечение, которые применяют Покровители. Я постараюсь поработать над этим 
вопросом, но думаю, здесь тебе следует подключить свои связи.
\\
Большое спасибо за присланные тобой чертежи. Теперь я не знаю наверняка, но всё 
равно уверен, что интегрирование в мою магнитную винтовку огнемёта, не раз 
спасло мне жизнь в Каньоне. Без тебя я бы точно не успел собрать оружие к сроку. 
Ещё раз спасибо.\\
\\
Сарн\\
Конец сообщения.
\end{mssg}

\noindent --- Паршиво выглядишь приятель, --- Расс получил ощутимый удар локтём 
по рёбрам, которым в последнее время и так нелегко жилось, --- жрице твоей 
наверняка пришлось здорово тебя подлатать. Как она кстати?

Двое высоких статных мужчин в форме шагали в полумраке узкого коридора. В этом 
крыле базы почти никого не было. Для большинства персонала вход сюда был 
запрещён. Лишь тем, кто работал непосредственно с такими как Расс, разрешалось 
находиться в этих стенах из бетона с начерченным языком пламени в кругу через 
каждый десяток метров.

\noindent --- Её отделение неделю назад перевели на дальнюю базу на севере. 
Связи нет уже два дня, --- мрачно бросил Расс. Пронаблюдав, как улыбка сползла с 
лица собеседника, он продолжил, --- На данный момент нас обслуживают жрицы 
непосредственно этой базы. Меня штопали в робкой тишине под испытующим взором 
Смотрящей, --- Расс позволил себе усмехнуться, --- От неё я, кстати, получил 
информацию, что скоро обещают прислать для нас новое отделение из Корпуса 
Милосердия. Так что не беспокойся на этот счёт, Тару.
    
Но Тару волновало не совсем это. И Рассу было об этом известно.

Они подошли к внушительным воротам из стали. Расс махнул рукой перевитой тугими 
мускулами в ту сторону где, как он знал, скрывалась камера. Створки ворот перед 
ветеранами послушно разъехались в стороны, скрывшись в стенах.

За воротами коридор стал шире и лучше освещён. Символ на стенах сменился на 
клинок с каплей крови. Здесь было несколько многолюднее. Двое часовых на посту 
отсалютовали ветеранам и снова замерли с оружием в руках. Впереди за поворотом 
скрылся отряд гвардейцев. Проигнорировав приветствие часовых Расс прервал 
возникшую в разговоре паузу, продолжив его с наиболее значимого для Тару места.

\noindent --- Север. Ты знаешь, что это значит. В командовании уже сформировали 
отряд. Они предполагают, что это очередная атака воплощений. Что справедливо. И 
шансов, конечно же, мало. Два дня\ldots\ --- Расс многозначительно замолчал.\\
--- Слишком много для устранения неполадки со связью, и слишком мало, для того 
чтобы выжить, - закончил за него Тару. Ему оставалось лишь хмуро добавить, --- 
знаю. \\
--- Знаю, что знаешь, но ведь желание прогуляться на север не отпадает. Верно?
    
Они свернули там же где и колона гвардейцев. Но дальше их с клонами маршруты 
вряд ли совпадали. Командование вызвало ветеранов в штаб.

Расс время от времени косился на собеседника. Плохие новости, всегда вгоняют 
Тару в состояние, слишком сильно схожее с ухудшением настроения. Странное дело 
для человека его ремесла. Впрочем, это база. Вне этой клетки из бетона и стали, 
Тару гораздо больше был похож на того, кем он был на самом деле. Здесь же в 
окружении всех этих людей, спокойно занимающихся своими делами, где не нужно 
всматриваться в кромешную тьму между деревьями, ожидая удара в спину или стоять 
под градом смертоносного металла, у Тару в голове начинали появляться всякие 
посторонние мысли. И Расс прекрасно был знаком с этим. Тоже общение со жрицами – 
их терапия, применявшаяся к нему между заданиями, была жалкой попыткой вытащить 
ветерана из мира бесконечного кошмара в мир обыденный. Но Расс слишком хорошо 
понимал, что это уже невозможно. То, что он видел и пережил, и то, что ему 
предстоит увидеть и пережить --- оставляет на человеке неизгладимый отпечаток. 
И жить после этого как все он точно не сможет. И Тару тоже должен был это 
понимать.

Чужие среди своих и свои среди чужих.

\noindent --- Ты меня знаешь Расс. Лучше уж на Север, в объятия самой смерти, 
чем бродить по смердящим лесам, выискивая деревушки этих жалких дикарей, --- 
Тару стал совсем серьёзен, --- Да и к тому же, я бы предпочёл быть как можно 
дальше от них.\\
--- В это сложно поверить, --- снова ухмыльнулся Расс, --- после таких слов 
тебя с трудом можно назвать выжигателем. Собираешься тащиться неизвестно куда, 
неизвестно зачем, да ещё и боишься.\\
--- Ты ведь знаешь, они меняются, --- доверительным голосом произнёс Тару, --- 
и причём, неимоверно быстро. Я всё чаще слышу, что Каньон Смерти уже не может 
защитить ни Анклав, ни тем более весь Север. Что-то происходит и уже не там, на 
границе, а здесь. Прямо у нас под носом, --- закончил он.

Настала очередь Расса хмуриться. До него тоже давно стали доходить тревожные 
вести с дальнего юга. Пока он сомневался в их достоверности. Но все, же не 
следует забывать, что если речь идёт о них, то возможно всё.

\noindent --- Поэтому нас сейчас и собирают? --- сказал Расс, скорее утверждая, 
нежели спрашивая.

Тару кивнул.

\noindent --- На брифинге будут не только рядовые члены Культа, но даже 
некоторые Властители. В свою очередь сверху выделили людей, по званию близких к 
Командующему Южного Фронта. Явный признак того, что дело связано с чем-то 
серьёзным.

Они обменялись многозначительными взглядами. Расса мучило желание закурить, но 
вне крыла Выжигателей это запрещалось. Мириться с возрастающим напряжением ему 
предстояло один на один, без помощи раскрепощающего дыма. Тару же пытался 
сбросить напряжение, делясь всей информацией, что знал сам.

\noindent --- Сюда стянули пушечного мяса с ближайших баз. Будут маскировать 
операцию под обычную чистку дикарей. Но в деле обязательно присутствие одного из 
нас. Возможно больше. Зачем пока не знаю, но и имеющихся данных хватит, чтобы 
заключить, что происходит что-то необычное.

\noindent --- Держать Выжигателей на юге --- вот что уже необычно само по себе, 
--- подумал Расс, --- с дикарями отлично справляются Каратели и присутствие 
здесь ещё и нас, явно излишне.

Но они всё-таки были здесь. Уже несколько месяцев как. Расс, Тару и ещё пара 
членов Культа Пламени. За это время ему довелось поучаствовать лишь в двух 
карательных походах на варваров. Оба были непримечательными для Расса 
мясорубками. Остальную часть времени он торчал на базе, беседуя со жрицами о 
вере и идеалах Человечества. Такие разговоры вгоняли Расса либо в тоску, либо в 
сон. Так что неудивительно, что Тару такой. Его все эти беседы видимо задевали 
за что-то живое, если оно у него ещё осталось.

Да и недели безделья тоже кого хочешь с ума сведут.

Но вот, наконец, наступает момент истины.

На юге действительно что-то происходит. И ему, Рассу, предстоит с этим 
разобраться.\\
Они обсуждали с Тару последнее задание Расса, пока лифт спускал 
их на этаж командования. Больше всего внимания в рассказе об очередной чистке 
варварского племени привлекло описание огромного демоноподобного дикаря 
разорвавшего напополам Карателя. На вопрос как же Рассу удалось убить тварь, тот 
лишь ухмыльнулся.

\noindent --- Выжигатели не выдают своих секретов, - ответил он, оскалившись, - 
но с тобой я, может быть когда-нибудь, и поделюсь своим опытом.\\
--- Поскорей бы! --- Тару улыбнулся в ответ, --- эти бездушные не горят с моём 
пламени. Я уже и не знаю что делать!\\
--- Ты всё ещё веришь во всю эту чушь с душами? --- Расс бросил в сторону 
собеседника скептический взгляд. Как некстати, тот упал прямо на ожерелье из 
костей, болтавшееся на шее у Тару. Расс скривился, --- разговоры со жрицами 
плохо на тебя влияют.\\
--- Не начинай, ---  отмахнулся Тару. Эту тему они обсуждали не раз. Обмен 
мнениями на счёт религии Анклава и выжигателей, в затяжное тоскливое время 
между заданиями, было одним из их основных развлечений помимо ухода за бронёй и 
оружием.

Выжигатели в отличие от других ветеранов Гвардии были весьма независимы во 
многих аспектах жизни. Их верования и мировоззрение зачастую даже шли вразрез с 
идеалами Анклава. Тару верил в души. Такая точка зрения поддерживалась Жрицами и 
Покровителями. Они считают что эссенция --- особая материя, текущая в жилах 
народа Анклава, наделяет его <<душой>>. Часть оружия, которым вооружались 
Выжигатели и другие ветераны, была создана для уничтожения этой самой <<души>>. 
И это оружие не оказывало ровным счётом никакого эффекта на дикарей, ведь 
внутри них не было эссенции, а значит и <<души>>. Анклав боготворил технический 
прогресс и научный подход, так присущие падшему когда-то Человечеству. И 
конечно официально отрицалось существование душ, духов и любой другой мистики. 
Но для фанатичного идеализма, которым были охвачен народ Анклава, называть 
людей отличных от них бездушными было очень удобно. Физическое различие, 
подобное цвету кожи или разрезу глаз, становилось причиной для вознесения одних 
над другими. Людям становится на удивление легко ненавидеть друг друга. И если 
эту ненависть направить в нужное русло как это делал Анклав, можно добиться 
весьма впечатляющих результатов.

Единственное чего не понимал Расс, как Тару повёлся на всё это. Культ Пламени 
проповедует силу очищающего пламени, которое оставляет лишь золу истины. И 
какими бы разными и самобытными Выжигатели не были, их всех отличало то, что они 
всегда пытались смотреть правде в лицо. Хотя может быть самобытность Тару как 
Выжигателя и заключается в этой странной наивности.

Оставшееся время мерного движения лифта вниз они провели молча.

На этаже командования была подготовлена тёплая встреча. В распахнутую кабину 
лифта устремились безжалостные лучи прожекторов, заставляя ветеранов прикрыть 
глаза руками.

После кровопролитных диверсий баз на Севере, было принято усилить их защиту 
повсеместно. Сейчас перед ослеплёнными Рассом и Тару должен был быть прекрасно 
укреплённый кордон, с дюжиной гвардейцев и тяжёловооружённым огневым расчётом. 
Ветераны даже могли услышать мерное гудение гвардейских Мэггов, готовых в любой 
момент изрыгнуть шквал смертоносных игл.

\noindent --- Мои извинения, ветераны, --- грубый металлический голос гремел 
словно гром, --- стандартная процедура.

Лучи прожекторов оставили кабину лифта в покое и Расс наконец смог оглядеться. 
Они с Тару были под прицелом у десятка вооружённых тяжёлыми Мэггами гвардейцев 
облаченных в облегчённую экзоброню. У всех на левой наплечной бронепластине был 
изображен, пронзённый мечём череп.

\noindent --- Ты говорил, что пушечное мясо собрали с ближайших баз, но 
насколько я помню, Северный Фронт лежит на другом конце Анклава, --- спокойно 
произнёс Расс. Тару был в замешательстве.\\
--- Может их транспортируют из Каньона Смерти через нас?\\
--- Из Каньона отряды направляются прямиком в столицу. Я не помню, чтобы хоть 
раз наш отряд совершил где-то остановку.\\
--- Я тоже. Но тогда как\ldots

Их разговор прервал всё тот же металлический голос.

\noindent --- Вас уже ждут в центре связи. Вас сопроводят, --- в говорившем 
можно было распознать ветерана Северного Фронта. Он возвышался рядом с 
остальными гвардейцами закованный в полную чёрно--серебрянную экзоброню. Его 
режущий слух голос раздавался из-под трофейного черепа, искусно закреплённого 
поверх стандартной боевой маски, к которой тянулось множество кабелей и тросов. 
Правую бронепластину заменяла схожая с человеческой грудная клетка, так же 
искусно интегрированная в броню. Место одного из наколенников занял ещё один 
череп, но уже явно не человеческий.

Ветераны Северного Фронта свято соблюдали традицию, заведённую ещё после Войны 
Смерти. Кости и черепа, поверженных врагов использовались ими как средство 
устрашения. Ходили слухи, что самые испытанные ветераны вместо штандартов, 
крепили за спиной целые костяки. Вокруг таких методов ходило огромное число 
споров, как в командовании, так и в кастах Защитников и Советников. Но, 
насколько было известно Рассу, совсем недавно Мудрецы--учёные смогли доказать 
правомерность таких <<украшений>>. Мёртвые останки, добытые ветеранами в бою, 
действительно являли собой нечто схожее с действием оберега каких--нибудь 
дикарей. Эманации эссенции, которые излучаемые этими костями, могли отпугнуть 
воплощения --- одних из самых опасных врагов Анклава на Севере. Выжигатели так 
же иногда использовали части тел поверженных противников, но делали это из 
несколько других соображений.

Оставшуюся часть пути ветераны проделали, молча, в сопровождении двух 
гвардейцев с Севера. Расс раздумывал над причинами появления здесь столь 
неожиданных гостей.

Выжигатели уже давно установили связь между воплощениями и тем, что Анклав 
всеми силами пытается удержать в Каньоне Смерти. Ветераны Северного Фронта, 
закалённые во множестве сражений солдаты, уже давно привыкли подавлять свой 
страх. Когда сталкиваешься с воплощением лицом к лицу бояться --- значит 
умереть. Схожий приём используется и в противостоянии с ними. Так не присутствие 
ли наиболее подходящих бойцов свидетельствует о наличии рядом их главного 
врага? Командованию было что-то известно. И уже давно. Оно будет пытаться 
использовать Расса вплоть до самого последнего, скрывая правду. А значит, он 
должен быть готов к самому худшему.

Всегда быть начеку. Выжить любой ценой.

\newpage

\begin{mssg}{%
Запись 310\\
\\
Датировано 256 годом со дня падения Человечества\\
Отправитель: Расс Ветеран Восточного Фронта Выжигатель\\
Получатель: Сарн Ветеран Восточного Фронта\\
Тема: Продолжение исследования\\
\\
Текст сообщения:
}%
Благодарю за смелость, а так же прошу выполнить ещё одну просьбу. Пока ты 
находишься в столице, то можешь посетить одного моего знакомого из касты 
Мудрецов. Он располагает средствами для проведения анализа эссенции в твоём 
теле, что поможет выяснить тонкости блокировки памяти. Если захочешь, то вполне 
возможно он поможет тебе снять барьер. Я в свою очередь обещаю поделиться 
информацией и своими догадками по данному исследованию.\\
\\
Итак, если согласишься, то присылаю тебе необходимые координаты и коды.\\ 
\\
Ещё раз поблагодари своего Карателя, за информацию.\\
\\
И ещё кое-что. Несмотря на то что, твой Мэгг после слияния с огнемётом стал 
более массивным и прочным на вид, не следует пытаться остановить им топор вождя 
какого-нибудь варварского племени. Удар наверняка выведет из строя регуляторы 
оружия, что существенно понизит точность. Так же станет невозможной генерация 
вспомогательного магнитного поля, что в свою очередь значительно уменьшит 
дальнобойность. Огнемёт при этом будет работать исправно, но эссенциальное 
пламя, к сожалению, не очень эффективно в сражении с вождями племён. На починку, 
скорее всего, уйдёт половина недели.\\
\\
Проверено на личном опыте.\\
\\
Расс\\
\\
Конец сообщения. 
\end{mssg}

\noindent --- Эти свежие, --- Расс стоял над телами двух убитых гвардейцев, 
выпуская облака сизого дыма. Он помянул павших, и теперь выкуривал самокрутку 
собственного изготовления. Рецепт ему удалось выведать у дикарей из одного 
северного племени, а на юге было предостаточно ингредиентов.\\
--- И всё пока на месте, --- задумчиво проговорил ветеран, изучая взглядом 
трупы. \\
Их маленький отряд вновь замер на небольшой поляне. Расс затянулся и нехотя 
перевёл взгляд на ближайшие заросли какого-то громоздкого кустарника.\\
--- Выходите, прятаться нет нужды, я давно вас почувствовал, --- лениво бросил 
он, --- к тому же здесь все свои.

Подлесок за спиной Расса затрещал, и оттуда вышло трое гвардейцев, облачённых в 
такую же экзоброню как и на мёртвецах. Стволы Меггов были направлены на 
выжигателя и его подчинённых.

\noindent --- Программа распознавания не получила от вас ответа. У вас и ваших 
клонов отключены маячки! --- раздался гулкий голос из-под маски одного из 
троицы, по-видимому, их ветерана командира, --- к тому же вы совершаете 
несанкционированные действия на задании.

Расс усмехнулся. Называть курение несанкционированным действием на задании? Это 
можно счесть неплохой шуткой. Хотя уже порядком изъезженной.

\noindent --- Ваши клоны не следуют боевым протоколам, --- перечисление 
нарушений Расса продолжалось, --- не было произведено протоколирование смерти 
солдат Анклава\ldots

Расс устало отбросил окурок в сторону и развернулся к лицом, ко всё ещё 
отчитывающему его ветерану. Избавив свои лёгкие от остатков дыма, выжигатель как 
всегда отметил, что кустарная дрянь невероятно забористая, но с сигарами, ни в 
какое сравнение не идёт. Бархатный вкус свободы был всего один.

\noindent --- Пренебрегаете защитой головы, --- отсутствие на покрытом шрамами 
лице боевой маски было тут же отмечено.\\
--- Интересно, как же я буду курить через слой стали толщиной в палец? --- Расс 
был уверен, что вопрос будет проигнорирован. Так оно и вышло.\\
--- Вы Выжигатель, --- прогудело из-за маски командира, --- многочисленные 
нарушения базовых протоколов и несоблюдение субординации. Мне было бы проще 
догадаться, если бы мы сейчас были не на юге.\\
--- К счастью именно для этого у меня на плече и красуется соответствующий 
символ, --- съязвил Расс, --- и ещё он там начертан для того, чтобы никто из 
верных слуг Анклава не наводил мне на грудь ствол магнитной винтовки.\\
--- Что вы здесь делаете? --- невозмутимость командира, не знала границ. Уж не 
решил ли ветеран решил поиграть с выжигателем в допрос. Расс решил настоять на 
своём.\\
--- Мне будет проще отвечать, если вы опустите оружие.

\noindent --- Отказано. На данный момент вы расценивайтесь как потенциальный 
предатель и дезертир, --- такое серьёзное обвинение могло обернуться не менее 
серьёзными последствиями. У членов Культа Пламени и так было шаткое положение в 
обществе. Постоянный надзор со стороны Покровителей, только из-за одного доноса 
мог превратиться в арест. 

Расс отбросил шутки в сторону.

\noindent --- И какие же имеются основания полагать такое?\\
--- Система не может распознать вас. У вас отключен маячок, --- Расс это уже 
слышал. Да, экзоброня гвардейцев распознаёт друг друга в пределах видимости 
вспомогательных камер, чтобы составлять адекватную картину <<друзья/враги>>. 
Кроме того по запросу могут быть высланы данные о её носителе. Возможно, 
предатели и дезертиры и отключают маячки, но сейчас выжигателю было плевать на 
это. Он собирался обеспечить себя максимальной свободой действий. А если о его 
передвижениях будет известно, достичь этого будет сложно.\\
--- Маячок теперь без надобности, --- ответил Расс и тут же спросил сам, 
отметая все попытки со стороны командира задать очередной вопрос, --- Сколько 
вам ещё поступило сигналов о смерти?\\
--- По моим данным почти все патрульные группы были уничтожены, - неохотно 
раздалось из-за маски. Расс кивнул, услышав подтверждение своих мыслей.\\
--- Корректировка действий из лагеря?\\
--- Не поступала, --- последовал ответ.\\
--- Буду удивлён, если связь с ними не пропадёт в течение десяти минут. --- 
Расс был уверен, что теперь невозмутимость ветерана дала брешь, --- как вы 
охарактеризуете трупы?\\
--- Смерть явно насильственная. На телах несколько открытых ран нанесённых 
холодным оружием кустарного производства. Скорее всего, использовали топоры. 
Судя потому, что следов борьбы практически нет, атака была произведена внезапно. 
Тактика очень похожая на дикарей, --- заключил ветеран. Но у Расса было 
припасено ещё несколько вопросов.\\
--- Почему здесь только два тела? Где командир этих клонов. И есть ли на него 
информация?

\noindent --- С экзоброни трупов можно получить необходимые данные, --- голос 
из-за маски прервался на мгновение, --- командир отряда --- старший сержант 
Маннос ветеран Южного Фронта. Месторасположение определить не удалось. Маячок 
отключен.\\
--- Вот и ещё один предатель и дезертир. Бросил подчинённых, отключив перед 
этим боевые протоколы и маячок. Как умно, --- проворчал Расс.\\ 
Паршиво. К уравнению добавлялось всё больше и больше неизвестных.\\
--- Эссенция всё ещё в трупах. Нужно уходить. Мы и так здесь задержались, --- 
выжигатель потерял интерес ко всяческим разговором. Пришла пора действовать.\\
--- Получён ещё один сигнал о нападении на патрульных, командир, --- прогудел 
один из подчинённых ветерана, --- жду приказов.\\
--- Нужно отступать в лагерь. Дикари хорошо подготовились, раз так легко 
вырезают наши разведгруппы. Командование наверняка прикажет перегруппироваться 
и зачистить этот район. Возвращаемся.

Таким приказам Расс подчиняться не собирался.

\noindent --- Похоже, здесь кто-то не очень хорошо понимает, что 
происходит\ldots\ --- начал он.\\
--- Да, и это ты выжигатель, --- с вызовом бросил ветеран, --- Я выше тебя по 
званию и ты, и твои люди обязаны мне подчиняться. Ты возвращаешься в лагерь 
либо как верный солдат Анклава, готовый служить Человечеству, либо как 
изменник, попытавшийся сбежать! Выбор за тобой.

Брови Расса сошлись на переносице, а изуродованное шрамами лицо само по себе 
стало напоминать жуткую маску. Последователь Культа Пламени начинал злиться.

\noindent --- Тебе не стоит пытаться угрожать мне солдат. Я выжигатель, и служу 
Анклаву так, как сам считаю нужным, --- Расс был на голову выше благодаря своей 
массивной модифицированной броне, и сейчас он взирал на ветерана сверху вниз. 
Гневный взгляд выжигателя казалось, проплавит в маске командира дыру. Дуло 
магнитной винтовки прямо пред своим лицом служитель Культа словно бы не замечал.

\noindent --- Я слышал, что выжигателям достаточно посмотреть на своего врага, 
чтобы обратить его в кучку пепла, но даже если эти слухи правдивы, поверь, я 
успею заставить тебя пожалеть о вовремя не надетой маске!

За спиной Расса раздалось гудение Мэггов. Проклятый ветеран перехватил 
управление клонами. Он действительно старше выжигателя по званию. Подчинённые 
командира тоже взяли потенциального предателя на прицел. Расс был окружён.

Но испугать последователя Пути Огня, участвовавшего в сотнях сражений и 
прошедшего через все круги ада, было не так легко.

Они стояли друг напротив друга. И не смотря на все нацеленные на него винтовки, 
Расс не сделал ни единого движения. На его лице не дрогнул ни один мускул. Ему и 
не нужно было ничего делать. Он просто сказал:

\noindent --- Мы можем простоять так ещё десяток минут, и тогда мне даже не 
придётся убивать вас самому.\\
И этого оказалось достаточно.\\
С ледяной яростью выжигателя мало что могло сравниться. И мало кто мог 
устоять перед его горящим взором. Ствол магнитной винтовки опустился.\\
--- Они тоже, --- прохрипел Расс, не меняя выражения лица.\\
Клоны подчинились. Ветеран и выжигатель продолжали смотреть друг на друга.\\
--- Я пойду с тобой в лагерь.

\newpage

Мы все из одного племени. Мы все когда-то были безличной серой массой. Клонами. 
Мы жили, так как нам велели навязанные нам законы, через которые мы были не в 
силах переступить, и умирали за то, за что нам приказывали умереть. Те, кому 
повезло выжить, прожив эту счастливую простую жизнь, те, кого сочли достойными, 
прошли инициацию. В награду за службу Анклав подарил им личность, подарил им 
душу. Им позволили выбрать себе имя, заменив кодовый номер. Им позволили выбрать 
себе новое лицо, заменив прежнее одинаковое для всех. Они переродились, стали 
Ветеранами. Элитой.
\\

Полноценными членами общества. Но ничего не изменилось. Всё что они могли 
делать, это снова идти на смерть, но, уже не понимая во имя чего. Они всё так же 
оставались оружием в руках всемогущего Анклава. Они умирали в бесконечных войнах 
и конфликтах, не в силах обрести другую жизнь. Они всё так же оставались рабами.
\\

\noindent Властитель Огня Эсхо.

\newpage

\noindent --- Враг наступает! Оружие к бою! Ветераны Гвардии занять позиции!
    
Чёрная волна закованных в экзоброню воинов прильнула к окопам и укреплениям. 
Их подняли на рассвете, едва разведчики засекли движение на подступах к первой 
линии обороны. Слева на востоке огромной выжженной равнины кроваво-красным 
заалел горизонт. Лишь в одном месте в Ксерне можно увидеть багровую зарю, так 
похожую на закат.\\

Каньон Смерти.\\

На юге во тьме, которой всё ещё была охвачена линия горизонта, появились тени. 
Расс отвлёкся от проверки снаряжения, и принялся всматриваться туда, где всё ещё 
царила ночь. Горизонт притягивал к себе взгляды. Треугольник рассвета, 
надвигающийся с востока, плавно отсекал звёздное небо от поверхности земли. На 
юге оно неохотно меняло чёрный цвет на тёмно--зелёный. Но скоро лучи света 
дотянулись и туда. Над горизонтом впереди поднималась золотая дымка. Но тьма 
исчезать не желала. Наоборот, в то время как очистившееся небо наливалось 
краской, по земле растекалась чернота.

\noindent --- Ползут, --- донеслось до Расса справа. Он повернул голову и 
первым делом увидел начертанный на наплечнике кулак в круге --- символ Гвардии 
Восточного Фронта. Ветеран не использовал маску, здесь на Каньоне Смерти это 
допускалось. Их враг, отнюдь не стремился пробить броню. От его оружия не было 
защиты. 

Ветеран с Востока, щурясь от солнца, нервно вглядывался вдаль, постоянно 
повторяя про себя какие-то слова. В нём читались волнение и страх. Здесь всегда 
было так. Сам воздух Каньона Смерти был пропитан тревогой и напряжением. Ожидать 
нападения каждую секунду было невыносимо.

Расс снова перевёл взгляд на юг.

И не поверил своим глазам. Сколько бы он не слышал историй о бескрайней Зелёной 
Волне, надвигающийся с дальнего юга, увидев её наяву, он всё равно не мог 
поверить своим глазам. Все воспоминания переживших этот кошмар блекли перед 
жуткой действительностью.

Огромный шелестящий ковёр, казалось, накрыл собой всю равнину от края до края.

Несокрушимая Зелёная Волна надвигалась, заняв собой горизонт. Десятки сотен 
тысяч невообразимых существ так непохожих на человека, устремлялись вперёд в 
едином животном порыве. Ведомые примитивным инстинктом они ползли вперёд, вселяя 
неведомый доселе ужас в сердца всего живого разумного. Расс попытался сосчитать 
их, но даже не смог различить, где начинается одна тварь и заканчивается другая. 
Огромная шевелящаяся волна готова была накрыть их с головой. Она заполнила собой 
всё. Весь горизонт от края до края, насколько хватало глаз. Смерть. Смерть 
встала перед Рассом во весь рост.

Он понял что дрожит.

\noindent --- Не бойся, --- голос шёл откуда-то слева.

Как же было странно слышать эти слова тому, кто не раз рисковал своей жизнью, 
пройдя через множество тяжёлых боёв на Востоке и Юге. Но до этого момента Расс 
даже не представлял, что может быть так страшно. Враг, которого невозможно 
победить, одним своим видом приводящий в бессилие. И жуткая участь, ожидавшая 
глупцов вставших у него на пути.

\noindent --- Чтобы выжить здесь, ты должен забыть о страхе, --- продолжил 
голос, --- они хищники. Они питаются нашими эмоциями, нашими чувствами, тем, 
что у нас внутри. От нашего страха у них лишь разыграется аппетит, усилится их 
и без того бесконечный голод. Ты должен перестать бояться. Перестать 
чувствовать. Очиститься. Лишь так ты сможешь пережить этот день.

Он говорил уверенно и спокойно. Расс не видел его символа на плече и мог лишь предположить, что, наверное, он был с Севера. Он тоже был без маски. В огромной экзоброне, и с тяжёлой винтовкой в руках, он стоял и спокойно смотрел в лицо приближающейся смерти, не отводя взгляда. В его глазах пылал лишь огонь.

Динамики внутри шлема Расса ожили.

\noindent --- Солдаты Анклава! Защитники Человечества! Пришла ваша пора отдать 
долг перед родиной!

Там далеко за спиной Расса в хорошо укреплённом бункере их вдохновляли на победу. Но даже в этом голосе какого-нибудь генерала, можно было различить страх. Только в одном месте в Ксерне ты никогда не сможешь почувствовать себя в безопасности, и не важно, где ты находишься, на поле боя или в окружении толстых стен и могучих солдат. Здесь страх найдёт тебя везде.

Здесь.

В Каньоне Смерти.

\noindent --- Ветераны приготовиться! Огонь открывать по команде! --- донеслось 
из динамиков.

Расс покрепче сжал Мэгг.

\noindent --- Целься!

Винтовка приятно загудела, набирая снаряд для магнитного поля. Тяжёлые иглы 
одна за другой перекочёвывали из обоймы в ствол Мэгга. Оставалось только дать 
программе управления оружием короткий приказ, и винтовка изрыгнёт из себя 
смерть. Расс прицелился, прижав приклад к плечу. Это конечно не имело смысла, 
промахнуться, когда на тебя надвигается живой ковёр голодных тварей, было 
невозможно.

Волна подбиралась всё ближе, но приказа не поступало. Расс почувствовал, как по 
лицу струится пот, но, ни волнения, ни страха почему-то уже не было. Он 
покосился на Ветерана с Востока, и заметил, что тот перестал бормотать. Он 
держал свой Мэгг с отсутствующем выражением лица, и руки у него больше не 
тряслись. Расс хотел посмотреть налево, но не успел…

\noindent--- ОГОНЬ!

Бой начался.

С шипением из винтовки вырвались копившиеся внутри иглы. Шквал холодной стали 
устремился вперёд, предвкушая кровавую жатву. Несокрушимая Зелёная Волна 
натолкнулась на стену. Огромная коса прошлась по первым рядам тварей сметая их 
под чистую. Все эти причудливые клубки из лоз и лиан, так жадно стремившиеся 
добраться до вожделенной эссенции внутри Расса, крошило на куски. Тяжёлые иглы 
прошивали сразу несколько тел, жалкая органика не могла остановить свистящую 
сталь. Ветераны стреляли из окопов и Волна остановилась.

А потом начали стрелять те, кто был за укреплениями.

И Волна откатилась.

Металл разрывал зелёный ковёр на куски, подхватывал лоскуты и нёс его над землёй 
обратно к горизонту.

Они должны были ликовать, но над полем боя царило безмолвие. Лишь гудение 
винтовок и шелест вылетающих игл нарушали эту мёртвенную тишину.

И тут Расс понял, что твари уже добрались до них. Его уже едят, пожирают заживо. 
Чувства, эмоции, всё исчезло. Остался лишь сбитый с толку напуганный разум, 
забившийся в угол и равнодушно бросивший Расса на произвол судьбы.

\noindent --- Стрелять солдаты! Стрелять! Не прекращать огонь! --- кричали 
динамики.

И они стреляли. И не прекращали огонь. Тупо и отстраненно. Пока на поле перед 
ними не осталось ничего. Лишь зелёная обездвиженная биомасса, когда-то так 
желавшая добраться до них.

\noindent--- Кажется, мы их отбросили, - лениво подумал Расс.

И тут тишину очистившуюся, наконец, от звуков стрельбы разорвал крик. В него, 
казалось, были вложены все чувства и эмоции, которых лишился Расс и другие. Он 
резал не только слух, но и сознание, переворачивая всё с ног на голову.

\noindent--- ВОЗДУХ!

В небе над полем боя что-то промелькнуло. Расс поднял голову и увидел, как 
множество снарядов, рассыпаясь в небе на сотни кусочков, исчезает за насыпью.

\noindent --- Проклятье, они же нас\ldots\ бомбят, --- послышался чей-то голос.

И тут мир взорвался звуками. Из-за укреплений позади Расса начали раздаваться 
истошные крики ничего общего не имеющие с человеческими. Так кричат загнанные 
звери, которых вот-вот разорвут на части. Тут же их сменили крики, разрываемых 
на части людей. Затем к ним присоединились звуки беспорядочной стрельбы.

Динамики разом заговорили десятками разных голосов. Просьбы о подкреплении, 
сменяло гудение Мэггов и предсмертные хрипы.

\noindent --- Нас отрезали от основных сил! Пришлите подкрепление к зелёной 
метке! --- к каналу подключалось всё больше и больше командиров групп. Расс 
понял, началась паника.\\
--- База у нас потери. Отходим в каньон.\\
--- Отойди от них, они заражёны! Пшшш\ldots\ --- звуки стрельбы, --- Все назад! 
К воротам! Отступаем!\\
--- База мы попали под обстрел заразителей! Повторяю! Нас обстреливают 
заразители! Требуется перегруппировать силы! Немедленно!

Небо прочерчивали всё новые и новые снаряды.

\noindent --- В перегруппировке отказано! --- Расс узнал голос, вдохновлявший 
их перед боем. Говорил командующий, --- удерживайте позиции! Локализуйте 
заражение! Ветераны передового рубежа, не прекращайте вести огонь! Всем 
стрелять!\\
--- Локализовать зону обстрела невозможно! Пшшшшш\ldots\ --- шелест вылетающих 
игл смешался с диким рычанием, --- Во имя Светлейшей, не подпускайте их ближе! 
Пшшшш\ldots\ --- снова стрельба, --- Локализовать очаг заражения невозможно! 
Зона обстрела обширная! Они накрыли нас почти полностью! Основные силы либо 
связаны ожесточённым ближним боем, либо уничтожены! Нужно уводить людей в 
каньон!\\
--- Отказано. Не отступать! Не отступать! --- крики командующего снова 
сменились стрельбой. Расс отключил связь. Все необходимые приказы были уже 
получены.

Он посмотрел вперёд и увидел, что Волна снова надвигается. В сплетении с 
клубками из лиан, к окопам теперь устремлялись и более крупные твари. Живой 
ковёр готов был накрыть охваченные паникой войска Анклава. Расс попытался 
оглядеться вокруг, и тут заметил, как сверху на него плавно опустилось несколько 
листьев.

\noindent --- Те самые кусочки, на которые распадались снаряды, --- пронеслось 
в голове.

Расс потянулся рукой стряхнуть их с плеча, и тут с листьев что-то соскользнуло.

Жуки. Блеснув в лучах солнца своими панцирями, они исчезли под наплечником брони.

И сразу же пришла боль. Словно в плечо вонзилось несколько игл, через которые 
из Расса начали вытягивать само его естество. Теперь его начали, пожирать на 
самом деле. Что же чувствовали те, кто был рядом с упавшим снарядом? Расс снова 
различил жуткие крики, в бесконечной какофонии звуков вокруг.

С неба падало всё больше и больше листьев. Над полем боя царил листопад смерти. 
Новые вспышки боли пронзили руку, спину, шею. Расс зарычал. Стоять здесь значило 
умереть в невыносимых муках.

\noindent --- Болит? Это хорошо! --- прокричал кто-то слева. Расс повернул 
голову.

Ветеран с неизвестного фронта стоял к нему лицом, держа свой тяжёлый Мэгг одной 
рукой и поливая подбирающихся к окопам тварей потоком игл. Расс разглядел, 
наконец, символ, начертанный на его левом наплечнике. Язык пламени в круге. 
Кажется, таких как он называют выжигателями.

\noindent --- Твоё имя, брат? --- бросил он Рассу.

Грохот взрыва из-за насыпи. Где-то сзади, за укреплениями рванула граната. И их 
обоих накрывает дождём из земли и крови.

\noindent--- Расс!\\
--- Хорошо Расс! Я Конрад. Руки жать не будем. Слушай меня. Порождения питаются 
эссенцией, текущей внутри нас, а заодно пожирают наши чувства и эмоции, --- 
выжигатель внимательно посмотрел на Расса, --- если ты сохранил в себе ещё 
что-то, то пришло время выпустить это наружу! Используй боль, она поможет тебе 
развести огонь внутри! Покажи им свою ярость! Лишь её пламя они не в состоянии 
сожрать!

Ручной излучатель на его свободной руке засветился. С шипением из сопла над 
запястьем вырвался клинок, сотканный из огня. Одним прыжком выжигатель выбрался 
из окопа. Сделав несколько шагов вперёд, он развернулся спиной к подступающему 
врагу. Его грозная фигура чёрнела на фоне восходящего солнца, а свет, исходящий 
от пламени клинка мог поспорить со светом светила. Так тогда показалось Рассу.

\noindent --- Братья! --- воскликнул выжигатель и его громовой голос перекрыл 
все звуки сражения, --- командование решило, что нам  всем пришло время 
умереть! Но может, среди вас ещё найдутся те, кто хочет жить?! Кто хочет выжить 
вопреки приказам, вопреки всему! Я не предлагаю вам сражаться за Человечество! 
Я предлагаю вам сражаться за себя! --- в небе над ним распыляли сотни листьев 
всё новые и новые снаряды, а он стоял в лучах света окружённый падающей листвой, 
и во взгляде его горел огонь, --- Да, я знаю, позади нас смерть. И впереди 
тоже. Но я укажу вам путь к спасению, братья! За мной! Вперёд!\\
--- Вперёд! --- зарычал Расс.\\ 
--- ВПЕРЁД!!! --- подхватили его крик. 

Один за другим ветераны покидали окопы. Один за другим зажигались клинки боевых излучателей.

Извергая яростный рык, Выжигатель повёл их в бой. Они оставляли позади бьющиеся 
в предсмертной агонии укрепления, и устремлялись вперёд, прямо навстречу 
несокрушимой Зелёной Волне.

Стена тварей приближалась. Ближе. Ещё ближе. Ещё. Всего мгновение до 
столкновения. Уже можно было расслышать как там, среди извивающихся лиан, за 
бурыми панцирями, копошатся крошечные вечно голодные жуки. Но их жалкое 
копошение заглушает топот десятка закованных в сталь ног. Их жалкое копошение 
заглушает единый хор десятка глоток.

И вот одно мгновение, длиною в вечность наконец-то заканчивается, и чёрная линия 
закованных в броню ветеранов сталкивается с бескрайней Зелёной Волной.

Расс слышит свой дикий крик. Видит перед собой врага, уже тянущегося к нему 
сонмищем лоз увенчанных чёрными когтями. Чувствует как энергия из его тела, его 
ярость перетекает в горящий клинок, вырывающийся из сопла над запястьем. Рука 
уже занесена для удара.

Снова пришло время убивать.

Чтобы выжить.

\newpage


\emph{%
Остерегайтесь слова выжигателя. В пламени его сгорит последняя надежда. 
Давно сошли служители Культа Пламени с пути истинного, вступив на ложный Путь 
Огня. И выжигая врага Человечества, они сожгут заодно и его идеалы, развеяв по 
ветру пепел нашей веры.
\begin{flushright}
Командующий Восточного Фронта
\end{flushright}
}


\noindent --- Всё в порядке, выжигатель? --- донёсся до Расса приглушенный 
маской голос его нового командира. Они возвращались в лагерь, пробираясь сквозь 
лесные дебри напрямик. Кустарники, заросли и сплетения веток не могли 
остановить закованных в сталь воителей. Шли колонной. Впереди прокладывал путь 
один из подчинённых командира, следом Расс, за ним сам командир, замыкали 
движение клоны.\\
--- В порядке. Так вспомнил кое-что, --- нехотя ответил тот.

Свой первый бой в Каньоне Смерти Расс помнил как сейчас. Хотя там каждый раз 
как в первый раз. Но всё же, тогда он почти ничего о них не знал. Теперь он 
выжигатель. И они его главный враг, которого он знает в лицо.

Старина Конрад. Указавший Рассу путь Огня и сделавший последователем Культа. 
Первый встреченный им выжигатель. Он имел странную привычку называть людей 
братьями. Из-за этого в Культе ходили слухи, что пройдя клоном инициацию и став 
Ветераном, Конрад мечтал вступить в ряды Защитников. Но мечте было не суждено 
сбыться. Его ждал Путь Огня.

А три года назад Конрад сгинул. Собрав отряд, он отправился в заснеженные 
северные земли, проверить давно расползавшиеся среди выжигателей истории об 
огромных некрополях в ледниках на полюсе. По мнению многих именно оттуда 
исходила угроза Второго Пришествия Пожирателя. Первое Пришествие в своё время 
привело к жуткой Войне Смерти, и едва ли не крушению всего Анклава. И вот шли 
годы, а Конрад так и не возвращался. Как и никто из его отряда. Рассу было 
трудно поверить, что такого могучего воителя как Конрад, могло что-то 
остановить. В конце концов, он не раз был свидетелем проявлений безграничной 
силы и мужества выжигателя, его необычайной воле к жизни. Поэтому Рассу было 
проще верить, что старый наставник до сих пор где-то там, на севере, ищет 
истину, скитаясь по ледяной тундре.

Тем временем Расс вынужден бродить под кронами зеленеющих лесов юга в компании 
Ветерана и четверых клонов. 

\noindent --- Я попытался связаться с лагерем, --- вновь заговорил командир, 
--- ты оказался прав выжигатель, связи больше нет.\\
--- Знаю. Я слежу за состоянием дел, --- бросил Расс в ответ. Он пробовал 
связаться с лагерем ещё тогда, когда напротив его лица появилось дуло Мэгга. 
Нужно было убедиться, что связь отсутствует, и этот новоявленный командир не 
доложит о нём генералу, как о предателе. Случись так ситуация бы сильно 
осложнилась.\\
--- Чего ты боишься выжигатель? --- вдруг спросил Ветеран.\\
--- Я? Ничего, --- ответил Расс.\\
--- Я не верю во все эти слухи о выжигании страха, --- в голосе, раздававшемся 
из-под маски, читались нотки усмешки.\\
--- Это не слухи.

Расс говорил серьёзно. Выжигатели действительно не ведали страха, но они 
использовали очень сложную методику, основанную на силе воли и твёрдости духа. 
Среди служителей Культа пламени не было ни одного труса, но не каждый мог 
вступить в его ряды. Расс знал это по себе. Но существовали и иные, более 
действенные пути.

\noindent --- Ты же знаешь о Шести Корпусах касты Защитников? --- продолжил 
Расс.\\
--- Конечно, но только их пять.\\
--- Ах, ну да. Я ошибся, --- Расс позволил себе усмехнуться уголками губ, --- 
так вот, один из них, Корпус Смерти практикует методику эссенциального 
воздействия известную как обратная инициация. Корпус Смерти один из тех, где 
могут служить Ветераны. Методика разработана специально для них. Она вычленяет 
из сущности чувство страха, заменяя их боевыми протоколами клонов.\\
--- Поэтому методика и называется обратной инициацией?\\
--- Да. Ветеран сохраняет личность, но чувство страха он уже испытать не может, 
вместо него могут быть задействованы особые боевые протоколы, побуждающие его к 
определённым действиям в схватке. Именно это и делает Смертоносцев такими 
эффективными в сражениях с воплощениями. Правда, методика эта довольно опасная. 
Обратная инициация может вызвать необратимые изменения в личности. В результате 
можно снова стать клоном, но уже навсегда, или, же и вовсе, превратиться в некое 
подобие Техночеловека. 

Расс замолчал. Видимое сказанное им заставило Ветерана задуматься.

\noindent--- Тебе не следует, так внимательно прислушиваться к словам 
выжигателя, если ты не хочешь чтобы твоя вера в идеалы Истинного Человечества 
пошатнулась, --- между делом заметил Расс.\\
--- Не беспокойся о моей вере, --- отозвался Ветеран, --- она прошла множество 
испытаний не только временем.\\
Расс ухмыльнулся. Похоже, он не зря согласился идти в лагерь.\\
--- Так я и думал, --- начал выжигатель обличение, --- это сложно заметить, но 
твой меч с каплей крови на левом наплечнике, начертан поверх ещё какого-то 
символа. Похоже, изначально там красовался Кулак Восточного Фронта. Да и 
свежесть краски не соответствует возрасту брони. Ты совершил очень серьёзный 
проступок, и был переведён на Южный Фронт.\\
Ответа пришлось ждать какое-то время, но Расс и без того знал, что оказался 
прав.\\
--- Я выявил предателя в командовании и взял правосудие в свои руки, --- 
сказал, наконец, Ветеран. Кажется в его голосе звучала горечь.\\
--- И раз ты здесь, значит, он остался жив, а ты в то время был выше званием, 
чем сейчас. Оставили в живых и не сослали в Каньон Смерти\ldots\ --- Расс 
сделал вид, что обдумывает варианты, --- хм, выходит: ты герой, --- озвучил 
выжигатель, напрашивающийся сам собой, вывод, --- нужно иметь впечатляющие 
заслуги на Фронте, чтобы заслужить такое снисхождение у трибунала.\\
--- Суд постановил, что на мой рассудок пагубно повлияла война. В последнем 
сражении, когда нас штурмовали войска Имперского Легиона и Чёрной Гвардии выжил, 
лишь я один. Силы, которые должны были поддержать нас в бою, неожиданно 
отступили, прихватив с собой нашу же артиллерию. В итоге мы напрасно стягивали 
на себя большую часть войск противника в том секторе. Город, который мы 
обороняли, был захвачен, и наша жертва оказалась напрасной. И я знал, кто во 
всём этом виноват.\\
--- Так значит, --- бросил Расс за спину, --- ты так ничему и не научился?\\
--- Что?! --- начал было Ветеран.\\
--- Все уже давно уверены, --- резко прервал его выжигатель, --- что Анклав 
нерушимый монолит, способный справиться чем угодно. Но далеко не до всех пока 
дошло, что этот уже монолит дал трещину. Разваливается изнутри сам по себе, --- 
глаза выжигателя сузились. Загляни бы туда Ветеран, он прочитал бы в них гнев, 
--- Ненависть, так усердно взлелеянная в нашем обществе, ко всему, что лежит за 
пределами границ Истинного Человечества --- хорошо удобренная почва для 
предательства. Люди становятся настолько озлобленными, что уже перестают 
понимать, где друг и кому следует доверять, и где враг, заслуживающий их 
ненависти. Исход один. Они оказываются в одиночестве, окружённые предателями и 
ненавидящие всех вокруг. В таком состоянии ты сам придумываешь для себя идеалы 
своего Истинного Человечества и сам же следуешь им, --- Расс выдержал паузу, 
--- Тот, кого ты считаешь предателем, всего лишь результат работы нашей же 
системы.\\
--- Хочешь сказать, Анклав разрушает себя сам? --- прогудело у выжигателя за 
спиной.\\
--- Верховный Совет, управляющий нами ещё со времён Войны за Независимость, уже 
давно прогнил насквозь. Предателей следует искать не в командовании, не в касте 
Мудрецов или в рядах Гвардии.\\
--- Ты что, обвиняешь его?! – вот теперь Ветеран действительно был сбит с толку.

Но Расс не дал ему время на изумление сказанным словам. Вот уже пять минут они 
спускались с холма, свободно шагая по довольно широкой прогалине. В этом месте в 
окружении небольших камней вниз в долину, где был расположен лагерь, стекал 
небольшой ручей. Деревья стеной встали слева и справа, образуя своеобразный 
коридор, по которому сейчас и вынужден был спускаться отряд.    

Выжигатель резко поднял руку, давая знак остановиться. Шедший впереди всех 
гвардеец, не успел сделать и пары шагов, как размытая тень, вырвавшаяся из леса, 
с запредельной для человека скоростью, едва не сбила клона с ног. Но выжигатель 
оказался быстрее. Короткая очередь из Мэгга, иглы для которой Расс собирал 
последний десяток секунд, отбросила нападавшего обратно в лес, со скоростью 
ничуть не меньшей, чем тот двигался.

\noindent --- Круговая оборона! Всем огонь! --- раздался приказ Ветерана. Но 
гвардейцы всё уже сделали сами. Им было не особо важно, о чём всё это время 
разговаривали выжигатель и командир. Их боевые протоколы были начеку и 
моментально среагировали на опасность. Уже через секунду четверо клонов 
прикрывая собой Ветеранов, поливали лес вокруг серебристыми потоками металла. С 
глухим стуком древесина принимала в себя тяжёлые стальные иглы, а молодые 
деревья и вовсе разлетались в щепки. Воздух наполнился гудением Мэггов и свистом 
разрезающей пространство стали. После минуты интенсивной стрельбы тени в лесу 
прервали свои размытые движения и бесследно исчезли.\\
--- Плохо, --- Расс разглядывал следы крови, на траве оставленные нападавшим, 
--- готов поспорить, я попал в сердце.\\ 
По чёткому багровому следу он прошёл до окраины леса.\\
--- Жди здесь, нужно кое-что проверить, --- сказал он Ветерану, --- не 
беспокойся, я не сбегу. Они пока отошли, но не далеко, и боюсь ненадолго. В 
любом случае мы в западне.\\
Он исчез за деревьями. Прошло не больше минуты, и он вернулся. Ветеран встретил 
его вопросом.\\
--- Как ты узнал, что они нападут?\\
--- Можешь называть это чутьём, --- хмуро ответил Расс. Шрамы на его лице 
делали его ещё более мрачным. Он замолчал на какое-то время, а затем решил 
всё-таки поделиться своими результатами проверки. \\
--- В лесу ни одного трупа. Там, --- указал подбородком на противоположенную 
стену изуродованных выстрелами деревьев, --- думаю, тоже самое.\\
--- Система наведения подтвердила попадание по трём целям. У клонов тоже 
приемлемые результаты. Неужели они все ушли живыми? --- недоумевал ветеран, --- 
Проклятые дикари. Как всегда бьют из засады.\\
Расс кивнул.\\
--- Согласен, похоже на них. Вот только это не обычные люди.\\
--- Благодаря этой варварской магии крови, в них  уже не осталось ничего 
человеческого, --- вступил в спор ветеран, --- конечно, это не обычные люди. Та 
скорость, с которой они двигаются, она явно за пределами человеческих 
возможностей. \\
--- Это верно. Но магия крови, это в каком-то смысле дело рук эволюции, пусть и 
несколько ускоренной и направленной. Она раскрывает внутренние человеческие 
резервы. Это же\ldots\ --- начал рассуждать Расс, --- в движениях этой твари не 
было животной пластики присущей дикарям. Они были резкими и спонтанными, 
словно\ldots\ --- выжигатель подбирал слова, --- словно внутри него сидело 
что-то, что им управляло. Всё это кажется мне очень знакомым.\\
Гвардейцы снова заняли оборонительные позиции вокруг ветеранов, окружив их. 
Глаза Расса опасно сузились. В такой ситуации он побывал совсем недавно.\\
--- Ты уже два раза уходил от моего главного вопроса, выжигатель. Так может 
хоть сейчас, ты мне скажешь\ldots\ --- ветеран снова встал напротив Расса, --- 
чего же ты боишься? И что ты здесь делаешь?

\newpage

\begin{mssg}{%
Запись 308\\
\\
Датировано 256 годом со дня падения Человечества\\
Отправитель: Канро Охотник на нежить из Клана Часовых\\
Получатель: Расс Ветеран Восточного Фронта Выжигатель\\
Тема: Предостережение\\
\\
Текст сообщения: 
}%
Приветствую Выжигатель. Думаю, ты не очень удивлён этому письму, учитывая 
происходящее. Не пытайся сменить коды и шифры, твой канал уже рассекречен. Итак, 
скоро тебе предстоит принять участие в событиях, от которых зависит судьба 
Севера, и я считаю, тебе следует знать кое-что. Каким бы свободным ты не был 
Выжигатель, помни, тебе не укрыться от неустанного взора Анклава. Ты всегда на 
виду, и каждый твой вздох, каждый удар сердца эхом отражаются от стен залов 
Совета. Не доверяй никому и не забывай о Пути Огня.\\
\\
Предостерегающий тебя \\
Канро\\
\\
Конец сообщения. 
\end{mssg}

  
Это произошло сравнительно недавно, и было вполне логичным последствием 
развивающихся на протяжении многих веков событий. Во время Первого Пришествия 
Пожирателя Анклав Истинного Человечества вёл ожесточённую войну с Империей 
Высшего Разума. Тогда мир сотрясала громовая поступь Стального Марша, и 
окончательная победа одной из сторон конфликта, казалось, была уже близко. Но 
угроза с севера, которую так долго игнорировал Анклав, в результате едва не 
уничтожила его и все идеи Истинного Человечества. Стальной Марш пришлось 
окончить, и победоносные войска Гвардии в спешке возвращались домой. Империя уже 
второй раз избегала полного поражения. И Анклаву было трудно с этим смириться. 
Поэтому когда на его границах появились полчища порождений, это было воспринято 
как ответный удар на Стальной Марш и начало новой войны. Народ Анклава с 
легкостью поверил, что Зелёная Волна была делом рук Империи. Число людей 
вступающих в касту Защитников и Покровителей начало расти, и конфликт разгорелся 
с новой силой. А Гвардия с тех пор держала извечную оборону в Каньоне Смерти.

Эту историю знал каждый в Анклаве. И новоявленный командир Расса не был 
исключением. Вот только всю правду о появлении порождений, наследникам Истинного 
Человечества так никто и не открыл. Но от Культа Пламени истину сокрыть было не 
так легко.

\noindent --- На самом деле порождения, --- Расс выдохнул, и струйка дыма 
устремилась в чистое голубое небо, --- наше оружие. Оно разрабатывалось ещё до 
Стального Марша, с расчётом на полное уничтожение уже побеждённого 
противника\ldots\\
--- Но\ldots\\
--- Ты задал мне вопрос, и я согласился на него ответить, --- прервал всяческие 
возражения Расс, --- я делюсь с тобой информацией и делаю в это время что 
захочу, --- выжигатель замолчал, чтобы втянуть в лёгкие ещё дыма, --- а ты, --- 
два пальца с жатой между ними самокруткой упёрлись командиру в грудь, --- 
слушаешь и помалкиваешь!

Из-за маски донеслось приглушённое ворчание.

\noindent --- Так вот, --- ветеран выпустил остатки дыма, и снова затянулся, 
--- по планам Стальной Марш должен был измотать Империю до предела, а потом 
Анклав нанёс бы несколько точечных ударов по всем крупным городам противника. 
Страшно подумать какого врага мы получили бы взамен, но нам повезло, спасибо 
Пожирателю, --- Расс оскалился, --- порождения разрабатывались специально 
против Империи, но время показало, что Мудрецы явно просчитались, и выкованный 
ими же меч, оказался у нашего же горла. Каньон Смерти явное тому доказательство. 
Вообще это довольно символично, --- выкуриваемая Рассом дрянь, явно оказывала 
на него пагубное воздействие, не ускользнувшее от ветерана. Маска снова издала 
ворчание.\\
--- Выжженная равнина перед Каньоном Смерти, известная сейчас как Мёртвые 
Пустоши, тоже последствие применения самого мощного оружия, созданного 
Человечеством до своего Падения. Кстати оно и послужило одной из главных причин 
заката людской расы. Забавно, не правда ли? --- выжигатель мрачно усмехнулся 
--- Мы сами создаём то, что будет держать нас и наших потомков многие столетия 
в священном трепете. Вы называете это Великим Прогрессом. Как по мне, так это 
Великий Идиотизм.\\
Остаток самокрутки он докурил в тишине. Когда тлеющий окурок исчез в траве, Расс 
продолжил.\\
--- Теперь вернёмся к нашему концу света, --- продолжил выжигатель уже более 
серьёзным голосом, --- народ Империи связан с эссенцией гораздо сильнее, чем 
мы, и ты как бывший ветеран Восточного Фронта должен прекрасно это знать. 
Справедливо говоря, этот народ и есть эта самая эссенция. Так вот, порождения 
это лишь побочный эффект действия нашего оружия. У оружия прошлого это было 
смертельное для всего живого излучение. У нашего оружия это порождения. По 
большей части побочный эффект оказывался ещё более разрушительным, чем само 
оружие. Последствия его применения ещё долго дают о себе знать. ---
Расс выдержал паузу.\\
--- Каста Мудрецов разработала новый вид жуков. В их основной рацион входила 
эссенция, и они были оснащены всеми необходимыми средствами добыть её. Когда 
Война Смерти закончилась, Анклав с помощью специальных машин перебросил капсулы 
с жуками далеко на юг, в самое сердце Империи. Что точно там произошло 
доподлинно пока неизвестно. Но, то, что порождения являются побочным эффектом 
применения этих жуков, неоспоримый факт. Всем известно, что весь Север облучён. 
Благодаря этому облучению возник и Анклав и Пожиратель. И жуки чувствуют это 
огромное количество разлитой в пространстве эссенции. Они были созданы вечно 
голодными, и всеми силами, они стремятся на Север, туда, где их создали и там 
где они возможно наконец-то насытятся. А по дороге они уничтожат и Анклав, и 
Империю, и Круг Бессмертных, и вообще всех кто встанет у них на пути.\\
--- Но разве мы не сдерживаем их уже несколько десятков лет в Каньоне Смерти? – 
спросил ветеран.\\
--- Мы сдерживаем не жуков, мы сдерживаем лишь одно из их известных орудий --- 
порождения. Они всего лишь растения, подвергнутые особому виду мутации. А теперь 
мы собственно подошли к сути, --- Расс замолчал и посмотрел на командира. Тому 
потребовалось совсем немного времени,  чтобы понять к чему клонит выжигатель.\\
--- Эти жуки перебрались через горы, минув Каньон, и здесь и сейчас мы 
столкнулись с их новым оружием?\\
Расс молчал. Ему почему-то не хотелось говорить об этом.\\
--- Отвечай же выжигатель. --- с нажимом повторил ветеран.\\
--- Я не знаю. Многое сходится. Я нашёл двухдневные трупы гвардейцев и не 
почувствовал в них ни следа эссенции. Теперь это нападение, за которым я вижу 
лицо своего главного врага, --- Расс выдохнул, --- мы слишком долго считали 
горы Пояса Ксерна неприступными, а Каньон Смерти единственной дорогой на 
Север.\\
Выжигатель знал, принять такие новости будет тяжело. Бич всего Севера 
подбирается к цели всё ближе. \\
Мир катиться к чертям слишком быстро.\\
--- Поэтому ты и не хотел идти в лагерь, --- догадался ветеран, --- Ты думаешь, 
что они ударят именно туда, и мы вряд ли сможем пережить этот удар.\\
--- Удар уже нанесён, --- мрачно произнёс выжигатель, --- мы потеряли связь, мы 
потеряли все разведгруппы. Обычные дикари так быстро и чисто справиться бы не 
смогли. И это ещё одно подтверждение.

Было и ещё одно подтверждение, но не такое явное. Старший сержант Маннос 
ветеран Южного Фронта, исчезнувший во время задания при странных 
обстоятельствах. Но об этом сейчас напоминать не стоило. К тому же для Расса это 
было пока не подтверждением, а загадкой, которую он связывал с происходящим 
интуитивно.\\
--- Выходит, в лагерь действительно нет смысла идти? --- спросил ветеран.\\
--- Есть. Я планировал сразу же возвращаться на базу, но ты мне помешал. Теперь 
нам уже не добраться до неё без транспорта, --- возразил Расс. Он смотрел на 
ветерана, и никак не мог отделаться от мысли, что что-то не так.\\
--- Значит нужно торопиться? Возможно, в лагере ещё идёт бой, тогда будет проще 
уйти, --- прозвучало из-за маски. Расс нахмурился. А ведь на секунду ему 
показалось, что всё так удачно сложилось. Но нет. Слишком просто.\\
--- Если кому-то в лагере и дадут уйти, то нас они ждать не будут. Надеяться на 
то, что там ещё кто-то жив тоже глупо, --- снова возразил выжигатель, --- 
поверь, если сейчас нужно было куда-то торопиться, меня бы здесь не было, --- 
спокойно произнёс Расс, --- я уже говорил. Мы в западне. И они, и дикари 
прекрасные охотники. Вполне возможны их совместные действия. Делать резкие 
движения сейчас глупо. Нас сразу же сожрут. Нужно снова дать возможность им 
напасть, --- Расс оскалился, --- и показать, что жертва им не по зубам.\\  
--- Тогда у меня остался, только один вопрос, --- сказал командир.\\ 
--- Почему я должен верить предателю?\\


\begin{center}
 ***
\end{center}

\emph{%
\ \\
Защитники на страже Человечества! Да не убоимся смерти!
\begin{flushright}
Безымянный Сержант\\
\end{flushright}
}

\begin{center}
 ***
\end{center}

\noindent --- Выжигатель Расс! --- полумрак центра связи содрогнулся от 
громового голоса, --- властью данной мне Культом Пламени постановляю, избрать 
тебя на это задание!\\
--- Да Властитель, --- выжигатель склонил голову, --- я выполню вашу волю.\\
--- И волю нашего народа, --- добавил второй голос, не такой грозный, --- не 
забывай Расс, на этом задании ты подчиняешься генералу Туфусу. Как бы цели и 
средства миссии не противоречили Пути Огня, помни, ты должен сохранять 
лояльность Анклаву и Человечеству.\\
--- Да Властитель, --- ответил выжигатель, не поднимая головы, --- я служу 
Человечеству.\\
--- Тогда решено! --- возвестил первый голос, --- командующие, детали за вами.\\
--- Генерал Туфус. Генерал Хэйро. Наше почтение, --- попрощался второй, и 
голографические лица Властителей рассеялись. Работа освещения вернулась в 
прежнее русло, и на смену полумраку, пришёл свет. Расс и Тару стояли напротив 
стола, за которым восседали генералы. Справа от них на смену голограммам 
Властителей Пламени, уже появилась объёмная карта местности.\\
--- Итак, выжигатель Тару, --- слово взял генерал Хэйро, --- вы остаётесь на 
базе и не задействованы в операции, поэтому попрошу, --- он кивнул гвардейцам и 
те послушно встали за спиной выжигателя.\\
--- Служу Человечеству, --- отсалютовал Тару. Гвардейцы вывели выжигателя из 
центра связи. Расс хмыкнул.\\
--- Что ж Выжигатель Расс, --- продолжил Хэйро, когда створки двери плавно 
сомкнулись за ушедшим Тару, --- надеюсь, у нас с вами не будет проблем. Как вы 
поняли ситуация серьёзная, --- он перевёл взгляд на Туфуса.\\
--- Операция естественно секретная. О ней знают только те, кто находился на 
брифинге и самые верхи --- перенял слово второй генерал.\\
--- Конечно же, секретная, --- подумал про себя Расс, --- кто же будет 
объявлять во всеуслышание, что главная угроза всего Севера, уже перешагнула 
порог.

Итак, командование Южного Фронта зашевелилось. Значит, откуда-то поступили тревожные звоночки. Причём, подкреплённые какими-то фактами. Возможно не такими уж и весомыми, раз в дело вступает лишь один выжигатель, с очень размытой задачей, но все, же эти факты заслужили внимание сверху. Значит, поводы для беспокойства вполне обоснованы.

\noindent --- Легенда такова, --- продолжил Туфус, --- я беру под своё 
командование крупный отряд гвардейцев, ветеранов и лёгкой бронетехники и мы 
продвигаемся в этот сектор и разбиваем небольшой лагерь, --- он обозначил на 
карте предполагаемую точку дислокации, --- дальше всё так же стандартно. Поиск 
противника несколькими разведгруппами, и подготовка сил лагеря к броску.

Генерал говорил чётко и быстро. Сразу видно, у старика был хороший опыт в таких делах. Расс хмыкнул. Наверное, только такими вот разговорами Туфус и занимался последние несколько лет.

\noindent --- Насколько я понимаю в этом секторе, нет значительных группировок 
противника? --- спросил выжигатель, --- то есть мы, вряд ли, сможем встретить 
ожесточённое сопротивление дикарей.

Хэйро кашлянул, прикрыв рот кулаком. Видимо это означало --- ты позволяешь 
себе слишком много, выжигатель. Туфус тоже наградил Расса недовольным взглядом. 
Но согласиться ему все же пришлось.

\noindent --- Да, но, несмотря на это операция будет маскироваться под большой 
карательный поход на предполагаемое скопление крупных сил противника, --- 
неохотно ответил командующий, --- теперь непосредственно о вашей задаче.\\
--- Вот она секретность, --- подумал Расс. Кучу народа посылают в пустой сектор 
искать потенциального противника. Похоже, там наверху торопятся, раз не могут 
придумать что-нибудь более убедительное.\\
--- На самом деле, --- вернулся в разговор Хэйро, --- совсем недавно в этом 
секторе при довольно странных обстоятельствах пропал отряд, который по плану 
должен был находиться в другом месте.\\
--- Время? --- снова встрял Расс с вопросом. На этот раз глаза Хэйро гневно 
сверкнули.\\
--- Полтора дня назад, --- ответил генерал, недовольно кривя губы. Он был 
моложе Туфуса, что наводило Расса на мысль о том, что Хэйро скорее всего был из 
Касты Советников. Такие командующие хорошо если находились где-то неподалёку от 
места боевых действий, не говоря уже о непосредственном участии. Зато спеси, в 
них было порядком, и кричали о защите Человечества они громче всех.\\
--- Теперь с ними нет связи, и они, скорее всего, мертвы. Вы будете возглавлять 
одну из разведгрупп. Задача предельно простая --- найти хоть какие-то следы 
отряда, и сразу сообщить о результатах командующему операцией.\\
--- А их не могли просто убить те же дикари, например? --- спокойно спросил 
Расс. \\
--- Исключено, --- ответил Хэйро. На сей раз, он остался спокоен. Лицо его 
стало серьёзным, а взгляд испытующим.\\
--- Отряд возглавлял один из вас, --- продолжил Туфус.\\

Вот оно что. Тогда понятно, почему Властители всё доверили им. Так было всегда, 
когда по неизвестным причинам с одим из последователей Культа, случалось то, что 
в принципе с ним произойти не могло. Выжигатели всегда были на особом счету в 
Анклаве. За их свободомыслие их недолюбливали в обществе, и относились к ним с 
подозрением. Отсюда вытекала изоляция от остальных ветеранов и постоянный надзор 
со стороны Смотрящих. И хотя случаев предательства среди выжигателей ещё не 
случалось, к любой ситуации подразумевающей оное относились с предельной 
серьёзностью.\\

Пока только было непонятно, как это всё было связано с прорывом на Север 
порождений. Расс собирался было задать новый вопрос, но в очередной раз нарушить 
субординацию ему не дали. Все свои страхи и тревоги генералы изложили ему сами, 
в форме приказа.

\noindent --- Найди его, выжигатель, --- с нажимом произнёс Хэйро.\\ 
--- У него должна была быть очень важная для всего Севера информация, 
собиравшаяся нами на протяжении последних месяцев, о которой пока никто не 
должен знать, но от которой зависит судьба всего человечества, --- продолжил за 
него Туфус, тоже ставший предельно серьёзным. Лицо старого Защитника стало 
похоже на каменную маску, --- Найди его. Забери информацию. И если потребуется, 
убей.

Расс нахмурился. И Властители допускают подобное? Что вообще происходит? 
Выжигатель добывший важную информацию не вернулся. И в последний раз он 
находился там, где быть не должен. Вышел на чей-то след? Или его самого 
преследовало что-то? Нет сомнений, добытые данные как то связаны с порождениями. 
Но что же это за данные если выжигатель не вернулся?
Слишком много загадок.

\noindent --- Вы послали с важной миссией одного выжигателя, и он не 
справился. Теперь вы посылайте разобраться с этим ещё одного, --- Расс смотрел 
прямо в глаза генералу. Но Туфус не отвёл взгляда.\\
--- Если ты не справишься, --- с неподдельной тревогой в голосе произнёс 
генерал, --- проблема будет решена другими методами. Сюда пришлют Безумца, и 
тогда\ldots

Глаза Расса расширились. Туфус не стал озвучивать, что произойдёт тогда, но 
слышавший истории, от которых кровь стыла в жилах, Расс и сам мог догадаться. 
Это подействовало на него гораздо лучше, чем какое-нибудь пресное <<миру придёт 
конец>> или <<Человечество обречено>>. Генерал ясно дал понять, что 
командование готово действовать. И если он Расс, не справиться, то здесь 
окажется самое страшное оружие Анклава из существующих, затмевающее собой даже 
порождения. И это могло значить только одно --- он должен был справиться.

Обсуждение закончилось. Расс поднимался в лифте в сопровождении двух клонов. А в 
голове у него всё ещё звучала фраза, брошенная Туфусом напоследок.\\

Найди его, выжигатель.\\

И убей.\\

\newpage

\begin{mssg}{%
Запись 309\\
\\
Датировано 256 годом со дня падения Человечества\\
Отправитель: Расс Ветеран Восточного Фронта Выжигатель\\
Получатель: ??? статус в обществе неизвестен\\
Тема: Секретность\\
\\
Текст сообщения:
}
Проверь что там у нас с системой.\\
\\
Расс\\
\\
Конец сообщения.
\end{mssg}

Превращать врага в пепел одним лишь взглядом --- способность внушающая трепет. 
И Расс ей не обладал. Зато он, как служитель Культа Пламени, знал, что это всё 
не просто слухи. Любая выдумка на чём-то основана. И для Выжигателей не секрет, 
что некоторые Властители Пламени действительно умеют разжигать огонь, сотканный 
из эссенции, не прибегая к использованию оружия. Но это был совершенно другой 
уровень, до которого рядовым служителем Культа было далеко.

Но сейчас Расс не испытывал потребности в испепелении противника без каких либо 
телодвижений. Перед его лицом не чернело дуло магнитной винтовки. И в уши не 
заползало гудение готовых выплюнуть металл Мэггов. У Расса были развязаны руки. 
И он собирался этим воспользоваться.

Ему даже не придётся убивать всех самому.

С тех пор как Расс встал на путь Огня, он обзавёлся невероятным чутьём на 
опасность. Выжигатель всегда готов к сражению за свою жизнь. Он собран, напряжён 
и ждёт удара с любой стороны. Выжить где и когда угодно, даже в самом пекле --- 
вот одно из главных правил Пути Огня.

Где бы Рассу не пришлось сражаться на изъеденных туманами рубежах Севера, под 
обстрелом Чёрной Гвардии на Восточном Фронте или в густых лесах юга, его чутьё 
никогда его не подводило.

Вот и сейчас, за несколько секунд до того как серебристая тень стрелы впилась в 
плоть чуть ниже маски одного из клонов, Выжигатель уже знал что делать. Стоило 
ему почувствовать, что отряд окружили, как ощущение затягивающейся на шее петли 
уже не отпускало. И конечно, он успел просчитать всё ещё до начала нападения.

Едва гвардеец со стрелой в горле стал оседать на землю, Расс сделал свой ход. 
Новоявленный командир полетел на землю от прямого удара, закованного в сталь 
кулака Выжигателя. Расс вложил в удар всю силу, в надежде оставить на маске 
Ветерана приличную вмятину, а его самого вывести из игры на некоторое время. В 
отличие от клонов, которых служитель Культа собирался пустить в расход, командир 
мог ещё пригодиться.

Рука Выжигателя ещё завершала движение, а ручной излучатель на запястье уже 
засиял лазуревым светом. Чёткая мыслительная команда Расса завершила начатое, и 
на смену свету из сопла вырвалось яркое пламя.

Выжигатель резко развернулся. Пылающий клинок без труда прошёл сквозь тела трёх 
гвардейцев, стоявших секунду назад за спиной Расса. Громыхая доспехами, они с 
глухими стонами рухнули на землю, корчась от невыносимой боли. Из щелей в броне 
показались алые языки пламени. Эссенциальное пламя разъедало их изнутри, выжигая 
само естество.

С шипением огонь развеялся, и свет излучателя погас. Расс снова резко 
развернулся, на этот раз, чтобы перехватить занесённый для удара топор. Тут же в 
другой руке Выжигателя загудела магнитная винтовка. Скупые выстрелы куда-то во 
тьму между деревьями. Но Расс знал, что именно там затаился лучник. Он видел, 
откуда прилетела стрела, и не давал стрелку ещё одного шанса.

Дикаря набросившегося с топором, Выжигатель отбросил ударом ноги. Раздался 
характерный хруст ломающихся костей. Нападавший с переломанными рёбрами полетел 
на землю, а его оружие осталось в руке у Расса. Перехватив топор поудобнее, 
Выжигатель с удивительной лёгкостью снёс головы ещё двум дикарям, перебравшимся 
через ручей. Но из леса со всех сторон к нему устремлялись всё новые войны. И 
служитель Культа явил им своё умение убивать.

Топор в руке Расса очень быстро окрасился кровью по рукоять, а Мэгг не затихал 
ни на секунду. Выжигатель зашёлся в безумном кровавом танце, с презрением разя 
дикарей одного за другим. Чёткие быстрые движения закованного в громоздкую броню 
служителя Культа, казались невозможными. Он бил именно тогда, когда враг был 
наиболее уязвим, не давая никому даже шанса нанести удар. Любое движение, любой 
взмах оружием были тут же разгаданы. Выжигатель словно знал, что произойдёт в 
следующую секунду, он будто видел будущее. И двигался при этом с нечеловеческой 
лёгкостью, игнорируя массивную тяжёлую броню, в которую был облачён. За всё 
сражение он так и не сошёл с места, пока землю вокруг него не устлали 
распростертые тела.

Расс обладал оружием, которое дикарям и не снилось. Этим оружием была 
технология.

Под чёрной сталью доспеха скрывалось множество искусственных мышц слитых в 
единый экзоскелет. Всё что нужно было делать Рассу --- это отдавать короткие 
команды управляющей бронёй системе. Его тело двигалось, словно само по себе, без 
устали вращая вокруг себя топор, уворачиваясь от стремительных выпадов дикарей и 
наводя Мэгг на новую цель. Множество крошечных камер расположенных на экзоброне 
Выжигателя исправно предоставляли ему всю нужную информацию. Расс прекрасно 
знал, что происходит вокруг: за спиной, под ногами, над головой. Он без труда 
различал тени среди деревьев, оценивал расстояние до сантиметров и при этом мог 
слышать стук сердец врагов.

Правильно и быстро обработать всю эту информацию и отдать нужные команды 
экзоброне, ему помогала эссенция. Эта необыкновенная материя, текшая у него в 
жилах, напрямую соединяла его разум с управляющими доспехом, оружием, камерами и 
датчиками программами. Эссенция позволяла ему, не задумываясь, ориентироваться 
во всём этом безграничном потоке информации. И чем лучше Выжигатель управлял её 
потоками в своём теле, тем чётче и быстрее отдавались команды, тем больше 
информации было обработано. Технология в руках Расса, затмившая наработки 
доминировавшего триста лет назад Человечества, не оставляла дикарям ни единого 
шанса.\\

Вода в ручье окрасилась красным.\\

Крутясь в воздухе, топор с гудением исчез среди деревьев, чтобы сразить 
последнего дикаря. Стрела так и не сорвалась с тетивы. Бой был окончен.\\

Топор достиг своей цели, а излучатель на запястье Выжигателя, снова изрыгнул 
пламя. Яростно горящий огненный клинок засверкал прямо перед лицом попытавшегося 
подняться с земли Ветерана. Скоротечные пять минут боя, слившиеся в одно 
мгновение, дали командиру время оправится от тяжёлой руки Расса. Но теперь 
лишившись верных гвардейцев, он, лёжа на спине, также лишился права отдавать 
приказы.

\noindent --- Даже не думай, --- холодно произнёс Выжигатель, заметивший 
попытку Ветерана дотянуться до валявшейся рядом винтовки. Пламя послушно 
запылало сильнее. Языки огня принялись лизать нагрудную пластину Ветерана. 
Отблески пламени мрачно мерцали багровым на чёрно серебряной броне Выжигателя. 
Его испытующий взгляд был направлен на маску командира.\\
--- Сними её, --- сказал Расс. Ветеран повиновался.

Теперь Выжигатель мог наконец посмотреть ему в глаза. Их взоры столкнулись.

\noindent --- Я слышал о тебе, --- медленно проговорил Расс, --- единственный 
выживший при штурме Диксейта, --- пламя в глазах Выжигателя могло поспорить с 
огнём его клинка, --- но руку ты потерял не там, верно?\\
--- Отсекли на Севере, когда я был гвардейцем, --- не отводя взгляда, ответил 
Ветеран.\\
--- И в награду инициация и Восточный Фронт? --- скривив губы, бросил 
Выжигатель, --- повезло. Обычно таких забирают либо в Корпус Милосердия, либо в 
Технолюди.\\
--- Повезло, --- согласился Ветеран. Он чувствовал, что взгляд Выжигателя 
пытается проникнуть в самую его суть, словно в стремлении найти там что-то. И 
хотя по его телу бежал холодок, а мысли в голове тревожно метались, не желая 
подчиняться, он не отводил взгляда и продолжал смотреть в лицо изрезанное 
шрамами.\\
--- Имя? --- вдруг спросил Выжигатель. Огонь в его глазах медленно угасал. Он 
вынес вердикт.\\
--- Старший сержант Энрот, --- ответил Ветеран. Кажется мысли в голове 
приходили в порядок.\\
--- Оставь звание своим подчинённым и командирам, Энрот, --- бросил 
Выжигатель\\.
Пламя излучателя погасло, и на смену сжатому кулаку к Ветерану потянулась 
ладонь.\\
--- Меня можешь назвать Рассом,\\
Ветеран ухватился за ладонь и под скрежет доспехов поднялся с земли.\\
--- Хорошо Расс.