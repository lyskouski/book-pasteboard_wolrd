\custompart{Путь Мауи}{melkij\_bes}{http://creativity.by}

\customsection{Путь Мауи}{melkij\_bes}{IKE~--- мир такой, каким вы его себе 
представляете}

\noindent --- И в подарок от магазина~--- этот пробник, реклама новой 
линии\ldots

Ее здесь узнавали. Молодая, в меру красивая, достаточно, судя по всему,
состоятельная женщина. Постоянная клиентка~--- появляется регулярно, но не чаще
чем раз в неделю. Обычно~--- по понедельникам, но не обязательно. От услуг
консультанта не отказывается, но предпочитает ходить по залу сама. Гуляет.
Однако никогда не выходит без хотя бы мелкой покупки.

Теперь~--- домой. Из офисной пыли сквозь осеннюю сырость в домашнюю\ldots\ А вот 
этого
не надо! Дома светло, свежо, чисто и уютно. И никакой затхлости. Так что~---
домой, из тусклого офиса по пасмурным улицам, с привалом в полном праздничных
красок и ароматов салоне. Так лучше! И пробничек подарили.
Салон косметики~--- для женщины если не рай на земле, то уж всяко~--- идеальное
средство подсластить начало рабочей недели. Пусть и к работе своей она
отвращения не испытывает, возможно, даже наоборот: большую часть времени вполне довольна. 
И даже не для того, чтобы себя побаловать. Просто\ldots\ Должен
же кто-то любить женщину, хотя бы она сама. Домой! Дома светло, уютно и хорошо.

Хо-ро-шо!

Включить в коридоре свет, повесить пальто, разуться. Пройти в комнату, включить,
компьютер, раздеться. Накинуть халатик, пройти в ванную, повесить халатик.
Открыть воду. Отдать тело струям. Отдаться душу.

Женщина. Молодая. Да, молодая, конечно. Не возраст. Но молодость категории <<чуть
за тридцать>>~--- архитектурный шедевр. Туристы, проходящие мимо, заметят и,
возможно, оценят, но окна его выходят на улицы с пасмурной осенью, холодной
зимой и пыльным летом. А те, кто стоит у окон, не видят фасада, и даже,
возвращаясь домой, привычно не замечают его шедевральности. Зато~--- по обоям
нет--нет да проползет набухшая влагой трещинка, то ли выцветшей добела листвой,
то ли снегом, то ли тополиным пухом опустится на пол штукатурная крошка. И
постоянно пытается атаковать забившаяся до поры в углы комнаты затхлость. Но в
целом, дом еще ничего, да и удобен. Как ни крути, архитектор~---
гений. И жить хорошо, и со стороны, говорят, красиво.

Красива. Конечно. Безусловно. Самая -- не самая, но и обаятельная, и
привлекательная, точно. А пяточки можно и пемзой\ldots\ Да, пяточки?
Состоятельная\ldots\ Вряд ли. Прилично зарабатывающая. Возможно~--- 
обеспеченная. Но
не больше. Впрочем, больше и не надо. Не в деньгах счастье. Зато очень похоже,
что в вечернем душе его элементы присутствуют.

Остатки дня, закручиваясь, исчезали в сливе. Персикового цвета халат мантией лег
на плечи, голову повенчало закрученное полотенце. Входила простой, пусть
красивой, молодой, да и бог с ней, состоятельной женщиной, выходит~--- королева.
Регина.

Теперь ревизия почты и за ней~--- ревизия кухни. Салат. Готовить самой, раз
королевским поваром не обзавелась. Зато креветки королевские. 
Сок, вино, чай? Казнить? Нельзя? Помиловать? Нельзя\ldots\ Зато можно решить иначе:
интернет, фильм, книга. Фильм, плед, вино~--- такова будет королевская воля.
Казнить нельзя. Помиловать.

И не забыть про будильник! В студенческие времена можно было проснуться, умыться
и выпорхнуть. Теперь\ldots\ Теперь любой полет возможен лишь с разрешения
диспетчерской, которая, как верная жрица красоты требует для своей богини жертв.
Лучше всего~--- минутами сна. И ей не докажешь, что коварный будильник сам
зазвенел на десять минут позже. И еще на пять. И еще на три\ldots\
Так, что уж теперь\ldots\ Раз--два--три~--- поднялись. Раз--два--три, 
раз--два--три,
раз--два--три~--- ванная. Раз--два--три~--- сок, раз--два--три~--- хлебец, 
раз--два--три~---
гардероб, раз--два--три~--- макияж, раз--два--три, раз--два--три, раз--два--три 
--- все,
борт <<Регина-30-с-мелочью>>, взлет разрешаем! Раз--два--три\ldots\ И хруст французской
булки.

Яркими цветами начавшийся день, продолжался соответствующе. День на работе, 
которую она ни то чтобы не любила, скорее напротив~--- большую часть времени 
была 
вполне-и-вполне довольна, но\ldots\ День не просто как обычно <<начался и 
закончился>>, он прошел весело и почему-то радостно. А возвращаясь, заметила, что 
утреннее ощущения чужого внимания осталось. Хотя вечерами люди слишком устали, 
чрезмерно спешат вернуться в светлый дом, туда, где можно до завтра сбросить плеч 
тяжелый, набухший от сырости, чуть потертый день~--- слишком, чтобы еще и 
замечать 
других. И чересчур, по себе знала, чтобы еще и чувствовать чужое внимание. 
Нет,~--- 
домой, из тусклого офиса, по пасмурным улицам. Всегда было так. А сегодня, 
почему-то, иначе.

При внимательном изучении пузырек оказался <<совсем пробничком>>~--- на нем не 
обнаружилось ни описания, ни вразумительного названия бренда. Только номер и 
аббревиатура из двух букв: 5AL. И тысячи результатов в поисковике, не имеющие 
ничего общего с парфюмерией. Впрочем, не страшно. Всегда можно зайти в салон и 
узнать. А можно и понедельника дождаться. В общем~--- не беда. Душ. А потом~--- 
ужин, интернет, форумы, сайт знакомств, планы на субботу.

Перед тем как сесть за компьютер вспомнила утреннее <<гадание>>. Почему бы сейчас 
не попробовать? Вода смыла дневную энергичность и чужие запахи~--- можно 
спокойно, 
вдумчиво прочувствовать. Просмаковать. Капнула. Прислушалась к ощущениям.

Душ, забрав дневные впечатления, то ли подарил, то ли выпустил чувство, схожее с 
усталостью. Может и усталость, но не та, от которой хочется отрешиться, 
спрятаться под одеялом, переждать. Скорее~--- умиротворенность. Борт <<Регина 
Умиротворенная>> вернулся на базу, экипаж сообщает о успешно выполненной миссии и 
отправляется на заслуженный отдых. Регина умиротворенная. Умиротворенная 
королева.

Все-таки правильно: первая волна~--- кофейная. Запах кофе, того, что <<в 
постель>>. 
Того, что первым приветом реального мира встречает твое возвращение из снов, 
превращая ужасы пыльных офисов и пасмурных улиц в нечто совсем нестрашное и 
далекое. Оно, конечно, есть, оно будет, но ведь это всего лишь часть дня, 
которая быстро пройдет и, к тому же, нескоро начнется. Зато уже сейчас есть это 
тепло, этот аромат, он окутывает тебя, все сильнее насыщая красками предстоящий 
день, окутывает тем сильнее, чем ближе к тебе оказывается чашка, протянутая тем, 
для кого ты~--- королева. Умиротворенная королева. Королева мира.\\

Regina Pacis.\\

Так, интернет. А то уже латынь в голову лезет. Откуда? Почта, форумы, 
знакомства. И все-таки первый аромат~--- кофе.\\

\noindent --- Регина, королева, твой мир заждался тебя. Утро!~--- Но глаза 
открывать нельзя, 
там~--- вечер, форма ответа пользователю <<Алекс\_Андр>>, дом. Но аромат~--- 
все еще 
кофе, однако появляется что-то еще~--- не домашний. Деревня, или, нет~--- 
охотничье 
угодье, но не лес. Лес вокруг, он рядом, но сразу за стенами спальни, стенами 
дома (может замка)~--- сад. Природа дикая смешивается с природой очеловеченной 
и, 
проникая через открытые окна, открытые тем же, кто разбудил ее, тем, для кого 
она~--- королева; проникая~--- сплетается с запахом кофе. Не олеандр, как бы он 
ни 
пах, но можжевельник возможно присутствует. Или просто лес хвойный? Открыть 
глаза, улыбнуться, принять чашку из его рук и проверить\ldots\ Будильник.

На этот раз встала вовремя, так что осталось время даже на утреннюю проверку 
почты. Почему-то очень тяжело вспоминалось как она договаривалась о свидании с 
Алекс\_Андр, ожидаемо оказавшемся Александром, но почта свидетельствовала: 
договорилась. Время и место уточняются до четверга. Четверг~--- завтра. Сегодня 
среда, вторник~--- вчера.

Надо полагать, вечерняя умиротворенная усталость все же превратилась в обычную 
сонливость, заставившую на автомате, почти бессознательно заканчивать дела 
вечерние. Зато снилось что-то хорошее. Кажется. В уме всплывала Бразилия и сцены 
европейской средневековой охоты. Как минимум~--- креативно. Кстати, как 
соберется 
в косметический, нужно выяснить, что это за духи и, наверное, купить флакончик. 
Капнула. Пора на работу, но не хотелось. А чего хотелось бы? Чего хотелось бы 
вместо работы, которой она, в общем и целом, вполне и вполне довольна? И вместо 
светлого и уютного, возможно даже вызывающего восторг туристов и случайных 
гостей дома, где в углах комнат прячется затхлость? Она не знала, и вообще~--- 
мысли эти были надежно заперты и запечатаны, если и проникали вовне, то по 
прихоти эха, отзвуком перестукивания где-то там в глубоких подвалах подсознания. 
Подобно вчерашней латыни. Подобно следующей: <<Не знаешь~--- зачем идешь? Сядь, 
подумай и реши\ldots\ Королева>>,~--- мысли. Никакого <<сядь>>! Надо!

На работу она пошла, но уже к обеду стало ясно о чем трезвонило подсознание. 
Осень. Покорившая улицы города, наполнившая их холодной слякотью и обжигающими 
ветрами осень добралась и до Регины. Еще не насморк, но первые признаки слякоти 
в носу. Еще не жар, но уже румянящиеся, словно обветренные щеки. Еще не очереди, 
но уже обжигающие легкие одиночные выстрелы кашля. Действительно, зачем шла? 
Теперь оставалось только доползти до начальства, пообещать, что 
завтра--послезавтра, в крайнем случае~--- к началу следующей недели будет как 
огурчик, и отправиться домой, залежать болезнь в зачатке. Так и поступила.

Удивительно, но стоило закрыть дверь и остаться в привычных домашних запахах как 
болезнь\ldots\ Может не прошла, но отступила, словно осталась за дверью. Или осталось за дверью то, чему противился организм. Не важно. Чай с медом, одеяло, 
сон. Вирус нужно давить в зародыше. 

Ни мед, ни одеяло, ни волевое внушение не помогли. Сна не было. Но и ничего из 
стандартного набора <<фильм, книга, интернет>> не хотелось. Организм, сбросив 
маску умирающего, столкнулся с проблемой безделья в рабочее время. Регина 
откровенно злорадствовала: то--то, милый, захотел отпуск себе внеочередной 
устроить? Получай! Забыл, что отпуска у нас распланированы заранее, за месяц--два 
билеты куплены, номера гостиничные заказаны, и даже сказки и мифы тех мест, в 
которые отправляемся, перечитаны. Вот и скучай теперь дома.

Впрочем, даже если организм и получил по заслугам, скучать приходилось самой 
Регине, так что радоваться было нечему. Хоть действительно~--- садись и думай. 
Только не садись, лежи. Мало ли, что не спится, мало ли, что кажется. Приболела 
--- выздоравливай. А думать\ldots\ Думать, в любом случае, полезно.

Чего хотелось бы? А чего вообще хочется, если не стабильности? Мальчикам~--- 
подвигов и открытий, девочкам~--- быть медсестрами, проводницами, актрисами и 
певицами. Два последних пункта актуальны и для девочек, не выросших к тридцати. 
А тем, которые выросли? Кто-то, идя против струи, цепляется за искусство: 
музыка, поэзия, фотография. Она в детстве неплохо рисовала. Хочется рисовать? 
Вряд ли. Интересно, дома есть хотя бы парочка цветных карандашей? Красок точно 
нет.

Чего еще? Еще девочки хотят быть принцессами, а девочки постарше~--- вдобавок 
принца на белом коне. Вот только она и так королева, и принц ей без надобности.

\noindent --- Королева, примите!~--- Что, откуда? Микстура ощутимо, хоть и не 
неприятно, горчила хвоей, хотя основными оставались вкус и запах чая и меда. 
Откуда, кто? Впрочем, какая разница. Ты хотела заснуть~--- засыпаешь. Это не 
горячечный бред~--- с чего ему быть таким приятным? Да и горячки у тебя нет.\\
--- Чего королева изволит?

Королева изволит посмотреть наконец на\ldots\ На него. Вот только открыть 
глаза\ldots\ И тяжело, и страшно. И не хочется сбить сон~--- он нравится. Когда 
ей снились повторяющиеся сны? Все в том же детстве, когда медсестрой? Про 
<<летать>>? А 
потом?\\
--- Сказку.

Почему вдруг? А почему нет? В детстве, раз уж оно сегодня вспоминается, засыпала 
под сказки. Сейчас~--- видишь сон о том, как под нее засыпаешь. Будешь 
засыпать. 
Ведь будешь?\\
--- А почему бы нет? Слушайте, королева\ldots

Проснулась резко и тревожно. Секундная паника: время~--- будильник~--- 
опа\ldots\ Опа! Время~--- вечер, будильник~--- не включен и не надо, опаздывать 
некуда. Болеть 
изволите, королева. Так что и волноваться, за исключением здоровья, не о чем. Не 
о чем.

Тревога не исчезла. Что-то было не так. Не правильно. Или правильно, но 
непривычно. Снова непривычно. Снова запахи.

Мед~--- или все-таки микстура?~--- сделал свое дело: жар, если он и был, ушел в 
простыню. Но помимо неприятного, но все-таки естественного и привычного пота 
простыня впитала и запах не успевшего выветриться и не попавшего под безжалостные 
струи воды парфюма. А казалось бы~--- всего две капли. Стоило найтись причине, 
как 
тревога исчезла. Две капли. Еще чаю. И можно назад в постель. Слаб все-таки 
человек духом~--- хватило нескольких часов дневного сна чтобы Регина вошла во 
вкус. Впрочем, раз уж болеть~--- то по полной. Сменить белье или дождаться 
выздоровления?\\
--- Слушайте, королева, слушайте\ldots\ В маленькой рыбацкой деревне, где всегда пахло морем\ldots

Чайная ложка меда, таблетка, две капли, растереть за ушами, сон. Сны были 
странными, о чем-то совсем чужом и далеком, но, вместе с тем, переплетающимся со 
своим. Сокровенным. Просыпаясь и засыпая, Регина оказывалась в воспоминаниях о 
том, что было и не случившемся, казалось, что и во сне она рассказывает о себе и 
выслушивает в ответ сказки о рыбацкой деревне, о поселке где-то в горах, о ветре 
в пальмах и пожаре в лесу. Присутствие того, кто слушал и рассказывал 
становилось все ощутимее. В какой-то момент Регине почудилось, что и не засыпая, 
и не закрывая глаз она видит, нет, ощущает, силуэт возле кровати. В какой-то 
момент Регине показалось, что она сходит с ума. В какой-то момент, Регина 
вспомнила о свидании, и подумала, что стоит его отменить. В какой-то момент 
случилась пятница и Регина окончательно проснулась.

От проведенных в постели двух дней осталось послевкусие бреда то ли горячечного, 
то ли наркотического. Но, не пугающее, скорее~--- смутное воспоминание о чуде. 
Да, 
королева, кто там говорил, что организм симулировал? Эк тебя\ldots\ Интересно, а 
реши ты не потакать организму и вернуться? Что бы тогда коллектив подумал? А 
если собраться и таки написать Александру? Кстати, нужно. Извиниться и или 
договориться окончательно, или перенести. Будет он?

Александра онлайн не оказалось, зато, просмотрев переписку, Регина обнаружила, 
что о месте и времени встречи они уже договорились. Переписка убедительно 
доказывала, что Регина была в сознании, вполне остроумно флиртовала, полушутя 
пожаловалось на <<подцепленную болячку>>. Переписка убедительно присутствовала. 
Вот только воспоминания о ней отсутствовали напрочь. Ну и черт с ними. 
Договорились и ладно~--- пусть болезнь все спишет, а нам остается только 
порадоваться, что не стали пересказывать парню сюжет о ушедших в море братьях. 
Тогда бы свидание точно накрылось. А так~--- пятница, восемь вечера, заедет. 
Полчаса на душ, час на подготовку, чистый остаток~--- шесть часов свободного 
времени. Постелить свежее белье и\ldots\ Хм, так все-таки есть дома карандаши?\\
--- Проходи, проходи. Я не буду кокетничать о <<бардаке в доме>>, у меня~--- 
светло, 
чисто и уютно. Комната~--- там, а я ненадолго отлучусь.

Освежиться. Сменить выходное на халатик. И, пожалуй, две капли. Да. Две капли. 
Посмотрим. Открыть дверь. Войти. Почувствовать восхищенный взгляд, взгляды\ldots\\
--- Регина!..
--- Королева!..

Отдаться~--- чувствам, рукам, губам, пальцам, прикосновениям, поцелуям, 
языкам\ldots\ 
Краскам, запахам, звукам\ldots

Проснуться~--- утром, одной, разнеженной и опустошенной. Когда ушли? Ушел. 
Ушли. 
Ушли?\\

Вскочить.\\

Дома было светло и уютно. Пахло любовью, приходившим мужчиной, недавней болезнью 
и совсем чуть-чуть~--- хорошим парфюмом. Хорошим, но не ее. Не то чтобы 
неприятным 
или отталкивающим, просто~--- чужим.

\newpage

Нельзя чтобы реальность не совпадала с ожиданиями. Это неправильно и не хорошо. 
Так просто не должно быть. Но так есть. От полного возвращения он ждал восторга. 
По меньшей мере~--- чувства удовлетворения. В худшем случае~--- немедленной 
битвы. 
И все равно восторга. Этот было бы естественно. Это казалось естественным. 
Казалось, той части его, которой 
еще могло что-то казаться. И что он имеет в итоге? Усталость, раздражение и 
дискомфорт от чужого тела. Александр. Века прошли, а имена остались. Кстати, 
нужно узнать сколько прошло. И вообще~--- покопаться в теле и освоить его 
память. 
Это~--- первое. Ну, после того, как отойдет подальше.

Из двери подъезда вышел пошатывающийся мужчина в неплохом, явно выходном, но 
как-то очень нелепо выглядящем пальто. То ли мужчина надел его в первый раз, 
хотя новым оно не казалось, и теперь не знает как нести его миру, то ли 
нелепость заключалась в перепутанных, затянутых не в свои петельки пуговицах. 
Впрочем, судя по озабоченному выражению лица мужчины, волновали его проблемы 
куда более глобальные. Например, кто он, где он и какой сейчас год, или, хотя 
бы, день. День был выходной, а утро ранним. Так что большинство обитателей двора 
насладиться или хотя бы просто удивиться видом утреннего выходца не смогли. 
Только традиционно выгуливающая оскорбительно прозванного Тушкой кота баба Лида 
проследила со своей скамьи как, отбросив что-то в сторону, мужчина прошел через 
детскую площадку и скрылся за углом соседнего дома. Хорошая девка~--- Регина, 
вот 
только с мужиками ей не везет.

Пожалуй, достаточно. Хорошая скамейка, подходящая. Теперь нужно сесть и 
расслабиться. Если уж смог взять чужое тело, то и разобраться с ним силенок 
должно хватить. Хотя, ослаб знатно~--- это чувствовалось. Ладно, силы~--- дело 
наживное. Надо надеяться. Что тут? Год 2012 нашей эры. Если наша эра~--- это от 
рождества, значит не так все и страшно. Язык, возраст, жена, дети\ldots\ Жена, 
однако? Не было печали. Ладно, потом. Работа~--- без надобности. Зарплата, 
валюта, 
правительство, строй, страна. Нанотехнологии, технологии, компьютер, интернет, 
почта. Машина, автобус, велосипед, лыжи, если зимой удастся в отпуск. Не 
удастся, то есть удастся, конечно, уже удалось.

Толку-то тебе. Зима, Новый год, бросить, наконец, курить, смысл жизни, религия. 
Интересный список приоритетов. Есть над чем подумать. Но сначала~--- нормально 
зашнуровать ботинки, перестегнуть пальто, с ширинкой, все-таки не ошибся~--- ее 
нужно застегивать. Лучше? Да, но не до конца. Со внешним дискомфортом 
разобрался, но внутренний никуда не исчез. Не нравится он телу. Не вмещает оно 
все величие такой личности. А там где вмещает~--- не может передать. Должно 
быть 
решаемо.

Осень в этих местах холодная и мокрая. Луж хватает. Любая подойдет. Эта, 
например. Только не рябись так, не надо. Вот, вот, успокойся. Успокойся, 
хорошая. Хорошо. Вот так.

Так, что можно сделать~--- чтобы и самому легче было, и местные не шугались. 
Жалко, конечно, что линии не сошлись где-нибудь южнее и восточнее, но~--- что 
есть, то есть. Брови развести чуть-чуть и или глаза вжать, или дуги выдвинуть. 
Лучше дуги~--- и привычнее, и красивее. Кстати, брови могут быть и гуще, и 
вообще 
--- волос на теле явно не хватает. Холодно же! Нос~--- увеличить, но, пожалуй, 
самую 
малость. Не стоит идти на поводу у традиций, если ради этого приходится 
жертвовать красотой. И челюсть выдвинуть, а подбородок, наоборот, убрать.\\

Начали.\\

Молодой, приличный по виду мужчина дремавший до этого на скамейке~--- резко 
встал 
и сначала склонился перед ближайшей лужей, а потом и вовсе, почти упал в нее 
лицом, в последний момент, опершись руками об асфальт. Перепил? Или что-то 
серьезное? Для сердца, вроде, слишком молод. Хотя\ldots\ Вон, главбух рассказывала, 
как у ее соседки сына на днях увезли с инфарктом, а парню и сорока не было\ldots\ 
Этому тоже нет. Подойти? Все-таки вроде прилично выглядит\ldots\\
--- Мужчина, вам плохо?\\
Ветра и воды, да! Но, что на такое принято отвечать? Как-то так, наверное:\\
--- Все хорошо.

Отшатнулась. Голос хриплый, запах гадкий. Это понятно~--- распитие одеколонов, 
как 
помнится, ни в одной из известных миров хорошим тоном не считается. А что 
делать, если <<королева>> использовать всю жидкость не успела. Ждать~--- 
сложно, 
особенно, когда ждать остается считанные дни. К тому же\ldots\ Ветра и воды! У этого 
Александра что~--- в роду ни одного оборотня или шамана не было? Но слушается. 
Медленно, больно, тяжело и со скрипом, но слушается. А боль~--- ее нет, есть 
только страх тела. Не бойся, все хорошо. Вот так. Успокойся, да. Видишь, как 
славно.

Получилось и вправду хорошо. Кожа, возможно, в результате оказалась слегка 
светловатой, но заниматься еще и пигментацией~--- сил не оставалось. Сил вообще 
осталось лишь на скамейку вернуться. Однако. О том чтобы превратиться в 
ближайшее время в ястреба или там голубя~--- можно и не думать. Остается 
надеяться, что сил хватит сознание перенести, если придется.

Все, полегче. Что теперь? <<Ступая на тропу, не забудь отряхнуть песок>>. Что с 
песком? Женщина увидела выходящего мужчину и узнала в нем мужчину входившего. 
Узнала~--- для того чтобы прочесть содержание пойманного взгляда не нужно сил, 
нужно умение. Значит, остаются жена и дети. Дочке~--- пять, мальчику~--- 
полтора. 
Еще работа\ldots\ Нет уж, с работой пусть сами разбираются.

Район телу был незнаком, так что пришлось идти просто вдоль по улице, пока оно, 
наконец, не нашло подходящую дверь. Оставалось довериться, позволить ему войти, 
приблизиться к девушке за прилавком, попросить открытку. Хорошо~--- скоро можно 
будет обойтись и без отстранения.

Ручка нашлась во внутреннем кармане пальто~--- содержимое карманов, кстати, 
нужно 
изучить. Но это потом. <<Дорогая Таня! Прости, но я понял, что моя жизнь зашла в 
тупик. Ухожу..>>~--- нет,~--- <<Уезжаю, чтобы разобраться в себе, куда-нибудь 
подальше 
от всего. Не прошу дождаться, так как не знаю~--- вернусь ли. Прошу об одном 
--- 
прости, если сможешь. Передай детям, что я их люблю\ldots>> Роспись, число. 
Открытка, ты правдива, ты дышишь искренностью, никто не будет в тебе 
сомневаться. Не сомневайся и ты в этом! Открытку убедил, тексту поверят. Он бы 
поверил. Не сомневайся!

Почтовый ящик обнаружился рядом с магазином. Еще раз пожелав открытке удачи и 
убедившись в ее уверенности, опустил. И найденный во время инспекции карманов 
телефон~--- в урну рядом. Отряхнулся.

Теперь можно подумать о дальнейшем. За четыре часа его существования в этом 
мире, его присутствием никто не заинтересовался. Значит, и дальше его не 
обнаружат как минимум до тех пор, пока он не полезет в сферы. Другой вопрос~--- 
что он может, если в сферы не лезть. Судя по всему, немного. Выпрыгнуть отсюда 
явно не удастся~--- почувствовать даже самый широкий коридор он не в состоянии. 
Вот лужу удалось успокоить, и письму, вроде бы, уверенности придать, но даже это 
стоило стольких сил, что о том, чтобы поднять или хотя бы утихомирить ветер 
остается только мечтать. С Региной получилось и с этим\ldots\ Телом. Но опять 
же~--- 
дело в навыках и долгой, очень долгой подготовке. Вывод: с прямыми возможностями 
не густо. Уровень не то чтобы минимальный, но близкий к этому. Чтобы 
восстановиться самому~--- понадобится много времени и упражнений. Для этого 
нужен 
дом. А лучше~--- пещера, где-нибудь в лесу, но не слишком далеко от небольшой 
деревушки. Вот только времена явно не те. Значит, дом.

Или сферы. Если пытаться идти туда, то\ldots\ Плюсы. Прояснится обстановка. 
Возможно, удастся связаться с кем-нибудь дружественным. Возможно, удастся 
восстановиться быстрее. Минусы. Засекут и с очень большой вероятностью кто-то 
недружественный, возможно даже тот, страшный, а достойно завершить прервавшийся 
несколько веков назад бой не получится. Все? Если так, то плюсов явно больше. К 
тому же, если бы кто-нибудь ждал его возвращения чтобы добить, то заметили бы 
его еще во время первого контакта с Региной. Значит, скорее всего, не ждут. 
Значит, даже если недружественные~--- могут не добивать сразу, могут просто 
выкинуть в темную. А это ничуть не страшнее, чем собирать себя по атомам. Значит 
--- сферы.

Стоило бы отойти подальше. Люди, как показало утро, тут весьма наблюдательные, и 
странно ведущие себя незнакомцы внимание привлекают. Не агрессивное --– пока, но 
даже и сочувственное вмешательство было бы сейчас нежелательным. Вот только тело 
не знало где тут рядом можно найти лес или, хотя бы, парк. Пришлось 
довольствоваться сквериком, замеченным по дороге. Скверик, сквер. Забавное слово 
для обозначения уголка живой природы на отобранной у мира людьми территории.\\
--- Здравствуй, угловая скамья. Здравствуй трава, если ты, пожелтевшая, меня 
слышишь. Здравствуйте, клены. Здравствуй, сквер. Здравствуй, город. Здравствуй, 
мир\ldots

И ничего. Выходит, сил еще меньше, чем казалось. В иные времена хватало простого 
приветствия, чтобы переступить порог. Сейчас не хватает~--- что ж\ldots\\
--- Благословляю тебя, скамья, на которой сижу. Благословляю тебя, трава у моих 
ног, пусть спокойны будут твои последние дни. Благословенны будьте и вы, клены, 
стоящие тут дольше всех идущих мимо вас, опытом старшего~--- благословляю! 
Благословляю тебя сквер, счастливым случаем ставший первым моим порогом. 
Благословляю тебя город, принявший меня. Здравствуй, благословленный до меня мир 
--- я, Мо, приветствую тебя\ldots\ Я, Мауи Моук, зовущийся Мо, прозываемый странником 
Мо, нареченный\ldots\ Нареченный так давно, что горы, на которых встречал солнце 
давший мне имя, стали дном рек\ldots\ Я, Мо, благословляю тебя и прошу твоего 
благословения\ldots

Каждому свое. Каждый из тех, кого он знал, кого помнил, о ком слышал видел свои 
пути к сферам. Кто-то умел нырять в бездну мира, кто-то равнодушно врезался в 
реальность словно хирургический скальпель, кто-то слышал музыку сфер и сплетал с 
нею свою мелодию\ldots\ Он --– он умел любить мир. Любой, каждый, все. 
Главное~--- 
искренне. Главное~--- не лицемерить, не каяться в том, что кто-нибудь мог бы 
счесть вредом, принесенным тобой миру. Миру, который ты, как говоришь, любишь\ldots\ 
Какая разница? Что миру до жизни одного, двух, многих~--- зверя, утеса, 
материка, 
человека~--- если они мешали твоему счастью. Ведь ты его любишь, а он любит 
тебя, 
а любить\ldots

Услышал. Понял. Захотел ответить. Захотел принять. Заворочался~--- тяжело, 
словно 
сонный, еще плохо стоящий на ногах щенок, стремящийся подставить бок под теплую 
ладонь хозяина, завозился.\\
--- Здравствуй, мир, здравствуй. Здравствуй, мой хороший. Я тут, я с тобой\ldots\ Я же 
тебе нравлюсь?

Он нравился. Мир, он чувствовал, готов был сказать ему это, готов был отдаться 
его ласкам, растаять в них. Мир урчал, мир, будь щенком, с радостью лизнул бы 
его, мир пытался, но не мог~--- словно молодой неуклюжий щенок. Мир хотел бы 
объяснить это, хотел бы раскрыться, встретится взглядом, пуская в душу\ldots\ Мир 
урчал, мир пытался повернуться, но никак не мог. Словно молодой неуклюжий щенок. 
Словно\ldots\

Хватило сил попытаться~--- неудачно, но попытку мир почувствовал~--- 
приблизиться 
самому. Хватило сил на еще несколько ласковых слов, ласковых мыслей, теплых 
чувств. Хватило сил чуть успокоить запаниковавшее от непривычных ощущений тело. 
Когда-нибудь оно привыкнет и к ощущениям, и к тому, что не все действия его 
обладателя касаются тела напрямую и на них вообще можно не реагировать. 
Когда-нибудь. А пока справившееся с паникой, но по-прежнему ошалевшее тело, 
пользуясь чужой, во всяком случае~--- не своей, слабостью достало из кармана 
пальто, сунула в рот, зажгла огонь. Гадость, хоть телу, как ни странно и 
полегчало. Новый год, бросить, наконец, курить\ldots\ Предсмертное желание почти 
что. Бросай.

Наблюдая за тем, как тело свыкается с фактом некурения, пытался думать. Проще 
всего с <<урчанием>>. Информационные поля засорены и перегружены. Можно пытаться 
ими пользоваться, если пройти глубже. Можно пытаться их чистить. Интересно, это 
обитатели мира их так засорили или хранитель?

С хранителем до сих пор непонятно. Плотно закрыть сферы от наглецов и паразитов 
вроде него сильный, укоренившийся в мире и следящий за миром хранитель, конечно, 
может. Чаще всего~--- так и делает. Вот только не может быть в этом мире 
настолько 
сильного! Не должно. Ни один из кинувшихся на эту кость не смог бы, пусть 
кое-кому и хватило сил растереть в порошок не самого слабого конкурента. Ладно, 
воспоминания потом. Итак, хранитель. Очень похоже. Скорее всего. Набрался? 
Появился новый? Вернулся? Последнее, все-таки, вряд ли. Проверить. Жаль, что 
напугавшая за миг до того, как понял: <<Всё, нужно возвращаться!>>~--- догадка 
не 
вспоминалась. Значит, вспомнится позже.

Тело послушно, но не слишком уверенно, встало. Однако\ldots\ Ладно, ничего тяжелого 
сегодня больше не будет. Просто прогулка. Тело, а хоть святые места в своем 
городе ты знаешь? Не обязательно рядом~--- вообще\ldots\ Покопался в чужой 
памяти~--- 
храмы, церкви? И то хорошо.

Ко времени, когда попал к <<святому месту>>, начинало темнеть. Конечно, осенью 
здесь темнеет рано, но все-таки\ldots\ На самом деле, по дороге, встречались и другие 
--- те, которые и он, по ощущениям, и чужая память классифицировали как 
подходящие, но если уж решил довериться телу\ldots\ Зато познакомился со всеми этими 
автобусами, троллейбусами, метро. Не самый плохой способ массового движения. Не 
<<коридоры пространства>>, конечно, но для технологичной цивилизации~--- вполне 
и 
вполне. Хотя тело вспомнило о оставленном у дома автомобиле, вспомнило с явной 
ностальгией по комфорту.

Святое место оказалось средних размеров церковью. Еще открытой, но~--- 
по-видимому, неурочный час~--- почти пустой. Впрочем, люди ему ни к чему. Ни 
приходящие к богу, ни помогающие жаждущим к нему прийти. Бог~--- не бог, но на 
стук ему уже не открыли. Так что теперь~--- по-простому, по-свойски, без доклада 
и 
без сопровождения. Лишь бы разобраться.\\

Осмотреться. Прислушаться. Понять.\\

Расслабился и отдался ощущениям. Хорошо~--- в отличие от скверов и проходных 
дворов~--- здесь погрузившийся в себя или напоминающий такового человек не 
должен 
вызывать недоумения. Тут все такие.

Так и есть. Так и было, во всяком случае, до того как его~--- впрочем, 
аккуратно, 
надо отметить, ненавязчиво~--- потряс за плечо мужчина в ритуальной одежде. 
Ряса. 
Священник. Священник в рясе. Церковь. Дом божий. Служитель бога.\\
--- Простите, что помешал, просто я уже собираюсь церковь запирать\ldots\ У вас все 
хорошо?\\
Церковь. Дом божий. Поднялся.\\
--- Бога нет. Ушел и не сказала когда будет. И что теперь делать,~--- Нужное 
обращение в чужой памяти нашлось не сразу,~--- батюшка?\\ 
Как ни странно, проводив его к выходу, священник дал вполне здравый ответ:\\
--- Ждать. И делать, то что должно. Хорошего вечера\ldots

Совет был дельный, но что должно~--- непонятно. Плохо, когда реальность 
отказывается соответствовать возложенным ожиданиям. Если мир не таков, каким его 
хочется видеть, значит слишком много людей видят его по-другому. Это 
несоответствие, не фатальное в других ситуациях, в этот раз разрушало все планы, 
мало того~--- оно мешало составлять новые. Во всяком случае, пока. Не удавалось 
ни 
придумать, ни предугадать, ни вспомнить. Зато тело напомнило о собственных 
надобностях~--- надобностях, о которых он за века забыл. Тело заслужило, но 
немедленно помочь он мог ему справить только самые мелкие. Одно, оно же, 
впрочем, и самое спешное, прямо в церковном дворе, отойдя за дерево, другое~--- 
в 
каком-нибудь заведении, вроде харчевни. Таверна, кабак, бар, ресторан, 
бистро~--- 
на выбор тела. Оно заслужило.

С остальным~--- сложнее. Разомлевшее от тепла и заказанного, дабы не выбирать 
из 
неизвестного, <<дежурного блюда>> тело требовало сна. При этом, подсовывая понятие 
<<гостиница>>. Вот только имевшаяся в памяти информация утверждала, что в 
гостиницах чаще всего просят документ, а память собственная, начавшаяся с утра, 
подсовывала воспоминания о утреннем превращении и связанных с ним ощущениях. 
Пробовать вернуть внешность Александра не хотелось совершенно, к тому же здравый 
смысл подсказывал, что сил на это просто не хватит. Как и на то, чтобы убедить 
<<ресепшн>>~--- слово показалось чужим, но к образу, просящего документ 
прислужника 
на постоялом дворе приклеилось именно оно~--- в том, что они (она, он?) не 
видят 
несоответствий между ним и его паспортом. Идея найти подходящую скамью или кучу 
листьев в очередном сквере вызвала у тела волну негодования, хотя, казалось 
бы~--- 
климат родной. Это он мучился с непривычки, и то готов был терпеть. Но тело 
нашло союзников. Унаследованная память и собственный, видимо, тоже очухавшийся в 
тепле, здравый смысл в один голос заявили, что в сквере еще больше шансов 
нарваться на проверку документов, а убедить в чем-то проверяющих там будет еще 
труднее, если вообще возможно, чем в гостинице.

Принесли кофе. Оказывается, он попал в <<вечерние часы>>, когда вместе с <<дежурным 
блюдом>> клиент получает в подарок от заведения чашку <<фирменного чая или кофе>>. 
Кофе, подарок от заведения. Кажется, мир пошел на мировую. <<Глупо пренебрегать 
течением, спускаясь по реке, смертельно опасно пренебрегать направлением ветра в 
открытом море>>. Да будет так!

Обрадовавшись ответному шагу, мир продолжал демонстрировать готовность к 
диалогу. Нужные окна темнели, а за дверью было тихо. Для того чтобы явиться 
именно так как ему виделось~--- недоставало малого: убедить дверь открыться, а 
спящую Регину~--- не просыпаться. Но тут оставалось надеяться, что проведенные 
здесь и, как ни крути, оставленные за дверьми частички себя помогут ему быть 
убедительнее, а другим~--- сговорчивее. Откройся, хорошая, свои!

Открылась. Послушалась, может, не с такой готовностью как хотелось бы, может 
быть, и без особой радости~--- но почти сразу. Это радовало. С Региной должно 
было 
быть еще легче.\\
--- Спите, королева, спите спокойно. Ваш паж разбудит вас, когда будет нужно. 
Спите, и пусть вам снятся море и небо\ldots

Королева хотела кофе, она его получит. Мир должен быть таким, каким его хотят 
видеть. И кофе должно быть таким, каким его хотят пить, а не тем, который 
приходится терпеть. Очень простая истина. Кроме того~--- и улыбка мира, и 
правильный напиток улучшают настроение человека и, следовательно, облегчают 
беседу с ним. Спите, королева! Будет вам, кофе. Но~--- утром. И кофе, и 
серьезные 
разговоры хороши на свежую голову. Спите. Простите, что придется воспользоваться 
вашим гостеприимством тайком от вас, хочется верить, что завтра удастся обо всем 
договориться. Обязательно удастся. Спите.

Устроился прямо на полу. Тело ожидаемо осталось недовольно, но не слишком~--- 
все 
же пол в теплой квартире предпочтительнее опавших листьев в осеннем парке. Ну и 
славно. Часа четыре, не больше. Не больше пяти, точно.

\newpage

\noindent --- Регина, королева, вставайте! Ваш мир заждался вас!

Первая волна пробуждения~--- кофейная. Запах кофе, того, который <<в постель>>. 
И 
голос~--- тот, который грезился. Не открывать глаза, если открыть~--- пустой 
дом, 
ушедший~--- и от тебя, и из онлайна Саша, и пропавший, закатившийся куда-то 
пузырек.\\
--- Регина, вставайте! День поистине чудесный, королева. И кофе, смею надеяться, 
тоже\ldots~--- Какой назойливый сон. А голос~--- все-таки не совсем тот. В 
голос 
вплетались другие нотки, тоже знакомые. Этими нотками общался с ней Саша 
позавчера вечером, вчера ночью. А утром ушел, не попрощавшись.\\
--- И в третий раз взываю к вам, королева! Проснитесь!\\
--- Да проснулась я\ldots~--- Лениво потянулась, открывая глаза, и прыгнула на 
кровати. 
--- Вы кто? Вы как\ldots\ Вы откуда?

Странно, всегда думала, что в такой ситуации она заорет и начнет кидать в 
грабителя чем-то тяжелым, хотя ничего тяжелого рядом с кроватью никогда не 
держала. Тумбочку~--- но ее попробуй кинь, она не считается. А оказалось, нет 
--- на 
повышенных, конечно, нотах спросила, можно сказать~--- крикнула, но не заорала. 
И 
страха не то чтобы много. Скорее удивление. Может, потому что не проснулась 
толком, а может, потому что ситуация все-таки на взлом с ограблением и 
изнасилованием не так похожа, как показалось. Странный вор, если он хозяйку 
будит и кофе~--- кофе он действительно принес~--- поить собирается. Насильник, 
конечно, может~--- мало каких маньяков есть, но и то\ldots\ И знакомый он. Или, 
точнее, кого-то напоминающий\ldots\\
--- Мы с вами встречались?\\
Пришелец~--- ну а как его называть прикажете?~--- кивнул:\\
--- Теперь вы проснулись. Возьмите ваш напиток, он, на мой взгляд, удался. Я 
вам 
все объясню.\\
--- Спасибо,~--- Нет, ну в самом деле. Пах кофе вкусно, а если бы ее 
хотели\ldots\ Если 
бы с ней хотели что-то сделать, то и будить не стали бы. Кофе и на вкус 
оказался 
изумительным, а Регина вспомнила кого ей напомнил этот\ldots\ Пришелец.\\
--- Вы какой-нибудь двоюродный брат Александра? Он вчера сделал слепок замка и 
сегодня вы пришли меня грабить?~--- Кофе действительно был бесподобным, Регина 
отхлебнула еще.~--- Или нет. Если бы вы пришли меня грабить~--- вы бы меня не 
будили. Вы, наверное, будете меня пытать и требовать переписать на вас 
квартиру? 
Напиток, кстати, божественный, спасибо.\\
--- Спасибо, что оценили,~--- Пришелец подтвердил свои слова кивком. А когда 
поднял 
голову, взгляд его стал\ldots\ Гуще, насыщеннее~--- так это охарактеризовала 
себе 
Регина.\\
--- Регина, пожалуйста, поверьте мне. Верьте мне, я не буду вам врать и вы 
знаете 
это. Верьте мне, хорошая, вы в безопасности, вам ничего не угрожает, и я 
принесу 
вам только добро. Верьте этому.~--- Очень проникновенно произнеся этот бред, 
пришелец вздохнул и заговорил уже по-человечески. Почти.~--- Нет, я не 
родственник 
Александра, до вечера, когда вы привели его в квартиру~--- я даже не знал о его 
существовании. Вместе с тем, можно сказать, что я и есть Александр. Я~--- тот, 
кто 
пробыл с вами большую часть недели~--- пробыл запахом, голосом, ощущением. Тот, 
кто рассказывал вам о море и солнце, ветрах и волнах. Я забрал тело вашего 
любовника и сейчас мне нужна ваша помощь.~--- После этого, он вздохнул и теперь 
уже вполне по-человечески улыбнулся.~--- Ну как, верите?\\
--- То есть духи не закатились куда-то под ванну?\\
--- Нет,~--- бывший Александр, если ему верить, конечно, развел руками.~--- Их 
остатки 
я выпил, а сосуд выкинул.\\
--- И Александра вы убили?\\

Замялся.\\

\noindent --- Ну\ldots\ Очень возможно, что да. На самом деле, теоретически, 
когда я оправлюсь 
настолько, чтобы создать собственную плотность~--- я отпущу это тело. И тогда 
Александр может вернуться. Другое дело, что я, к сожалению, не слышал ни одной 
истории о том, что бывает с теми, чьи тела забирают надолго и не для 
конкретного 
единичного поступка. Так что~--- не знаю, но я бы готовился к худшему. Он был 
вам 
очень дорог?\\

Регина поперхнулась.\\

\noindent --- Шутите\ldots\ Вы говорите, что были со мной всю неделю. То есть, 
видели~--- или как 
это назвать?~--- как мы договорились о встрече, как мы первый раз встретились в 
живую\ldots\ Вы, правда, считаете, что люди становятся близкими после 
одноразового 
секса?\\
--- Я встречал и такое,~--- гость пожал плечами.~--- Но я рад, что это не ваш 
случай. 
У него, кстати, жена и дети, так что\ldots\\
--- Я ни на что серьезное и не рассчитывала,~--- Регина вернула жест, но улыбка 
получилось не совсем искренней. Чтобы гость не успел заметить, быстро спросила: 
\\
--- Их вы тоже\ldots\ Того?\\
--- Того?~--- На миг нахмурил брови, а вот интересно~--- это он сознательно их 
себе 
так выпятил? На взгляд Регины раньше было лучше.\\
--- А\ldots~--- Рассмеялся.~--- Нет, письмо отправил. Я, конечно, думал над 
таким 
вариантом, но, с одной стороны~--- я все-таки слишком слаб, с другой~--- у вас 
тут 
слишком, на мой взгляд, тотальный контроль за всем. И тут мы, кстати, и 
подходим 
к моей просьбе о помощи. Так вы мне верите?\\
--- Во всяком случае убивать и насиловать вы меня не собираетесь,~--- Регина 
протянула мужчине пустую кружку.~--- Так что, давайте, вы пока сделаете еще 
кофе --- 
у вас он чудесно получается. А я оденусь и тогда уже мы с вами поговорим\ldots\\

Одеваясь, собралась с мыслями. Гость, судя по шуму, ориентировался на кухне 
вполне по-хозяйски, что может как говорить в пользу его бреда, так и не 
говорить 
ничего. Утренний кофе он как-то делал. Интересно, как она его проспала? Ладно, 
списать на мистику пока, раз уж это объяснение подсовывают. Далее. С ума она 
либо все-таки не сошла, либо сошла гораздо основательнее чем казалось после 
горячечного бреда с галлюцинациями. Голоса, запахи, прикосновения~--- это одно, 
а 
вот вести осмысленную беседу с кем-то, кто объясняет, что был <<духами>>, а 
теперь 
похитил чужое тело\ldots\ На Александра он похож, спору нет, но при этом и 
разница 
такая, что никакой мимикой и сменой сознания~--- о, как!~--- не объяснить. 
Чисто 
физиологическая разница. А вот это уже хорошо, Регина. Регина~--- королева 
дедуктивного метода, во как!\\
--- Так, перед тем как мы с вами сядем пить кофе, разденьтесь, пожалуйста.

Гость, успевший за это время, сервировать на двоих стол и даже приготовить к 
кофе несколько бутербродов из содержимого холодильника, смотрел в окно. В ответ 
на ее требование обернулся и уточнил:\\
--- Вам совсем?\\
Регина задумалась:\\
--- Сначала спину. Потом разберемся.\\
Спина исцарапана. И пятно родимое, под левой лопаткой. Оно еще\ldots\\
--- Так, повернитесь ко мне, пожалуйста.\\
Повернулся. Подошла, обняла. Да, оно еще чуть-чуть выпуклое, как жировик.\\
--- Можете одеваться.\\
--- Спасибо, доктор,~--- Гость, секунду подумав, разрешение проигнорировал и 
сел за 
стол так как был.~--- На самом деле, вы правы, конечно, королева, но не до 
конца. 
Если у меня получилось лицо чуть-чуть подправить, то и пятна-царапины я снять 
мог. Или, наоборот, воссоздать\ldots\\
--- Могли,~--- Регина, устраиваясь напротив него, кивнула.~--- Но будем 
считать, что 
для меня это слишком сложно. А теперь все-таки объясните мне~--- кто вы такой и 
что вам от меня нужно.\\
--- Давайте я со второго начну?~--- Гость разлил кофе по кружкам.~--- Первое 
слишком 
сложно, а снова убеждать вас поверить\ldots\ Я, конечно, в вашем доме себя 
получше 
чувствую, но все-таки не хотелось бы силы тратить. Зовите меня Мо, кстати.\\
--- Хорошо, Мо. Так что вам от меня нужно?\\
--- Понимаете\ldots~--- Регина с удовольствием отпила кофе, гость же пока 
задумчиво 
вертел свою кружку.~--- Все получилось немножко не так, как я ждал. И мир не 
соответствуем моим ожиданиям, и расстановка сил в нем вызывает некоторое 
недоумение. Поэтому мне крайне необходимо кое-куда смотаться, попытаться 
выяснить, что у нас вообще происходит, и вернуться. Вот только сделать это 
привычным образом я, к сожалению, не могу. В результате мне нужно открывать 
коридор\ldots\ Ммм\ldots\ В общем, провести небольшой обряд для которого нужно 
спокойное 
место и еще один человек в помощь. Человеку ничего делать не придется 
практически, только быть. И так получилось, что мне больше некуда идти и не к 
кому обратиться\ldots\\
--- Хорошо.~--- Кивнула.~--- Проводите. Я буду.\\
--- Так просто?~--- Увидев непритворную растерянность Мо, Регина рассмеялась.\\
--- Понимаете, Мо. У меня на этих выходных, видимо, период такой~--- период 
мужчин, 
уходящих по-французски. Ушел Саша, ушел тот, из пузырька\ldots\ Если верить 
этой 
тенденции, то и вы, раз грозите вернуться, уйдете с концами. А меня это более 
чем устраивает. Мне завтра на работу и работать я бы хотела, не отвлекаясь на 
мысли о собственном бреде, пусть даже и готовящим столь вкусный кофе. Так что 
проводите свой ритуал~--- скатертью вам дорожка.\\
--- Королева, вы мне определенно нравитесь!~--- Мо залпом проглотил свой 
напиток и 
запихнул в рот бутерброд. Прожевав, встал из-за стола.~--- У вас дома есть мел? 
На 
худой конец можем и карандашами обойтись, конечно, но мелом удобнее\ldots\\
Регина задумалась:\\
--- Только <<Машенька>>. Против тараканов.\\
--- Пойдет.~--- Получив мел, Мо повертел его в пальцах и зачем-то лизнул, 
заставив 
Регину содрогнуться.~--- На самом деле совершенно не важно чем рисовать. Во 
многих 
традициях считается, что для подобных действий необходима кровь, зачастую кровь 
главных обитателей мира. Глупости. Просто нужно уметь. Другое дело, что четкие, 
хорошо видные линии облегчают процесс, а когда сил мало\ldots~--- Вздохнув, 
гость 
присел на корточки и провел мелом по плитке.~--- Замечательно! Теперь нужно 
место\ldots\ Королева, простите, у вас здесь плитка выложена и достаточно 
мелкая, а 
в комнате, я заметил, тоже мелкие доски\ldots\\
--- Ламинат.\\
--- Да, так это называется. Есть ли у вас где-нибудь в квартире однородное 
покрытие, диаметром хотя бы метр? Ну, или сантиметров семьдесят\ldots\\
--- Коридор,~--- Регина встала со стула и пошла за гостем.~--- А однородное 
покрытие~--- 
это, в отличие от крови, важно?\\
--- Да тоже, в общем-то, нет.~--- Мо выбрал в коридоре место и очертил круг 
вокруг 
себя.~--- Просто, должен признаться, я не так сильно уверен в себе, как 
хотелось 
бы. А это плохо. Вот и приходится облегчать процесс как только можно. Ну, это 
же 
стыдно, в конце концов, не верить, что я даже при таких условиях не смогу его 
провести! Так, королева,~--- Он еще раз осмотрел круг, в котором оказался и, 
видимо, остался доволен.~--- Поставьте, пожалуйста, себе стул где-нибудь в паре 
шагов от круга. И сядьте, конечно. Потом, все что вам нужно будет делать~--- 
это 
наблюдать, а когда я исчезну~--- уйти заниматься своими делами, только стул 
оставьте на месте. А если я через пару часов не появлюсь~--- вам придется снова 
сесть на стул и трижды сказать: <<Мо, пора!>> Справитесь?\\
--- Справлюсь,~--- Регина удобно устроилась на стуле.~--- А если вы и тогда не 
вернетесь, что делать?\\
Мо пожал плечами:\\
--- Ну, что делать, если не вернусь\ldots\ Протереть пол, унести стул на кухню 
и 
вздохнуть с облегчением, видимо\ldots\ Ну что, начали?

Дальше началось странное. Мо несколько раз покрутился в круге вокруг своей оси 
и 
плавно сел по-турецки. Эта деталь удивила Регину~--- она ждала, что гость, если 
не 
повиснет в воздухе, то примет хотя бы воспетую эзотерикой позу лотоса. Нет, 
вполне себе типичное <<по-турецки>>, причем, приняв это положение, Мо еще и 
повозился, устраиваясь. Затем сказал, негромко и ласково:\\
--- Я, Мауи Моук, прошу тебя, коридор, соединись и откройся.

На самом деле, то что он сказал прозвучало примерно так: <<Аш, Мауи Мо, паршау 
атверсти си ши кяля>>, но Регина была уверена, что сказал он именно то, что она 
поняла. Удивилась и этому и еще внимательнее стала следить за дальнейшим. 
Регина 
смотрела, Мо ждал. Через какое-то время он снова повторил слова, а еще через 
какое-то удлинил фразу, добавив несколько эпитетов как себе, так и 
<<коридору>>, 
помянув какого-то Берса и, почему-то Харона. После этого картина перед глазами 
Регины, утратив четкость, поплыла. Правда, ненадолго. Стоило ей проморгаться, и 
окружающий мир вновь стал отчетливым, а Мо, выйдя из круга, прошел на кухню. 
Повернувшись на стуле, Регина видела, как гость, налив себе остывший кофе, взял 
кружку обеими руками и уставился на ручку. Задумался.\\
--- Может, все-таки, крови?~--- Не выдержала.\\
Мо усмехнулся:\\
--- И вы туда же\ldots\ Каждый норовит пнуть сдохшего шакала\ldots~--- 
Запнулся, словно 
вспомнил о чем-то важном, но, не поймав мысль, вздохнул.~--- Нет, кровь 
все-таки 
не поможет. Но вы правы, нужен усилитель. У вас найдется три шеста или чего-то 
палкообразного, длинной примерно метр~--- полтора?\\
--- Лыжи подойдут? Я не любительница и вообще предпочитаю отдыхать летом, но 
иногда отпуск выпадает на зиму, да и корпоративная этика обязывает\ldots

Лыжи вместе с одной лыжной палкой подошли. Составив их над своим кругом 
пирамидой, Мо вздохнул и жестом предложил Регине вернуться на стул. Вернулась. 
Тогда Мо повторил свою речь, на этот раз еще и бурно жестикулируя. Теперь 
Регина 
не сомневалась~--- воздух в круге действительно начал рябить, картинка плыла, 
даже 
сам Мо превратился в большое пятно, которое, судя по колыханиям, продолжало 
махать руками.\\
--- Если хочешь остаться~--- останься просто так\ldots\\
Квартира наполнилась звучанием русской попсовой песни.\\
--- Что это?~--- Пятно покинуло пределы круга и стало обескураженным 
Мо.\\
--- Это\ldots~--- Регина задумалась.~--- А давайте проверим\ldots

На кухне стоял старенький еще кассетный магнитофон. Пользовалась она им очень 
редко, но иногда, зависая ночью на кухне, включала радио~--- чтобы не включать 
музыку в комнате, не тревожить соседей. Правила общежития.

Включила и медленно повела колесико настройки, пока шум радио не 
синхронизировался с музыкой круга:\\
--- Пусть будут все шептаться~--- утром что-нибудь соврем\ldots\\
Это была <<Дискотека Авария>>, а с вами я\ldots~--- Регина приглушила звук.\\
--- Поздравляю, вы поймали <<Русское Радио>>,~--- Иронично пояснила она. Потом 
повторила про себя: <<Вы поймали русское радио>> и опустилась на стул. Кажется, 
она только что стала свидетелем чуда. Не того, что можно объяснить чудом, или 
что пытаются им объяснить~--- а просто чуда. Ну, если не предполагать, что ее 
спортивное снаряжение обладает встроенной фм-функцией. Не обладает.~--- Вы 
знаете, 
по-моему, я готова узнать кто вы и что все-таки происходит.

Мо отошел от продолжавшего транслировать <<Стол заказов>> сооружения и вернулся 
за 
стол.\\
--- Я попаданец, Регина~--- Сказал он все на том же непонятно как понятном ей 
языке.\\
--- Вы представить себе не можете, как я попал\ldots\\
--- И кстати, объясните мне заодно~--- почему я понимаю то, что вы сейчас 
сказали.\\
--- Вы понимаете?~--- Мо усмехнулся.~--- Я так и думал, что нужно было брать 
вас, а не 
этого\ldots\ Александра. Это долгая история.


%\section{Kala~--- пределов нет}
\customsection{Путь Мауи}{melkij\_bes}{Kala~--- пределов нет}


Это долгая история. Долгая и начавшаяся очень и очень давно. Впрочем, не нужно 
забывать, что время относительно~--- и то, что может казаться давним одному, 
для 
другого случилось только что, а для третьего~--- и вообще только случится. Во 
всяком случае, такая гипотеза наиболее непротиворечива и кажется наиболее 
достоверной.

Эта долгая история. И вся она построена на гипотезах. Пусть достоверных и 
непротиворечивых, но~--- гипотезах. Не нужно забывать, что истина~--- это 
всегда 
лишь то, что кажется наиболее на нее похожим. С другой стороны~--- то, что 
человек 
готов принять за истину и является таковым. Для него и для всего участка 
реальности, на которое распространяется влияние человека. Впрочем, это тоже 
гипотеза.

Когда рассказывают эту историю, обычно начинают с того, что реальность вышла из 
хаоса. Когда-то в то самое <<очень давно>>. Однако, возможно так же и то, что 
она 
продолжает выходить до сих пор, возможно так же, что никуда она из него не 
вышла, а продолжает быть его частью. Пусть себе~--- на общую картину это не 
влияет.

Итак, реальность вышла из хаоса. О том, что такое <<хаос>> достоверных и 
непротиворечивых гипотез не существует. Нельзя строить достоверные гипотезы о 
том, что нельзя исследовать и даже с чьим влиянием нельзя столкнуться напрямую. 
Не важно, что такое хаос. Важно, что он есть и он или однажды исторг или, 
скорее, исторгает из себя то, что можно называть участками упорядоченной 
реальности. Или, как предпочитают называть это любители дешевых каламбуров, 
места интегрированной реальности. Миры.

Согласно другой гипотезе, хаос выкидывает не миры, а существ, которые потом эти 
миры создают. Творцов. Спорная и не выдерживающая критики теория~--- уж слишком 
много факторов, ограничивающих их возможность <<создавать>>, да и сама их 
способность к чистому творчеству вызывает много сомнений. Так что теория 
<<истинных творцов>> может существовать лишь как теория <<первоначальных>>~--- 
то 
есть, поскольку известному нам порядку вещей такая теория не противоречит, 
объясняя при этом некоторые мифы и предрассудки, то допустимо предположить, что 
на заре возникновения <<миров>> появлялись и <<истинные творцы>>. Что случилось 
с 
ними и их мирами~--- неизвестно, и разве что Харон мог бы сообщить 
что-то по этому поводу. Но и сам он~--- практически миф. Итак, если <<истинные 
творцы>> и были~--- они ушли. Сейчас уместнее говорить о том, что <<миры>> сами 
каким-то образом создают персонификацию своего порядка~--- хранителя, демиурга, 
бога, который поддерживает функциональность текущей системы миропорядка.

Миров много. Согласно <<Общему полному реестру>> известных миров насчитывается 
порядка 232 тысяч. <<Критический анализ полного реестра>> утверждает, что более 
семидесяти процентов указанных в нем случаев~--- повторения и с уверенностью 
можно 
говорить о существовании шестидесяти~--- семидесяти тысяч миров. При этом самые 
опытные из составителей подобных реестров и из тех, для кого такие списки 
составляются, собственными глазами видели несколько сотен уникальных миров. В 
совокупности. Короче, миров много. В некоторых из них развивается то, что 
принято называть <<разумной жизнью>>. А ее представители, судя по всему, 
обладают 
потенциальной способностью перемещаться между мирами.
Как и за счет чего эта способность становится активной~--- сказать сложно. 
Любая 
типология спотыкается об один простой факт: иногда это просто происходит, 
вопреки всем теориям. Если же все-таки пытаться типологизировать и не учитывать 
при этом случаи, когда кто-то из опытных обучает новичка, то вот стандартные 
объяснения:
\begin{quote}
По тем или иным причинам зарождающаяся разумная жизнь, или отдельные 
представители разумной расы могут быть неугодны миру или, если это не одно и то 
же, его хранителю. Тогда этот неугодный может быть просто выкинут за пределы 
мира.

На определенном этапе представители <<разумной жизни>> начинают задумываться о 
том, как устроен их мир и вообще вселенная. Поиск ответов на эти вопросы 
приводит их к идее <<множественных миров>>. Экспериментируя с ней, они, чаще 
всего 
случайно, находят способ покинуть родной мир.

Задумываясь о возможности существования других миров или невидимых граней 
родного мира, существо начинает представлять эти возможности, то есть мысленно 
творить их. Если в его мире сила мыслетворчества является значимой, или если 
сам 
он почему-то обладает сильной способностью к нему~--- он неосознанно призывает 
к 
себе способ увидеть то, о чем он думает.

В результате попыток познать мир и себя, разумная раса создает множество 
теорий, 
как сугубо естественнонаучных, так и тех, что можно назвать эзотерическими. 
Некоторые из последних совпадают с реальными, но еще не познанными, 
возможностями представителей расы или с возможностями некоторых мутационных ее 
ветвей и, если такой теорией увлекается подходящее существо~--- его опыты 
приводят 
к неожиданным последствиям, в результате чего он проваливается куда-либо в 
другое место.

Вслед за <<разумной жизнью>> в мир приходит цивилизация ее представителей. 
Поскольку созидательная цивилизация возникает лишь путем проб и ошибок, то чаще 
всего она разрушительна для мира. В результате разрушений возникают дыры в его 
изоляции, в которые и попадают везучие или, наоборот, невезучие индивидуумы.

Или же, в процессе развития цивилизации ее представители могут действительно 
дойти до истинного устройства реальности и научиться осознанно путешествовать 
по 
мирам. Эта возможность обусловлена <<множественностью миров>>, однако на 
практике, 
в таком случае, выходцы подобного типа стали бы подавляющим большинством, в то 
время как на самом деле о них ничего не известно.

Наконец, согласно все той же теории множественности, должны существовать миры, 
для разумных представителей которых, другие миры и возможность путешествовать 
между ними является таким же аксиоматичным знанием, как для других~--- сила 
тяжести или возможность полета. Впрочем, и о таких известно крайне мало, хотя 
расу <<стадных метеоритов>> почти со стопроцентной уверенностью можно отнести 
именно к этому пункту.

Ну и, стоит подчеркнуть еще раз, иногда это просто случается.
\end{quote}

\noindent --- И, когда случается, оказывается, что <<попаданцы>> обладают 
невообразимыми 
силами или, на худой конец, везучестью. Знаю, у нас девяносто процентов дешевой 
литературы на этом стоит\ldots\ То есть ты, действительно, попаданец, это был 
не 
<<дешевый каламбур>>?

Надо же. Неделя, проведенная буквально душа в душу, ночь, проведенная, она же 
должна это понимать, втроем~--- но <<вы>>. А стоило провести неудачный сеанс 
связи, 
создав в результате отвратительный музыкальный фон и начать речь о том, что 
лежит за гранью ее мира, так сразу <<ты>>. Забавно.\\
--- Забавный термин, правда? Удивительно, но он у нас действительно 
используется, 
во всяком случае более близкого перевода я не нашел. Я~--- не попаданец, но 
некоторые попаданцы, действительно, везучи. Я же говорил~--- это долгая 
история\ldots

Итак, существо разумной расы выпадает из своего мира. Что дальше? Во-первых, 
то, 
что он выпал откуда-то~--- не означает, что он куда-то попадет. Путешествие 
между 
мирами~--- это вообще отдельный разговор и единственное, что можно здесь 
сказать наверняка, это то, что даже для опытных путешественников такое 
путешествие 
является сложным и слабо контролируемым процессом. Шансов же не раствориться в 
хаосе для случайного попаданца меньше, чем для обитателя данной планеты, 
например, пройти наудачу лесные болота~--- без знаний маршрутов, тропок и 
элементарных навыков прохождения.

Во-вторых, даже если он попадает <<в мир>>, совсем необязательно этот мир будет 
пригодным для жизни. Мир Рамазки, например, представляет из себя замкнутую 
систему постоянного распада. Сложно объяснить, что это значит, но оказаться 
вросшим в обломок скалы и чувствовать при этом, как тают твои кости~--- 
удовольствие то еще. В-третьих, это может быть пригодный для жизни мир, но не 
для жизни конкретного <<попаданца>>. Сухопутная форма жизни может оказаться в 
мире 
водоплавающих и наоборот. Наконец, даже в мирах с более менее схожими 
параметрами жизнь может развиваться иными путями и, хотя внешний вид и язык~--- 
проблема мнимая, но, не сориентировавшись, попаданец точно так же рискует 
обрести смерть или провести следующие двадцать лет в лабораториях каких-нибудь 
медузообразных драконов. Вероятность любого из этих пунктов крайне велика, так 
что тех попаданцев, кому удалось избежать всего этого и успешно интегрироваться 
в местную реальность действительно можно назвать везунчиками.

Что же касается остального\ldots\ Никакой сверхмощи у них, разумеется, нет. То 
есть, 
возможны варианты. Например, если существо расы бойцовых киброидов попадает в 
мир разумных кроликов-пацифистов, то тогда у него есть все шансы стать великим 
полководцем~--- если, конечно, хранитель мира не заметит его пришествия и не 
решит, что этому миру он не нужен. В целом же, при прочих равных~--- удачливый 
попаданец просто проведет остаток жизни в поисках способа вернуться домой, либо 
сделает карьеру сапожника или маэстро раскраски гнезд. Если ему не повезет еще 
раз и он опять не провалится.

Утверждение о том, что искусный оратор приковывает к себе внимание слушателей, 
ловит их в сеть~--- не совсем верно. Мо чувствовал как Регину оплетает его 
паутиной внимания, чтобы ловить, уплетать вылетающие из его рта слова, фразы, 
мысли, чтобы не пропустить не одну из роя и переварить, откусывая по кусочку, 
позже тех, которых, пресытившись, не удастся заглотить сразу. Забытое чувство, 
забытый способ. Встал со стула и, размяв ноги, остался стоять у окна, 
обратившись лицом в кухню.

Природа этого вторичного везения, равно как и всех последующих, до тех пор, 
пока 
его перемещения станут не <<провалами>>, а осмысленными <<переходами>> 
заставляет 
задуматься о природе разумной жизни вообще. Слишком уж эти провалы напоминают 
процессы взросления, то, что в вашем мире можно было бы сравнить с первыми 
шагами ребенка, с тем, как он учится держаться на ногах. Первый шаг~--- 
неосознанный, заставить ребенка повторить его~--- невозможно и, если забыть о 
примере других детей, нет никакой уверенности, что рано или поздно он этот шаг 
повторит. Но повторяет. И еще раз, и еще~--- пока, наконец, не начинает 
чувствовать себя на ногах вполне уверенно, пока, наконец, прямохождение 
становится для него нормой, одной из видовых характеристик. Пожалуй, даже не 
<<одной из>>~--- первой, инициирующей развитие остальных. Многочисленные, 
проводившиеся разными исследователями наблюдения позволяют уверенно выделить 
четыре таких универсальных характеристик <<разумной жизни>>: навык осознанного 
перемещения в пространстве, мышление, речь, экзистенциальная тоска. Интересно, 
что последнее является наиболее универсальным~--- и способ передачи информации, 
и система мышления, не говоря уже о способах перемещения у представителей 
разумной жизни разных миров могут расходиться кардинально. Но тоска, поиск 
места 
как отдельного индивидуума, так и всего вида в своем мире~--- общая.

Теперь вернемся к <<попаданцам>> и попаданиям. Как нельзя научиться ходить 
иначе 
как шаг за шагом, причем каждый сделанный шаг лишь подтверждает возможность 
ходьбы, но не обещает следующего, как постепенно, шаг за шагом, зарождаются у 
человека мышление и речь, так и здесь~--- если после первого попадания следует 
второе, если после него происходит третье, однажды существо понимает не как это 
делать, нет. Однажды существо понимает, что оно умеет это делать. И вместе с 
этим постепенно приходит то, что мы, дабы не дробить теории, называем 
<<наследием 
хаоса>>. Сюда входит владение, то есть способность не только понимать, но и 
пользоваться, универсальным языком, навыки маскировки адаптации, умение слышать 
и творить музыку сфер. Последнее~--- самое важное. Так называют общее умение 
тех, кто уже не является <<попаданцами>> ощущать структуру мира и уметь 
изменять ее и действовать на разных ее уровнях.

Крепчает ли паутина вместе с вниманием слушающего, или внимание крепчает вместе 
с паутиной? Чем крепче паутина, тем крепче связь между ее создателем и ее 
целью. 
Чем крепче паутина, тем меньше тому, кто не создал, но вызвал ее нужно искать 
слова. Чем крепче паутина, тем лучше слушающий слышит сами мысли.

Ощущение от первого прикосновения к <<сферам>>, а у меня это все же 
<<прикосновение>>, незабываемо. Его возможно сравнить разве что с тем, что все 
равно будет непонятно неиспытавшему~--- просветлением, нирваной. Его сложно 
рассказать. Это выход. Выход за рамки движения, мышления, речи. Исчезновение 
тоски. Ощущение себя миром. Отсутствие тоски. Ощущение мира собой. Осознание, 
нет, осознание~--- продукт мышления, все еще ощущение своего места.


Вот только место это обычно занято. И владелец не спешит уступить его. 
Наоборот, 
пользуясь тем, что ему оно привычнее, роднее, что он, в отличие от пришельца, 
создан не для таких мест вообще, но конкретно для этого места, владелец делает 
все, чтобы избавиться от конкурента. Выгнать со своей территории или 
уничтожить, 
в зависимости от агрессивности хозяина, в зависимости от того спасет ли случай 
пришельца очередной раз. Если спасает, то он продолжает свое скитание по мирам, 
но уже зная для чего это делает и что ищет. Мир, который по причине отсутствия 
ли хранителя или по причине слабости хранителя, он может занять.

\noindent --- Так ты сюда и попал\ldots~--- Сказала, спросила, закончила за 
него Регина. Ораторы 
умеют вызывать вокруг себя такую паутину. Чаще всего бессознательно, но умеют 
пользоваться ею пророки и учителя. Такие как он умеют сохранять ее и управлять 
ею. Не создавать, но пользоваться чужой, уже созданной. Чувствовать, когда тот 
ради чьего существования она создана, кого они кормят, кормясь при этом сами, 
устал, пресытился, хочет спать.\\
--- В общем и целом. А попав сюда~--- попал в неприятную ситуацию. В кому, 
можно 
сказать. А очнувшись, очутился в прихожей дома, двери которого~--- и входные, и 
в 
жилые помещения~--- заперты. Ни выйти, ни обустроиться.

За его спиной~--- темнота. Не такая, впрочем, как в бесконечных, вневременных 
коридорах, и не такая, какой она была вечерами за пределами отцовской хижины. 
Темнота нового мира, их мира, мира Регины~--- с фонарями, светом окон в домах 
напротив. Электрические провода опутывали этот мир, напряжение в них заставляло 
его дрожать, не давало уснуть. Электрические провода опутывали этот мир своей 
сетью, своей паутиной. Регина~--- из угла своей~--- кивнула, переварив 
очередную 
муху.\\
--- Хорошо рассказываешь, я себя, прям, студенткой на лекции почувствовала.\\
--- Был опыт,~--- кивнул в ответ.~--- Читал в одном месте несколько десятилетий 
<<Теорию и практику сновидчества>>.

Тогда и понял~--- еще сам, без обсуждений, сопоставлений, долгих споров,~--- 
что 
сеть, объединяющая тебя с разумной жизнью мира, может замещать сеть 
непосредственно с миром. Не полностью, как горсть ягод может притупить голод 
поджидающего у водопоя охотника. И только потом услышал о тех, кто тянул и 
тряс эту паутину так, что со своей паутины слезал большой паук. Хозяин. А потом 
и сам научился. Вот только в этом мире нет паука, бог ушел и непонятно кто 
остался. И потому\ldots\ Мо неожиданно оттолкнулся от подоконника, словно 
разрывая 
ткань прозвучавшего повествования.\\
--- Вот такая, словом, история. И теперь, Регина, я вынужден просить тебя о 
приюте. Хотя бы на несколько дней~--- мне нужно время чтобы разобраться в том, 
что 
происходит и что я, вообще, могу. Пока я точно знаю одно~--- отблагодарить тебя 
я 
смогу наверняка, в накладе не останешься\ldots\\
Регина рассмеялась:\\
--- Неужели ты выполнишь три моих желания?\\
--- Все зависит от желаний, и если я смогу~--- то смогу и больше трех. Но, на 
самом 
деле, я имел в виду то, чем у вас принято расплачиваться~--- деньги\ldots~--- 
Серьезно 
начал отвечать Мо, но, запнувшись на полуслове, улыбнулся.~--- Нет, я не джин и 
не 
золотая рыбка, если что.\\
--- Не знаю, кажешься ты весьма сказочным. И я,~--- Регина задумчиво пожала 
плечами, 
--- я буду дурой, если откажусь твою сказку дослушать. Оставайся, конечно. А 
что 
касается денег\ldots\ Что, наш Александр оказался миллионером и у тебя теперь 
есть 
доступ к его счетам?\\
Смысл этой шутки не пришлось искать в закромах чужой памяти, ее Мо понял 
сразу.\\
--- То есть, по-твоему, я не просто джин, но еще и грабитель с магическими 
способностями~--- убил бедного Сашу в корыстных целях? Нет, я не предлагаю тебе 
его деньги.\\
--- А у тебя самого они разве есть?\\
--- И у меня их нет,~--- Мо картинно развел руками.~--- Я же говорил, у меня в 
этом 
доме нет ничего. Только несколько отмычек~--- и если двери открыть пока не 
получается, то уж какой-нибудь почтовый ящик вскрыть я сумею.\\
--- Ясно. Ты не просто грабитель, ты медвежатник с магическими способностями. 
Ладно, <<маг в законе>>, надеюсь, что хотя бы посуду помыть тебе твои 
<<понятия>> не 
помешают.~--- Регина встала.~--- А я в душ, и вообще\ldots\ Мне завтра на 
работу.

На работе ей обрадовались. И даже не из-за накопившейся за несколько дней 
отгулов работы, а просто, по-человечески. В других обстоятельствах Регина не 
преминула бы отметить этот факт большим плюсиком в своих мысленных счетах с 
миром. А то и несколькими. Но не сейчас. Последние дни были слишком странными, 
а 
слово <<мир>>, даже произнесенное про себя, вызывало во рту вязкую горькую 
слюну. 
Интересно получится, если первое, что она сделает, вернувшись с больничного, 
это 
предоставит для ознакомления коллективу утренний, кстати, приготовленный Мо, 
кофе. Пожалуй, так можно получить и еще неделю отдыха. И репутацию человека, 
совместной поездки в лифте с которым лучше избегать. И, значит, кончатся, не 
успев начаться, комплементы от Олега\ldots

Нет, шутки~--- шутками, но нужно успокоиться, доулыбаться коллегам, получить ЦУ 
от 
генерального и, закрывшись в кабинете, разобраться. Работа, будь честной, может 
подождать, а вот с тобой явно что-то происходит. С твоим миром. Бррр\ldots 

Мо, проводив Регину, налил себе еще кофе, почти привычно откинув гласящий, что 
кофе без сигареты~--- это неправильно, импульс тела. Почти привычно. Вчерашний 
день, неудачный опыт с коридором, паутина, рассказ, который и самому помог 
расставить вещи по местам, спокойный сон~--- все это помогло ему окончательно 
почувствовать себя собой. И, следовательно, понять~--- проблема не в бессилии и 
не 
в утраченных навыках~--- нельзя разучиться ходить, даже если ноги отрезаны по 
колено; проблема в мире. И, это тоже нужно было признать, чтобы двигаться 
дальше, понять что с миром не так и как с этим бороться ему не удается, пока. 
Если не можешь разбить кокос, займись пока костром и лежанкой. Он допил кофе и 
принялся тщательно, неторопливо растирать руки. Они медленно~--- нет, ну надо 
же 
было попасть на такое неприспособленное к простейшим вещам тело~--- наполнялись 
теплом, разгорались, насыщались светом. Теперь разлить это тепло по всему себе, 
дать себе наполнится этим светом. Кажется, получается.

Кажется\ldots\ Кажется, уже решил, что обойдемся без <<кажется>>~--- и так 
тяжело, не 
надо усложнять все еще и недоверием к собственным силам. Получается.

Получается. Убедившись, или же попросту убедив себя в этом, Мо прошел на 
балкон. 
Открыл там окно, разрывая границы между миром и домом Регины, и долгим взглядом 
поприветствовал небо.

Осеннее небо стелилось над городом серым замызганным пледом. Грубым, плотным. 
Узоров не было~--- были пятна. Облаков не было~--- был скатавшийся ворс, 
пролитое 
вино, выцветшая ткань. Что можно увидеть в таком небе? Ничего. Увидеть нельзя. 
Можно прочитать, можно почувствовать~--- но не увидеть. Потянуться к небу, 
ощупать 
его ткань глазами как пальцами, и забыть про глаза. Лишиться зрения и словно 
слепой пальцами ощущать выпуклость точек. Не видеть ничего, ничего не слышать и 
почувствовать тогда как шероховатость неба превращается~--- не в слова, не в 
буквы. В образы в голове.

На носу драккара, поглаживая глядящего в тьму морского горизонта дракона, стоит 
капитан. Шлем в руках, густые светлые волосы развеваются на ветру, короткий 
одноручный топор заткнут за пояс. Капитан отдыхает. Глаза закрыты и, кажется, 
он 
не замечает сейчас ничего: ни ветра, ни бьющих о борт корабля волн, ни возгласы 
буйствующих~--- азартно, но беззлобно, в торжестве победы над стихией~--- 
воинов, ни 
стоящего в двух шагах от него помощника.\\
--- Сколько?~--- Нет, помощника он, может и не замечает, но знает, что кто-то 
из них 
всегда должен быть рядом.\\
--- Шторм унес семерых. Осталось шестьдесят девять.\\
--- Семеро\ldots~--- Капитан открыл глаза. Море отражалось в них. Море смотрело 
на 
море.~--- Семерых мы отдали тебе, Тор, владыка бурь, семерых славных сыновей 
Тюра. 
Прими же их, пусть займут они достойные места на пиру. И пусть сопутствует нам 
твой благосклонный взгляд\ldots\\
--- Шестьдесят девять\ldots~--- Капитан повернулся к помощнику. Теперь во 
взгляде его 
играл огонь.~--- Хорошее число.

И безмолвствует море. И небо снова гладко. И снова ведет по нему палец взгляда 
--- через километры, через года, через миры, через эпохи. И спотыкается в 
сплетении 
тумана, там, где находит черная туча на белокурое облако, и звучат в голове 
вздохи двух, мысли двух, мечтающих быть одним, мечты двух о том, что их будет и 
трое, и четверо. Мечты, которым не суждено сбыться, ибо уже недолго осталось 
скакать личному отряду князя, и во главе которого~--- сам князь\ldots

И будет страшный крик, и оборвется он, подарив кому-то вечную тишину. И 
растворится тишина в небе. В небе, медленно и неохотно скользящем над городом, 
в 
небе, обволакивающем землю, в небе, в котором тонет земля.

Остров тонет. Целиком, весь. Это, теперь уже, факт. Ученые разводят руками, 
маги 
спешно строят порталы на дальние расстояния, но не успевают и тоже разводят. 
Давно забытый бог, основатель рода~--- молчит. Значит, остаются считанные часы.

Сотни лет никто из его семьи не спускался в круглый зал в <<Доме матери>>. 
Сотни 
лет никто из его рода не приходил в круглый зал в подвале царского дворца. Он 
первый~--- за сотни лет. Сидит у массивного, покрытого пылью стола и ждет. 
Убрать 
пыль можно одним щелчком, но пусть увидят остальные. Ждет, когда девять дверей 
в 
разных концах острова откроются, ждет, когда шагнут сюда те, кого традиция 
требует называть <<младшими братьями>>. Девять дверей открылись одновременно. 
Братья пришли. Такими, какими он помнил их с прошлогодних празднеств, но и 
такими, какими на портретах сохранились те, первые, братья. Исчезла праздность 
из взглядов, пропала изнеженность движений~--- девять атлантов, девять царей 
великого острова пришли разделить последние минуты со старшим. Первым не 
выдержал Диапрер~--- потомок самого младшего.\\
--- Ты оставил все это чтобы показать нам, как пали мы, забыв традиции рода? 
Не слишком ли поздно, брат?

Он покачал головой и щелкнул пальцами. Пыль исчезла, стало видно как 
переливаются на стенах выложенные из драгоценных камней узоры, в кабинете 
запахло свежестью и вином. Вино и десять фамильных кубков возникли тут же~--- 
прямо на столе.\\
--- Нет, братья. Я оставил все это, чтобы вы разделили со мной охватившие меня 
чувства. Это не укор, поздно винить кого-либо. Рад видеть вас.

Все расселись. Молча. Выпили. Молча. Все смотрели на него. Он улыбнулся.\\
--- Я не знаю, что сказать вам, братья. Все. Тот, кого легенды зовут нашим 
отцом, призывает домой нас и наш народ. Маги обещают, что через час начнет 
работать портал в царском дворце~--- но он один и не устойчив. Куда выкинет 
тех, кто пойдет через него~--- они не знают.\\
Братья молчали. Наконец, один из них уточнил:\\
--- Ты пойдешь?\\
Покачал головой:\\
--- Я отдал приказ принимать людей на все, оставшиеся у берегов корабли~--- 
торговые, военные, не важно. С собой брать только самое необходимое, тюк на 
семью, независимо от ее многочисленности. Уверен, что и вы распорядились так 
же. 
Многих мы спасем. Но многие и останутся. Я, верховный царь девяти царств, 
останусь с ними.\\
Выпили. Молча. Выпили еще. Не сразу, но зато почти одновременно все девять 
кивнули. Он улыбнулся:\\
--- Я приглашаю вас, братья, разделить со мной последние часы нашего острова. 
Да останется он в памяти людей и богов.\\

\noindent Выпили.\\

\noindent Остров тонул. Весь. Совсем. Все девять царств.\\

Отдающая синью серость неба замерла, растворяя в себе случайную тучу, и 
однородной густой массой двинулась дальше. Солнце упорно пыталось пробиться 
сквозь нее. Кололо, в поисках слабых мест, насыщало полуднем стершиеся, 
истончившиеся участки. Красные прожилки, незамкнутым кругом проявлялись на 
небе. 
Бутун \cyrotld лк\cyrschwa л\cyrschwa рин пролетарлары, 
бирл\cyrschwa шин\ldots\ Visu zemju prolet~{a}rie\v{s}i,
savienojieties\ldots\ Бутун дунё пролетарлари, бирлашингиз\ldots\ Небо 
боролось. Ветер наложил на истончившееся, поддающееся теплу и свету место 
заплату~--- иссиня--черную, прочную, не порвешь.\\

\noindent Словно заплата\\

\noindent Туча висит на небе\ldots\\

\noindent Осень здесь правит.\\

Ветер прилаживает тучу к новому месту, словно лаская ее. Туча дрожит от этих 
ласк. Никто из них не знают, что именно так, именно сейчас зарождается дождь. 
Никто не знает. Просто туча начинает набухать, просто кажется, что в ней 
начинает шевелиться что-то новое, растет, заполняет ее и наконец пытается 
прорваться, разрывая тучу. И прорывается~--- бурным осенним дождем. Дождем, 
заливающим город. Дождем, чьи плети хлещут по стеклам окон. По стеклу балкона 
ползут две капли. Одна~--- маленький цилиндр с тянущимся за ней, расходящимся 
на шероховатости стекла хвостиком отростков. Вторая~--- большая, массивная, 
огромная по сравнению с ней~--- напоминает формой рыбу. Она стремительна, она 
нагоняет первую. Капли стекают по окну, кончая у рамы свой путь.

Мо моргнул. Ладони пульсировали и горели. Обе. Он улыбнулся. Высунул руки в 
окно, представил, как крутым мостом соединяет ладони яркая радуга.\\
--- И снова здравствуй, мир. Прими от меня этот подарок, пусть будет он знаком 
того, что я не желаю тебе зла.

Легким движением подкинул радугу. Оторвавшись от ладоней, она стремительно 
полетела вверх, теряя насыщенность цветов но разрастаясь, разрастаясь\ldots\ 
Тучи, разродившись дождем, прохудились и солнце все-таки нашло себе дорогу. Над 
городом взошла радуга.

Полюбовавшись ею, Мо закрыл окно. В тот самый миг, как ручка окна повернулась 
до 
предела~--- радуга погасла, а сам Мо почувствовал, что сделал шаг. То ли 
все-таки вступил на путь, то ли во что-то вляпался. Интересно, но по-прежнему 
непонятно. Вернулся в квартиру.

Сев за компьютерный стол Регины, пододвинул к себе лист бумаги и карандаш. 
Интересно. Подаренные небом образы~--- сплошь человеческие. Их история, их 
культура, их физиология. Интересно. Впрочем и спрашивалось о людском, так 
что\ldots\ 
Карандаш заштриховывал угол листа. А сам мир разговаривать не хочет. Ни 
разговаривать, ни отвечать. Только о людях и только через людей. Интересно. 
Значит, придется соглашаться с его условиями, значит, придется слушать людей.

Быстро записав на листе одну строчку, Мо отодвинул лист и нажал кнопку 
включения 
на компьютере. Тело утверждало, что навыки работы с компьютером у него есть. 
Будет видно.

Регина стоит у окна. Вверху темнеет серое небо, явно собираясь подарить городу 
очередной дождь. Он никому не нужен, но дареному коню\ldots\ Внизу, по серому 
асфальту ползет поток машин, тоже преимущественно темных. Зеленые, синие, 
черные, но с высоты Регины все они, за исключением редких пятен красного, 
кажутся серыми. Да и красные~--- в пыли, в грязи\ldots\ Некрасиво. Непрактично. 
Поэтому их и не любят. За дорогой по серой плоскости плитки перекатываются 
серые 
крошки гравия. Движение серого на сером создает рябь, как бывает в жаркий день 
на поверхности озера, когда ветра хватает лишь на то, чтобы слегка поглаживать 
воду, словно он~--- любовник, способный после бессонной ночи разве что на 
ленивые ласки. Сейчас осень. Вода забросана опавшими листьями, возлюбленные 
расстались, дом, что должны были заселить еще летом, возвышающийся серым рифом 
над серым озером городского асфальта через дорогу от офиса, недопостороен до 
сих пор. Того не довезли, то не доделали, то не запустили\ldots

По площадке, против ленивого движения строительной пыли, неторопливо, как все 
вокруг, идет мужчина в изначально белом, но посеревшем от пыли, шлеме~--- 
прораб. 
Торопиться некуда: с одной стороны, все сроки уже пропущены, и никого уже не 
волнует~--- сдадут объект завтра, через неделю или через три месяца, с 
другой~--- рабочий день будет тянуться до пяти вечера, когда прервется, чтобы 
возобновиться 
в восемь утра следующего. С одной стороны~--- забор, с обращенной к городу 
рекламой строительной фирмы, с другой~--- магазин, где мужики из бригады 
покупают 
пирожки на обед и пиво на завтрак. Впрочем, и на обед тоже~--- не есть же им 
пирожки всухомятку. Разве что сегодня~--- в виде исключения, из-за приезжавшей 
инспекции. Инспектора побродили по объекту, указали на одно, другое, третье, 
покивали о чем-то своем и уехали инспектировать дальше.  Но пива из-за них не 
купили, а до конца рабочего дня еще вечность, и ничего интересного больше 
сегодня не намечается. Делать, несмотря на то, что недоделано одно, недовезено 
другое, не запущено третье~--- нечего. Прораб вышагивает по периметру вокруг 
зависшего между будущим и настоящим дома, чернорабочие, подменяя будущих 
жильцов, курят, высунувшись в еще голые~--- без стекол~--- окна, каменщики 
лепят из раствора пирожки и замки, сварщики разыгрывают сценки из <<Звездных 
войн>>. 

Серый густой туман заслоняет им, тем, кто внизу, плотное серое небо. 
Подкосившееся небо, медленно наползающее на землю.

В городе почти не осталось воробьев. Который год их не видно. Вороны есть, 
сороки, галки, а воробьи~--- крайне редко. Вслед за прорабом, под взглядами 
собирающихся у окон рабочих, один такой~--- редкий~--- перескакивает по серой 
строительной площадке с места на место, пытаясь найти в серой гальке что-то 
съедобное. Серый.

Серый. Самый практичный цвет. Самый привычный цвет в городе. А уборщица 
халтурит. Постучала пальцем по стеклу. Ситуация не поддавалась анализу. Думать 
не получалось. Точнее получалось, но не о том. Итак, еще раз. Сначала она 
заболела, причем болезнь сопровождалась бредом, или тем, что логичнее всего 
было 
бы принять за бред. Потом в ее квартире оказался человек, знающий о том, как 
она 
провела последнюю неделю, напоминающий персонажа ее бреда манерами, 
напоминающий 
парня, с которым она переспала за день до этого~--- внешне. Дальше этот гость 
сообщил, что он пришелец из других миров, рассказал фантастическую чушь и 
попросился остаться. Да, показал странный и, если честно, не слишком 
убедительный фокус с радио. Да, приготовил очень вкусный кофе. Да, машина 
Александра стоит у нее во дворе. Что из этого следует?

Что же должно следовать из всего этого ведущий аналитик российского филиала 
крупного европейского банка не понимала. Она должна была испугаться за свою 
психику и вместо работы пойти сегодня к психиатру~--- но сумасшедшей она себя 
не ощущала. Она должна была допустить, что с Александром действительно что-то 
случилось и позвонить в полицию~--- но этого она не сделала, и даже мысли такой 
до сегодняшнего дня не возникало. Она должна была решить, что к ней в квартиру 
проник сумасшедший, и опять же вызвать полицию~--- но и этого она не сделала и 
делать не собирается. Что следует из этого? Из этого, можешь не соглашаться 
сколько тебе угодно, следует только одно: поверила. Вот так просто~--- взяла и 
поверила. Почему?

Небо, наконец-то, лопнуло. И через секунду после начала дождя возникла радуга 
--- яркая, большая. Возникла и почти сразу же погасла, растворившись в 
городской 
серости. Да, взяла и поверила. Вот так. Просто. Регина отошла от окна. Он, 
конечно, пообещал много денег, но обещанного три года ждут. Так что, надо 
работать.

Радугу, озарившую в начале дождя небо, во всем городе не увидел больше никто. 
Почти никто.

\newpage

\noindent --- Я дома!

Давно не приходилось возвращаться не в пустой дом. Давно не приходилось 
возвращаться в дом, где горит свет еще до твоего прихода. Давно не приходилось 
говорить это: <<Я дома>>. Было странно.\\
--- Здравствуйте, королева, я рад.~--- отозвалось из комнаты. Разулась, прошла. 
Мо сидел на полу, сложив ноги по-турецки, и перекидывал из руки в руку 
маленький резиновый мяч. Мяча в ее квартире в жизни не было.\\
--- Где ты его нашел?\\
--- Что?~--- Не отвлекаясь от процесса, Мо посмотрел на нее.~--- А\ldots\ 
Регина, что это, по-твоему? Лови!\\
Поймала. Посмотрела. Подтвердила.\\
--- Мяч.\\
--- Интересно\ldots~--- Мо поднялся и подошел к ней.~--- Вообще-то, это 
энергетический шар, спроецированный мною для тренировки ощущения и управления 
собственной энергетики. И для легкой медитации хорошо. Но теперь я согласен. 
Это мяч. Интересно\ldots~--- Мо взял мяч у Регины, повертел в руке и положил на 
пол. Мяч откатился в сторону.\\
--- Интересно\ldots\ Впрочем, ладно. Регина, вот, возьмите.\\
--- Что это?~--- Регина приняла протянутый жильцом листок. Помимо неуклюжей 
графики, там было выведено: <<Викинг 69(+7) - 2, 9, 15 (16), 17, 40, 42>>.\\
--- Моя часть уговора.~--- Отдав ей лист, Мо снова вернулся к мячу, и теперь 
отвечал, склонившись над ним, катая мяч по полу.~--- Лотерейные номера. Лотерея 
<<Викинг>> действительно есть, я проверил, и следующий розыгрыш действительно 
шестьдесят девятый. Вот только тут я не уверен~--- то ли он будет счастливым, 
то ли один из следующих семи. И насчет пятнадцати сомневаюсь. Возможно 
шестнадцать. Так что придется тебе два варианта заполнять.\\
Опешившая Регина не нашла ничего лучшего как спросить:\\
--- Ты веришь в лотереи? Серьезно?\\
--- Я верю во все, что может оказаться полезным. Но вопрос хороший. Я когда 
гулял по городу, видел рекламу одной лотереи, так сначала попытался загадать на 
нее. И почему-то вместо цифр или хоть чего-то близкого к этому, у меня в голове 
все время всплывало непонятное Шамиль Бара Барактян. Не знаю\ldots\\
--- Я знаю,~--- хмыкнула Регина. Телевизор она смотрела редко, в основном~--- 
КВН, 
спортивные и познавательные каналы, считая, что все остальное можно либо 
скачать 
с интернета и смотреть без реклам, либо в том же интернете прочитать без 
цензурных закосов в ту или иную сторону. И если в телевизоре фамилия крупного 
бизнесмена Барактяна могла звучать разве что в новостях экономики, то в 
обсуждении статей на криминальную тему в новостных порталах Бару упоминали 
очень 
часто. Кроме того, Бара был в числе важных клиентов их банка и частью 
информации 
о его экономической деятельности она владела, и ее компетентности как аналитика 
хватало, чтобы верить тому, что писали в комментариях.\\
--- В общем, я тогда просто на выигрыш запросил. И вот результат.~--- Мо 
поднялся в полный рост.~--- Прости, я тут увлекся, в магазин не сходил, ужин не 
приготовил. Сейчас что-нибудь из запасов твоих придумаю\ldots

Проводив взглядом удаляющегося на кухню Мо, Регина хмыкнула. Надо же, и ужин ей 
приготовят. Потом посмотрела на лист и хмыкнула еще раз. Подняла мяч. 
Нормальный 
обычный мяч, такие на пляж часто берут. Может и правда у нее откуда--нибудь 
завалялся. Странно все это. И такое ощущение, что странно что-то еще, что-то, 
что она пропустила. Обвела взглядом комнату. Стол, включенный компьютер, шкаф с 
телевизором, кровать, окно, дверь на балкон, книжный шкаф. Вроде бы все 
нормально. Обвела еще раз, в обратном порядке. Коснулась стены. Решительно 
прошла к компьютеру.\\
--- Мо, подойди на секунду, пожалуйста.\\
--- Что случилось, королева?~--- Правильно услышав гневные нотки в ее голосе, 
Мо мгновенно появился рядом. Регина повернула монитор в его сторону.\\
--- Вот смотри. Это фотография из каталога строительного магазина. Я там 
покупала обои, когда ремонт делала. Эти. Долго выбирала.\\
--- Красивые обои\ldots\\
--- Рада, что тебе нравятся. Только скажи мне, если уж мы оба согласны с тем, 
что они красивые~--- почему, и главное как они, провисев на моей стене три 
года, вдруг изменились?\\
--- Изменились?~--- Мо перевел взгляд с монитора на стену.~--- Интересно\ldots

Регина действительно долго выбирала обои. Хотелось чего-то невычурного, но 
изящного. Светлого, но не яркого. Своего. И, остановившись на голубых обоях с 
золотистым вертикальным, напоминающий схематический рисунок деревьев, узором, 
была довольна своим выбором. Все три года, вплоть до того, как вернувшись 
сейчас 
домой не обнаружила, что голубизна ее стен удивительно похожа на цвет молодого 
леса, а золотистый рисунок потемнел и теперь, то, что напоминало деревья 
действительно кажется их рисунком.\\
--- Красота\ldots~--- Выдохнул, наконец, Мо, но почувствовав на себе взгляд 
Регины, тут же добавил.~--- Я ничего не делал. Интересно\ldots\\
--- Интересно ему\ldots\ Вот уж не знаю тогда, кто это сделал, но точно знаю 
кто будет возвращать как было!\\
--- А зачем, красиво же\ldots~--- Мо провел по стене рукой.~--- Я вообще 
зеленый люблю. Наверное потому что на природе вырос\ldots\\
Не дав Регине придумать чем и куда его стукнуть, помчался на кухню:\\
--- Чего-то я засиделся тут с тобой\ldots\ Через десять минут прошу к столу!

Регине осталось очередной раз хмыкнуть, вздохнуть, еще раз сравнив выведенную 
на 
монитор фотографию с тем, что завелось в ее комнате, и пойти умыться и 
переодеться.

Когда она появилась на кухне, стол, как и обещал Мо, был накрыт. Сам Мо, 
дожидаясь ее, стоял возле стола. Села. Попробовала. Удивилась. Содержимое 
своего 
холодильника она знала, но определить по вкусу, что из этого вошло в блюдо~--- 
не 
смогла.\\
--- Очень вкусно, но что это?\\
Довольно улыбнувшись, Мо сел напротив.\\
--- Упрощенная версия фирменного блюда одного маленького, но популярного в 
своем мире трактира. Чужая память предлагает мне для определения слова 
<<салат>> и <<рагу>>, но в общем по термической обработке~--- это что-то 
среднее. А так\ldots\ У меня это называлось <<Хозяйские закрома>>.\\
--- У тебя?\\
--- На Кладре я держал трактир, достаточно неплохой, как я считаю\ldots~--- 
Мо, убедившись в том, что Регина есть с искренним удовольствием, удовлетворенно 
кивнул и сам принялся за еду. Регина же, найдя возможность сделать паузу между 
вилками, торопливо спросила:\\
--- Где? Еще один из множества миров?\\
--- Можно и так сказать, но он не совсем <<из множества>>. Кладра~--- это 
легендарный 
мир, найти который~--- мечта любого <<попаданца>>. Пожалуй даже не попаданца, а 
опытного путешественника, странника. Это мир, где обитают самые прославленные 
из 
них, легендарный мир живых легенд. Открытый мир~--- его хранитель дает почти 
полную свободу каждому, кто дошел туда\ldots\ Впрочем, это неважно,~--- Мо, 
которому удивительным образом разговор не мешал поглощать <<Закрома>> отодвинул 
пустую тарелку:\\
--- Скажи, что ты слышала про конец света и календарь Майя?\\
Регина удивленно посмотрела на него:\\
--- В декабре этого года, кажется. Очередной конец света, вроде миллениума и 
прочих\ldots\ Их постоянно придумывают\ldots\\
--- Ага,~--- Кивнул Мо.~--- Я тут посмотрел сегодня, с этого самого 
<<миллениума>> вы их 
себе уже больше двадцати спрогнозировали. Что, кстати, тоже показательно. Но с 
2012, увы, все чуть-чуть сложнее. Имя Кинич-Ахау тебе тоже ничего не говорит?\\
Регина покачала головой.\\
--- У вас~--- он бог пантеона Майя, что само по себе забавно, учитывая, что 
пантеон этого народа формировался около четырех тысяч лет назад, если верить 
вашим ученым, а Кинич прибыл в этот мир вместе со мной. Что ж\ldots\ Значит, он 
просто оказался удачливее.~--- Мо встал из-за стола, забрал пустые тарелки и 
переложил их в раковину.~--- А вот вне вашего мира, он один из сильнейших 
известным мне <<скитальцев>>, помимо прочего~--- блестящий ученый. Изначально 
--- представитель разумной расы кошачьих, в вашей фауне наиболее соответствует 
ему ягуар. И как бог он, кстати, тоже в эту кошку обращался. Так вот, этот 
самый календарь обещает конец света вместе с концом эпохи Ягуара\ldots\ В 
общем, знаешь, лично я бы к этому пророчеству прислушался.~--- Завершив 
монолог, Мо занялся посудой, оставив Регину переваривать и его слова, и его 
блюдо.\\
--- И что ты\ldots~--- Начала, наконец, она, но вдруг встрепенувшийся Мо 
не дал ей закончить.\\
--- У нас, кажется, гости.\\
--- В смысле?~--- удивилась Регина. Мо вздохнул.\\
--- Таня идет. Жена Александра\ldots\\
--- Ой,~--- Регина осела. Нет, она, конечно, не монашка и особого стыда даже, 
узнав, что Саша оказался женатым, не испытывала. Семью разбивать она не 
собиралась, мужчину этого видеть в дальнейшем~--- пожалуй, тоже, так что и 
переживать ей нечего. Но общаться с женщиной, чей муж побывал в твоей постели и 
исчез~--- увольте. А если еще вспомнить\ldots\ Ой, а если еще вспомнить, что 
на данный момент этот муж пребывает в состоянии <<не совсем живой>>, его 
машина стоит под ее окном, а тело моет ее посуду\ldots~--- Значит так, 
посуду я сама домою. Иди, разговаривай, похититель тел\ldots\\
Мо поморщился:\\
--- А ведь была идея просто зайти и убить всех. Как спокойно сейчас было 
бы\ldots

Прошел в коридор и остановился возле двери. Повторять смену внешности не 
хотелось. Сил, наверное, должно хватить, с ними, учитывая обои королевы, вообще 
все интересно, но слишком уж болезненная процедура. Значит\ldots\ Если уж 
память 
тела позволила уловить ее приближение, можно попробовать по-другому. Представил 
поднимающуюся по лестнице женщину, позволил чужой памяти придать ей вид 
невысокой темноволосой дамы тридцати девяти лет. Ухоженной и настроенной на 
скандал в лучших традициях стервозных дам, но со следами утренних слез на лице. 
Протянулся мысленно к ее сознанию, прошептал со всей возможной убедительностью: 
<<Таня, мужчина, который встретит тебя~--- это твой муж, Александр. Ты узнаешь 
его и увидишь его таким, каким привыкла>>. Повторил для эффекта еще два раза и 
вышел в коридор, встретив, как раз подходящую к квартире женщину.\\
--- Что?~--- Посмотрев на Мо, хмыкнула та.~--- Курить в квартире не разрешают, 
бедному? В коридор бегаешь?\\
--- Вообще-то бросил,~--- Мо сердито цыкнул на тело, которое, обрадовавшись 
предоставленному поводу, уже готово было спросить у женщины, нет ли у нее с 
собой сигареты.~--- Здравствуй, Таня\ldots\\
--- Здравствуй,~--- Сигареты, кстати, у женщины были. Она закурила.~--- Бросил 
даже\ldots\ Серьезно новую жизнь начинаешь, да? Вот только не очень далеко ты 
уехал, я смотрю!\\
--- Разве расстояние играет роль?~--- Начал было, но почувствовав на щеке ожег 
пощечины решил, что образ глубокомысленного путника сейчас не уместен.~--- 
Прости, Тань\ldots\ Наверное, это было трусливо с моей стороны уйти так, с 
открыткой. А тебе, наверное, не стоило искать меня\ldots\ Как ты меня нашла, 
кстати?\\
--- Как--как?~--- Женщина посмотрела на него как на полного идиота, с которым 
она непонятно почему провела столько времени.~--- Ты правда считаешь, что я не 
знаю ни твоих электронных адресов, ни твоих паролей? Хотя для человека, который 
не только трахает всех подряд, но и всерьез решает уйти из семьи, держать 
паролем дату рождения дочери~--- странно, по-моему.

Мо вздохнул\ldots\ А что сказать, Александр действительно гулял часто, а 
уходить из 
семьи сам не собирался. Так что права женщина, странно это и нелогично. С 
другой 
стороны, весь его опыт говорил, что поступки тех, кого он привык называть 
<<разумной жизнью>>, вообще не слишком стремятся к постоянной логичности. 
Вздох, в данной ситуации~--- идеальный ответ. К тому же и Таню он устроил. 
Нервно затушив сигарету о перила, она продолжила.\\
--- Знаешь, я почему-то даже поверила твоей открытке. Мол\ldots\ Кризис 
среднего возраста, знаешь\ldots\ Кардинально все поменять решил. 
Уехать\ldots\ В паломничество\ldots\ Я и сюда-то ехала так, последней телке 
твоей в глаза посмотреть, ну, может космы повыдирать\ldots\ Может полегчало 
бы\ldots\ А тут машина твоя, ты в коридоре\ldots\ Противно\ldots

Противно, в этом Мо ее понимал и даже готов был согласиться, что Александр 
поступил как последний подлец. И даже мельком пожалел человека: если он 
когда-нибудь вернется в мир, то объяснить супруге, что он лучше, чем показался 
в 
этой ситуации~--- будет очень непросто. Но на самом деле, эти проблемы Мо не 
касались. А вот, если верить взгляду, которым одарила его Регина, если 
разъяренная жена встретится с разлучницей~--- проблемы будут уже у него. 
Мысленно, он успокаивал Татьяну, странно, но хотя сейчас он видел ее, а не 
просто представлял, работа с ее сознанием давалась тяжелее, а вслух сказал:\\
--- Не надо тебе с ней встречаться. Да и не при чем тут она\ldots\ Просто 
так получилось, Тань. Прости. Скажи лучше, ты сюда как приехала?

Пусть тяжелее чем должно было, но переключить сознание женщины ему удалось. 
Продолжения претензий не последовало.\\
--- Как-как\ldots\ На такси.\\
--- Возьми,~--- Он протянул ей ключи от машины.~--- Тебе нужнее. А я\ldots\ 
Может, и в самом деле уеду, кто знает\ldots

Женщина молча приняла ключи. Убедившись, что его воздействие подействовало и 
скандала уже не будет, а запасной линии поведения у нее заготовлено не было, Мо 
открыл дверь.\\
--- Все будет хорошо, Тань. Поверь, так лучше нам обоим\ldots~--- И 
перешагнув через порог вдруг повернулся и удивленным голосом добавил,~--- Все 
действительно будет хорошо. Через полгода ты встретишь человека, который примет 
и тебя, и нашех детей и очень возможно, что вы будете счастливы до конца жизни. 
А еще через год тебя назначат коммерческим директором вашей фирмы\ldots\ Все 
будет хорошо\ldots~--- Скрылся. Женщина некоторое время рассматривала дверь, 
потом вздохнула:\\
--- Сволочь ты все-таки, Сашка\ldots\ И даже поругаться не получилось\ldots

А Мо по другую сторону дверей сосредоточенно думал. В самом предсказании 
будущего ничего удивительного для него не было. Это он еще до начала своих 
путешествий мог, школа отца. Разве что редко предсказания бывают такими 
четкими, 
но и тут понятно: не ему самому, так используемой памяти этот человек очень 
хорошо известен. Странно другое. Почему вся информация обрушилась на него 
именно 
тогда, когда он оказался в квартире? И ведь он даже не искал ее\ldots А еще 
радуга. Обои. Интересно\ldots\\
--- Что, дыхание там, в дверях, переводишь?~--- Отвлек его голос Регины.~--- 
Неужели даже многовековому существу тяжело сладить с разъяренной и оскорбленной 
женщиной?

Интересно. Даже использованные Региной духи, запах которых остался в этой 
квартире, не должны давать такого перепада сил. Мо рассмеялся\ldots\\
--- А вы как думаете, королева? Если женщина своя и ты серьезно к этому 
относишься, то конечно сложно. Один мой друг ради того чтобы убедить любимую 
простить его, однажды даже алмазное небо создал в чужом мире. А это, скажу я 
вам, задача не из легких\ldots~--- Войдя в комнату, поднял с пола мяч. 
Подбросил его.\\
--- С этого все и началось, в общем-то.\\
--- Что все?~--- Успевшая совершить последнюю метаморфозу одомашнивания Регина, 
закутавшись в халат, сидела в компьютерном кресле. Мо остановился напротив нее 
и 
еще раз подбросил мяч. Мяч завис в воздухе и, повинуясь небрежному жесту Мо, 
закрутился на месте.\\
--- Все\ldots\ Казавшийся невозможным до этого побег из Тюрьмы, война с 
хранителями, приход Темного Властелина, новая эра, приход нескольких из нас в 
этот мир, возможно и обещаемый Киничем конец этого мира\ldots~--- Мо 
усмехнулся.~--- В общем, не все удачно получилось.\\
--- Да уж\ldots~--- Вспомнив, на чем они оборвали разговор, Регина 
нахмурилась.~--- То есть ты серьезно про этот конец света?\\
--- Вполне.\\
--- И что делать?\\
--- Тебе~--- тебе выигрывать свои миллионы и с удовольствием проживать 
оставшуюся жизнь, сколько бы ее не осталось. А мне\ldots~--- Мо резко 
дернул рукой и мяч полетел в стену. Но не отскочил от нее, а, расплющившись, 
остался на ней светящимся пятном.~--- Так лучше?

Регина проследила как по стене, в стороны от странной золотистой кляксы, 
разливается голубизна.\\
--- Да.\\
--- Все-таки чувство прекрасного в тебе слабо развито\ldots~--- 
Усмехнулся.~--- А мне думать. Что не так с этим миром и почему, скажите на 
милость, твоя квартира ему не принадлежит?\\
--- Это как?\\
--- Пока не знаю, но выясним\ldots~--- Мо, проследив как обоям 
возвращается оригинальный цвет, повернулся к Регине.~--- Нужно одну гипотезу 
проверить, так что схожу я, королева, прогуляюсь. Вам, кстати, что-нибудь нужно 
--- я все равно и в магазины заглянуть хочу?\\
Уловив вопросительный взгляд Регины, пояснил.\\
--- Буду Сашу нашего грабить. Сколько можно в безвкусных обносках ходить? 
Машину я в семью вернул, но карточки-то у меня остались.\\
Регина фыркнула:\\
--- Все-таки, вор. Вор и убийца.\\
Мо от дверей обернулся:\\
--- А что делать\ldots

Для того чтобы успешно справиться с задачей, человеку нужно определиться какие 
задачи он решает, чьи~--- свои, людские, вселенские. И тогда~--- все 
получится. Но 
человек, он же все смешивает в одну кучу: вселенная для него~--- мир себе 
подобных, а сам он~--- центр мирозданья. Вот и не видит он разницы, вот и 
смешивает\ldots\ А трудно прийти куда-нибудь, если сам рвешься сразу в три 
стороны. 
Крутишься, в результате, как волчок, как карусель~--- сам ни с места, а детали 
снашиваются. Подобно этому пальто.


\noindent ---  Девушка, покажите мне эту куртку, пожалуйста.



Прибыл в этот мир он, решая проблемы вселенной и, отчасти, собственные. 
Оказавшись запертым в этом мире, вынужден решать проблемы сугубо свои, 
возможно, 
то есть даже очень вероятно, решая ради этого проблемы мира. Ни в том, ни в 
другом случае, это никак не касалось проблем местных обитателей. Все умрут~--- 
непреклонность этой истины, оказывается, окончательно доказал папа. В интернете 
нашел. Что ж, кем бы он на самом деле ни был, но его мирок отошел большому 
этому 
миру, и стать богом уже его~--- отцу не удалось.


\noindent ---  Рубашку этой модели на меня посмотрите, пожалуйста. И предложите мне 
какие-нибудь штаны легкого покроя.



А уж куда до отца Александру. К тому же\ldots\ Он говорил Регине~--- пусть 
маленький, 
но шанс, что изначальный носитель тела возродится~--- есть. А если и не 
возродится, то тем более~--- зачем покойнику деньги. Никаких угрызений совести 
по 
этому поводу Мо не испытывал. Разве что жалел, что не хватило терпения слиться 
с 
Региной~--- она казалась более приспособленной для нужд странника. Попаданца. 
Но 
что уж теперь\ldots



\noindent ---  И еще, пожалуй, возьму эти мокасины. Да, плачу карточкой.



И тем более он не переживал о Татьяне и не думал о том, что карточки нужно было 
вернуть ей. Зачем? У нее, к тому же, и так все будет хорошо~--- вернувшись в 
квартиру, он это понял отчетливо. Вернувшись в квартиру\ldots

Мо вышел из магазина. Мешок со старыми вещами положил возле урны. Итак, в 
квартире он может спокойно пользоваться всеми способностями. Причем, порог 
между 
квартирой и остальным миром, он же~--- между квартирой и лестничной клеткой, 
проходит так отчетливо, что при его пересечении происходит что-то вроде взрыва 
внутри Мо, способности активируются сами по себе. Он не собирался гадать на 
будущее Татьяны, он просто его увидел. Действительно ли это так? Почему это так?



Мо остановился напротив разрисованной граффити стены жилого дома. Потер друг о 
друга ладони, почувствовал закипающее в них тепло, развел их. 
Представил-увидел, 
как это тепло концентрируется между ними сгустком энергии, сгустком его воли. 
Все получалось так же легко как в квартире. Направил сгусток в стену. Между 
изображением человечка в синем комбинезоне с одобрительно поднятым большим 
пальцем и ярким рисунком дракона с гиперболизированно--выраженными мужскими 
половыми признаками~--- да, не видели они драконов~--- появилось изображение 
золотого замка с центральной башней по центру, от которой в сторону расходились 
широкие стены. Мо опустил руки. Замок, проявившийся на стене яркой картинкой, 
стремительно тускнел, оставшись, через несколько секунд всего лишь плохо 
различимым контуром. На Мо нахлынуло чувство, подобное тому, что он испытал 
закрыв окна балкона. Только, если тогда он почувствовал смутную угрозу, 
вызванную его действиями, сейчас это был крик инстинктов о надвигающейся 
опасности. Резко обернулся.



За спиной у него, просто из ночного воздуха вышел высокий худой мужчина в 
спортивном костюме. Желтая кожа туго натянутая на череп, казалось, не оставляла 
возможности для морщин, поэтому они, исхитрившись, остались на лице мелкой 
сеткой шрамов. Лысый. Тонкие губы дрожат, зрачки неестественно широки. Взгляд 
отсутствующий.


\noindent --- Что, сука, нашел игрушку, решил поиграться\ldots~--- Словно превозмогая 
боль, 
выплюнул пришелец.~--- Ну, так доигрался\ldots



Зрачки, глаза\ldots\ Мо почувствовал, что через миг он оцепенеет. Рванул.


\noindent --- Шустрый, гаденыш\ldots\ Ну, побегай, побегай\ldots~--- голос за спиной 
звучал 
смертельно спокойно. Словно и не гнался этот тип за Мо, словно спокойно, не 
торопясь шел. Мо на бегу обернулся. Так и было. Нападавший именно шел~--- 
медленно, тяжело, шатаясь. Но расстояние между ними не увеличивалось. Мало того~--- Мо чувствовал как холод сковывает его ноги. Очень медленно, но 
распространяется по телу. Вопрос времени. На улице противопоставить ему нечего.


Прямо перед Мо упал кирпич. Мо обдало волной мелкой крошки. Прибавил скорости. 
До дома Регины оставалось не так много. Должен успеть. Было бы глупо\ldots


\noindent --- Прыткий\ldots


И, кажется, не стоит пытаться завязать с ним разговор. Впустую. Отвлечься, 
сбиться со скорости~--- и все. Дома поговорим.



На то чтобы открыть дверь подъезда потерял две секунды и ощутил на шее дыхание 
догоняющего. Дыхание было тяжелым, видимо погоня и ему давалась не слишком 
легко. Это хорошо. Плохо, что в эти секунды и его тело налилось холодной 
стягивающей тяжестью. Из последних сил взметнулся по лестнице, в дверь квартиры 
Регины прыгнул, не открывая ее. На мгновение ощутил сопротивление дерева, 
испугался, что теория неверна и уже не хватит времени создать новую, провалился 
в коридор квартиры Регины. Поднялся и повернулся к двери, не сомневаясь, что 
преследователь пройдет тем же путем. Оцепенение спало, дыхание восстановилось. 
Когда преследователь появился в квартире~--- таким же способом, пройдя сквозь 
дверь, будто ее и не было, перед ним стояла не жертва, а охотник, причем 
охотник 
многоопытный, охотник на львов, которому точно не страшна какая-та бешеная 
собака. Впрочем, преследователь этого не заметил. Сплюнул на пол, тяжело 
выдохнул, прикрыл глаза.


\noindent --- Все, набегался.

Вышедшая на шум в коридор Регина обнаружила спину Мо в дутой светло-зеленой 
куртке и стоящего рядом с ним то ли наркомана, то ли алкоголика. Хотела 
возмутиться~--- мало того, что сам тут поселился, так еще и приводишь непонятно 
кого, но подавилась словами, когда у нее на глазах куртка Мо начала покрываться 
мелкими, словно от тысячи иголок, дырочками, а выглядывающая из-за воротника 
шея~--- кровавыми пятнами. Еще через секунду поняла, что не <<словно от сотни 
иголок>>: вдруг увидела, как пошел рябью воздух вокруг Мо и через такую же как давеча 
мутную пленку разглядела, что воздух~--- уже не воздух, а именно сотни мелких 
тонких иголок. И все они впиваются в Мо.


\noindent --- Не боись, лярва, не трону.~--- Заметил ее пришедший с Мо.~--- Сейчас с 
хахалем твоим разберусь и свалю\ldots


Договорить он не успел.


\noindent --- Регина, прикройся,~--- Мо, наконец, взмахнул рукой, в руке блеснул прозрачный 
длинный меч, и разрубая шипы воздуха, рассек поперек тело мужчины. То есть, по 
мнению Регины, должен был рассечь. Во всяком случае, мужчина рухнул. 
Одновременно с ним звонко упали на пол, раскрошившись на тысячи осколков 
воздушные иглы. Регина вскрикнула.


\noindent --- Я же говорил~--- прикройся.~--- Мо повернулся к ней. Одежда спереди выглядела 
не лучше чем сзади, лицо и руки покрыты кровавой пленкой.~--- Сейчас, потерпи 
чуть-чуть.

Стряхнул кровь с рук и протер руками лицо. Кровь исчезла. Подошел к Регине, 
провел ладонью ей по лицу.

\noindent --- Больше не больно?


Ощущение, будто бы в лицо впилось множество иголок~--- впрочем, почему будто, 
кажется, так и было~--- прошло.


\noindent --- Не больно. А он~--- мертв?


\noindent --- Не должен,~--- Мо присел на корточки и, поднимаясь, провел руками вдоль 
одежды. Дырки исчезли.~--- Да и не хотелось бы, я думал, что такие люди в этом мире 
вообще невозможны.


\noindent ---  Но\ldots\ Ты же его\ldots~--- Регина махнула рукой, изображая удар Мо.


Мо пожал плечами:


\noindent ---  Ой, подумаешь~--- энергетический поток разрубил. Он срастается быстро.~--- Мо 
шагнул в сторону лежащего у порога тела, но, услышав хруст осколков под ногами 
остановился. Повелительно щелкнул пальцами, превратив осколки назад в воздух. 
Или во что-то. Во всяком случае, покрывшая пол острая пыль, пропала прямо у 
Регины на глазах.~--- Королева, простите, что требую от вас столь недостойной 
вашего положения работы~--- тут придется подмести или что-то в таком духе. Я 
чуть-чуть прибрался, конечно, но мало ли\ldots


Подошел, наконец, к телу.


\noindent --- Так, давай мы тебя сначала посадим и озаботимся тем, чтобы, очнувшись, ты не 
думал ни сбежать, ни напасть, ладно?~--- Тело, видимо, не желая возражать, 
поднялось в воздух, медленно пролетело на кухню, приземлившись на выдвинувшуюся 
табуретку. Мо удовлетворенно кивнул.


\noindent --- Хорошо.


Улыбнулся Регине:


\noindent --- Вы не бойтесь, он очнется совершенно обездвиженным. Простите за беспокойство, 
ладно?~--- и не дожидаясь ответа, прошел вслед за телом.~--- А сейчас мы тебя 
послушаем\ldots

\newpage

Невозможно избавиться от зависимости. Любой~--- наркотической, никотиновой, 
алкогольной. Даже от зависимости грызть ногти. Невозможно. Просто некоторые 
умеют жить в вечной ломке.


Михаил Горелов знал это лучше кого-либо другого. Он уже родился с зависимостью. 
Отец~--- давний любитель винта сначала приобщил к своему хобби мать и только 
потом ее обрюхатил. Обрюхатил, если верить откровениям матери во время наиболее 
яркого за всю ее память совместного прихода. Для отца он чуть ли не кончился 
передозом. Чудом очухался. А Миша чудом родился~--- недоношенный, слабый, уже изначально 
больной. Чудом выжил. Впрочем, сначала благодаря экспериментальной советской 
медицине, на которую долго не думая подписались родители, а уже потом~--- чудом 
выжил после многолетнего курса экспериментальных лекарственных препаратов. 
Единственный из всех тех, на ком их пробовали. Видимо, сказался полученный от 
родителей иммунитет. Спасибо, дорогие.


Программу свернули через десять лет. Лечебный курс признали завершенным, 
лекарства давать перестали. Миша впервые познал ломку. Мама, папаня к тому 
времени благополучно скончался, зная со слов врачей, что такое может быть, а с 
собственного опыта~--- как это бывает, не пустила его в школу, держала в 
постели, 
ставила компрессы на лоб и кормила бульоном с ложечки. Нихрена не помогало, но 
те три дня остались в памяти Миши как самые счастливые. Обычно ему столько 
внимания не перепадало.



Через три дня~--- нет, не отпустило, но все как-то притупилось. Во всяком 
случае, 
казалось, что с этим можно жить, даже встать и пойти в школу. Ну, или маме 
надоело с ним нянчиться, и она решила, что он может. Уже не вспомнишь. В школу 
его отправили, в школе стало хуже. Мутило, болело все, что может болеть, 
особенно голова, руки дрожали, в глазах темнело. Язык отказывался выдавать 
связанные фразы, ноги~--- идти, мочевой пузырь~--- терпеть. Обоссавшись прямо 
на 
уроке, сквозь туман услышал естественный смех одноклассников. К боли 
прибавились 
обида и стыд. Втроем они родили первую за день четкую мысль, пульсирующую 
тоненькой змейкой вены у виска: <<Вот бы они почувствовали какого мне. Вот бы 
они 
почувствовали. Вот бы\ldots.>> Что конкретно случилось в классе, Миша не знает 
до 
сих пор. Вроде бы никто не умер. В школе объявили эпидемию, а за ним приехали 
знакомые с первых лет жизни врачи. Вкололи лекарство.



От мамы его забрали. Она, кажется, расстроилась, но не сильно. Из школы 
перевели. Теперь он жил в специальном интернате, ходил в специальную школу. 
Ежедневно получал лекарства. Память о том, первом разе, притуплялась.


После одиннадцатого класса их всех выставили за дверь. Ему выдали запас 
лекарств 
на месяц, белый билет и сказали к концу месяца прийти за рецептом. И адрес 
мамы, 
которая, хоть и ни разу за это время не появилась, но сына приняла и, кажется, 
даже обрадовалось. Мама прозябала в коммуналке, мыла полы в новомодных 
кооперативах полы, и по вечерам варила для всей квартиры мульку~--- папина 
школа. 
Миша присоединился к ней. Разве что наркота его поначалу не интересовала. А 
потом кончился месяц, он получил рецепт и узнал цену лекарства\ldots\ Как 
оказалось, 
мулька с винтом справлялись не хуже и были куда как доступнее.



Мама померла через полгода. Из комнаты его выперли~--- он ей, оказалось, с 
десятилетнего возраста никто и никаких прав чтобы жить здесь не имеет. На 
следующий день он сидел на скамейке у дома и подыхал от забытого ощущения 
бесконечной ломки. Двое довольных жизнью парней в варенках из соседнего 
подъезда 
узнали его:\\
---  Что, хреново, нарик сраный?\\
Не то слово. Поднял на них мутный взгляд.\\
---  Чтоб вы сдохли, суки\ldots

И следующие две минуты изумленно смотрел как парни судорожно глотают воздух и 
хватаются за горло. Наконец, рухнули посиневшими лицами в асфальт.\\
--- Вот же херня,~--- прошептал пересохшими губами Миша. Потом здравый смысл 
возобладал: обыскал карманы, и, прикинув, что на пару доз хватит, побежал, 
окрыленный предвкушением, к знакомому варщику.

Взяли его через два часа. Менты решили не заморачиваться со странными 
показаниями свидетелей и предъявили ему, что тех парней он задушил. Было, в 
общем-то, пофиг.

Отходняк начался в КПЗ, но там добрые люди сунули в рот папироску, спросили 
канает ли фарш и, получив утвердительный ответ, пообещали помочь. Получив дозу 
у варщика, он в общем-то понял, что падение двух гаденышей на глазах всего дома 
для него фатально, и к аресту подготовился. Выблевав в парашу содержимое 
желудка, нашел в блевоте пакетик с оставшейся после шмона карманов тех 
двоих~--- минус доза~--- добычей. Отдал пакет <<доброму человеку>>, получил одобрительное 
<<Толковый>> и несколько папирос авансом. Папиросы пусть не спасали, но хотя бы 
помогали терпеть.

На следующий день ломало по новой. Спросили: <<Отработаешь?>>\\
--- Как?\\
--- Короче, есть маза, что та паскуда у окна~--- наседка. Сделаешь его~--- 
считай, на неделю экспонатом обеспечен.

Кивнул. После смерти тех лохов~--- много думал. Точнее, думал столько, сколько 
физически мог~--- не так уж и много, но все-таки. И был он сейчас не 
десятилетним пацаненком, и видел случившееся вполне отчетливо. Выводы напрашивались. Почему 
бы не попробовать. Тем более~--- надо.\\
Отыскал глазами указанного чела и прошептал:\\
--- Умри, падла.


Мужчина у окна дернулся и потер висок. Миша услышал как тот сказал собеседнику: 
<<Блин, как сейчас в голове стрельнуло>>. Но умирать, кажется, не собирался. 
Предложивший дело покрутил у виска:\\
--- Короче, хочешь страдать~--- страдай. Условия, если чо, знаешь.

До того как получилось он пробовал еще трижды. В третий раз чел рухнул на пол и 
секунд десять бился в судорогах. Но встал. От момента первой попытки прошло 
часа два, ломка усиливалась, несмотря на почти одна за другой идущие папиросы~--- 
дали вчера <<на сдачу>>. Выплюнул, притушил ботинком и подождал пять минут.\\
--- Умри, падла.

На этот раз получилось. На зону Миша ехал со сроком за убийство и ограбление, 
запасом наркоты и погонялом Шаман. Ехал, крепко задумавшись.

\newpage

\noindent Удар, догнавшим давешним кирпичом, обрушился на затылок. Мо поморщился.\\
--- На самом интересном месте\ldots\ Оживает наш гость. Может его еще раз 
выключить?~--- Он повернулся к Регине. Она, прибрав коридор, пришла на кухню и, пока Мо сидел 
на корточках перед обездвиженным <<гостем>>, изучала лицо пришельца. 
Бесчувствие изменило его лик~--- сейчас он уже не выглядел агрессивным почти сумасшедшим 
наркоманом, нет, на стуле <<сидел>> измученный жизнью и болезнями, но сильный и 
готовый продолжать борьбу и с первой и со вторыми мужчина. Достаточно красивый, 
если не брать в расчет желтизну кожи. Он вызывал жалость и что-то, что могло бы 
развиться в уважение.\\
Она покачала головой:\\
--- Не надо.\\
--- Ну не надо, так не надо~--- Легко согласился Мо.~--- Основное я понял, а 
подробности он нам сам расскажет. Однако сила воли у человека\ldots\ Еще толком 
в себя не пришел, а блоки уже включились. Впрочем, не удивительно.\\
Мо прошелся по кухне.\\
--- Регина, ему сейчас нужно будет что-то выпить. Очень. У вас что-нибудь есть?

Регина встала с места и достала из шкафа открытую бутылку вина. Вина там 
оставалось на неполный бокал. Мо скептически взял бутылку, изучил этикетку, 
понюхал.\\
--- Насколько я понимаю, спирт подошел бы лучше. Ладно, пусть сначала заслужит.\\
Вылил вино в стакан и, оставив стакан на столе, повернулся к Шаману:\\
--- Ну, ты уже должен ожить.\\
Словно повинуясь приказу, губы гостя разжались и выплюнули чуть слышное:\\
--- Падла\ldots\\
И затем:\\
--- Курить\ldots\\
Мо огорченно вздохнул:\\
--- Не угадал. Вот с <<курить>>~--- оно тяжелее будет. А вина могу предложить для 
начала.

Гость открыл глаза. Во взгляде в равных долях смешались боль и ненависть. Судя 
по пробежавшим по телу судорогам, он попытался двинуться. Безуспешно. Закрыл 
глаза, повторил:\\
--- Падла.\\
Регина, вскочив, поднесла стакан к губам того, кого Мо стал называть Шаманом.\\
--- Пейте.

Шаман губами нашел край стекла. Регина подняла дно стакана. Вино потекло в рот. 
Открыв глаза, Шаман сфокусировал взгляд на Регине:\\
--- Спа\ldots\ Спасибо.


Перевел взгляд на от стены наблюдавшего за сценой Мо и попытался изобразить 
ухмылку.\\
--- Сильный, сученок\ldots\\
Мо ответил открытой широкой улыбкой.\\
--- Спасибо. Да и ты не слабак, Шаман.\\
--- Расколол уже\ldots\\
--- Не до конца.~--- Мо подошел к столу и сел рядом с Шаманом.~--- Я бы очень 
хотел понять две вещи: как ты меня вообще обнаружил и зачем тебе понадобилось меня 
убивать?\\
Шаман улыбнулся. Вино помогало~--- в этот раз улыбка получилось живее.\\
--- Радуга. В парке бухал. Смотрю~--- радугу кто-то повесил. Вот и\ldots- Он 
тяжело вздохнул.~--- Слышь, если я еще на сушняке останусь~--- не проканают твои 
браслеты. Порву, нафиг.\\
--- Не порвешь,~--- Мо пренебрежительно взмахнул рукой.~--- Ты, конечно, даже по 
нашим меркам сильный, но в этом конкретном месте против меня не протянешь. Но мучить 
гостей не вежливо. Подождешь пять минут или тебя отключить?\\
--- Протяну.\\
Мо поднялся со стула и обвел глазами кухню. Растер руки.\\
--- Ладно, посмотрим.

На разделочном столе возле плиты образовались три бокала. Странные, Регина 
таких никогда не видела~--- сужающиеся у основания причудливо изогнутые толстые 
неровные стенки из мутного стекла или, поправила себя она, вещества напоминающего 
стекла, широкие изрезанные края, грубые загнутые ручки с двух боков. Почувствовавший ее 
интерес Мо обернулся через плечо и подмигнул. В этот момент на дне бокалов 
появился белый дымящий порошок, над ним вырос слой какой-то грязно-коричневой 
жидкости, над ним~--- слепяще--белый туман и еще выше, заполнив почти до самых 
краев бокалы, изумрудно--зеленая пена. Мо повернулся к остальным. Бокалы, бешено 
вращаясь, пролетели от одного стола к другому. В приземлившемся перед ней 
Регина обнаружила однородный напиток насыщенно--красного цвета. В двух других 
содержание не отличалось. Мо двумя руками поднял свой, пригубил.\\
--- Конечно, лучше использовать натуральные ингредиенты, а не синтезированные, 
однако\ldots\ Вполне ничего, пейте смело.\\
Опустив свой бокал, Мо посмотрел на гостя. Щелкнул пальцами.\\
--- Предупреждаю\ldots
Договорить не успел~--- Шаман, пошатнувшись после жеста Мо, сохранил равновесие 
и рванул бокал на себя. Уполовинил. Повернулся к Мо.\\
--- Понял я. Шаг влево, шаг вправо, попытка к бегству\ldots\ Значит будем 
разговоры разговаривать.\\
--- Будем,~--- Согласил Мо, наконец, садясь за стол.\\
Регина отважилась попробовать. Напиток оказался терпким на вкус, с горчинкой и 
детской мечтой.\\
--- Я мечтала быть феей,~--- Наконец вспомнила и к собственному удивлению вслух 
сказала она. Мо улыбнулся.\\
--- Неудивительно. У меня для вас, королева, есть новость, но это\ldots\\
--- А я~--- о теплой обуви\ldots~--- Задумчиво перебил его Шаман.~--- Постоянно 
ноги мерзли. Что это?\\
--- Облегченная <<Истина в вине>>. Она, правда, и в этой версии запрещена в 
большинстве тех миров, где о ней вообще слышали.\\
--- Почему?~--- Регина поставила бокал. Мо рассмеялся:\\
--- Да вы пейте. Из-за эффекта, из-за привыкания, из-за безумно страшного 
похмелья\ldots\ Да вы пейте, пейте\ldots\ Но у <<легкого рецепта>> эффект 
сведен к минимуму, так~--- чуть-чуть лучше себя вспоминаешь. А от остальных последствий я вас 
избавлю. Пейте.\\
--- В большинстве миров\ldots~--- Шаман уже допил свой бокал и теперь смотрел, 
как он медленно наполняется, поднимаясь от дна.~--- Ты кто, паря?\\
--- Мауи Моук,~--- Мо, улыбаясь, склонил голову набок.~--- Гражданин Кладры, 
профессор сновидчества, дух безмятежной жизни, демон путешествий, трактирщик, рыцарь, 
странник Мо. Но об этом потом. Так зачем ты меня убить хотел?\\
Гость кивнул.\\
--- Будем знакомы, Мо. Я~--- Миха Шаман.

Где-то на втором-третьем плане сознания Регину изумляла реакция <<Михи>>. 
Сидит, знакомится, как ни в чем не бывало, как будто ничего необычного в таких 
посиделках, в таких разговорах, в таких знакомых. Впрочем, и сам гость, судя по 
сцене в коридоре, не слишком обыкновенный. Впрочем, и сама она сидит, слушает, 
смотрит и даже не слишком удивляется. Глотнула. Отпил и гость, обтер внешней 
стороной ладони губы, продолжил.\\
--- Я с детства знаком с магией. Я знаю цену чудесам. Тот, кто готов совершать их 
чисто показухи ради~--- либо псих, либо маньяк. Мне стало страшно за город.\\
Мо кивнул.\\
--- Примерно так я и думал. А ты, значит, для города не опасен?\\
--- Уже нет.~--- Шаман пожал плечами и добавил.~--- Надеюсь. Я, кажется, научился 
с этим справляться и жить. После дебильных попыток завязать, после трех 
наркодиспансеров и двух психушек, после <<белочек>> и передозов\ldots\ Кажется, 
научился. А ты?

Шаман и Мо взглянули в глаза друг другу. Регина со стороны видела безмятежный, 
дух безмятежной жизни, да?~--- взгляд Мо, и жесткие, колючие, полные решимости 
глаза Михаила. Шаман отвел взгляд первый. Присвистнул.\\
--- Тебе не надо?\\
--- Неа.~--- Мо покачал головой.~--- Во всяком случае, обычно~--- нет. Правда, в 
этом мире за пределами этой квартиры я и не могу ничего, по большому счету. И, до 
встречи с тобой, думал, что никто не может.\\
Шаман горько рассмеялся.\\
--- Давно понял, что я~--- выродок. Спасибо, что подтвердил.\\
--- Ты не выродок\ldots~--- Мо поднялся с места, и с бокалом в руках прошелся по 
кухне.\\ 
--- Тут другое. Есть у меня одна идея, но пока еще я в ней не уверен. В любом 
случае\ldots\\
Мо остановился над Шаманом. Они снова смотрели в глаза друг другу.\\
--- Я не желаю зла этому миру. Я не несу ему зла. Я, странник Мо предлагаю тебе, 
Михе Шаману, дружбу и помощь пока наши дороги совпадают и прошу у тебя того же.\\
--- Заметано.~--- Шаман кивнул.~--- Только\ldots\ Ты уверен, что они совпадают?\\
--- Уверен,~--- Мо, держа свой бокал двумя руками, протянул его к Шаману. Шаман 
поднял свой.\\
--- Так,~--- В торжественную процедуру вмешалась Регина.~--- Вы как-то забыли, 
что вы сидите у меня на кухне, что вы, причем оба, без спроса вторглись в мою квартиру 
и мою жизнь, что вы, в конце концов, пользуетесь моим гостеприимством. 
Поэтому\ldots\\
--- Миха,~--- Шаман протянул свой бокал в сторону Регины.~--- Прости за 
беспокойство, хозяйка. Все недосуг узнать было~--- тебя как звать-то?\\
--- Регина,~--- Она церемонно коснулась протянутого бокала своим.~--- Приятно 
познакомиться.\\
--- И мне\ldots\ Будем.\\
--- Простите, королева.~--- Мо присоединил свой бокал к остальным.~--- Я так 
привык к тому, что мы с вами~--- одна команда, что не подумал запросить ваше на это 
разрешение\ldots\\
--- Так что там, с путями?~--- Когда выпили, спросил Шаман.~--- И, раз пошла 
такая пьянка, может меня кто-нибудь в конце концов сигаретой угостить?\\
--- Это сложнее.~--- Мо отставил свой бокал.~--- Я эти ваши сигареты один раз в 
руках держал. Ну, попробуем.

Закрыл глаза, расслабился и попросил тело вспомнить. Ощущение фильтра на губах, 
щекочущий, чуть дерущий горло горький дым. Кружится голова, расширяются 
капилляры, насыщается кровь. Вытянув губы трубочкой он подул. Выдуваемый воздух 
замирал на губах, собираясь в желто-коричневый пятнистый фильтр, продолжаясь от 
него белой бумажной гильзой, гильзой, внутри которой сворачивался мелкой 
табачной крошкой.\\
--- Оцени,~--- Мо вытянул из-за рта сигарету и протянул Шаману. Шаман повертел 
сигарету в руках:\\
--- Можно?\\
Регина посмотрела на Мо:\\
--- Ты, маг вне категорий\ldots\ Обещаешь, что моя квартира дымом не провоняет.\\
Мо, улыбаясь, кивнул. Тогда Шаман тяжело выдохнул, вложил сигарету в рот и 
щелкнул пальцем по кончику. Сигарета загорелась.\\
--- Пойдет,~--- Выпустив дым первой затяжки, резюмировал Шаман.~--- Валяй, 
рассказывай.

\newpage

\noindent --- Итак,~--- Мо хлопнул ладонями и перед собравшимися выросло по фирменному 
кубку трактира.~--- Берс улетел, но обещал вернуться. В отличие от ставших 
свидетелями его самоубийства демиургов. Кто знает точно кто там был?

Мо хлопнул еще раз и стол, вокруг которого расселись собравшиеся, заполнился 
всевозможной едой. Все одобрительно закивали, но к еде пока не потянулись. 
Кинич мокнул хвост в кубок и облизал его.\\
--- Фантастика\ldots\ Спасибо, Моу\ldots\ Ну, точно не знает никто, но я, 
кажется, уходил последним. Тогда оставалось тринадцать. Если никто не успел сбежать\ldots\\
--- Не успел,~--- голос раздался из пустого угла зала. Вслед за голосом из угла 
потянул густой темный дым. Медленными клубами переместившись к столу, дым стал 
столбом. Верхушка его приобрела вид знакомого всем собравшимся лица.\\
--- Берс!\\
--- Ура!\\
--- Ты спасся!\\
--- Ты справился!\\
--- Как ты?~--- Последний вопрос задал Мо.\\
Берс улыбнулся:\\
--- Лучше чем можно было ожидать. Монетки сработали, но не проверяйте их на себе, 
мой вам совет. Простите за внешний вид, пока меня на более материальный облик 
не хватает\ldots\\
Лицо исчезло и дым расстелился по полу. Голос остался.\\
--- Никто не сбежал. Харон любезно позволил мне полюбоваться из своего окна 
последними минутами Тюрьмы. Так что, на данный момент, мы имеем тринадцать 
лишенных демиургов миров. И в наших интересах занять их до того как туда придет 
кто-то другой или сам мир породит нового Хранителя.\\
--- Возражаю!~--- Раффа, первым среди собравшихся, лишь только выяснилось, что 
Берс жив, принявшийся за еду, оторвался от трапезы.~--- Теория рождения Хранителя 
миром, так же как теория Сказочника является недоказанной\ldots\\
--- Однако не противоречит ни одному из известных нам фактов\ldots~--- Включился 
в спор Ноябрь. Дым вокруг собравшихся задрожал мелким смехом. Улыбнулся и Мо.\\
--- Да, именно это сейчас важно. Занять их, если мы следуем плану, в любом случае 
надо. Итак, у нас есть список?\\
--- Есть.~--- Кинич, удовлетворенный дегустацией, обхватил кубок хвостом и теперь 
потягивал напиток, оставляя лапы свободными. Сейчас в них появился свиток 
пергамента, который Кинич, не торопясь, развернул.\\
--- Значит так,~--- Сделав большой глоток, начал он.~--- Зачитываю. X-373, X-375, 
X-579\ldots\ Берс, все так?~--- Завершив перечень, Кинич сложил лист семь раз. 
На восьмом сгибе пергамент исчез.\\
--- Так,~--- Подтвердил дым.~--- Теперь давайте выяснять~--- кто куда, или кто 
кого может предложить.\\
--- Я на 579,~--- Вдруг сказал Мо.\\
Дым свернулся в очертания гуманоидного облика Берса.\\
--- Уверен? Я думал ты со мной останешься?\\
Мо осушил кубок.\\
--- Прости, Берс. Раз уж этот мир без присмотра~--- это личное.\\
На короткий миг дым приобрел плотность. Берс крепко обнял Мо:\\
--- Тогда удачи, дружище. Только очень прошу тебя~--- не застревай там пока 
надолго, ладно.\\
Мо не успел ответить. Сильные лапы еще секунду назад сидевшего за столом Кинича 
обхватили его.\\
--- У меня плохое предчувствие,~--- Тихо, так чтобы никто больше не услышал, 
мяукнул Кинич.~--- Держись там.

\newpage

\noindent --- Предчувствия его не обманули,~--- Мо очередной раз опустошил бокал и оставил 
его наполняться.~--- Я, конечно, ожидал встретить тут других претендентов, но с 
ними я мог и должен был справиться. А тот, кого я встретил\ldots\ У меня не было ни 
малейшего шанса на победу. Чудо, что я вообще выжил. А выжив, обнаружил, что 
этот мир закрыт от всех возможных путей, Хранителя как не было так и нет, мир 
не идет на контакт, не отвечая и не позволяя воспользоваться его силами, а Кинич 
не только приходил, но и оставил четкое указание <<срока годности>> мира. И этот 
срок истекает.\\
--- Удачно зашел, ничего не скажешь,~--- Ухмыльнулся Шаман. Мо, не отвлекаясь от 
повествования, произвел на свет целую пачку, и теперь Шаман курил, не прекращая.\\
Мо пожал плечами:\\
--- Ну, не все так плохо. Зато теперь я знаю какая из теорий о разумной жизни 
мира правдива и, по всей видимости, у меня есть неплохие аргументы в пользу теории 
сказочника. А мир\ldots\ Мир, я думаю, мы спасем.\\
Когда проводили Миху, условившись, что завтра он вернется сюда, Регина спросила 
о том, что зацепило ее в рассказанной истории:\\
--- Ты сказал там, что для тебя это личное. Почему?

Мо, точно так же как он сделал с Шаманом при прощании, обнял Регину. Потом, 
отстранившись, обвел ладонями контур ее силуэта, дернул руками, словно 
стряхивая что-то.\\
--- Вот, теперь все будет хорошо. Почему\ldots\ Мой отец\ldots\ В общем, я сам из 
этого мира, из одного из его сегментов.

\customsection{Путь Мауи}{melkij\_bes}{MAKIA~--- энергия устремляется вслед за вниманием}

Поговорить с Киничем о его опасениях и предчувствиях тогда не успели. А жаль. 
Возможно, успей он сказать, что именно его беспокоило~--- для Мо все сложилось 
бы иначе. Впрочем\ldots\ Разве даже если бы Кинич рассказал ему подробно как все 
произойдет~--- Мо отказался бы? Нет. Когда прозвучало <<x-595>>, просто стало 
ясно: пора навестить дом, пора навести порядок дома. Вот и все. И даже если бы Кинич 
предупредил, что конкретно ждет его~--- что бы это изменило. Ничего. Против 
того, кто шел тогда по площади\ldots\\
По мосту\ldots\\
По огню\ldots\\
По воздуху.\\
Тот, одетый в элегантный костюм эпохи Ренессанса этого мира\ldots\\
В кожаные штаны и кожаную куртку с демонами на плечах\ldots\\
Завернутый в кусок шкуры мамонта\ldots\\
Голый.\\
Тот, кто сказал тогда: <<Ну уж нет, теперь этот мир мой>>\ldots\\
--- Прости, так получилось.\\
--- Умойся.\\
--- Глупая смерть.\\
-- Гы.\\
И моргнул\ldots\\
Щелкнул пальцами\ldots\\
Развел руками\ldots\\
Улыбнулся\ldots\\
Продолжал идти\ldots

Против него у Мо не было шанса. Просто. У отца против смерти шанс был, помешала 
случайность. У него против этого~--- не было, и случайность спасла от 
абсолютной гибели. Состояние, в котором Мо провел следующие несколько веков после той 
встречи, мало способствовали сознательному анализу, но когда сознание 
вернулось~--- первое, что понял Мо, понял абсолютно точно: у него не было шанса. У него не 
должно было быть доли секунды, позволившей поймать направление удара и принять 
его. И если Мо верно оценивает собственные силы, а от излишней скромности он 
никогда не страдал, то вопрос о том, откуда же взялось это случайное время~--- 
это еще один пунктик в списке вопросов, ответы на которые необходимо найти. Но, 
пока не самый главный. Пока множество более общих <<почему?>> выигрывают у одного 
личного <<откуда>>.

Регина спала. Забавная женщина. Зацепившись о <<личное>>, о новости, о которой 
он почти проговорился, уже не вспомнила. Впрочем, понятно~--- она сейчас и так в 
новостях недостатка не испытывает. Вот и славно. Не стоит говорить человеку 
вещи, способные заставить его полностью пересмотреть свою жизнь, если сам в 
этих вещах не уверен. А уверен ли он?

Последние два часа Мо сидел в центре кухни, примерно на высоте полутора метров 
от пола. Закон притяжения не возражал. Вывод, подтвержденный уже 
экспериментальным путем: в этой квартире законы этого мира не столь 
категоричны, как за ее пределами.

Почему? А почему на островах отца при должном умении можно было говорить с 
огнем и ветром? И это умение мало было связано с музыкой сфер, скорее и ветра, и 
огонь были на тех островах живыми. Потому что острова еще не стали частью общего 
большого мира, потому что эта квартира не является однородным кусочком 
окружающей ее реальности. Почему? Еще одно <<почему?>> и на него Мо тоже мог 
попытаться ответить\ldots\ Вот только ответ, который он мог предложить, был слишком 
революционным, а Раффа тогда был прав: теория не просто не доказана: ни разу 
никто не находил аргумента, который, хотя бы с допустимой натяжкой, можно было 
бы счесть аргументом в пользу ее истинности. Хотя в целом, она не противоречит 
общепринятой системе. Искать и найти доказательства истинности или ложности ее. 
Завтра. А пока~--- откатиться на несколько пунктов назад.

Сферы этой квартиры поддается чужому влиянию. Сферы этого мира~--- нет, или, во 
всяком случае, минимально. При этом есть Миша Шаман, способный усилием воли 
заставлять мир меняться. Что это значит? Во-первых, это дает неплохое 
подтверждение еще одной спорной теории, теории разумной жизни. Если 
воспринимать мир с его законами, с его материальностью как физическое тело, а Хранителя~--- 
его мозгом, посылающим телу импульсы, следящим за состоянием тела, то вполне 
возможно, что разумная жизнь~--- что-то вроде души данного мира. Воля. Если 
так, то Миха обладает таким запасом ее, что на практике способен в одиночку творить 
то, что в теории могут делать с миром обитатели единым, чаще всего 
бессознательным порывом. Во-вторых\ldots\ Во-вторых, учитывая, что пойманный 
отрывок истории Миши всплыл в сознании Мо не мысле--образами, как это обычно бывает, а 
текстом, может получиться совсем интересно. Но, впрочем, неудивительно. Если ты 
открыт всем ветрам~--- попутный ветер обязательно найдет тебя. Значит, 
завтра~--- проверить. А сейчас убедиться в том, что квартира действительно подвластна ему.

Мо потянулся и нащупал ногами пол. Встал. Встряхнул на весу руки. Потер ладони. 
Улыбнулся. Подошел к двери в комнату Регины, встал к ней спиной. Вытянул перед 
собой руки.\\
--- Ike~--- мир таков, каким вы его представляете!

Стены под его взглядом ожили, узоры побежали по ним радужной рябью, собрались в 
форме пальм и лиан и замерли. Ламинат на полу почернел, пожелтел, наконец, 
налился сочным зеленым цветом, с тонкой линией цвета утрамбованной земли. Белый 
потолок навис синевой.\\
--- Kala~--- пределов нет.


Узоры деревьев перестали быть узорами. Стены исчезли, исчез и пол и потолок. Мо 
стоял посредине тропического леса, опираясь спиной на непонятно откуда 
взявшуюся тут дверь. Повернулся к лесу спиной, повернул ручку двери.\\
--- Вставайте, королева! Сказка ждет вас!~--- голос, тот самый голос из недавнего 
странного полубреда, голос, погружавший ее в странные сновидения, голос, 
затихая, заставлявший ее выныривать обратно в реальность. Не открывать глаза, а 
то он исчезнет, пусть сниться\ldots~--- Вставайте!

Ну вот\ldots\ Открыла, и хотя голос никуда не исчез материализовался и его 
обладатель~--- Мо. Регина посмотрела на часы у изголовья кровати:\\
--- Мо, мне же на работу завтра\ldots\\
--- Регина, ты выспишься, это я тебе обещаю,~--- Мо склонил голову в легком 
насмешливом поклоне: мол, это даже не чудо, так.~--- А я хочу показать тебе 
сказку. Старую--старую, такую, что даже я сам ее уже плохо помню. Решай.\\
Протерла глаза. Вздохнула.\\
--- Отвернись, халат накину.\\
Перед тем как выполнить ее просьбу, Мо повторил свой поклон. Кажется, еще 
насмешливее. Зараза.\\
--- Готова?~--- Мо оглянулся через плечо и, удостоверившись, что Регина 
действительно готова, подошел к двери. Нежно погладил ручку:\\
--- Makia~--- энергия устремляется вслед за вниманием!

\newpage

Гневная мысль: <<Что с моей квартирой?>> не успела обрасти словами. Выйдя вслед 
за Мо за порог, Регина вдохнула пестро насыщенный теплый воздух. Ароматный, чуть 
пряный, но при этом непривычно, до головокружения чистый.\\
--- Что это?~--- тихо спросила.\\
--- Мауи,~--- так же тихо отозвался Мо.~--- Один из нынешних Гавайских Островов, 
только много--много лет назад.\\
Много-много\ldots\ Регина помяла мясистый пальмовый лист.\\
--- Это иллюзия?\\
--- Это~--- память мира. Это не иллюзия, но мы не сможем никак соприкоснуться с 
местными обитателями. Они нас просто не увидят и не услышат, а если попытаемся 
коснуться их~--- или не заметят, или почувствуют легко шевеление ветра. Но сам 
по себе~--- это настоящий Мауи несколько--тысячелетней давности.

Казалось, что голос Мауи дрожал. Регина оторвалась от созерцания окружающего 
пейзажа и посмотрела на спутника. Он улыбался, но уголки губ предательски 
дергались, а в весело--отстраненном обычно взгляде струилась сейчас странная 
печаль.\\
--- Что с тобой?\\
Мо тряхнул головой.\\
--- Все в порядке, королева.\\
Поднял голову и посмотрел на солнце.\\
--- Пора идти, жалко будет, если опоздаем.


Только сейчас Регина заметила, что они стоят на протоптанной тропинке, уходящей 
куда-то в глубину леса. С сомнением посмотрела на дверь~--- сказка--сказкой, но 
удаляться от единственного очевидного входа домой было страшно. Мо, уже 
успевший пройти несколько шагов, обернулся. Уразумев причину ее сомнений, рассмеялся:\\
--- Ага, вот такой оригинальный способ от тебя избавиться я придумал. Идемте, 
королева, обещаю, что через час-полтора вы будете дома, и даже, более того, 
дома за это время пройдет от силы пара минут.

Несколько секунд Регина смотрела в спину удаляющегося Мо. Потом, решившись, 
последний раз коснулась двери, словно ее прикосновение должно было придать 
двери сил выстоять, и побежала за ним.\\
--- Мауи\ldots\ Ты, тогда в коридоре, говорил, что ты~--- Мауи Моук, да?\\
--- У тебя хорошая память,~--- Не останавливаясь, Мо коснулся пальцами ветки 
дерева, мимо которого проходил.~--- Здравствуй, хорошая\ldots\\
--- То есть, ты из этих мест?~--- Регина попыталась понять чем это конкретное 
дерево отличается от других, растущих вдоль дороги, но разницы не заметила.~--- Что-то 
вроде <<Толян Питерский>>?

Мо хмыкнул. Если Регина правильно понимала происходящее, то с каждым часом он 
все лучше владел чужой памятью. Во всяком случае, она была уверена, что еще два 
дня назад шутки про <<Толяна>> он бы не понял.\\
--- Не совсем, тогда уж <<Данкан Маклауд из клана Маклаудов>>. Мауи~--- в данном 
случае, это что-то вроде родового имени. Мауи Моук, если перевести~--- сын 
Мауи. Папа не слишком озабочивался проблемой имен, решив, что и для меня, и для брата 
его имя будет достаточным оберегом. В целом, так оно и должно было быть, и у 
меня даже нет оснований считать, что так не было.~--- Мо подмигнул Регине.~--- 
Я, как-никак, до сих пор жив\ldots\\
--- А кто такой твой Мауи тогда? Какой-то вождь-царь местного\ldots\ населения?\\
--- Племени Тумба-Юмба,~--- Мо правильно понял заминку. Положительно, она права 
насчет его памяти.~--- Сейчас сама все узнаешь. Вот он.

Тропинка, резко свернув, уперлась в лужайку в центре которой стояло странной 
формы\ldots\ Здание, Регина не подобрала другого слова. Назвать хижиной это 
сплетение живых пальм, лиан и каких-то неизвестных ей растений~--- она, 
впрочем, и насчет пальм с лианами не была уверена,~--- не поворачивался язык. Возле жилища 
полулежал смуглый темноволосый папуас. Рядом с ним стоял деревянный, 
напоминающий кувшин, сосуд.\\
--- Усаживайтесь, королева,~--- Мо подал пример и по-турецки уселся на траве в 
метре от Мауи.~--- Сейчас будет интересно.

Регина села рядом. Мужчина потянулся к кувшину, отхлебнул из него. Потянулся и 
резко сел на земле. Интересно~--- сел в ту же самую позу, которую на несколько 
секунд раньше принял Мо.\\
Мауи хлопнул в ладоши.\\
--- Ну, появляйтесь уже, оболтусы,~--- крикнул лениво.\\
Регина удивленно посмотрела на Мо:\\
--- На каком языке он говорит?\\
Мо, не сводя пристального взгляда с отца, отмахнулся:\\
--- Не важно. Память мира понятна всем, кому она доступна.\\
Не то чтобы Регина поняла объяснение, но переспросить не успела. С приглушенным 
травой ударом на землю упало двое юношей-туземцев.\\
--- Пап, ну мы бы и сами пришли\ldots~--- обиженно протянул тот из них, кто 
встал на ноги первым и теперь потирал рукой ушибленный зад. Второй, поднявшись с 
небольшим опозданием, согласно кивнул. Мауи фыркнул.\\
--- Я вас три часа назад звал, так что нечего теперь. Признавайтесь, что еще 
натворить успели.\\
Парни переглянулись и пожали плечами. Мауи хмыкнул.\\
--- Не признаетесь. Ладно, тогда рассказывайте: вы зачем дяде мною подаренный 
посох испортили?\\
Парни снова переглянулись и несмело хмыкнули. Тот, что поднялся первым, ответил:\\
--- А мы не портили. Он, когда тебя рядом не было, все жаловался, что ты ему 
посох слишком тяжелый подарил, что ему и так по земле ходить тяжело, а с ним вообще 
неудобно\ldots\ Мол, век бы он его в руки не брал, но тебя обижать не 
хочется\ldots\ Ну, 
мы ему и помогли\ldots\\
Мауи весело рассмеялся.\\
--- Помогли они\ldots\ Мне потом каждый житель деревни рассказывал: идет брат мой 
тучно--важный, на посох опирается по тропке из лесу. Идет, брови по обыкновению, 
хмурит, и вдруг раз~--- и падает. Лицо~--- в зверином помете, в руке~--- 
обрывок лианы. И как подгадали-то?\\
--- А зачем подгадывать?~--- Разговорчивый парень удивленно поднял брови.~--- Мы 
просто с посохом договорились, что он лиану вспомнит именно тогда, когда дядя 
рядом с той конкретной тучей на него обопрется. И с кучей договорились, что она 
дядину улыбку поймает. Угадать сложно\ldots\\
--- Сложно им. Неучи!~--- ворчливо бросил Мауи и потянулся к кувшину. Отхлебнул. 
Поставил его на землю.~--- Ладно, а что там с проливом было?\\
И снова парни переглянулись. И тот же ответил:\\
--- А что было? Мы как лучше хотели. Они все ворчали: мол, хорошие острова, 
ничего не скажешь, одно плохо~--- на соседний остров перебирать тяжело, каждый раз 
лодку брать надо. А если двумя-тремя семьями в гости собраться, то и несколько\ldots\\
--- Вот вы им и помогли\ldots~--- Кивнул Мауи.\\
--- Помогли.~--- Согласился парень.~--- Попросили воду ниже опуститься, землю 
подняться и обняться берегам\ldots\\
--- Ладно, берега сцепили и сцепили~--- лес с вами. Но зачем 
опускать-поднимать-то было? Как им сейчас до воды добираться прикажете?\\
--- Пусть колодцы копают,~--- буркнул, наконец, второй. А первый согласно добавил:\\
--- Мы им так и сказали, да. А там зато, теперь, такой красивый обрыв 
получился\ldots\\
--- Так вы, значит, местность украшали,~--- Отсмеявшись, Мауи снова хлебнул.~--- 
Про остальное, я так понимаю, спрашивать бессмысленно? Вы, одним словом, всем 
помогали?\\
Парни синхронно кивнули.\\
Мауи вздохнул.\\
--- Оболтусы. И что мне с вами делать?\\
На этот раз промолчали оба. Мауи устало протер лицо.\\
--- Скажите мне, сыны мои,~--- Неожиданно серьезно спросил он.~--- Я~--- кто?\\
--- Ты~--- великий Мауи, Великий Шаман, бог и создатель, творец и хранитель этого 
мира.~--- Мгновенно, словно машинально ответил тот из парней, что все больше 
молчал.\\
Говорливый добавил:\\
--- Ты тот, кто придумал наши острова и один из тех, кто может придумать весь 
мир, включая бескрайнюю воду и земли, на других краях ее, если таковые появятся.\\
--- Когда таковые появятся.~--- Поправил Мауи.~--- В том, что это произойдет~--- 
не сомневайся. Я это знаю. И вы бы знали, если бы что-то кроме примитивных приемов 
из моих уроков усвоили. Впрочем, лучше чем ничего. В остальном~--- все так. И 
что, как вы думаете, я, как творец и хранитель, должен делать с нарушителями порядка 
моего мира?\\
Парни синхрона опустили взгляд на землю. Мауи вздохнул.\\
--- На этот вопрос у вас нет ответа. Может, тогда, на другой ответите: что я, как 
отец и наставник, должен делать с двумя оболтусами-переростками, которые 
усвоили все, что удосужились перенять из моей науки и теперь шатаются по моему дому без 
дела и дурят, не зная куда силу своей молодости девать?\\
Парни молчали, но в этот раз Мо не стал перебивать их молчания. Наконец, первый 
спросил пересохшим голосом:\\
--- Куда?\\
--- Жить.~--- Коротко ответил Мауи и щелкнул пальцами. И снова Регине показалось, 
что она уже видела как абсолютно так же делает это Мо. Перед парнями появилось 
по кувшину, такому же как перед Мауи, только в несколько раз меньше. Мауи 
поднял свой:\\
--- Удачи вам, мальчики. Пусть любой ветер будет для вас попутным.~--- Отпил 
несколько глотков и резким движением выплеснул остатки напитка между собой и 
сыновьями. По воздуху прошла рябь, между Мауи и сыновьями возникла прозрачная, 
переливающаяся радугой прозрачная стена.\\
--- Ты разрешаешь нам пользоваться коридорами?~--- удивился первый. Мауи 
отрицательно покачал головой:\\
--- Я заставляю вас уйти в коридоры.\\
--- Куда?~--- Спросил второй.\\
Мауи улыбнулся.\\
--- К звездам\ldots\ Я не знаю. Сами разбирайтесь. Все, побежали, или мне вас 
туда закидывать прикажете?


Парни повернулись лицом друг к другу. Минуту, две, во всяком случае~--- очень 
долго~--- пристально смотрели друг другу в глаза. Наконец, обнялись и 
повернулись к отцу.\\
--- Удачи, папа!~--- сказал один и ступил в рябь.\\
--- Увидишь, на самом деле мы многому научились,~--- улыбнулся другой и шагнул 
следом. Радужная стенка задрожала и взорвалась желтоватым фонтаном. Регину 
обдало жидкой влагой с приторно--сладким запахом. Мауи между тем улегся на землю 
перед жилищем. Потянулся к возникшему рядом с ним кувшину, отхлебнул.\\
--- Надеюсь, надеюсь. И вам удачи.~--- Отхлебнул еще раз и вдруг 
расхохотался.~--- Значит, говорите, посох ему мой не понравился, по земле ему ходить 
тяжело\ldots\ Ну, как знает\ldots\\
Некоторое время ничего не происходило. Регина повернулась к Мо~--- тот сидел, 
словно застыв, глядя прямо перед собой. \\
--- Мо, ты как?\\
Встрепенулся. \\
--- А? А\ldots\ Все нормально, королева. Хотите увидеть, что отец с дядей сделает?\\
Регине хватало впечатлений и, к тому же, она была уверена, что хватило их и Мо 
на сегодня. Она покачала головой:\\
--- В другой раз, если ты не против. Спасибо тебе за сказку, но я бы, пожалуй, 
вернулась домой.\\
Мо поднялся на ноги и склонился перед Региной в поклоне:\\
--- Ваше слово, королева, для меня закон.~--- Протянул ей руку.~--- Тогда вставай.

Назад возвращались молча. Мо шел ровным шагом, рассматривая тропу у себя под 
ногами, а Регина, понимая, что ей дали прикоснуться к личным и очень 
сокровенным воспоминаниям, заговорить не решалась. Только когда впереди стала видна дверь 
отделяющая мир жизни Регины от мира воспоминаний Мо~--- спросила:\\
--- А кто из них ты?\\
--- Наглый,~--- Не доходя до двери Мо остановился.~--- Хотя Микала тоже наглый, 
но по-другому. Он, когда за нас отец брался, предпочитал отмалчиваться, считая, 
что оправдываться~--- ниже его достоинства. А я все сводил в шутку, надеясь на 
любовь отца к юмору. Получали, правда, в результате одинаково\ldots\\
--- Микала\ldots\ Тоже красивое имя. А что с ним стало?\\
Мо пожал плечами:\\
--- Я не знаю, Регина. Для нас обоих это был первый опыт перехода по коридорам в 
другие миры, до этого отец нам разрешал только в другие секторы этого мира 
перемещаться. Я выпал в мир Пирпити прямо в центр схватки двух прайдов 
арахнольвов, и это первый среди бесконечного числа раз, когда в моих 
путешествиях я выживал чудом. Что ждало Микала на его выходе~--- я просто не 
знаю и, учитывая, что больше я его не видел, боюсь представить.\\
--- Прости,~--- Регина погладила Мо по плечу. Он вздохнул:\\
--- Все хорошо. Все это было очень давно, так давно, что даже в памяти местных 
народов осталось смутными легендами~--- и создание мира, и наше изгнание, и 
гибель отца\ldots\ Просто увидеть это сейчас, это как услышать колыбельную, которую 
пела тебе кормилица~--- светло и грустно. А в том, что люди теряют братьев, 
переживают родителей~--- в этом нет ничего удивительного. Это нормально.\\
--- Мауи умер? А\ldots\ А где в этой истории ваша мать?\\
--- Мауи умер. Во всяком случае, если верить остаткам мифов, которые мне удалось 
найти в интернете. А матери в этой истории просто нет. Подозреваю, что придумав 
острова, придумав людей, придумав огонь и язык, отец просто придумал и нас~--- 
то ли для развлечения, то ли ради каких-то своих целей.\\
--- Да, точно,~--- Регина отшагнула назад, чтобы иметь возможность смотреть Мо в 
глаза.~--- Вы там говорили что-то такое. Что значит <<придумал мир>>? Просто 
взял и придумал?\\
--- Возможно,~--- Мо огляделся, вздохнул и щелкнул пальцами. Регина ощутила в 
руках тепло от уже знакомого кубка, правда на этот раз в нем был темно-коричневый, 
почти черный напиток, запахом напоминающий кофе. Чуть позже она поняла, что 
сидит, сидит на привычном кухонном стуле, а вокруг нее не тропический лес, а 
родные, собственноручно обклеивавшиеся моющимися обоями, кухонные стены. 
Напротив нее, с таким же кубком в руке, сидел Мо:\\
--- Я полагаю, что в своем случае, отец просто использовал такую терминологию, 
чтобы рассказать нам о Хранителях, но и просто придумать мир, или, даже точнее, 
новый сегмент мира~--- теоретически возможно. Людей, способных на это, мы 
называем Сказочниками, и хотя все выкладки показывают, что они действительно могут 
существовать, до сих пор все это оставалось теорией. Во всяком случае, до сих 
пор единственным примером такого сказочника, имеющимся у меня, был мой отец, но 
и то~--- я же не знаю, как все это было на самом деле. Во-первых, он создал 
мир, или, во всяком случае, мир возник до того, как в нем появились мы с Микала. 
Во-вторых, тогда у меня просто знаний не хватало оценить и понять все то, что 
вокруг меня происходило. Так что, до сих пор у меня не было возможности 
убедиться в том, что Сказочники действительно существуют.\\
--- До сих пор\ldots~--- Регина попробовала напиток. Действительно, похоже на 
кофе. А еще спать\ldots~--- Ты это <<до сих пор>> несколько раз повторил. Надо 
полагать, что-то изменилось? С сей поры?\\
--- Подловила,~--- Мо улыбнулся.~--- С сей поры у меня появилось предчувствие, 
что я найду Сказочника в этом мире. Как минимум, я всерьез намерен найти его. А пока, 
как видишь, сам пытаюсь понемногу исполнять его функции\ldots

Регина хотела что-то ответить, поблагодарить за рассказанную-показанную сказку, 
но голова так сладко тонула в подушке, одеяло так тепло и нежно накрывало тело, 
в голове под медленную баюкающую мелодию щел слайд-показ картинок из давешнего 
теплого леса. Она заснула. На кухне Мо неторопливо допивал свой напиток.\\
\\

\noindent --- Ну чо, стою перед тобой как лист перед травой, как стукач перед прокурором. 
Учи меня, фигли!

В квартире запахло табаком и перегаром. Миха появился прямо в прихожей, 
пропуская, по вчерашнему образцу, этап со звонком, открытием дверей и прочей 
необязательной, видимо, ерундой. Не дождавшись ответа, заглянул на кухню, и, не 
найдя там никого, направился в комнату.\\
--- Ау, начальник!\\
Мо, не отрывая глаз от монитора, поднял руку в приветствии.\\
--- Сейчас-сейчас, вот, закончу и сразу же буду тебя учить. Подожди чуть-чуть.\\
--- Ну--ну,~--- Шаман осмотрелся и плюхнулся на край кровати.~--- 
Интернет--зависимость --- ничуть не меньшее зло, чем любая другая наркота. Как спец говорю.\\
Мо повернулся на стуле и устало протер глаза:\\
--- Ну, раз как спец~--- то поверю. Мне просто в голову мысль одна пришла, пока я 
нам с тобой учебный план разрабатывал. Но вот не получается пока ее в жизнь 
воплотить, может, с твоей помощью\ldots\\
--- А чо за мысль?\\
--- Вот смотри,~--- Мо встал со стула и остался стоять, сложив руки за спиной, 
напротив Михи.~--- Как нам наглядно демонстрируешь ты, если мозг и физическая 
сила данного мира отказали, то воля все--таки жива и способна заставлять мир 
функционировать. Значит, то, что нельзя сделать с помощью сил самого мира, 
можно попытаться осуществить человеческими силами.\\
--- Это как?\\
Мо развел руками:\\
--- Очень верный вопрос: <<Как?>>. К сожалению, у меня пока нет на него ответа. 
Но, давай попробуем порассуждать. Возьмем твой феномен: ты собственной силой воли 
способен заставлять мир подчиняться своим желаниям, так?\\
--- Ну.\\
--- Так,~--- Мо кивнул.~--- Однако, вряд ли ты способен изменить что-то в мире в 
глобальных или не затрагивающих тебя на прямую масштабах. Или я ошибаюсь?\\
Миха задумчиво почесал подбородок:\\
--- Ну\ldots\ Хрен знает, заставить исчезнуть всю Америку у меня не получится. И 
если по телику Обаму увижу, тоже могу тужится сколько угодно~--- ничего с ним не 
случится, пробовал.\\
--- Что и требовалось доказать. Почему?\\
--- Хилый я, по ходу, для таких дел~--- Миха закурил.\\
--- Неверный ответ. Теперь уже я тебе говорю как спец~--- ты неотесанный, но 
очень сильный маг, давай это так называть для простоты. Но даже если тебя отесать~--- 
ты все равно, пользуясь только своими силами, не сможешь сделать с этим миром 
что-то глобальное. Потому что оно так не работает. Потому что воля мира~--- эта 
воля его разумных обитателей вообще, а не отдельных их представителей. И вот 
эта общая воля людей, как мне кажется, вполне должна работать. И снова вопрос\ldots\\
--- Как?\\
--- Именно. Расскажи мне, представитель разумной жизни, что соединяет всех вас 
или хотя бы большую часть, образуя из ваших сознаний единый фонд, единое поле?\\
--- Ну и?~--- Миха осмотрелся в поисках пепельницы, и в результате просто 
стряхнул пепел на пол.~--- И чо ты хочешь от интернета?\\
--- Собственно того, чем он и так является~--- связи. Но, кажется, мне нужен 
кто-то вроде оператора, кто-то, кто смог бы перенаправлять уже существующие потоки 
вашей воли в нужные мне русла.\\
--- Или просто твоя идея~--- бред.\\
--- Или так,~--- Согласил Мо.~--- Впрочем, к этому мы еще вернемся. А пока\ldots\ 
Позволь спросить: я правильно понимаю, ты можешь совершать чудеса только будучи в 
состоянии абстинентного синдрома? То есть\ldots\ Когда нет ломки и тебя всего 
не клинит на единственной цели, ты~--- магический ноль?\\
--- Ну, теперь уже нет.~--- Шаман затянулся, затушил двумя пальцами окурок и 
кинул его на ковер. Через пять секунд окурок и кучка пепла вокруг него 
испарились.~--- Теперь уже по мелочи могу и так. Хотя знаешь,~--- Он горько усмехнулся.~--- Я 
на такой куче разного говна сидел в разные периоды жизни, что у меня теперь всегда 
ломка. Просто больше или меньше приглушенная.\\
--- И чем больше она приглушена, тем меньше ты можешь?\\
--- Ну да.\\
--- Очень хорошо.~--- Чему-то обрадовался Мо и протянул Михе кубок.~--- Тогда пей.\\
--- А сейчас что?~--- Недоверчиво спросил Шаман. Мо улыбнулся:\\
--- Да то же самое. <<Истина в вине>>, только в этот раз не <<лайт>>, а 
оригинальный, разве что чуть адаптированный под наши нужды, рецепт. То, что доктор прописал, 
уверяю тебя. Пей.

Миха понюхал. Новая версия вчерашнего напитка пахла спиртом и мороженным. 
Вдохнул, опустошил бокал тремя большими глотками, выдохнул. Ничего не 
происходило. Ничего. Ничего не болело, не щипал табачной горечью язык, не 
ломило виски от постоянного похмелья, не крутило мышцы. Голова была ясная как\ldots\ 
Пожалуй, как никогда. Разве что в нос ударил букет слабых ароматов, постоянно 
присутствующих вокруг, но давно уже проходивших мимо рецепторов Шамана.\\
--- \ldots пожалуй, единственный, кто может приготовить все 148 разновидностей 
<<Истины>>. И если мне вообще есть чем гордиться в моей жизни, то, наверное, 
этим.\\ 
--- Мо, оказывается, все это время говорил что-то. \\
--- Прости, начальник. Прослушал. \\
--- О, ты снова с нами,~--- Отмахнулся от извинения Мо.~--- Нормально все, первые 
секунды эффекта они погружают тебя с головой в окружающий мир, но ты сам 
слишком занят собственными ощущениями чтобы еще на него как-то реагировать. Кстати, как 
ощущения?\\
--- Ништяк.~--- Ответил Миха. Прислушался к себе и повторил еще раз.~--- Ништяк.\\
--- Вот и славно.~--- Мо посмотрел на ковер. Несколько минут назад исчезнувший 
окурок вновь появился там.~--- Повтори фокус.

Шаман пожал плечами и прикрыл глаза. Бычок остался лежать. Секунда, две, три. 
Вены на лысом черепе шамана стали набухать кровью. Шаман открыл глаза~--- бычок 
по-прежнему лежал на ковре.\\
--- Факир был пьян и фокус не удался. Мо удовлетворенно кивнул.\\ 
--- А теперь начнем. Ложись, закрой глаза, расслабься.\\
Миха с сомнением посмотрел на кровать Регины. \\
--- Ложись-ложись.\\
--- Как скажешь.~--- Лег. Закрыл глаза. Расслабиться\ldots\ Когда, интересно, он 
последний раз был на расслабоне? Даже в полубессознательном состоянии 
прихода~--- нет. Потому что~--- нельзя. Потому что он~--- выродок, потому что он не знает, 
что будет если отпустить вожжи, если дать зверю свободу. Может, он спокойно 
останется дремать в норке сознания Шамана, а может\ldots\ Шаман давно разучился 
расслабляться. Оказалось, не до конца. Сейчас, то ли под влиянием Мо, то ли под 
влиянием напитка~--- он поплыл. Нет, без вертолетов или каких-то глюков, 
просто~--- тело опустилось в кровать, как в воду, и осталось качаться на волнах. Мышцы 
расслаблены, голова пустая, и только голос Мо ведет его извне.\\
--- А теперь почувствуй себя\ldots~--- Мо вдруг замялся.~--- Стоп! А скажи-ка 
мне, друг любезный, тебя вообще в этом мире что-нибудь держит?\\
--- Меня?~--- Шаман вынырнул на поверхность мира и неожиданно честно ответил.~--- 
Нет.\\
--- Я так и думал\ldots~--- Мо кивнул.~--- Тогда просто имей в виду, что если все 
у нас получится, то я познакомлю тебя с виновником всех твоих бед. Так что, не забудь 
вернуться\ldots\\
--- А он есть, этот виновник?~--- Миха в такое не верил. Просто он~--- выродок, 
просто в его жизни было слишком много наркоты. Разве в этом виноват кто-то, кроме него 
самого?\\
--- Есть-есть,~--- Кивнул Мо.~--- Можешь мне поверить. А теперь\ldots Расслабься и 
почувствуй себя. Почувствуй\ldots\ От кончиков пальцев до кончиков волос. Ну, 
каких-нибудь. Печень, почки, микрофлору кишечного тракта. Все это ты. Все это 
работает потому что ты заставляешь это работать. Все это подчиняется тебе. Той 
части твоего сознания, с которой сам ты не сталкивался. Почувствуй и ее. Включи 
ее. Завладей ею и пусть она завладеет тобой. 

На зеленом широком листе замысловато танцевала сороконожка. Ее спросили\ldots\ 
Кончики пальцев вечно мерзнут~--- кровь еле доходит до них. Она вообще еле 
ходит по его организму~--- слишком грязная, слишком тяжелая. Ее бы прочистить, но 
фильтры засорены, почки не справляются, а грязь все прибывает и прибывает. 
Слишком много грязи, слишком большая нагрузка насосу, а сердце качает изо всех 
сил и нельзя требовать от него больше. Где-то под кожей все еще ждут чего-то 
луковицы корней, давно уже не способные давать ростки волос. Но ждут, но\ldots\ 
Может быть могут? Может\ldots\ Может\ldots\\
--- Хорошо\ldots~--- Где-то на заднем плане прозвучал голос Мо. Кажется, чуть 
удивленный голос.~--- Очень хорошо. Оставайся таким, оставайся собой, оставайся 
в сознании. Увидь~--- почувствуй комнату вокруг себя, увидь~--- почувствуй, то, 
что находится за ее стенами, то, что находится за стенами за ее стенами, увидь~--- 
почувствуй двор, улицу, город, небо, облака, солнце, автостраду с летящими по 
ней машинами. Почувствуй свой мир как часть себя, как то, что работает по твоей 
воле, то, что подчиняется твоему сознанию. Прими его в себя и пусть он будет 
тобой\ldots

Гусеницу спросили\ldots\ Шаман обрастал комнатой, как щетиной, как памятью. 
Кровать прилегала к спине горбом, нет, удобным панцирем, спасающим, спасавшим от жары, 
от холода, от опасности. От грусти, от тоски, от похмелья, от болезни. Вдруг 
вспомнилось как заносили сюда эту кровать. В магазине, заполняя заявку на 
бесплатную доставку, уточнила у менеджера: не будет никаких проблем, и 
доставят, и соберут? Да, говорит, и доставят, и затащат, и соберут, даже застелить могут, 
если очень надо. Не, сама как-нибудь. Ну, значит, просто доставят и соберут. 
Ага. Сначала грузчики пытались отказаться заносить ее из-за отсутствия лифта. 
Ну, как отказаться\ldots\ Занесем, но в среду~--- сегодня нас только двое, по 
средам ездит бригада из пяти человек, для тяжелых заказов. Тяжелый заказ, тоже мне. 
Пришлось ругаться с ними, звонить в магазин, ругаться там\ldots\ Но ничего~--- 
занесли. Зато собирать отказались напрочь. Мол, дамочка, там все просто, 
инструкция прилагается, сами разберетесь. Пол вечера проигралась, пришлось еще в 
магазин ехать, отвертку искать, но ничего~--- собрала. Первая серьезная покупка 
в новую квартиру.

Чьи это воспоминания? Той женщины, Регины? Кровати? Его? Теперь~--- его. И всей 
комнаты, потому что откуда тогда он помнит, что именно кроватью, пока Регина 
собирала ее, придвигала ее, зацеплены были свеже--поклеенные обои? Такой хороший 
лоскуток оторвался. Потом, через несколько дней на то место встал шкаф, 
купленный с рук. Говорили, что буквально месяц назад сами купили, просто сейчас 
переезжать собираются, а в ту квартиру не подходит\ldots\ Но на самом деле шкаф 
достался им от ПалВаныча, на новоселье подарил еще за шесть лет до того как 
Регина объявление нашла, на свадьбу подарил, а у самого он на даче стоял еще 
лет десять. Но шкаф хороший, на самом деле, чудо, что так сохранился. Сам 
удивляется.

Шевеление в комнате. Что-то напряженное, точнее, нет~--- сосредоточенное. Мо. 
Мо отошел от него, сейчас садится за компьютер. Внимание устремлено к чему-то 
далекому, совсем далекому, непонятному. Это~--- чужое, туда не надо. Зато 
легким зудом прошло по коже свое~--- на кухне забурчал холодильник. Никому незаметные 
обычно вибрации разошлись по квартире~--- от кухни по стенам, до окон, до 
дверей. 

Что за дверью~--- за дверью подъезд, подъезд тоже мир Шамана, часть Шамана\ldots

\newpage

Вдалеке догорают костры окон. Они затухнут. Всегда затухают.

Вдалеке? Четыре этажа вниз, двести плиток по прямой~--- один дом, еще 
пятьдесят~--- второй. Еще тридцать~--- остановка, дорога, а за ней другие дома. По сравнению 
с звездами~--- совсем близко. Но это не важно: звезды тоже всегда затухают. 
Ежедневно, еженощно и еженощно же загораются. Вместе с фонарями.

Свет прилепленного где-то на уровне третьего этажа у среднего подъезда дома 
напротив пробивается сквозь струящуюся иву. Ива, зеленая днем, сейчас 
набирается серебра фонаря и синевы ночи. У торца дома ярким пятном среди тьмы горит 
желто-красная, еще неопавшая береза. Застанет их снег такими? Какая разница? 
Трескучие морозы зимы прогорят без остатка в костре масленицы. Так всегда 
бывает.

Каждый день, каждый год. Даже машины проносятся по опустевшей дороге с равным 
промежутком времени. А по утрам стоят в пробке на равном расстоянии друг от 
друга. Время движется, жизнь идет? Идут люди, а время\ldots\ Интересно, оно 
никогда не идет, или просто однажды один день закончился, а следующий не наступил? Если 
так~--- когда это было? Заметил кто-нибудь? Впрочем, говорят, заметил~--- один 
старик где-то в Латинской Америке. Давно.

Люди идут. Ходят. В гости, на работу, на свидания. Восходят на Олимп, 
вскарабкиваются на Эверест. Каждый своими тропками, каждый своим Путем, каждый 
к своему счастью. Ходят троллейбусы~--- последний на сегодня только что 
прогрохотал, разбрасывая обязательный сноп искр. Тоже~--- каждый по своим маршрутам, каждый 
по своему расписанию. Пока не снашиваются, не ломаются, не списываются и не 
сдаются 
на металлолом. Но даже если этот, проехавший только что~--- проехал не в 
последний 
раз за сегодня, но последний раз вообще: он все равно завтра проедет снова. И 
тот кто сядет в него не заметит разницы между вчерашним <<Шестнадцатым>> и 
завтрашним. Это все равно <<Шестнадцатый>>. И из всякого ли окна есть разница 
между деревом, в тени которого отдыхал Будда и тем, под которым сидел Ньютон?

Сигарета догорает. Если время и движется, то его движение можно ощутить только 
так~--- теплым, протекающим сквозь пальцы дымом, опадающим пеплом, исчезающими 
миллиметрами белой гильзы. Но на смену этой приходит следующая~--- ежечасно.

Затушить, отвернуться от окна, подняться в полный рост. Вверх по ступенькам, 
пройти мимо двух первых дверей, открыть третью. Через час выйти из третьей, 
пройти мимо двух, спуститься, открыть окно. Ничего не меняется~--- та же ночь, 
те же окна, те же редкие машины, тот же подъезд с теми же дверьми. Докурить. 
Подняться. Идти~--- потому что время не идет, жизнь не меняется, ходят люди. 
Ходят и открывают двери, как и троллейбусы. Вот только\ldots\\
--- А что если открыть другую?\\
--- Ничего,~--- Харон повернулся от окна и приветливо улыбнулся Берсу.~--- В моем 
подъезде все двери ведут именно туда, куда я собираюсь прийти. Вот только я 
никуда не собираюсь\ldots\ Какими ветрами?\\
--- Скорее затишьем. Как-то все как никогда спокойно,~--- Берс присел на корточки 
рядом с Хароном и посмотрел в окно. Там был вечер урбанистического мира.~--- 
Что за мир?\\
--- Один из тех, куда тебе нельзя, а мне только в форточку подсмотреть можно~--- 
Х-579. Что-то мне последнее время нравится им любоваться.\\
--- X-579\ldots~--- Берс наморщился и удивленно посмотрел на Харона.~--- Это не 
тот, где Мо пропал?\\
--- Кажется, тот~--- Харон проводил взглядом одинокого прохожего внизу.~--- Во 
всяком случае именно его, если я правильно помню, ты очень не хотел отдавать Хаосу.\\
Берс кивнул.\\
--- Странно\ldots\ Ты этим миром <<любуешься>>, я последнее время стал часто Мо 
вспоминать. Кинич тоже. Кинич, кстати, когда ходил печати ставить, весь тот мир 
перетряс~--- от Мо там ни следа не осталось. Но, может\ldots\\
--- Думаешь, явился и пытается достучаться?~--- На маленьком газетном столике в 
пролете между пятым и четвертым этажами, во всяком случае сейчас~--- пятым и 
четвертым, где у окна сидели Берс и Харон, лежала пачка сигарет, стояла 
пепельница, бутылка вина и стакан. Стакан задрожал и раздвоился. Бутылка 
поднялась в воздухе и поделилась своим содержимым теперь уже с двумя стаканами. 
Харон взял один.~--- Похоже, на самом деле, но слишком много времени 
прошло\ldots


Глотнул, покатав вкус на языке. Свободной рукой перелистнул пейзаж за окном. 
Теперь окно выходило на шумную площадь, в центре которой стояла башня с часами. 
Рядом с ней~--- электронный таблоид высвечивал: <<2012--11--17>>.\\
--- И слишком мало осталось,~--- правильно понял мысль собеседника Берс.~--- Но, 
зная Мо~--- надежда остается.

На лестничной клетке пятого этажа хлопнула дверь. В подъезд вышел маленький 
пузатый мужчина. Поежился, запахнул полы толстого махрового халата, спустился к 
окну.\\
--- Надежда остается всегда, а времени всегда бывает ровно столько сколько нужно, 
поверьте старому временщику. В случае с этим вашим Мо~--- даже чуть больше.~--- 
бесцеремонно взял со стола пачку Харона, прикурил. Посмотрел на бутылку.\\
--- Значит, господа хранители Вселенной изволили отдохнуть, принять за труды 
праведные, а о бедном незначительном Сеере забыли. Конечно, кому нужен Сеера, 
когда все хорошо, кто о нем вспомнит. Вот если беда случится, если кто-то из 
приближенных этих достославных господ в беду попадет, тогда господа вспомнят. 
Они, перед сном, ежедневно будут Сееру вспоминать~--- почему не помог, почему 
не спас. Так вспоминать, что Сеере потом годами икаться будет. А он что~--- он все 
что мог сделал, больше даже сделал, и что? Таки кто ему спасибо сказал? Может, 
кто-то ему девственницу прекрасно--ликую подарил в знак благодарности? Даже 
рюмочку не поднесли, что вы~--- самому наливать придется\ldots

Задрожала бутылка, меняя очертания. Рядом с двумя обычных размеров стаканами, 
возвышалась гигантских размеров рюмка. Ее и поднял новоявленный.\\
--- Подожди, Се\ldots~--- Берсу понадобилось некоторое время чтобы понять что-то 
кроме ворчания в прозвучавшем монологе.~--- Кому ты помог, как?\\
Перед тем как ответить, Сеера уполовинил свой напиток. Довольно причмокнул, 
облизал губы.\\
--- Как мог, так и помог, а может слабенький Сеера немного, иначе к нему 
по-другому относились бы\ldots~--- Медленными глотками он допил свою <<рюмку>>. 
Сел на ступеньки.\\
--- Хорошее вино, Харон, где берешь?\\
Берс закусил губу.\\
--- В 228,~--- Харон, видя реакцию Берса, усмехнулся. Густые широкие усы спрятали 
усмешку от Берса, но внимательный Сеера подмигнул в ответ:\\
--- А, 228\ldots\ Помню-помню, хороший мир.~--- Выжидающе замолчал. Харон кивнул: 
мол, да, хороший и вино там отменное~--- и закурил. Берс не выдержал:\\
--- Сеера, пусть меня это не оправдывает, но я сам заглянул к Харону случайно. 
Прости, что не оповестив тебя о визите, который я вознамерился совершить, я 
оскорбил тебя. Мало кто мог бы простить такую обиду, но я уповаю на мудрость, в 
которой мало кто сравнится с тобой и на снисхождение более опытного наставника 
к молодому и зеленому мне\ldots~--- При последних словах, дабы усилить эффект, 
Берс и вправду позеленел. Харон, не скрываясь, расхохотался. Хмыкнул и Сеера:\\
--- Вот как запел\ldots\ Твой Мо сразу по прибытию нарвался на эмиссара Хаоса, 
тот его, походя, развеял чтобы под ногами не путался. Окончательно или нет~--- 
сказать не могу, я твоему Мо подарил тысячную долю мига, как он ей распорядился~--- я 
не знаю. Если потратил на то, чтобы вспомнить дополнительную пару веков из 
пролетающей перед глазами жизни~--- что ж, кто я такой, чтобы осуждать чужие 
решения\ldots\\
Берс встал и прошелся по площадке.\\
--- <<Энергия устремляется вслед за вниманием>>~--- После размышления сказал 
он.~--- Мо очень часто повторял это. Если так, будем считать, что он жив\ldots\\
--- Будем,~--- Харон и Сеера синхронно подняли вновь наполненную посуду и выпили. 
Берс на секунду замер, но не стал отвлекаться на паясничающих друзей.\\
--- Если так, то, что мы имеем\ldots\ Мо находится один в закрытом, 
обездвиженном, лишенном жизненных сил и хранителя законсервированном мире, который через два 
месяца по заключенному с Хаосом договору отойдет Хаосу. Нарушить договор я\ldots\\
--- Не можешь,~--- коротко сказал Харон, а Сеера шутливо изображая сожаление 
развел руками. Берс кивнул.\\
--- Не могу. Сам мир, как уже говорилось, скорее мертв чем жив и, следовательно, 
тоже не сможет воспротивиться Хаосу. К тому же все предыдущие передачи прошли 
без малейшего сопротивления со стороны миров. Из Хаоса, как мы все прекрасно 
знаем, не возвращаются. Значит, единственный способ спастись для Мо~--- спасти 
мир, и помочь ему в этом мы не можем. До дрожи обидно\ldots\\
--- Вздрогнули!~--- Моментально среагировали двое у окна. Теперь уже Берс 
повернулся к ним.~--- А если серьезно?\\
Сеера зевнул:\\
--- Ты же его все равно уже оплакал, так какая разница~--- сейчас твой Мо умрет 
или тогда умер.\\
Харон протянул так и оставшийся полным второй стакан Берсу:\\
--- А если серьезно~--- выпей. У тебя штиль, ты сам говорил, отдохни, 
воспользуйся 
моментом. У тебя появился повод для большей надежды, что твой друг жив~--- 
радуйся, воспользуйся поводом. У тебя выдался случай посмотреть как твои друзья 
справлялись с трудностями до того как встретили тебя~--- насладись 
представлением, 
воспользуйся случаем. Ты сам взял на себя ответственность за порядок во 
вселенной и теперь ты просто не имеешь права лично его нарушать.

Взгляды Харона и Берса встретились. Берс принял стакан из рук старшего друга, 
медленно поднес ко рту, резким движением опрокинул в рот. Снова посмотрел в 
глаза Харону:\\
--- То есть, вы знали?\\
Первым ответил Сеера:\\
--- Ой, дорогой ты наш черный властелинчик, мы с дедушкой Хароном живем так 
давно, что из всех лекарств от скуки нам и осталось-то~--- вино с мира 228 и 
наблюдение за тем как вы, молодые, шишки себе набиваете. Так неужели мы не почуем, если 
где-то, пусть и в самых далеких уголках вселенной чем-то интересным запахнет, 
если еще один герой бесшабашный--безбашенный голову свою в поисках неприятностей 
поднимет?\\
--- Подняли,~--- Харон выдержал взгляд Берса и разлил на троих.\\
--- Подняли!~--- Согласился и Сеера. Берс молча выпил. Харон неодобрительно 
посмотрел на него и вздохнул.\\
--- Не дуйся, поверь мне, твои друзья не многим хуже тебя умеют справляться с 
тем, с чем справиться невозможно. Показать тебе Мо?\\
Не дожидаясь ответа, обслюнявил пальцы и вернул за окно картинку двора.\\
--- Видишь окно прямо напротив нас? Смотри\ldots\\
Все трое прильнули к окну подъезда. За окном осенний дождь поливал 
возвращающуюся домой Регину. 

\newpage 

Мо, проведший последние часы среди выдаваемых поисковыми сайтами ссылок, 
удовлетворенно откинулся на спинке стула.\\
--- Можешь гадать по облакам, можешь читать следы, можешь слушать пение птиц, 
главное~--- помни, что нет такого вопроса, который мир оставил бы без 
ответа\ldots

Еще раз пробежался глазами по тексту на мониторе. Взгляд случайно зацепился за 
часы в нижнем углу. Мо присвистнул и рывком встал со стула. Подбежал к лежащему 
Шаману. Поднес ладонь к его голове, прислушался.\\
--- Я же говорил~--- ты очень сильный и талантливый маг. Теперь я даже 
понимаю~--- почему\ldots~--- Пробормотал себе под нос и уже громче добавил:\\
--- Шаман. Шаман, возвращайся. Пора.\\
Шаман очнулся не сразу.\\
--- Офигеть,~--- Сказал он, открыв глаза. Сел на кровати, потряс головой.~--- 
Офигеть. Правда, подчинить я себе ничего не смог, так что\ldots\\
--- Все нормально,~--- Мо улыбнулся.~--- Ты смог принять это все как себя. А 
подчинить\ldots\ Ты когда-нибудь пробовал подчинить собственный аппендикс и 
заставить его быть дополнительным сердцем?\\
Шаман хмыкнул.\\
--- Ну и фантазия у тебя\ldots\\
--- Ну а чего рудименту пропадать,~--- Мо пожал плечами.~--- Так вот, чтобы 
сделать что-то такое, нужно или менять структуру всего организма, либо обладать 
огромным запасом воли и уметь концентрировать ее. Я думаю, ты мог бы сделать это через 
силу, волей~--- так, как обычно ты подминаешь мир под себя. Но это не идеальный 
путь. А уметь читать всю структуру тебе еще предстоит научиться. Но обо всем 
этом позже, сейчас у нас неотложное дело\ldots\\
--- Како\ldots~--- Последний слог вопроса застрял в пересохшем вдруг горле. Не 
просто пересохшем, а словно кто-то прошелся и продолжал медленно и медитативно тереть 
его наждаком. Сглотнуть не представлялось возможным, мышцы~--- все, от 
собственно глотательных до сфинктера, сводило и разжимало в совершенно произвольном 
порядке, глаза налились кровью и пытались выскочить из глазниц, страстно 
пытались, так, что казалось~--- или им это удастся, или они просто лопнут 
сейчас к чертовой матери\ldots\ Голова\ldots\\
--- Неотложное,~--- Мо дал Шаману прочувствовать прелесть похмелья 
<<Истиной>>.~--- И заметь, ты, теоретически, можешь сейчас искать какого-то облегчения с помощью 
магии, но у тебя нет возможности даже подумать об этом. И это самое 
начало\ldots\ Шаман не слышал его. Не слушал. Он был занят~--- катался по полу, подвывая. 
Впрочем, сам он не знал, что он сейчас делает. Его просто не было~--- была 
боль, мучительная, страшная, всепоглощающая\ldots\ Мо щелкнул пальцами.\\
--- Спасибо\ldots~--- Медленно поднимаясь с пола протянул Миха.~--- Так плохо 
мне\ldots\\
--- Никогда не было,~--- закончил за него Мо. Шаман согласно кивнул. \\
--- Что это\ldots\ такое?~--- Яростно, почти свистяще спросила появившаяся на 
пороге Регина. Миха посмотрел на нее, огляделся вокруг, перевел взгляд на улыбающегося 
Мо:\\
--- Это все я?\\
Мо внимательно изучил пятна рвоты, мочи и кала на ковре и кровати и согласился:\\
--- Ага.\\
Потом повернулся к Регине:\\
--- С возвращением! Это~--- наглядная иллюстрация того, почему упоминавшаяся 
вчера оригинальная <<Истина>> повсеместно запрещена.\\
--- Вы, вы\ldots\ Вы охренели?~--- Регина прислонилась к косяку. Комната, всегда, 
даже тогда, когда она кокетливо называла ее запущенной, бывшая ухоженной, комната, в 
убранство которой она вкладывала столько сил, была безнадежно изгажена. Ковер, 
покрывало, постельное белье, наверное, даже матрас и шторы~--- все это придется 
выкидывать, менять\ldots\ Она посмотрела на Мо:\\
--- Ты, ты\ldots\ Это все\ldots\\
--- Это все, все--таки, он, которому я бы настоятельно рекомендовал принять 
душ~--- Мо, продолжая улыбаться, показал на растерянно замершего на месте Шамана. После 
слова <<душ>> до Михи вдруг дошло, что если комната выглядит так, как он ее 
видит, то и сам он\ldots\ Он посмотрел на Регину и~--- небывалое дело~--- покраснел. 
Мо продолжал:\\
--- Однако на самом деле, королева, я не понимаю вашей реакции. Все это~--- Он 
обвел рукой комнату,~--- решает одним движением руки. Только что произведенным, к 
тому же.

Комната вернулась в первоначальное состояние. Регина в последний момент 
сдержала плевок.\\
--- Делайте вы, что хотите~--- только и сказала она.\\
Мо кивнул Шаману:\\
--- Тогда пользуйся королевским разрешением и шуруй в душ. Я, конечно, и так тебя 
отмою, но само ощущение очищения от текущей воды\ldots\ Это важно.

Проводил Шамана к ванной, попутно мысленно попросив его одежду чтобы та 
очистилась и заросла в порванных местах к моменту, когда ее хозяин примет душ, 
и прошел на кухню, где, вжавшись в стул, билась в истерике Регина. Сел рядом:\\
--- Регина, честное слово~--- ничего с твоей комнатой не случилось. Точнее, 
случилось, но совсем обратное тому, что тебя разозлило~--- сейчас она вся как 
новая. Такой уборки, как я сейчас, поверь мне~--- ты в жизни не проводила\ldots\\
--- Ты\ldots\ Ты понимаешь, что я все равно видела это? Как я теперь спать там 
буду\ldots\ После\ldots\\
--- Прекрасно ты будешь спать, ну что ты\ldots\ И вообще~--- не хочешь, не спи. 
Ты билетик купила? Когда там тираж-то\ldots\\
--- А я не знаю\ldots~--- Регина против воли улыбнулась.~--- Купила\ldots\ Хоть и 
не верится\ldots\\
--- Вера~--- это важно,~--- Поучительно сказал Мо и аккуратно погладил Регину по 
голове.~--- Все будет хорошо, королева. Если вас это утешит~--- сегодня к 
вечеру вы думать забудете о том, что случилось в вашей комнате.\\
--- Это еще почему?~--- Настороженно спросила. Такое обещание вряд ли предвещало 
что-то хорошее.~--- Чем ты меня еще удивить собрался?\\
--- Да многим\ldots~--- Мо незаметно сжег собранную в ладонь негативную энергию 
с макушки Регины. Видя, что женщина в достаточной мере успокоилась, встал.~--- А 
начну, пожалуй, с ужина. Знала бы ты, как я, оказывается, соскучился по готовке 
за это время. Сам не думал\ldots\\
Регина фыркнула:\\
--- Что-то я тебя готовящим не очень видела. У тебя все готовка~--- пальцами 
щелкнуть или там в ладони хлопнуть. Была бы борода, так, наверное, еще бы и 
волоски на себе рвал\ldotsМо задумался и отбросил идею:\\
--- Больно же\ldots\ А что касается руками махнуть\ldots\ Ну, можно и не руками, 
но для этого мне все равно придется кучу оборудования и продуктов себе\ldots\\
--- Придумывать?~--- Подсказала Регина, вспомнив ночное приключение.\\
--- Можно и так сказать, да. Ты извини, но для серьезной кулинарии твоя кухня не 
подходит совершенно\ldots\\
--- Ой--ой, можно подумать\ldots~--- Вдруг обиделась Регина.~--- Никто на 
приготовленное на моей неподходящей кухне не жаловался почему-то. \\
--- Ну, я думаю, из твоих рук любой мужик и яд бы с радостью принял, да еще бы и 
добавки попросил~--- Из душа вышел освежившийся Шаман. Завладев вниманием и 
взглядом Регины, он неожиданно смутился.\\
--- Я это\ldots\ Извини, короче.\\
Регина вздохнула. \\
--- Да ну вас обоих, если честно. Я боюсь спросить, что ты вообще на моей кровати 
делал. Хотя, мне кажется, это Мо спрашивать надо и вообще~--- если кому 
отвечать то ему.\\
Регина перевела взгляд на Мо, который, против обыкновения промолчал. Хотя 
улыбнулся. Регина еще раз вздохнула и улыбнулась в ответ.\\
--- В общем, вы меня сегодня оба обижаете, в стряпню мою не верите, так что так и 
знайте~--- будете есть яд с моих рук, если хотите заслужить прощение. Исчезнете 
оба с кухни, пожалуйста!~--- И, видя, что Мо собрался возразить, добавила.~--- 
Исчезните, исчезните, дайте мне в себя прийти.\\
--- Как ты себя чувствуешь?~--- Спросил Мо, когда они с Шаманом вновь оказались в 
комнате. Шаман прислушался к ощущениям.\\
--- Привычно, разве что ломка чуть меньше обычного\ldots\\
--- Вот и славно, значит, будем сразу пробовать как ты можешь работать при 
сниженной возбужденности.\\
--- И на что ты меня уже запрячь хочешь?\\
--- На работу славную, на дела хорошие\ldots~--- Мо прошелся по комнате.~--- 
Помнишь, я тебе утром говорил про оператора? Так вот, судя по результатам экспериментов, 
сам я им для себя быть не могу. Подозреваю, что я все ж таки инородный элемент 
для местного общего сознания, а нужно, чтобы его частичка направляла общий 
поток. Вот мы и попробуем тебя ею поставить\ldots~--- Помолчав, добавил.~--- 
Ты, конечно, тоже не совсем однородный общей массе элемент, но все-таки\ldots\\
--- Это наезд?~--- Не понял Шаман. Мо отмахнулся.\\
--- Скорее анонс вечерней программы. А пока вернемся к текущему расписанию. Итак, 
хочешь попробовать поработать вместе?\\
--- А то,~--- коротко ответил Шаман. И добавил.~--- Чо делать-то?\\
--- Садись,~--- Мо показал за компьютерный стол.~--- На самом деле, ничего 
особенного.\\
Шаман скептически покосился на Мо, но молча сел. Мо встал у него за спиной.\\
--- Все приготовления я уже сделал . Все что тебе нужно~--- запустить <<Скайп>>, 
открыть блокнотный файлик\ldots\ Видишь, тут два имени. Верхнее~--- наш логин.

В открытом шаманом файле было две строчки. В обеих~--- непонятные иероглифы. 
Миха послушно перекопировал первый в окно ввода <<Имени пользователя>>. Мо кивнул:\\
--- Пароль: <<MAKIA>>~--- большими латинскими буквами. Знаешь, тут любая мелочь 
может помочь\ldots\ Хорошо, мы в сети, да? 

Знак скайпа загорелся зеленым. Шаман вопросительно повернулся к Мо. Мо стоял, 
закусив губу и вцепившись пальцами в спинку кресла. Взгляд был прикован к 
монитору. Наконец, почувствовав на себе взгляд Шамана, выдохнул:\\
--- Итак\ldots\ Теперь тебе нужно почувствовать сеть. Так, как ты чувствовал 
комнату и 
все остальное сегодня. Прими ее в себя.~--- Легко сказать\ldots~--- Буркнул 
Шаман, но 
закрыл глаза и попытался расслабиться. Получилось, пусть не так как после 
принятия <<Истины>>, но\ldots\ В любом случае, просить добавочную порцию не 
тянуло. 
Зато внутренний взгляд притягивал большой зеленой круг с белой галкой внутри. 
Круг приближался. Галка, вдруг из знака письма обернувшись взъерошенной птицей, 
посмотрела в сторону Шамана. Каркнула~--- в ее крики Миха услышал крик птиц в 
парке рядом с интернатом. Старшие пацаны стреляли по ним из рогаток. Однажды 
Миха, после забав старших, подобрал одну недобитую. Почти выходил~--- то ли 
настучал кто-то из <<сокамерников>>, то ли просто нянечка вдруг решила убраться 
по-серьезному. Нашла. Больше ту подбитую нетипично для своего племени белую 
галку он не видел. До сегодняшнего дня. Галка подмигнула Михе, вспорхнула, 
ловко подхватив клювом стремительно уменьшающийся круг и полетела куда-то мимо 
Шамана. В руки ему упал медальон с изображением все того же значка скайпа. Шаман открыл 
глаза и встретился с изумленно-восторженным взглядом Мо.\\
--- Ты действительно могуч,~--- Сказал тот. И добавил.~--- А я действительно 
гениален, очень может быть, что сработает\ldots\ Теперь\ldots\\ 
Мо перевел дыхание:\\
--- Теперь осталась малость. Вводи второе имя из блокнотика в <<Поиск 
контактов>>, жми <<звонить>> и, пожалуйста, очень желай, чтобы нам удалось дозвониться\ldots
Звонок пошел. Комната поплыла, Шаман ощутил это, но отвлекаться не стал. Так 
плыло пространство, когда он сам шагнул из своей квартиры в темный переулок, 
где пытался оставить свой знак Мо. Так плыло пространство на складе, где он 
работал, когда по просьбе однокашника Сереги-Коршуна, подобравшего его после зоны, 
гендиректора фирмы <<Феникс>> он\ldots\ Плывет~--- значит, все работает как 
надо.\\
Дозвонимся.\\

Мо напряженно вслушивался в гудки. Стало влажно~--- то ли потел сам, то ли 
комната так среагировала на выплеск энергии. Скорее~--- и то, и другое. Неважно. Когда 
гудки смолкли, сменившись чуть шипящей тишиной, спросил тишину:\\
--- Кинич?\\
Тишина отозвалась. Минуя динамики, звук~--- приглушенный, но отчетливый~--- 
пришел откуда-то из стыка стен с потолком. \\
--- Моук? Ты? Живой? Ты где, старина?\\
Мо протер руками лицо. Все-таки, потел он. Усмехнулся, ответил как ни в чем не 
бывало:\\
--- Помниться, последний адрес, который я тебе сообщал~--- x-579. Если бы он 
изменился, то будь уверен~--- я прислал бы тебе открытку.\\
Голос издалека рассмеялся~--- раскатисто, подмурлыкивая.\\
--- Я рад, что ты жив, старина.\\
Мо, не заботясь о том, что связь налажена только голосовая, пожал плечами:\\
--- Знаешь, я, может, и не бессмертен как Берс, но и меня убить не так уж и 
просто.\\
--- Я знаю\ldots~--- Согласился голос. И с интонациями занудного профессора по 
биологии, очередной раз объясняющего, что крокодилы не летают, а майоры просто 
слишком часто пьют неразбавленный спирт, добавил:~--- А Берс не бессмертный. 
Его просто невозможно убить без фатальных последствий для мира, где он в этот 
момент находится, и всех тех, кто находится там. Физиологический феномен. Ты знаешь.\\
--- Знаю\ldots\ Как он?\\
Кинич помолчал.\\
--- Хорошо\ldots\ Но, ты бы теперь не узнал в нем подобранного тобой талантливого 
молодого странника совершившего невозможный побег из Тюрьмы. И Черного 
Властелина, во главе сборища авантюристов ступившего в безнадежный поход против 
Хранителей, ты бы, наверное, не узнал в нем тоже. Он другой. Все 
изменилось\ldots\\
--- Изменилось\ldots~--- Повторил Мо и резко спросил.~--- То есть все 
получилось, план удался? Вы создали упорядоченную вселенную?\\
--- Он создал.~--- Все так же занудно поправил Кинич.~--- С нашей помощью, 
конечно, и не только с нашей, но я склонен считать, что создал~--- он. Во всяком случае, 
он несет на своих плечах все ответственность за нее, и тут уже мы ничем не можем 
ему помочь.~--- Все так плохо?\\
И снова Кинич ответил не сразу:\\
--- Не плохо, просто роль Хранителя Вселенной очень мало похожа на существование 
в образе Хранителя какого-нибудь мира и уж совсем далеко от привычного нам 
беззаботного порхания с одного мира на другой в поисках приключений. 
Кстати\ldots~--- В задумчивом голосе вдруг возник мотив деловитой заинтересованности.~--- Ты в 
курсе, что ты находишься в мире, который в ближайшее, если я не ошибаюсь, время 
поглотит хаос и с которого невозможно уйти? Я, в общем-то, не представляю как 
тебе хотя бы достучаться до меня удалось\ldots\\
--- Воля мира.~--- Коротко пояснил Мо.\\
-- О!~--- Оценил Кинич.\\
Мо согласился:\\
--- Ага. Но вот со всем остальным у меня пока тяжело. Ты не мог бы рассказать, 
что это вообще за номера с концом света и поглощением хаосом?\\
--- Мог бы\ldots~--- По шуршанию, прибавившемуся к шипению динамиков, Мо понял, 
что старый друг присаживается на упругий хвост. Улыбнулся. Кинич, наконец, 
устроился.\\
--- Если вкратце, то после того как мы систематизировали миры, к Берсу явился 
эмиссар Хаоса\ldots\\
Мо присвистнул.\\
--- Ага,~--- В свою очередь согласился голос.~--- Но не перебивай, пожалуйста. 
Так вот. Эмиссар заявил, что упорядоченная вселенная, в отличие от разрозненных 
упорядоченных миров~--- это прямая угроза Хаосу, а, значит, и вообще всему. 
Система упорядоченных миров будет стремиться к расширению, вытесняя Хаос. С 
другой стороны, если ограничить систему и создать таким образом определенную 
сбалансированную оппозицию Хаос~--- Порядок, Хаос не будет возражать. 
Вмешиваться во что-либо вообще не в его правилах, разве что, если действия <<порождений>>, 
если что~--- это цитата, угрожают вообще всему. Потом эмиссар убедительно 
продемонстрировал Берсу всю бессмысленность возражений, потом Берс с эмиссаром 
пришли к соглашению. В результате этого соглашения вселенная и расширяется, и 
сужается одновременно, оставаясь более менее постоянной пульсирующей кляксой 
посреди Хаоса. Расширяется~--- за счет новых рождающихся миров и миров, которые 
время от времени порождает Хаос. Сужается~--- благодаря передаче Хаосу 
определенных старых миров. В первую очередь~--- тех, чьих Хранителей мы 
уничтожили во время нашего похода. Порядок передачи строго запротоколирован и отход от 
протокола, что тоже запротоколировано, считается расторжением соглашения, что 
ведет к абсолютному уничтожению всей системы Порядка Хаосом.\\
--- Альтернативы?\\
--- Отсутствуют,~--- Моментально ответил собеседник. Потом пояснил.~--- Нам 
действительно убедительно продемонстрировали все безнадежность прямого 
столкновения с Хаосом и очень убедительно доказали бессмысленность нашей 
теоретической победы. И тот, и другой путь ведут к полному исчезновению 
всего\ldots\ Причем,~--- Усмехнулся чуть погодя голос,~--- включая и сам Хаос.

Мо молчал, просчитывая ситуацию. Слова Кинича под сомнения ставить 
бессмысленно --- другой информацией он все равно не располагает. Да и не привыкли они 
сомневаться в правдивости слов друг друга. Получается\ldots\ Мо сильнее стиснул 
стул, практически дырявя его спинку своими пальцами. Получается\ldots\\
--- Получается,~--- Медленно начал он,~--- Что я не просто сижу в обвешанной 
замками клетке с собранным зверю жертвоприношением, но и к тому же не имею права 
попытаться взломать эту клетку, если не хочу чтобы пострадали все остальные, 
включая опять же меня, включая опять же и тех, кто эту жертву решил принести? 
Так?\\
Теперь задумался Кинич.\\
--- После консервации мира и исключения его из системы 
порядка и до принятия мира Хаосом~--- мир существует автономно. Правда, по 
идее, пребывая при этом в анабиозе или близком к нему состоянии. Но~--- автономно. 
Включая всех и все, что, прости за каламбур, он в себя включает на момент 
консервации. То есть, к действиям совершенным миром или его населением система 
Порядка отношения не имеет\ldots~--- Кинич громко выдохнул и тепло 
добавил.~--- Так что, если ты знаешь, как убраться оттуда~--- убирайся. Или попытайся поймать 
момент, когда придет Хаос, но мир еще не отойдет ему. В это время мир будет 
открыт, как я понимаю, и ты сможешь выпрыгнуть из него. Я очень соскучился по 
твоему трактиру.\\
Мо, скрывая облегчение, усмехнулся:\\
--- Обжорой был, обжорой остался. Но, прости, трактир подождет. Сначала придется 
спасать мир силами самого мира.\\
--- Воля?~--- После непродолжительного молчания уточнил Кинич. Мо облизнулся, 
предчувствуя реакцию собеседника на его следующие слова:\\
--- Не совсем. У меня есть веские основания полагать, что в этом мире есть 
Сказочник.\\
После на этот раз долгой паузы из глубины комнаты раздалось просительное:\\
--- Расскажешь?\\
--- Конечно\ldots~--- Напряжение спало. Мо разжал пальцы. Встряхнул руками, 
восстанавливая нормальное кровообращение. Задел замершего перед монитором 
Шамана~--- тот покачнулся. Мо внимательно посмотрел на него.~--- Но позже. Прости, мне, 
кажется, пора разъединяться.\\
Кинич вздохнул и как-то очень по-кошачьи, вызывая в заимствованной части 
сознания Мо, очень определенные ассоциации протянул:\\
--- Вот так всегда, на самом интересном месте\ldots~--- Потом деловито 
добавил.~--- Так, Моук. Я до тебя не дотянусь, я сейчас весь сеанс пробовал двустороннюю связь 
наладить. Но если ты нашел способ~--- постарайся держать меня в курсе. Сам 
знаешь~--- совет лишним не бывает.\\
--- Не бывает,~--- Согласился Мо.~--- Постараюсь, Кинич. В любом случае, рад тебя 
слышать. Прости, мне действительно пора заканчивать.\\
--- Увидимся,~--- то ли спросил, то ли констатировал Кинич.\\
--- Обязательно,~--- Твердо отозвался Мо и потянулся к мышке. Нажал на рисунок 
красной трубки и, посмотрев на так и не поменявшего позы Шамана, выключил Скайп 
вообще. Шаман грузно повалился на ковер.\\
\\

\noindent --- Что ж, вынужден признать~--- ты, королева, все-таки умеешь готовить. Конечно, 
до меня тебе далеко, но потенциал несомненен.~--- Мо отодвинул от себя тарелку. За 
время, которые они с Шаманом потратили на звонок, Регина захваченная азартом 
сотворила маленькое чудо. А то, понимаешь, все вокруг волшебники, что, она 
хуже? Не хуже, пусть сейчас Мо иронизирует, но плов уплетал за обе щеки. А Шаман уже 
третью порцию ест и даже без перекуров. Хотя, выйдя из комнаты, смолил одну за 
другой, и вытащенную Мо из воздуха бутылку чего-то крепкого уполовинил одним 
движением. А сейчас ест, не отвлекается. Точнее отвлекается, но только на 
салаты. А еще их ждет чай заваренный по особенному рецепту и к чаю~--- свежие 
булочки, их, кстати, через три минуты из духовки вытащить надо. 

Сейчас самой удивительно: как все успела. Действительно~--- чудо. 

\noindent --- То--то,~--- Улыбнулась в ответ.~--- Знайте наших, товарищи иномиряне. Миш, 
тебе еще положить?\\
Шаман тяжело выдохнул.\\
--- Баста, и так пузо набил как никогда. Благодарствую, хозяйка, знатно угостила!\\
--- Это ты еще моего угощения не попробовал,~--- Мо поднялся из-за стола.~--- 
Регина, не переживай, моя заготовка твоим булочкам не помеха. Это, скорее, из области 
духовной пищи. После такого, оставившего слушателей в легком замешательстве, 
анонса, Мо прошел в комнату. Вернулся с несколькими листами распечатанного 
текста.\\
--- Я тут немножко по интернету полазил и нашел кое-что, что, как мне кажется, 
может быть интересным для вас.~--- Он разделил листы и вручил часть Регине и 
часть\\ 
--- Шаману.~--- А я пока посуду помою\ldots\\
На некоторое время кухня погрузилась в тишину, разбавленную прерывающимся 
журчанием воды. \\
--- Что это?~--- вопрос Регины прервал тишину, и, заставив Мо отвлечься от 
раковины, вернул монотонность потоку из крана. Мо встал сзади Регины:\\
--- Это? Насколько мне известно, такой значок используют для обозначения номера. 
<<Странный мир номер шесть>>,~--- зачитал он вслух.~--- <<- И в подарок от 
магазина --- этот пробник, реклама новой линии\ldots\ 

Ее здесь узнавали. Молодая, в меру красивая, достаточно, судя по всему, 
состоятельная женщина. Постоянная клиентка~--- появляется регулярно, но не чаще 
чем раз в неделю. Обычно~--- по понедельникам, но\ldots>>\\
--- Я умею читать,~--- Регина обернулась.~--- Что это?

Мо не рискнул посмотреть ей в глаза. Отошел к окну. Внимательно изучил ночной 
двор, чувствуя на себе пристальный взгляд. То ли со спины~--- взгляд Регины, то 
ли чей-то извне\ldots\\
--- Что ты там про сказочника и виноватых балакал?~--- Прокуренный голос Шамана 
прозвучал еще более хрипло чем обычно. Мо развернулся.\\
--- Умничка!~--- Воскликнул с наигранной веселостью, по-прежнему не решаясь 
встретиться взглядом с сидящими за столом людьми.~--- Сразу ухватил суть.\\
--- Не подмазывай\ldots~--- Миха шумно закурил.~--- Объясни просто, что это 
значит и откуда\ldots~--- Оборвал вопрос затяжкой.\\
Регина молчала. Впрочем, Мо хватало и ее прожигающего ему переносицу взгляда. 
Он поморщился.\\
--- В глобальном масштабе это означает, что теория Сказочника отныне доказана. 
Для этого мира~--- это значит хоть какую-то возможность спасения. Для вас\ldots~--- 
Он вздохнул, подбирая слова.\\
--- Какой-то хрен меня выдумал?~--- Пришел ему на помощь прямой вопрос Шамана. Мо 
наконец поднял глаза.\\
--- Да. С другой стороны, это значит, что и любой из нас может быть\ldots\\
--- Ну и будьте.~--- Миха поднялся.~--- Спасибо, Регина, было очень вкусно, но я, 
пожалуй, пойду\ldots\\
Не давая Мо возможности остановить его бегство, Шаман стремительно прошел в 
коридор. Уже открывая дверь, услышал голос Регины:\\
--- И меня?..\\
~\\

\noindent Вечер встретил порывом холодного позднеосеннего ветра. \\
--- Суки,~--- сплюнул Шаман, прикуривая на крыльце. Нужно было курить. Много. И 
еще --- нужно много выпить, нужно уколоться, иначе\ldots\ Иначе он за себя не 
отвечает. Где тут ближайшая <<точка>>? Или быстрее будет~--- домой?\\
--- Миша?~--- окликнули его голосом Регины и тронули за плечо. Обернулся. \\
--- Можно я с тобой прогуляюсь?~--- голос женщины дрожал. Вздохнул. Сглотнул и 
жадно затянулся. Сигарет хватает, пожалуй, он сможет себя контролировать. \\
--- Попробуй.\\
Двор пересекли молча. Вышли на дорогу, остановились у светофора.\\
--- Что у тебя?..~--- Регина не договорила, но Миха понял.\\
--- Моя жизнь. С чужими приписками. <<Невозможно избавиться от зависимости. 
Невозможно. Просто некоторые умеют жить в вечной ломке. Михаил Горелов знал это 
лучше кого-либо другого. Он уже родился с зависимостью\ldots>>~--- Михаил 
Горелов зыркнул на красный фонарь светофора. Светофор моргнул и воззрился на пешеходов 
тремя глазами. Зелеными. Шаман пошел. Регина пошла за ним.~--- Родился, 
блин\ldots\ А он представляет, этот\ldots\ Сказочник~--- что это, блин, такое? Придумал он 
меня\ldots\ А прожить всю жизнь полуинвалидом--полупсихом он пробовал? Не, нахрен нужно~--- 
Миха проживет. Блин\ldots~--- Миха замолчал и сунул в рот новую сигарету. Сигарета 
послушно задымилась.\\
--- А у меня~--- безысходность\ldots~--- Регина задумчиво поддела носком сапога 
камешек на тротуаре.~--- Безысходность и какая-то тоска о чуде какой-то совершенно 
серой и погрязшей в серости тетки. Моя тоска.\\
Камешек неохотно ускакал на проезжую часть. Чьи-то фары высветили: серая галька 
катится, замирая на сером асфальте. Миха пожал плечами.\\
--- Какая же ты тетка\ldots\\
--- Никакая,~--- Послушно согласилась Регина.~--- Серая.

Пошел дождь. Или он шел изначально, просто только сейчас Миха начал замечать 
его, в отличие от назойливо привлекающего к себе внимание колючего ветра. 
Поежился.

\noindent --- Я семь раз в дурке лежал, это не считая детского интерната. Зона, опять 
же\ldots\ Меня на зоне~--- этот у себя в рассказе тоже мельком упомянул~--- чел один 
учить взялся. Филин\ldots\ Не законник, но авторитетный мужик. И башковитый. Садист, 
правда, но кого волнует? Научил. Во всяком случае, я хоть понимать, что со мной 
начал и как-то справляться с этим научился. А еще~--- винт крутой варить, а не 
то говно, что мы с мамкой готовили. Полезная наука, спасибо\ldots\ Вот только не 
знаю теперь кому: Филину, или этому\ldots\\
--- А я мечтала быть феей\ldots~--- Регина усмехнулась.~--- Интересно, это для 
того чтобы показать, что в детстве я была ярче, ближе к чему-то волшебному, но\ldots\ Быт 
засосал, карьера, квартира~--- и все, нет мечты, нет волшебства. Все что 
осталось --- неосознанная тяга к чуду и то, ее еще разбудить надо\ldots

Пара прохожих, мужчина и женщина, оба задумчивые, оба придавленные неожиданной 
рухнувшей на них правдой, шли мимо кафе. У столика возле самого окна сидела 
пара --- мужчина и женщина, оба задумчивые, оба придавленные неожиданно вспыхнувшим 
между ними чувством. Мужчина неловко крутил на блюдце чашку. Женщина на улице 
усмехнулась, взглянув на мужчину за стеклом.\\
--- А еще я знаю свыше тридцати способов заваривания чая. Интересно, а это 
зачем?\ldots\\
--- А я травы заваривать умею, разные. И грибы. Хорошо, так, чтоб с любым 
эффектом\ldots\ Наш создатель, наверное, просто чаи уважает\ldots

Мужчины и женщина посмотрели друг на друга и вдруг рассмеялись.

Женщина за столом что-то сказала мужчине, мужчина ответил~--- и они 
рассмеялись, смехом рассеивая неловкость решающего свидания.

Рассмеялись, смехом развеивая гнетущий запах чужой воли. 

На мгновение пары увидели друг друга, но тут же отвели глаза и вернулись 
каждый~--- к своему. У каждого~--- своя история.\\
--- И чем твоя история оканчивается?~--- Спросила, отсмеявшись Регина.\\
Миха пожал плечами:\\
--- Зоной. Я, вообще-то, потом откинулся, устроился в фирму бывшего одноклассника 
кладовщиком работать, помог ему с бандюками разобраться, но надорвался и 
очередной раз в дурку попал с <<белочкой>>\ldots\ Но все это создателю, видимо, 
не так интересно, он меня к тому моменту уже бросил. А твоя?\\
--- Ничем\ldots~--- Отозвалась Регина и, подумав, добавила.~--- Сексом, 
некоторым образом, втроем. Наш создатель не только гурман, но еще и извращенец\ldots\\
Миха внимательно посмотрел на Регину:\\
--- Ну почему же\ldots~--- Закашлялся. Регина фыркнула, тряхнула головой и пошла 
вперед. Отщелкнув в сторону окурок, Шаман двинулся за ней. Оставшаяся позади 
них пара за столиком попросила счет.\\
~\\

\noindent --- Итак, сказочник. Теория сказочника возникла очень давно~--- наверное, еще 
тогда, когда странники вообще стали задумываться о принципах существования миров. 
Общая, подтверждающаяся в большинстве случаев схема выглядит просто: есть мир, 
есть Хранитель. Хранитель~--- то ли вызванная самим миром персонификация его 
законов, то ли действительный создатель и господь мира, то ли вообще занявший 
это место странник. Существует разумная жизнь~--- не всегда, что само по себе 
интересно, но достаточно часто. Разумная жизнь возникает в результате эволюции 
мира, согласно его, мира, законам, и в дальнейшем, в ходе собственной эволюции, 
начинает влиять на устройство мира. Чаще всего грубо насилуя его законы, иногда 
– находя возможность обходить их. Или, наоборот, развивать их. Кроме того, 
разумная жизнь способна общей волевой интенцией влиять напрямую на сферы мира, 
изменяя его порядок вплоть до изъятия из системы действующего Хранителя. Это 
утверждение тоже долгое время считалось спорным, пока не было убедительно 
доказано на практике Торквемадой. Его известный парадокс\ldots\ Впрочем, это 
тема отдельного разговора. Важно то, что доказано~--- разумная жизнь как явление 
действительно может оказывать влияние на породивший ее мир. Логично было 
предположить и дальнейшее: если такая способность есть у разумной жизни вообще, 
то среди ее представителей должны появляться концентраты этой способности, 
существа, обладающие возможностями Хранителей. Резервный вариант развития мира 
на случай, если с Хранителем что-то случится. Или, на случай, если тот порядок, 
который воплощает первоначальный Хранитель окажется недееспособным. Сказочники.

Теория эта, при всей ее логичности и даже находившихся косвенных подтверждения 
числится среди недоказанных. Числилась. До тех пор пока я не очнулся от забытья 
в квартире Регины. Меня, как вы знаете, должны были убить, развеять в ничто~--- 
и физическое тело, и энергетический слепок личности, все. Собственно, так и 
сделали. Единственное, что успел сделать я, защищаясь от удара~--- позаботиться 
о том, чтобы мой энергетический прах, назовем это так, остался в сферах данного 
мира, не сливаясь при этом ни с чем, кроме собственных частичек. Теоретически, 
это давало шанс, что когда-нибудь частицы меня соберутся воедино и я вернусь. 
Теоретически, с минимальными стремящимися к нулю шансами. Однако~--- я перед 
вами. 

В чужом, но подчиняющемся мне теле, в полном, кажется, сознании.

При этом, каюсь, королева, первой моей целью было тело Регины. Ну, зачем мне 
искать другую жертву, пытаться поработить чье-то еще сознание, если вот, прямо 
передо мной есть тело, чей хозяин впускает меня в себя, как принимает в себя 
море приношение Каналое? Но, в последний момент зацепился за Александра. 
Зачем?~--- сам не понял, и в первые часы полноценного существования был слишком занят 
радостью бытия, чтобы еще размышлять о нелогичности собственных поступков. А 
когда отрадовался~--- пришли другие вопросы. Например, почему я, не самый 
слабый среди странников, равный среди пришедших в Кладру, не могу достучаться до сфер 
этого мира, почему самые элементарные действия требуют от меня таких затрат 
сил, каких не потребовало бы и создание хрустального небосвода в любом из миров? И 
почему, при этом, я практически всемогущ на территории квартиры Регины? В 
последнем я окончательно убедился когда появился Миха~--- человек, способный 
изменять закостеневшие сферы этого мира. Сам. Один.

Миха, поначалу, я примеривал роль Сказочника на тебя. Смущало одно: когда я 
смотрел тебя~--- я видел не образы, не вспышки чувств и переживаний, как 
обычно. Я видел текст. Но, кто знает как оно со сказочниками бывает~--- может, так и 
надо. Потом, присмотревшись, решил: нет, не можешь ты им быть. Сильный~--- да, но 
сила~--- дикая и по большей части разрушительная. Ты ломаешь мир, ломая при этом себя, и 
хотя тебя можно научить обходиться без этого, сам по себе ты не созидатель. 
Опять же~--- где связь между тобой и Региной и ее миром-квартирой? А если нет 
такой связи~--- то не много ли неизвестных для одного уравнения? Не ты, но при 
этом и ты, и Регина с ним связаны. Иначе~--- бред.

Дальше просто. Я беру отрывок из твоего, Шаман, текстового содержания и помещаю 
его в поиск. Нахожу в каком-то блоге текст, абсолютно соответствующий тому, что 
я прочитал в тебе. В том же блоге нахожу текст про Регину\ldots\ Все, задача 
решена, теория доказана, все свободны.

Мо победно обвел глазами слушателей. Слушатели~--- Регина и Миха~--- уплетали 
булочки. Булочки, между прочим, тоже Мо спас~--- шеф-повар ушла вместе с 
Шаманом и вернулась вместе с ним же только под утро. Вернулись оба голодные, но 
успокоившиеся. Мо вообще не понимал их переживаний~--- ну, придумали их, 
подумаешь. Как они с братом на свет появились~--- это вообще неизвестно. Не 
общепринятым способом~--- точно. И ничего~--- жил Мо и на этом свете и на 
других, о том, что его папочка, образно говоря, щелчком пальцев на свет произвел не 
переживал. Так что, да, он заранее понимал, что реакция на эту новость будет 
бурной, но почему она будет такой\ldots\ И ладно, расстроились так 
расстроились, бывает. Но поблагодарить~--- за рассказ, за те же булочки, в конце концов~--- 
можно?\\
--- Вкусно\ldots~--- Протянул Миха.~--- Регина, ты золото, это очень круто, что 
тебя придумали\ldots

Оба захохотали, при этом Регина подавилась булочкой и, в результате, 
закашлялась. Мо вздохнул и постучал женщину по спине. Вот-вот, на него внимания 
вообще не обращают, а сами все утро так~--- один что-то скажет, а потом оба 
хохочут, как будто воздухом Ширилиты надышались.

Хлопки подействовали. Отдышавшись, Регина повернулась к Мо.\\
--- Это все, Моук Мауиевич, очень интересно и познавательно, но не могли бы вы 
перейти ближе к теме? Мы с, так сказать, братом,~--- Тут Миха хмыкнул,~--- 
хотели бы побольше о нашем, так сказать, родителе узнать. Будьте так любезны\ldots\\
Мо пожал плечами:\\
--- Пренебрежение теорий, поверьте мне, до добра не доводит. Но, как хотите. 
Имя~--- Борис, возраст~--- 26, и не смотрите на меня так, место проживания~--- Вильнюс, 
Литва.\\
--- И все?~--- Спросил Шаман после почти минутной паузы.\\
Мо кивнул.\\
--- Более менее. Могу добавить разве что, что Сказочник наш~--- кстати, как ни 
странно, в своем блоге он использует именно такой ник~--- по-видимому, является 
сильно рефлексирующим молодым человеком, недовольным миром, который он видит 
вокруг и своим местом в нем. Впрочем, насколько я понял~--- это общее для 
современных рефлексирующих молодых людей явление. За более подробной 
информацией нам необходимо обратиться к самому объекту. И теперь уже вопрос к вам, дорогие 
мои~--- как нам попасть в <<Вильнюс, Литва>>? Уточняю: попасть туда нам нужно 
как можно скорее, у нас ноябрь заканчивается, и мне бы не хотелось выяснять 
насколько точно Кинич указал дату\ldots\\
--- А мы хотим встретиться с ним?~--- С деланным удивлением спросил Шаман. Мо в 
ответ удивился вполне искренне:\\
--- А что, не хотите?\\
Переглянулись. Шаман закурил, Регина вздохнула.\\
--- Хотим\ldots\ Самое быстрое~--- это, конечно, самолетом, но загранпаспорта, 
оформление виз\ldots\ Это все даже в срочном порядке пару недель займет, не 
говоря уже о специфических обстоятельствах Мо\ldots\\
Мо, подумав, кивнул:\\
--- Ну да, весь паспортный стол сюда не приведешь, а смутного сходства меня с 
Александром на паспорте им может и не хватить. А назад кости вправлять~--- ой, 
как не хочется\ldots\ Больно. В любом случае, пара недель~--- это долго. Миха, твои 
предложения?\\
Шаман с размаху вдолбил дымящийся конец сигареты в тарелку.\\
--- Вмазать.\\
--- В смысле?\\
--- Вмазать, ширнуться, уколоться,~--- Он поднялся из-за стола.~--- Короче, 
будьте готовы. Послезавтра я прихожу к вам и мы двигаемся.\\
--- Добро,~--- Согласился Мо, а почувствовавшая неладное Регина не нашла ничего 
лучше как переспросить:\\
--- Послезавтра?\\
Из коридора послышался тяжелый вздох. \\
--- Послезавтра--послезавтра,~--- раздался раздраженный голос Шамана.~--- До 
завтрашнего вечера я буду накачиваться, ночью придут ломки, и к утру я как раз 
буду готов\ldots\\
\\
Хлопнула дверь, оставляя без ответа вопрос Регины: <<Готов~--- к чему?>>. \\
~\\

\noindent --- Готов~--- к чему?~--- Выпив, переспросил Шаман у Серенького. Тот, не отвечая, 
налил еще по одной, не дожидаясь удивленного Шамана, выпил и выдохнул. 
Выжидающе уставился на Миху. Миха задумчиво принял стакан~--- Серый обычно пил очень мало.\\
--- К жизни после склада,~--- Внимательно проследив за тем как одноклассник 
принимает на грудь, соизволил ответить Сергей Ковров, основатель и гендиректор 
логистической компании <<Синяя птица>>~--- <<Транспортировка и складирование 
ваших товаров~--- это наши проблемы>>.~--- Тебе жить-то есть где?

Теперь, не дожидаясь своего начальства, выпил кладовщик <<Синей птицы>>, Миха. 
От добра добра не ищут, и Серенькому, конечно, поклон уже за то, что он сделал, 
но\ldots\ Обидно. \\
--- Гонишь, да? Ну что ж, барин, спасибо за хлеб, за соль\ldots\\
--- Иди на\ldots~--- Коротко огрызнулся Серый.~--- На кой мне тебя <<гнать>>, где 
я по-твоему еще такого кладовщика найду? Вот только по ходу меня самого 
погонят\ldots\\
--- Рассказывай.~--- Бутылка кончилась очень быстро, но и Серый сегодня пришел на 
склад не пустым, и у Шамана в закромах всегда было чем догнаться. Иначе просто 
жить невозможно.

В общем, птица удачи от своей сестры отвернулась. Просто вдруг. Дела шли 
хорошо, круг клиентов разрастался, платили вовремя, услугами оставались довольны. Крыша 
брала по-божески, еще и пару раз помогала проблемы с разтаможкой решать\ldots\ 
Все было хорошо. Вот только однажды <<крыша>> взяла и не поделила что-то с другой 
конкурирующей организацией. Не поделила, правоту свою доказать не смогла и 
осталась, как водится, должна. Ну и долг Серегой отдала. В смысле не им, а 
<<Птичкой>> его <<Синей>>. А новым крышующим этот бизнес глубоко до лампочки, 
им еще 
один объект опеки не интересен, они из него все что можно выжать хотят~--- и 
все, 
пожар, банкрот, ликвидация фирмы, шестьдесят процентов от страховки. Это они 
прямым текстом сказали. И рыпаться, сам понимаешь\ldots

Шаман слушал Серенького, почесывая под правой лопаткой и вспоминал науку 
Филина. 
Филин учил его самообладанию и самоконтролю жестко, как и подобает 
авторитетному 
мужику. После его уроков вся спина, да и не только спина, должна была бы 
остаться в шрамах. Не осталась. <<Ты, паскуда, человека взглядом замочить 
можешь, 
а себя самого заговорить тебе, значит, впадлу? Не умеешь? Ну, раз не умеешь, 
так 
после меня уродом останешься, дело хозяйское>>\ldots\ Дело хозяйское. Спина 
зудела, 
опьяневший Серега всхлипывал, а Шаман вспоминал ухмыляющегося зэка. <<Ты, 
щенок, 
мне потом за каждый рубец спасибо скажешь, да спасибом не отделаешься. Запомни, 
падла, ничто не делается просто так~--- запомни, и сам помощи безвозмездной не 
принимай, и другим нахаляву не оказывай. Закон~--- не блатной, но жизненный>>. 
Шаман вздохнул, посмотрел на упавшего лицом на стол одноклассника и почесал 
исколотое запястье. Налил себе. С Серегой говорить будет завтра, но начать 
подготовку можно уже сейчас\ldots

\noindent --- Итак, чем ты расплатишься со мной, если я тебе помогу?~--- Похмельный Серега 
с трудом понимал: это у них сейчас серьезный разговор, или очередной заскок 
пившего всю ночь Шамана. Он-то отключился еще после первой бутылки, 
кажется\ldots\ А Шаман, Шаман и не ложился по ходу. Выглядит он~--- на грани коней, но голос 
трезвый. Начал аккуратно:\\
--- Миха, я реально ценю, но что ты можешь\ldots\\
--- Могу,~--- Шаман отмахнулся.~--- Мне интереснее, что ты можешь, пацанчик\ldots\\
--- А что тебе, дядя, надо?~--- Грустно улыбнулся Серый. Шаман оторопело 
уставился на него. Потом рассмеялся: \\
--- А хрен я знаю\ldots\ Но что-то надо~--- закон\ldots\ Во! Придумал! Ты меня 
давеча со 
склада гнал\ldots\ Так вот! Если я тебе помогу ща, то ты разрешишь мне жить в 
любом 
из твоих складов, где бы они не находились, с кем бы я не находился\ldots\ Во!\\
Довольный собой Миха гордо посмотрел на Сергея. Сергей, уверившись, что даже 
здоровья и опыта Михи на сегодняшнюю ночь не хватило, пожал плечами:\\
--- Пусть будет по-твоему\ldots\ А сейчас~--- я поеду, а ты спать иди. 
Послезавтра 
понедельник уже\ldots\\
--- Понедельник\ldots~--- Миха задумался.~--- Так, вызови кого-нибудь, я до 
четверга не работаю.\\
И пожал плечами:\\
--- Запой у меня. И вообще\ldots

Запоем, конечно, не ограничилось. Пропив субботу, в воскресенье утром Миха 
похмелился двумя стаканами и вмазал первый винт. До ночи принял еще две дозы, в 
понедельник увеличил общее количество до пяти. <<Главное~--- без 
передозы>>,~--- выныривая между приходами в жуткие ломки, одергивал себя.~--- <<Главное~--- без 
передозы>>\ldots\ С передозировкой Шаман был знаком и ничего хорошего об этом 
знакомстве в его памяти не осталось.

Сколько и чего он принял во вторник~--- не помнил. Но вечером почувствовал: 
<<Хватит>>. Почувствовал, заварил себе крепкий чифирь, выпил литр в два захода 
и завалился спать. Утром было хреново. Очень. Наверное так, как почти никогда не 
было. Так, как не было очень и очень давно. Долго гнал от себя мысль о дозе, 
стакане или хотя бы покурить. Прогнал. Попытался встать с кровати, но тело не 
слушалось. Сходил, с озлобленной радостью, под себя. Вздохнул и закрыл глаза: 
<<Ну, где ты, крыша босса моего?>>\ldots

\newpage

Миха появился рано утром. Руки дрожали, взгляд мутный, губы~--- ссохшиеся. \\
--- Готовы?~--- голос хриплый.

Регина, кинувшаяся было с расспросами, споткнулась о его голос и только 
кивнула. Мо повторил его жест.\\
--- Хорошо.~--- Миха снял куртку, майку под ней и повернулся к ним спиной. Со 
спины на Регину и Мо смотрела татуировка раскинувшей крылья птицы, на туловище 
которой умещалась покрытая точками карта Европы.\\
--- Найдите там Вильнюс, прикоснитесь к нему пальцами\ldots~--- Теперь дрожать 
начал и голос Шамана. Он чертыхнулся и пояснил.~--- Никогда этой фичи с другими людьми 
не пробовал, так что извиняйте, если что\ldots\\
Мир моргнул и превратился в радугу, причем радуга семью радостными поездами 
прокатилась по Регине. Прекратив кричать, она открыла глаза.\\
--- Где мы?

Все трое они стояли в маленькой пыльной комнатке. У стены стоял кривой стол и, 
рядом с ним, потрепанное кресло. Дальше угадывалась дверь. Света не было, 
только из маленького оконца под высоким потолком пролезали лучи солнца, но и они 
моментально разбивались о столбы пыли.

Шаман натянул майку и закурил.\\
--- В моей персональной коморке местного склада <<Синей птицы>>. Добро пожаловать 
в Вильнюс\ldots

\customsection{Путь Мауи}{melkij\_bes}{MANAVA~--- сейчас и есть Момент Силы}

Это странный мир. Мир, где до сих пор оплакивают самоубийство наркомана Кобейна 
и где почти сразу забыли о смерти его великого тезки Воннегута. Мир, где прошла 
незамеченной смерть Сарамаго. Мир, где даже Аксенов уже умер.


Я не знаю кого из живых еще можно читать. Поэтому слушаю. Шевчук, Сурганова, 
Арбенин~--- они, слава богу, еще живы. И даже, скорее всего, их смерть не 
пройдет совсем незамеченной~--- все же шоу-бизнес\ldots\ Оплачут, пусть в пределах 
только русской тусовки.

Сидящий напротив меня парень тоже что-то слушает. Слушает и подпевает под нос. 
Судя по движению губ~--- что-то из хип-хопа. Освобождаю одно ухо, разбавляя <<Я 
теряю тебя>> гулом троллейбуса. Так и есть~--- литовский хип-хоп, наверняка 
какой-нибудь очередной гениальный <<поэт улиц>>. Но парню нравится. Его глаза 
восторженно прикрыты, острые черты лица в ожидающем экстаза напряжении, губы 
сосредоточенно повторяют за <<гением>>\ldots\ Почувствовал мое внимание. Открыл 
глаза. \\
\\
Посмотрел на меня, потом в окно, потом опять на меня. Снял наушники:\\
--- Эй, а что бы ты сделал, если бы знал, что завтра умрешь?\\
Я пожимаю плечами.\\
--- Напился бы, наверное.\\
Парень кивнул.\\
--- Ясно\ldots\ А я бы~--- умер сегодня. Или послезавтра. Назло!


Троллейбус притормаживает у очередной остановки. Парень встает, отряхивает 
пальто, какое было в моде у определенной молодежи лет пять назад~--- узкое, с 
массивными пуговицами и острым капюшоном. На прощание кидает мне:\\
--- Бывай!


Никогда не любил всех этих случайных собеседников общественного транспорта. 
Нищего вида женщина, рассказывавшая мне от конечной <<двойки>> на станции и до 
самого Антакальниса о том, какой невкусный беляш на станции она купила, отдав 
последние~--- и так и не смогла его доесть; пьяный мужик в ночном 163-ем с 
повестью о том, как мать вылила всю заначку в раковину, но он не растерялся, 
пошел на точку, а возвращаться пешком так лень, еще один пьяный с 
альтернативной историей СССР, трезвый, но с еще одним вариантом этой истории, этот вот\ldots\ 
Забавно: узнав пальто, я, возможно, узнал и его владельца. Это может быть~--- 
Вуду, Видас, дай бог памяти, Овидиюс Дийунас. Я его придумал. Когда-то очень 
давно, когда еще верил, что могу писать. А потом Наташа~--- одна из немногих 
читателей той моей <<нетленки>>~--- увидела его в микрушке <<Висагинас~--- 
Вильнюс>>. 
Позвонила, я тогда уже не жил с ними, в тот же день: <<Борь, а я твоего Вуду 
видела>>\ldots\ Нет, я ей, конечно, не то чтобы поверил, но\ldots\ Было 
приятно. 
Знаменательно. Да, именно как к знаку я к этой новости отнесся: жизнь говорит, 
что могу, что стоит, что буду\ldots\ Ага.


Моя остановка. Мокро и холодно, а до собеседования еще полтора часа. И чего я 
постоянно заранее приезжаю? В офис не пустят, больших супермаркетов, по которым 
можно побродить часик~--- тут нет, разве что <<Senukai>>, но на них я по самое 
<<не хочу>> насмотрелся, спасибо; а этот часик мокнуть~--- тоже не хочется. Придется 
деньги тратить. Ну, в конце концов, неужели я дошел до 26 лет так, что не могу 
позволить себе даже чашечку чая в недорогой кафешке? Могу. Или не стоит? Могу. 
Стоит. Буду.


Буду--Вуду. Интересно, это тоже за знак посчитаем? Жизнь очередной раз 
налаживается~--- диплом филфака наконец-то придется кстати, креативность 
(вспомнить хотя бы еще одну <<нетленку>>:

Круто, ты попал на филфак --- Ты мудак\ldots)

\noindent оценят, и можно будет уйти с места мальчика--консультанта в <<Senukai>> и пойти 
на достойную работу в не самое последнее в Балтии рекламное агентство. Сам-то верю?

Не верю. Верил бы~--- то, получив приглашение от них, первым делом уволился бы. 
Ибо нефиг! Ибо возьмут по-любому, ибо <<воздух выдержит>>\ldots\ Не уволился. 
Даже не заикнулся никому~--- ни на работе, ни вообще. Только проследил чтобы выходные 
замечательно скользящего графика проскользнули на правильное число.

Принесший чай официант что-то спрашивает. Снимаю наушники:\\
--- Выбрали, что будете к чаю?\\
--- Спасибо, ничего\ldots~--- Тянусь вернуть наушники на место, но успеваю 
услышать: 
<<Видишь, Миш, я права~--- он чай заказал!>> Поворачиваюсь на торжествующий 
женский 
голос. А, эта троица вошла почти сразу за мной. Женщина как женщина, ничего 
особенного: в меру привлекательная невысокая блондинка средних лет. А вот 
спутники ее~--- да, своеобразные. Тоже невысокий мужчина с цепким, кстати, на 
меня 
направленным, взглядом, явно не европеец~--- то ли азиат, то ли еще кто-то. 
Узбек, 
что ли? Откуда у нас узбеки? И еще один~--- высокий, худой, лысый и страшный. 
Взгляд~--- злой. И тоже на меня. \\
--- Попробует у меня еще этот чаеман чифиря первоклассного, будет знать\ldots

Бред, лучший способ бороться с которым~--- отгородиться от него. Вернул музыку 
в уши, отвернулся. Но почему их так заинтересовал я? Может, действительно~--- мой 
день, и в результате и рекламщиков я заинтересую? Все-таки хотелось бы.

Замечаю, что троица направляется к моему столу, но продолжаю их не замечать. Не 
помогает. Усаживаются напротив меня. Еще минуту вслушиваюсь в рассуждения 
Шевчука о гражданской войне. Терпеливо ждут. Не выдерживаю. Избавляюсь от 
Шевчука в ушах.

Убедившись, что им удалось завладеть моим вниманием~--- еще бы не удалось, 
такими-то методами~--- вступили в разговор. Заговорил <<узбек>>.\\
--- Извините за беспокойство, но мои спутники очень хотят познакомиться с вами. 
Да и я, не буду скрывать, крайне заинтересован в знакомстве\ldots\\
Пожимаю плечами.\\
--- Борис. Что дальше?\\
--- Очень приятно, Борис. А меня зовут Моук, или Мо. А мои спутники\ldots~--- 
<<Узбек>> посмотрел на них и неприятно, во всяком случае, так мне показалось, усмехнулся.\\
--- Они, пожалуй, предпочли бы представиться самостоятельно.\\
--- Что ж, в таком случае предоставим им такую возможность,~--- Отвечаю в тон 
<<узбеку>> и слышу сипящий голос высокого, видимо его женщина называла 
<<Мишей>>:\\
--- Смелый, гаденыш\ldots\\
А почему нет? Они нагло пялятся, кидают угрожающие реплики, бесцеремонно лезут~--- 
а мне даже поерничать в ответ нельзя? Можно. Буду. А если что не нравится~--- 
идите откуда пришли. Мы вас не звали, здесь вам не рады. Смотрю на того, кто, 
получается, <<Мишей>>.\\
--- Миха,~--- Наконец, представляется он, давая мне повод поздравить себя с 
зачатками дедукции.~--- Шаман.\\
--- А я Регина. Если не помнишь~--- одинокая, погрязшая в серости быта женщина, 
зашедшая однажды в парфюмерный\ldots\\
--- Вуду пипл, мэджик пипл,~--- Всплыло и показалось очень уместным мне. Но 
троица, естественно, не въехала.\\
--- Что, простите? \\
--- Да ничего,~--- отмахнулся я от озадаченного <<узбека>> Мо.~--- Просто мне 
сегодня Вуду вспомнился, знаете, из <<Точки~--- бесконечности>> моей. Я писал, кажется, 
как мне однажды подруга рассказала, что его встретила. А тут вы трое\ldots\ Вот 
только 
вас я, простите, не помню. Вы тоже из моих рассказов, какой-нибудь 
дворник--эмигрант из <<Тоскливо>>? Я откровенно хамил, однако Мо после моего 
вопроса 
задумался. Я же, пока он думал, а двое других усердно пилили меня глазами, 
вспоминал: упоминал я в последних постах, что планирую сегодня днем в 
<<Северном 
Городке>> быть? Вроде бы нет. А говорил\ldots\ Говорил, кажется, только 
Арвидасу, но ему мои графоманско-блоггерские потуги по барабану. Мягко говоря. Или еще Игорю 
говорил, когда с ним пересечься договаривались?..\\
--- Нет, определенно не дворник,~--- Наконец родил Моук.~--- Но однозначно 
ответить по существу вашего вопроса, Борис, я не могу. Это вам лучше знать~--- сочиняли вы 
меня или нет. Сочиняли?

Если говорил~--- то, пожалуй, он мог бы озаботиться\ldots\ Пожимаю плечами:\\
--- Не знаю, да и, честно говоря, по-моему, шутка перенасыщена. На мой взгляд, 
одного Шамана хватило бы. Или одной Регины, но Шаман~--- он колоритнее, 
согласитесь. Они оба, они вместе~--- уже перебор. А вы тут вообще непонятно 
зачем\ldots\\
<<Шаман>> и <<Регина>> переглянулись.\\
--- Слышь, хозяйка, я <<колоритнее>>,~--- Зло расхохотался мужчина.~--- Конечно, 
блин, всю жизнь на наркоте провести~--- кто угодно колоритным будет\ldots\\
Женщина аккуратно погладила <<Шамана>> по руке.\\
--- Не верите?~--- Улыбнулась мне, заставив меня машинально улыбнуться в 
ответ.~--- Я, когда нам Мо про вас рассказал\ldots\ Хотя, что я говорю~--- поверила, еще 
понять не успела, а уже поверила. \\
--- Прям сразу?\\
--- Сразу. Правда, я тогда уже во все что угодно была готова верить~--- и Мо 
постарался и вообще\ldots~--- Женщина, замолчав, тряхнула волосами, а я вдруг 
подумал, что она действительно кажется мне какой-то очень знакомой, почти 
родной.~--- Вообще, вот, после того, что вы описали, у меня в голове 
перевернулось 
что-то. Как-то очень ярко стала осознавать, что если не верить в чудеса~--- мир 
грустен.

Мир грустен, говорите\ldots\ Грустен он, когда оказывается, что все, что ты о 
себе думаешь~--- завышенная самооценка и дутые розовые мечты, когда друзья, за 
которых ты в юности готов умереть~--- предают, несмотря на то, что ни умереть тебе за 
них не светило, ни для предательств у вас жанр жизни не слишком подходящий. 
Грустно, когда ты <<снимаешь>> однокомнатную квартиру одноклассника, который делает 
ремонт в собственном доме и пока живет с тобой, а по вечерам вы вспоминаете как он у 
тебя списывал все контрольные и даже экзамен по биологии. За нее, правда, оба 
по <<шестаку>> получили, но она и была шестым <<запасным>> экзаменом. Когда та, на 
которой всю студенческую жизнь мечтал жениться, после университета\ldots\ Ай. Я 
достал телефон, посмотрел на часы. Можно идти. Приду минут за двадцать до 
назначенного, но это уже в рамках нормы, так что должны впустить переждать. 
Махнул рукой официанту.\\
--- В общем\ldots~--- Я чуть не ляпнул привычное в блоге обращение к 
предполагаемым 
читателям <<дорогие мои>>, но осекся. В реале так не говорят.~--- В общем, 
простите, 
но мне пора идти. Передайте спасибо от меня Игорю~--- шутка позабавила. 

Всегда так. В продуманном эпизоде, после финальной реплики герою положено 
встать 
и уйти. Зрители в восторге, оппоненты в осознании превосходства героя, 
герой~--- в 
гримерке взбирается на пьедестал. А тут приходится ждать пока официант принесет 
чек, потом сдачу, зрители негодуют по поводу провисания сцены, а у оппонентов 
появляется возможность ненужной и даже лишней, учитывая, что финальная-то уже 
прозвучала, реплики. Ею воспользовался Мо:\\
--- Борь, это не шутка. Эти двое~--- это действительно созданные вами персонажи. 
Я проверял. Эти двое~--- чудо, которое вы сотворили. Точнее, два разных чуда.

Не таких уж и разных. В обоих рассказах писалось, в общем-то, об одном. Я 
вообще 
достаточно уныл и однообразен в своих творческих темах. Официант неохотно 
принес 
сдачу. Можно идти. Поднимаюсь, цепляю пальцами провода от наушников. Время для 
еще одной финальной реплики.\\
--- Чудес не бывает.

Ухожу. <<Случайный подбор>> выдает <<Сказки нашего двора>>. Положительно, мне 
сегодня везет на знаки и созвучные совпадения, но какие-то они все 
бессмысленные.

Бессмысленно потраченное время. Хм, может блог переименовать? Или даже нет: 
<<утраченное>>, собрать все дыбровые посты из него, и издать. Ну, собрать все 
дыбровые посты из него, скопить денежку, ибо ни одно издательство за свои 
деньги на такое не пойдет~--- и издать. <<Бессмысленно утраченное время>>~--- 
постмодернистский сиквел великого творения Пруста. Действие романа перенесено в 
современность и, соответственно под современный мир адаптированы и основные 
идеи романа. Теперь главный герой, пусть и так же пропускает время сквозь пальцы, 
увлеченно отмечая в окружающем мире случайные отсылки к собственному прошлому, 
однако не может на основе накопленного материала создать собственную 
эстетическую концепцию. Собственно он ничего не может~--- только бессмысленно 
тратить отведенное ему время\ldots>> Как-то так. Шедевр, я считаю~--- и даже 
специально писать ничего не надо.

Чудес не бывает~--- правильно я сегодня сказал этим. Да, на собеседование меня 
позвали. Да, вежливо поблагодарили за оказанное внимание к их фирме. Что-то 
поспрашивали. И вежливо извинились, что поскольку заинтересовавшихся вакансией 
оказалось на порядок больше, чем они ожидали, и конкурс получился очень 
большой, то они не могут уделить должное внимание всем кандидатам\ldots\ Однако они 
внесут меня в базу данных и в случае появления новых вакансий\ldots\ Козлы. Не 
говорили бы, ограничились бы простым <<Ждите, в случае если вы попадете во второй тур~--- мы 
вам позвоним>>, я бы сейчас скептически рассуждал, что <<не позвонят, конечно>>, но 
к моменту, когда действительно стало бы понятно, что уже точно не позвонят~--- 
смирился. А сейчас\ldots\ Обидно. Так обидно, что даже Игорю позвонил и отменил 
встречу~--- может и хорошо было бы сейчас через эмулятор во вторую Контру 
побегать, наверное, даже и правильно было бы\ldots\ Но не хочу. Хочется, 
несмотря на то, что до февраля еще долгих два с небольшим месяца, <<достать чернил и 
плакать>>, предварительно обратившись к буниновской методе топки камина.

Игорь к тому, что я в последний момент план срываю, нормально отнесся~--- 
привычный. Но вот на тему розыгрыша колоться не пожелал. <<Бо, вы, как 
соберетесь к нам~--- расскажете подробнее, интересно, что там у вас произошло такое, 
мистически-пиранделловское>>\ldots\ Ага, приедем, расскажем и посмотрим в вашу 
хитрую физиономию~--- глазки, небось, бегать будут. Если подумать~--- свинский прикол. 
Нельзя так человека по больному, особенно перед важной встречей. Может, если бы 
эти не выбили меня из колеи, то и перед рекламщиками я б лучше выступил. 
Наверняка. В общем, Игорь, с вас пиво. Или даже нет~--- пивом за испорченное 
будущее вы не отделаетесь, с вас~--- бутылка текилы, вот. А за пивом сегодня 
сам схожу. В любом случае~--- по пути от остановки.

Так и сделал. А возле дома попал в засаду. Недавние знакомцы, эффектно 
оставленные мною в кафе, ждали меня у подъезда. Исчадия, если верить им, моей 
фантазии~--- стояли у крыльца: <<Шаман>> курил и что-то рассказывал внимающей 
ему <<Регине>>. Пожалуй, он действительно похож на Шамана. А она? Не знаю, в 
общем-то, если у меня и был перед глазами какой-то образ, когда я писал ее~--- то 
исключительно на уровне типажа. Ну и, возможно, фотографии Регины в 
<<Фотогалерее>>. Типаж угадали, хотя, кажется, в том рассказе о внешности ее не 
было сказано ни одного конкретного\ldots\ Игорь-Игорь\ldots\ Третий~--- Мо, 
кажется,~--- расслабленно сидел на скамейке. В общем, просто пройти мимо них не получалось, 
так что я дошел до сидящего на скамье~--- он хотя бы не претендовал на мое 
авторство и обратился к нему:\\
--- А где букет?\\
--- Какой?~--- Он встал со скамьи и потянулся.~--- Задремал я, вас 
дожидаясь\ldots\ Так какой букет?\\
--- Ну как\ldots~--- Я пожал плечами.~--- В программе <<Розыгрыш>>, насколько я 
помню, в конце собственно розыгрыша его участники дарят объекту торт, букет, пьют 
шампанское и радуются\ldots\\
--- То есть вы до сих пор уверены, что это <<розыгрыш>>?~--- Мужчина выглядел 
явно расстроенным. \\
--- А что, собственно, изменилось? Тем более, вы, простите, <<спалились>> с этим 
торжественным ожиданием. Я в эту квартиру месяц как въехал, в блоге адрес не 
светил, так что теперь я более чем уверен, что вся эта постановка~--- дело рук 
Игоря.

На самом деле так уж уверен я не был. Перестал быть, когда их увидел. Слишком 
назойливый прикол, если честно. Мой собеседник махнул рукой:\\
--- Да ладно. Пока мы вас три дня искали Миха не пил почти~--- так что, 
соприкоснувшись с вами, найти где вы живете~--- для него было легче легкого. Он 
бы нас и прямо в квартиру провел без проблем, но мы решили, что это было бы как-то 
невежливо\ldots~--- Лицо мужчины просветлело.~--- Точно! Скажите, вас убедит, 
если мы прямо сейчас переместимся в вашу квартиру~--- без помощи лестниц, вашего ключа, 
вас?\\
--- Прям-таки <<без помощи>>?~--- Я позволил себе недоверчиво приподнять брови. Я 
вообще-то не знаю как это делается и почему это должно выражать <<недоверие>>, 
но в книгах так пишут. Впрочем, если оно и выглядело неестественно, то темнота и 
такт Мо позволили загладить это. Он повернулся к оставшимся двум:\\
--- Шаман, сможешь?\\
Второй мужчина прервал свой монолог.\\
--- Как два пальца,~--- Кивнул и для убедительности сплюнул.

<<Регина>> взяла его под руку. Странно они вместе смотрятся. Мой собеседник 
подошел к парочке и приобнял <<Шамана>> за плечи. Бред какой-то. Обернулся ко 
мне:\\
--- Вы с нами?\\
--- Бред какой-то,~--- Я, честно говоря, растерялся и не придумал ответа 
достойнее. \\
Мо пожал плечами:\\
--- Тогда встретимся наверху.

И они исчезли. Они исчезли. Просто. Вот как стояли~--- так и исчезли. Даже без 
какого-нибудь сияния, без постепенного превращения в прозрачные силуэты, 
которым 
еще через пару секунд будет суждено растаять в вечернем воздухе, без арки 
портала\ldots\ Просто. Исчезли. А я неожиданно вспомнил, что дома~--- 
беспорядок и я 
даже не успел за него извиниться. Потом, оценив комичность моих переживаний, 
растерялся. То есть\ldots\ Ну, что мне сейчас полагается делать? Рассмеяться и 
пойти 
домой? Просто пойти домой? Позвонить~--- а куда? В полицию или сразу~--- в 
скорую? 
<<Забирайте, товарищи дорогие, чего-то меня плоды собственной фантазии 
преследовать начали>>\ldots\ Или развернуться, позвонить, например, тому же 
Игорю, 
что снова передумал, приехать прям с пивом к нему, погамать, заночевать~--- и 
уже 
завтра вечером вернуться? Последний вариант казался наиболее привлекательным, 
но, к сожалению, был абсолютно неприемлемым. Конечно, хорошо, когда тебе 
удалось 
приучить друзей к тому, что планы могут меняться и, мало того, меняются. Но 
когда они меняются по десять раз на дню~--- это все-таки перебор. Представил, 
как 
на меня будет смотреть Игорь~--- и отчетливо понял: не буду ему звонить. А 
Арвидас 
эту неделю живет у своей~--- у нее родители в отпуск уехали. Отпадает. Я 
зачем-то 
пролистал список контактов в телефоне, хотя прекрасно знал, что больше мне, 
собственно, звонить некому. Значит~--- наверх. И будь, что будет.

Они~--- были. Не знаю, обрадовало меня это или нет. Всю дорогу наверх, а это, 
между прочим, пятый этаж, я пытался понять, чего я жду и что меня напугает 
больше~--- гости в квартире, или совсем уж взбесившиеся тараканы в голове. 
Тараканы у меня те еще, я знаю~--- мне говорили, однако до этого мы с ними 
сживались без особых проблем. С другой стороны, к тому, что они могут 
взбеситься 
– я морально готов. Это нормально. А чудес~--- чудес не бывает. Чудес не 
бывает, и 
я очень внимательно осмотрел замок. Не то чтобы я что-то там мог увидеть, но 
все-таки. Замок, насколько я мог судить, не взламывали.

Открылся он как всегда~--- плавный первый круговой поворот, толчок от себя~--- 
и заедающий второй. И плавность первого круга, и строптивое нежелание ключа 
поворачиваться на втором~--- совершенно обычные. Значит это, что никто его 
отмычками не открывал или не значит? В любом случае, они были внутри. Все трое 
чинно сидели на моей кровати и через дверной проем смотрели на меня. 
<<Миха>>~--- насмешливо, Мо выжидающе, <<Регина>>~--- с сочувствием.\\
--- Пиво будете?~--- А что еще им сказать? Впрочем, предложение выпить оказалось 
не самым плохим вариантом. Глаза <<Михи>> подобрели.\\
--- А курить у тебя можно?~--- Приняв бутылку, спросил он.\\
--- Кури\ldots\ По-моему, сейчас вы командуете\ldots\ Парадом\ldots\\
--- Самое сложное в охоте на тигра~--- убедить зверя, что это на него 
охотятся\ldots~--- Подал голос и Мо.~--- Только вы все-таки не правы, Борис. В нашей четверке 
самая важная фигуры~--- вы, и очень важно чтобы вы это поняли.\\
То есть мы уже четверка, да? Я сел за компьютерный стул.\\
--- Тогда вы, может, наконец объясните мне нормально~--- что тут происходит? 
Желательно без мистификаций и сказок про ожившие фантазии.\\
--- Я вам просто все расскажу~--- Начал Мо, но в разговор включилась <<Регина>>:\\
--- Борь, а можно я пока у тебя душ приму? А то мы три дня в походных условиях 
жили, а в самой истории я~--- персонаж второстепенный и без меня вы сейчас 
обойдетесь\ldots\\
--- Конечно, по общему коридору и направо.~--- Я кивнул. Женщина встала, но с 
места не двинулась. А\ldots\ Я понял и смутился:\\
--- Мыло, шампунь я сейчас дам, конечно\ldots\ Но сильно сомневаюсь, что у меня 
найдется чистое полотенце\ldots\\
--- Да, блин!~--- Это <<Миха>> отставил уже почти приконченную бутылку.~--- Дайте 
уже отдохнуть\ldots\ Любое давай. И мочалку,~--- прибавил, посмотрев на женщину.\\
--- Спасибо, Миш,~--- тихо ответила та. Ну да, спасибо ему, я тут вообще не при 
чем.
 
Мочалка с полотенцем оказались в руках у <<Шамана>>. Он повертел их, 
приговаривая: 
<<Да здравствует мыло душистое и полотенце пушистое и еще-что-то там, 
трам-парам, 
пам-папам>>\ldots\ Протянул <<Регине>>:\\
--- Готово.

И я поверил. Ну, почти. Фальшивые чудеса так не делаются. Когда зрителя хотят 
убедить в том, что у него на глазах свершается чудо~--- это чудо обставляют 
соответствующе. Антураж <<чудодейства>> никак не должен ограничиваться неточной 
цитатой из Чуковского. Тем более <<Регина>> явно обрадовалась, понюхав 
полотенце. 
Прям, можно подумать, оно было так страшно запачкано\ldots\ Две недели назад 
стирал.\\
<<Шаман>> потянулся.\\
--- Пойду я прошвырнусь. Мне все эти истории тоже осточертели\ldots\\
Мы с Мо остались наедине. Я открыл следующую бутылку.\\
--- Итак, слушаю\ldots

Мо встал с дивана. Прошелся по комнате, словно выбирая подходящее место, 
наконец, встал, прислонившись к шкафу. Я слегка повернулся чтобы оказаться 
лицом к лицу с ним. Он благодарно кивнул.\\
--- Давайте, я начну с самого начала\ldots

\newpage

Слушая рассказ Мо о странниках и их метаморфозах, о не до конца понятной мне их 
войне с <<богами>>, о возникшей в результате упорядоченной вселенной и ее 
противостоянии с хаосом, я думал, что сам бы не написал такой истории. Не 
потому 
даже, что она слишком невероятна или слишком хороша~--- на самом деле история 
вполне на уровне средненькой <<самиздатовской>> фантастики. Я просто не смог бы 
рассказать историю о чудесах. Чудес не бывает, а для того чтобы сочинить 
историю 
нужно хоть ненадолго в нее поверить. Мой предел: аномальный Шаман, чья 
способность к волшебству настолько противоестественна, что разрушает его 
самого. 
Чудес не бывает:

Изучение отделов эзотерики и магии в книжных и онлайн библиотеках убедило меня 
в этом окончательно. Так что, однажды перестав верить в них, я потерял и 
способность сочинять их. Но слушать было\ldots\ Интересно, само собой, но и 
приятно. 
Приятно даже в большей степени. Одно дело~--- читать фантастику и фэнтези, 
зная, 
что все это сказка, совсем другое~--- когда напротив тебя стоит человек и 
преподносит это тебе как реальность, ту часть реальности, которая от тебя 
ускользает. А это уже дело другое. Слышал же ты краем уха о квантовой теории 
полей и физике элементарных частиц и со школьной скамьи тебе известно, что 
большая часть звезд в ночном небе на самом деле давно погасла. Это тоже 
реальность, просто те ее уровни, которые тебе кажутся непостижимыми. Почему бы 
и тому о чем рассказывает Мо не быть правдой? Это уже не чудеса\ldots

Удивительно, как можно убеждать самого себя, подспудно убеждая, что ты не 
убеждаешь\ldots\ Когда мой рассказчик дошел до теории Сказочника, я пришел к 
выводу, что поверить ему~--- правильнее, чем не поверить. Во-первых, тогда не нужно 
придумывать объяснения тому, откуда вообще взялась эта троица, не нужно гадать 
о том, как они проникли в квартиру\ldots\ Во-вторых, так интереснее. Окей, чудес 
нет, я привык, я смирился. Но с какой стати это должно означать, что реальность не 
может быть несколько иной, чем я привык ее видеть? В конце концов, того, что в 
нашем мире, таком как он есть сейчас, чудеса практически невозможны, не 
отрицает и сам Мо. Он только говорит, что я могу изменить законы этого мира до того, как 
с ним случится что-то очень пугающее.\\
--- Ладно, допустим.~--- Я поставил на пол последнюю опустевшую бутылку.~--- И 
что я должен делать?\\
Мо вздохнул.\\
--- Я не знаю, Борис. Вы~--- первый встреченный мною Сказочник и вы первый 
запротоколированный Сказочник вообще. Я не знаю как это должно работать. 
Полагаю, что для начала вы должны поверить мне, потом должны поверить в ваши 
способности и выяснить как именно они работают\ldots\ С первым, насколько я 
понимаю, мы разобрались?\\
--- Более-менее,~--- Я пожал плечами.~--- Я не очень хорошо представляю как вы 
намереваетесь выманивать у меня деньги или еще что-то в дальнейшем и я 
прекрасно осознаю, что вообще не являюсь хоть сколько-нибудь интересным для аферистов. 
Поэтому наиболее логичным мне кажется допустить, что то, что вы 
рассказываете~--- правда\ldots\\
--- Что-то мне это напоминает, правда, Мо?~--- Вернувшаяся из душа где-то на 
середине рассказа Регина сейчас сидела на диване. Время от времени я поглядывал 
на нее и находил, или убеждал себя, что нахожу, все больше сходств с одной из 
двух виденных мною фотографий той, другой Регины.\\
--- Кстати, Регина, не сочтите за наглость\ldots\ Скажите, у вас на спине, 
справа, чуть выше талии, нет двух родинок~--- наискосок друг от друга, нижняя чуть 
больше верхней?\\
--- Действительно, напоминает,~--- Теперь, по непонятной мне причине рассмеялся 
Мо.\\ 
--- Вот они издержки общения персонажей с автором\ldots\ Ну что, королева, 
помните, как вы меня проверяли?\\
--- Да уж\ldots~--- Регина встала напротив меня.~--- Интересные у тебя способы 
создания персонажей\ldots\\
Она повернулась ко мне спиной и задрала вверх кофточку.\\
--- Такие?\\
Да, такие. Родинки были точь-в-точь как на второй фотографии.\\
--- Спасибо\ldots~--- Женщина поправила одежду и повернулась ко мне. Улыбнулась.\\
--- А что, у меня и прототип есть?\\
--- Да как прототип\ldots~--- Я вздохнул.~--- Скорее, случайная муза. В 
интернете пересекались\ldots


Я начал было рассказывать Регине про Регину, но вдруг запнулся. Что-то в этом 
было очень абсурдное, сюрреалистичное, но даже не это меня остановило. Я вдруг 
понял, что принял их историю. Что в моем сознании исчезли кавычки, и эта 
стоящая передо мной Регина~--- действительно Регина, что странный асоциальный тип с 
повадками блатаря~--- действительно Миха Шаман. Не то чтобы я перестал считать, 
что чудес не бывает, скорее Мо каким-то непостижимым образом удалось убедить 
меня, что в том, что я~--- Сказочник нет ничего чудесного. Просто еще одна 
грань реальности.\\
Сам Мо, кажется, поймал мое настроение. Он облегченно выдохнул:\\
--- Да, с первым пунктом разобрались.\\
~\\

Итак, я должен спасти мир.

Странное ощущение от непривычной утренней мысли. Слова твердыми комками 
перекатываются на языке, постепенно смягчаясь. Итак, я должен спасти мир. 
Словно карамельки.


Да, непривычная мысль. В рабочие дни я просыпаюсь с мыслью: <<Уже!?>>~--- она 
короткая и кисло-острая, как жвачка <<Шок>> из моего детства. Просыпаюсь от 
шока, просыпаюсь от навязчивого <<Johanna, why don't\ldots >>, выключаю будильник на 
телефоне, заканчиваю за него: <<Shut up>> и некоторое время слушаю тишину. 
Интересно, эта самая Джоанна почему-то представляется мне не красивой румынкой, 
а толстой, почти необъятной полячкой. Я одно время даже думал написать про нее, 
но так и не понял~--- кто она, где, зачем.

<<Зачем?>>~--- это вторая по частоте моя утренняя мысль. Она случается на 
выходных, 
когда не звонит будильник, но я просыпаюсь даже раньше, чем с ним. Она тягучая 
и горчит. Есть такие невкусные конфеты, обычно черные. Она заставляет понять, что 
я уже не усну, что придется вставать и занимать себя чем-то. На вечерний пост в 
блоге, скорее всего, никто еще не ответил, и вряд ли в ф-ленте появились новые 
интересные тексты. Значит, надо что-то придумывать. Читать? Писать? Завтракать? 
Зачем?

Итак, я должен спасти мир. Эта мысль отсылает к чему-то очень давнему, 
прошедшему. То ли к детству~--- когда каждое утро просыпаешься, осознавая, что 
тебя ждут подвиги и свершения, и мир прекрасен и крутится он вокруг тебя. То ли 
к ранней юности, когда ждешь, что совокупность шагов, начиная уже со спущенной 
с кровати ноги, которые ты ступишь сегодня, приведет тебя к будущему не просто 
светлому, но ослепительно сверкающему. Будущему, в котором ты сделал все, что 
должен и даже больше~--- и все это оказалось нужным кому-то. Даже не 
кому-то~--- 
всем.


А потом оказывается, что чудес не бывает.


А еще потом~--- наступает утро, и первое, что ты слышишь от себя: <<Итак, я 
должен 
спасти мир>>. Но, поскольку это утро выходного, вопрос <<Зачем?>> никуда не 
исчезает.


Зачем мне спасать этот мир? Хочу ли я его спасти? Нравится ли он мне? Бросьте, 
Борис, ваша мизантропия не распространяется так далеко. Это странный мир, да. И 
не все в нем меня устраивает. Но это не значит, что я не хочу чтобы он был. И 
тем более это не значит, что я хочу отказаться от такой возможности\ldots\ 
Возможности спасти мир.


С <<Зачем?>> разобрались. Хочу. Осталось чтобы кто-нибудь объяснил <<Как?>> 
Лидер 
команды гостей вчера развел руками: мол, сам знать ничего не знаю, ведать не 
ведаю, давай, Боря, утром порознь и вместе покумекаем\ldots\ Ну\ldots\ Давай. 
Значит, 
окончательно просыпаться, вставать, думать.


Странно, только сейчас заметил. Вместе с привычными утренними мыслями исчезла 
из 
квартиры и привычная утренняя тишина. Слышно постороннее дыхание, но это ничего 
– оно слышно и в те дни, когда здесь ночует Арвидас. Не в том дело. Жужжит 
компьютер. Стоило бы сказать: <<тихо жужжит>>, но компьютер у меня старенький, 
если не сказать дряхлый. Так что~--- жужжит, <<тихо>> в данном случае было бы 
недопустимым бессовестным преувеличением.


Сел на кровати. Привычно-аккуратно, чтобы не побеспокоить соседа. Квартирка эта 
слишком маленькая для двух постоянных лежбищ, так что мы с Арвидасом спим 
обычно 
на одном большом диване-кровати. Сами, конечно, на эту тему шутим, но другим 
стараемся не распространяться. А то мало ли\ldots\ Сегодня диван со мной делил 
Мо, 
Регине я надул гостевой матрас, а Шаман, сейчас оказавший на нем, к тому 
времени 
еще не вернулся. Или вернулся под утро, или в компьютерном кресле спал. Ну, 
хотя 
могли и вдвоем поместиться, почему нет?

Итак, компьютер. За ним сидела Регина. Сидела странно: руки положены на стол, 
но 
ни клавиатуры, ни мышки не касаются, спина оторвана от спинки стула и тело 
нависает над монитором, взгляд, насколько можно судить с моего места, прикован 
к 
одной точке. Если бы на женщине были наушники, то можно было предположить, что 
она сосредоточенно смотрит нечто очень захватывающее. Наушников не было, да и 
свечение от экрана шло постоянное, а не дерганное, как бывает при частой смене 
кадров.\\
--- Доброе утро,~--- Сказал вполголоса.\\
Регина встрепыхнулась и повернулась на мой голос.\\
--- Ой, доброе. Рано встаешь\ldots\ Ничего, что я твоим компьютером 
воспользовалась?\\
--- На здоровье,~--- Я окончательно покинул кровать и теперь быстрыми движениями 
натягивал штаны. Всегда раздражали такие ситуации. Вроде и просить отвернуться 
неловко: я, как никак, в белье, получается какая-то болезненная 
стеснительность, с другой стороны~--- неловко, все равно. В общем, чем быстрее я справлюсь со 
штанами, тем меньше пострадает мой душевный комфорт. При этом, желательно не 
показывать, что он вообще страдает.~--- А что ты там нашла такого 
поразительного, если не секрет?\\
--- Ой\ldots~--- Второй раз за короткое утро вздохнула гостья.~--- Не секрет, 
просто нечто, что я пока не могу переварить. Идем на кухню? Я как раз чайник ставила, 
но пока он закипал~--- наткнулась вот\ldots\ Так что и расскажу, и чаю заодно 
попьем.

Я кивнул. Утренний чай~--- это святое.

Пока успевшая освоиться на моей кухне~--- и когда успела?~--- Регина, доставала 
из шкафчиков кружки и заварку, быстро умылся. Смотрел: женщина готовит мне 
утренний чай. Идеальная картина. Разве что, пожалуй, ей не хватало халатика~--- на голое 
тело, или на какую-нибудь смешную домашнюю маечку. Да, тогда бы она выглядела 
совсем уютно. Совсем органично, если забыть о том, что женщина, готовящая мне 
утренний чай у меня на кухне~--- это неорганично по определению. Это что-то на 
уровне оксюморонов. Уже давно.\\
--- У тебя есть халат?~--- Спросил неожиданно для себя.~--- Какого цвета? Такой, 
светло-коричневый, почти желтый?\\
--- Ага,~--- Она разлила кипяток из чайника по кружкам и повернулась ко мне.~--- 
Откуда знаешь? Или ты и его придумал?\\
Я задумался.\\
--- Не знаю\ldots\ Вообще, насколько помню, у меня там была сцена, как ты после 
душа облачаешься в халат. Но на самом деле, просто представил тебя в таком халате 
сейчас, получилось очень естественно.\\
--- Тяжело с вами\ldots~--- Регина уселась на единственный кухонный стул, 
оставляя мне единственной альтернативой подоконник.~--- Один меня выдумал, причем во всех 
деталях, причем, получается, знает обо мне самые сокровенные подробности\ldots\ 
Другой превращает по ночам мою квартиру в допотопные джунгли, а днем 
наколдовывает мне невероятное количество денег, третий\ldots


Она шутливо махнула рукой и притворно вздохнула:\\
--- Мужчины\ldots\ Вечно с вами проблемы\ldots\\
--- Разве деньги~--- это проблема?~--- Искренне удивился я.\\
--- Да, деньги\ldots~--- Регина кивнула.~--- Как раз их я и нашла утром, когда 
ты меня позвал. Ну, точнее, еще раньше, но\ldots~--- Она потрясла головой и улыбнулась 
мне.~--- Мо тебе рассказал про лотерею?\\
Я покачал головой.\\
--- Ну да, конечно. Для него же это все было не очень важно, наверное. Так, 
мелкие чудеса ради великой цели. В общем, когда он ко мне заявился, уже в человеческом 
обличии, и если не убедил в своей истории, то все-таки убедил его оставить~--- 
он пообещал квартплату. Еще пошутил что-то насчет того, что хорошую, не факт, что 
я ее потратить успею.

А на следующий день дал мне номера, чтобы я лотерейный билет заполнила. Ну, 
понимаешь, мне же уже интересно было~--- что вообще происходит и разве бывает 
так, 
я заполнила. А потом закрутилось: Миша появился, Мо нам про конец света 
рассказал, про тебя, потом сюда поехали\ldots\ В общем, я, если честно, тоже 
про 
этот билет почти забыла. А сегодня утром проснулась~--- вы спите, Миша в кресле 
ворочается, а у меня сна ни в одном глазу. Ну, я его на мое место пустила, а 
сама к компьютеру села. Решила почту проверить. Проверила, так полазила\ldots\ 
Думаю, зайду на сайт этих <<Викингов>>, посмотрю. Вот и посмотрела\ldots\ 
Выиграл мой билетик главный приз прошлой недели~--- а это 7 миллионов долларов. Вот и 
зависла.\\
Я присвистнул.\\
--- И что теперь?\\
--- Теперь\ldots~--- Регина как-то испуганно посмотрела на меня.~--- Я не знаю, 
что теперь. По идее, нужно связаться с организаторами: или в местном, ну, в смысле, 
в нашем местном офисе, или по телефону, или зарегистрировать свой билет через 
интернет у них на сайте\ldots\ Но мне страшно как-то.\\
Я почувствовал, что мне нужно как-то подбодрить ее.\\
--- Денег бояться~--- в банк не ходить,~--- Дурацкая шутка, но сомневаюсь, что 
кто-нибудь смог бы придумать что-то действительно стоящее, услышав от 
собеседника, что тот только что стал миллионером. Причем, миллионером ни по 
российским и даже не по нашим стандартам. Считаю, в данной ситуации я еще очень 
неплохо справился.~--- Пойдем, твой счастливый билет зарегистрируем. Или хочешь 
дождаться, когда остальные проснуться?\\
--- Да не знаю\ldots~--- Регина пожала плечами.~--- Не думаю, что им это сильно 
важным покажется. Мо во всех отношениях ни от мира сего, его наши финансовые сложности 
мало интересуют, а Миша, судя по тому, каким он уже под утро был, будет со 
своими проблемами разбираться.\\
--- Пьяный? --- Зачем-то уточнил я.\\
--- В хлам.~--- Регина улыбнулась мне. Возможно, показалось, но я заметил в ее 
улыбке печальное осуждение, причем адресованное не Шаману.~--- Так что, 
наверное, ты прав. Идем, зарегистрируемся\ldots~--- Последнее слово прозвучало так, как, 
наверное, должно звучать слово <<Сдаемся>>~--- обреченно и с чуть заметным 
вкусом облегчения от того, что пусть так, но все кончится.\\
--- Идем.


Компьютер мирно трещал. Гости спали. Интересно, если кто-нибудь представляет 
время от времени, как он становится обладателем семи миллионов долларов, 
представляет ли этот кто-то такую картину? Вряд ли. Однокомнатная маленькая 
квартирка с дряхлым дребезжащим компом, видавшей виды мебелью и спящими на полу 
людьми как-то не укладываются в рисунок мига, после которого меняется твоя 
жизнь. С другой стороны, такая картина очень красочно передает то, почему ты 
стремишься чтобы твоя жизнь изменилась. Так что, ладно. Сочтем все это 
символичным.

Впрочем, и регистрация выигрыша не добавляла торжества. Обыденные поля для 
заполнения: номер билета, номер тиража, совершенно прозаичная таблица: 
<<Поздравляем, вы выиграли>>, снова анкета: номер билета, номер под штрих-кодом 
билета, номер тиража, дата покупки, место покупки. Подтверждение. Сообщение о 
том, что данные совпадают. Информационное сообщение о том, что нужно связаться 
с представителями лотереи в течение четырех недель, возможность зарезервировать 
время и выбрать офис, возможность оставить свои данные (что удивительно~--- не 
обязательная для заполнения), контактный телефон и <<мыло>> сотрудника. Все.

Зашевелились остальные, так что Регина, завершив манипуляции у компьютера, 
вернулась на кухню. Я сел на согретый ею стул. Почта, жж, фэйсбук, форум 
любимой игрушки. Антивирус выкинул сообщение о атаке. Так иногда бывает. Когда я 
некоторое время снимал комнату в студенческом общежитии, такие сообщения 
выскакивали постоянно. До тех пор пока не разобрался в настройках и не попросил 
антивирус не показывать мне их.

Потом, после очередной переписи Виндовс, уже не настраивал~--- все-таки, 
отдельная линия, и такие сообщения хоть и появляются почему-то иногда (и Игорь, и Арвидас 
пытались мне отдельно друг от друга объяснять почему и на что антивирус так 
реагирует, но я не слишком пытался понять), но не слишком часто. А тут уже 
второе за три минуты. И третье. Это что, Регина в свою утреннюю сессию где-то 
вирус поймала? Нужно будет полное сканирование поставить\ldots

Меня отвлек телефон. Посмотрел на дисплей~--- незнакомый иностранный номер. А 
<<+7>> --- это не Россия случайно? Поднял.\\
--- Алло?\\
--- Здравствуйте, уважаемый. Мы звоним вам по поводу вашего недавнего выигрыша. 
Не могли бы представиться?~--- Голос вежливый, но напористый. Собственно, как и 
первые его слова.\\
--- А сами-то вы кто?~--- Я отогнал промелькнувшую мысль отдать телефон Регине. 
Она заполняла билет при мне~--- ни ее, ни тем более моего телефона она не 
указывала. Мой номер она, вроде бы, вообще знать не должна. Как, соответственно, и 
звонящий.\\
--- Меня зовут Аркадий Викторович, я являюсь менеджером одной крупной фирмы, 
занимающейся планированием инвестиций и капиталовложений, и у нас есть для вас 
предложение, в котором, мы уверены, вы будете заинтересованы\ldots\\
--- Иными словами, от которого я не смогу отказаться?~--- Да, по телефону я 
смелый. Хотя дело даже не в этом~--- просто это вообще не мои проблемы, и, к тому же, 
вся ситуация слишком напоминает кино или плохую литературу, чтобы испугаться 
по-настоящему. Мой собеседник хмыкнул.\\
--- Пожалуй, можно и так сказать. Так не могли бы вы представиться, уважаемый 
Борис?~--- Ага, если я правильно понял жанр всего этого, то сейчас мы перешли к 
фазе завуалированных угроз. Кстати, мне полагается испугаться того, что они 
знают мое имя? Кстати, а откуда? Так, стоп, не сейчас.\\
--- Простите, вы меня с кем-то путаете. И я не знаю, о каком выигрыше вы 
говорите. \\
--- Кладу трубку. Оглядываюсь. Мо уже ушел из комнаты, наверное, пьет на кухне 
чай с Региной. А Шаман тут, смотрит на меня.\\
--- Проблемы?\\
--- Вроде бы. Только непонятно у кого.~--- Я криво улыбаюсь, мне кажется, что в 
таких ситуациях следует улыбаться именно так.~--- Там Регина на кухне чай всем 
делает. Я сейчас приду, если ты не против.\\
--- Твое место у параши, баклан, не мешай пахану~--- Перевел для себя Шаман. 
Вздохнул.~--- Интересно, я себе заначку вчера оставил?

Оставшись один в комнате, я задумался. При чем, что характерно, не о самой 
ситуации, а о том, что я должен о ней задуматься. Так полагается, правильно? Ты 
оказываешься втянутым в какую-то криминальную историю с большими деньгами, тебе 
звонят и завуалировано предлагают отдать деньги, при этом намекая, что у них 
хватает информации о тебе\ldots\ Ты пугаешься, начинаешь думать, начинаешь 
что-то хаотично делать, хватать вещи, складывать их по сумкам\ldots\ Снова зазвонил 
телефон. Отключив его, я пошел пить чай.\\
--- Доброе утро кого не видел,~--- Улыбнулся я Мо. Занявший мое место у 
подоконника Мо приветливо улыбнулся в ответ. Регина сидела на облюбованном стуле, Миха 
сосредоточил внимание на бутылке пива. Уже второй~--- первая, пустая, валялась 
у него в ногах. Я взял свою кружку~--- Регина не забыла и обо мне~--- и 
прислонился к дверному косяку.\\
--- Мо, я помню наш вчерашний разговор, но сначала я хотел бы сообщить вам, 
господа, пренеприятнейшее известие\ldots~--- Думаю, хотя бы двое из моих 
гостей должны были узнать эту сверхизвестную цитату, но ни один из них на нее не 
отреагировал. Что ж\ldots~--- Регина поделилась с вами радостной новостью?\\
--- Это не новость,~--- Мо пожал плечами.~--- Я уже достаточно давно обещал 
королеве, что так будет. \\
--- То есть, поделилась. Так вот, я только что получил пару звонков с российского 
номера. Звонивший, если я правильно его понял, предлагал мне поделиться 
выигрышем. Учитывая, что последний раз я выигрывал два лита в <<Теле-лото>>~--- 
это, конечно, не проблема. Но я подумал, что вам стоит знать.

Итак, я добился всеобщего внимания. Замечательно. Регина смотрела на меня 
испуганно, Мо~--- озабоченно, впрочем, не так озабоченно, как должно, 
столкнувшись с серьезной проблемой, скорее так, как смотрят родители, когда их в середине 
важнейшего рейда в PW, теребят дети с требованием помочь им правильно составить 
кубики. Ну, что-то в таком духе. В общем, единственную помощь, я, как ни 
странно, получил от оторвавшегося в конце концов от пива Шамана.\\
--- Быстро сработали. Я так мыслю: получили данные компа, по нему нашли 
интернет--компанию, у них получили телефон и паспортные данные. И все это в 
ритме марша. Серьезные пацаны. \\
--- Я тоже так думаю,~--- Я кивнул.~--- При этом не слишком скрываются~--- то 
есть им ничто не мешало скрыть номер, и, видимо, не сильно переживают из-за того, что 
объект разработки находится в другом государстве. Из чего следует\ldots\\
--- Линять надо. Сами они, конечно, ксивы так быстро себе не выправят, но 
звякнуть кому-нибудь из местных могут. Подставили мы тебя, Сказочник\ldots~--- Шаман хмыкнул.\\
--- Конкретно,~--- Ответил я в тон ему.\\
--- Круче всего было бы назад вернуться. Только я не верю, что прямо сейчас нас 
четверых потяну. Можно и на складе пока перекантоваться, тебя там искать не 
станут. Но там и втроем тесно, Регине не понравилось, а вчетвером\ldots\\
--- Причем тут <<не понравилось>>,~--- Регина, кажется, искренне обиделась.~--- Я 
тут человека так подвела, что я о комфорте своем думать буду\ldots\ Прости, 
Борь\ldots\\
--- Все нормально.~--- Я улыбнулся ей. Симпатичную все-таки женщину я создал, не 
думаю, что тот, кто делал Еву, сделал ее лучше. Черт, еще и это\ldots\ А я 
почти успел забыть. Я же мир спасти должен. Ладно. Потом.~--- Итак, куда?\\
--- В нижний мир.~--- Подал наконец голос Мо. \\
--- Это где?\\
--- Это что?\\
Мы с Шаманом спросили одновременно. Регина молча повернулась к Мо. Мо потянулся.\\
--- Это~--- Милу. Находится он в Ро. Вы довольны?\\
Шаман какое-то время смотрел на Мо, потом пожал плечами и взял очередную 
бутылку. Кажется, последнюю. Я счел нужным озвучить:\\
--- Нет.\\
--- Почему-то я не удивлен\ldots~--- Мо выразительно, то есть, выражая все свое 
презрение к нашей тупости, как я понял, пожал плечами.~--- Я большую часть ночи 
думал о том, что нам с тобой, Сказочник, делать. Долго думал, пока не вспомнил 
о Милу. Он нам подходит. Если коротко, Нижний мир~--- место в астральном мире, 
куда спускаются шаманы и прочие для решения своих проблем. В обычной практике~--- 
это что-то вроде воображаемого места, чтоб вам было проще. Во всяком случае, я не 
знаю почти никого, кто умел бы проникать туда в физическом мире. Почти. Потому 
что это умел делать мой отец, а он был кем-то вроде тебя. Или вроде Творцов и 
Хранителей, но я не уверен, что разница между вами принципиальна. Он иногда 
брал нас с братом, однако, хотя мы без проблем проделывали самостоятельные 
путешествия, оставляя тела и выходя в сферы, оставаясь в физическом облике, я 
ни разу не смог попасть туда. При этом, с другой стороны, я не знаю ни одного 
случая чтобы отец занимался чем-то таким, выходя из тела. Насколько я понимаю, 
для него это было если не невозможным, то, во всяком случае, очень неудобным: 
мир, тот мир, чьим Творцом он являлся, менялся согласно его воле, то есть, 
стоило ему подумать, что ему нужно проникнуть в сферу астрального и часть мира, 
его окружающего, становилась этой сферой. Мне кажется, в твоем случае, оно 
должно работать похоже. Ну да, конечно. Шаманские традиции, как я не 
подумал\ldots\ 
Впрочем, почему нет?\\
--- И как мы все туда попадем?\\
--- Обычно это делают через Сад\ldots~--- Мо пожал плечами.~--- Но поскольку мы 
это делаем не обычным способом, Сад нам не нужен. Наверное. Хотя\ldots\ Открой 
какую-нибудь из закрытых дверей, посмотри~--- не открылся ли там путь туда. 
Желательно при этом, чтобы ты хотел, что он открылся.\\
--- Ты серьезно?\\
--- Ага.~--- Мо вздохнул.~--- Энергия устремляется вслед за вниманием.\\
Вот прямо сейчас, ага. Пойди и открой. И пусть там будет сад. Совершить чудо 
очень просто.\\
--- Вот прямо сейчас возьмет и устремится?\\
--- Конечно,~--- Абсолютно серьезно кивнул Мо.~--- Сейчас и есть момент силы.

Первая, она же, впрочем единственная дверь, вела в общий коридор. Я открыл ее. 
Зеленые облупившиеся стены, мозаика из синих и коричневых квадратов плитки на 
полу, соседка. Поздоровался с ней, закрыл дверь. Повернулся на кухню:\\
--- Не сейчас.\\
--- Сейчас--сейчас. Вообще, в экстремальных условиях все работает лучше. Но я и 
говорю: сад не наш метод, по-видимому. Так что мы просто пойдем в нижний мир. 
Теперь насчет <<мы>>.~--- Обведя глазами всех присутствующих, Мо кивнул 
каким-то своим мыслям.~--- В Нижний мир, обычно, спускаются одни или, точнее, со 
спутником~--- зверем силы. Это что-то вроде воображаемых наставников--помощников, которые, 
при этом, являются частью тебя. В этом смысле, идеально подходит Шаман\ldots

Если мне не показалось, то Регина поморщилась. Плечами она, во всяком случае 
передернула. \\
--- Не бойтесь, королева,~--- Не показалось. Или показалось не только мне, но и 
Мо.\\ 
--- Миха подходит идеально, но он ничего не знает о том месте, куда мы 
отправляемся. О конкретном его воплощении, конечно, никто не знает, но хотя бы 
общие принципы\ldots\ Поэтому я надеюсь, что твое сознание согласиться принять 
меня твоим спутником, тем более, что оставшись, я буду почти совсем бесполезен. В 
отличие от Михи, который, я уверен, в случае чего сможет и защитить Регину и 
переправить вас обоих назад. Я думаю, второй вариант предпочтительнее\ldots\\
--- Снова через склады?~--- Глухо спросила Регина. Мо пожал плечами и показал на 
Шамана. Миха вздохнул\ldots\\
--- Да, в общем-то без разницы. Это в другие страны-города мне шастать без маяков 
тяжко, а в наш город\ldots\ Что к Серенькому на склад, что к тебе домой, 
хозяйка\ldots Одинаково хреново, в общем-то.\\
Регина кивнула и повернулась к нам.\\
--- А как же вы?..~--- Спросила-то Регина, конечно, взволнованно, но я-то видел, 
что слова Мо ее успокоили. Интересно, можно ли рассматривать возможные отношения 
между двумя сотворенными мною персонажами как инцест? Я посмотрел на Шамана. 
Тот выглядел как обычно: злым, подвыпившим и не особо интересующимся тем, что 
происходит вокруг.\\
--- Я думаю, королева, что после того как мы с Борисом совершим это путешествие, 
проблем с поиском вас и возвращением к вам уже не будет. Я уверен в этом. Так, 
что~--- решили?\\
Шаман нагнулся, составляя на полу аккуратный ряд из пустых бутылок. Потом 
закурил, хм, кажется, первый раз за день.\\
--- Вы банкуете, вам виднее. Собирайся, Регина, чем раньше отсюда свалим, тем 
целее будем.\\
--- Угу\ldots~--- Растерянно кивнула Регина. Потом бросилась к Мо.\\
--- Держитесь, мальчики\ldots\ Я\ldots\ Простите\ldots\\
--- Ты не причем.~--- Мо отстранил женщину от себя и ласково улыбнулся ей.~--- 
Королева, все идет хорошо. Верьте мне.\\
--- Угу\ldots~--- Она всхлипнула. Повернулась ко мне.~--- Борь\ldots\\
--- Увидимся, Регина. В любом случае, мне интересно посмотреть как выглядит 
придуманная мною квартира вживую\ldots~--- И тоже ласково ей улыбнулся. Ну а 
что, я хуже что ли?\\
Через пять минут они ушли. Мы с Мо смотрели друг на друга.\\
--- И что теперь?~--- Наконец, спросил я.\\
--- Не знаю.~--- Просто ответил он.~--- Это твое путешествие.\\
Здорово! Просто замечательно! Я вздохнул:\\
--- Советы, пожелания, напутствия? Как хоть вход должен выглядеть?\\
Мо потер ладони одну о другую.\\
--- Как угодно. В тех декорациях, в которых жили мы с отцом, чаще всего вход был 
норой или пещерой. Но на самом деле~--- как угодно. В целом же\ldots\ Я не 
знаю, что тебе сказать, я говорил~--- в такие путешествия сам я отправлялся только 
пассажиром. Наверное\ldots\ Наверное, тебе стоит помнить, что этот мир 
умещается в твоем сознании, и повинуется твой воле. Наверное, ты должен хотеть чтобы вы с 
миром~--- для начала с Нижним~--- услышали друг друга. И, наверное, нам тоже 
пора.\\
Я стоял и смотрел как Мо обувается.\\
--- Странно,~--- Зачем-то сказал.~--- Я люблю писать тексты, и иногда даже льщу 
себе 
мыслью, что делаю это лучше большинства графоманов. Я даже пару романов 
написал\ldots\ А сейчас чувствую себя так, будто сам нахожусь в тексте, 
написанном 
кем-то совершенно некомпетентным. Точнее даже в двух, которые зачем-то 
совместили под одну обложку\ldots\\
--- Тогда еще вот что,~--- Мо разобрался со своими мокасинами.~--- Помни. Ты~--- 
не в тексте. Ты~--- пишешь текст. Создаешь его. Сюжет, дополнительные планы, мелкие 
детали, все.\\
--- Все?~--- Грустно переспросил я и прошел в прихожую. Конечно\ldots\ Когда-то я 
даже сам так думал. Когда-то, до того как не оказалось, что работа в Senukai, 
одинокие, или, иногда, разбавленные Арвидасом, вечера в маленькой квартире в 
общежитии~--- мой потолок. До тех пор, пока не осталось всего несколько 
человек, которые даже не друзья, которых, просто, мне не стыдно видеть. Может, потому 
что они сами так же переживают то, что из них не получилось того, чего от них 
ждали. Что им не удалось наполнить яркими красками ни свою жизнь, ни жизнь других. Что 
из их жизни не получилось чуда.

Я пишу этот текст\ldots\ Мы вышли в коридор, я запер дверь. Если бы я писал 
этот текст, коридор выглядел бы иначе. Еще более мрачным, подчеркнуто тоскливым. 
Желтым, наверное\ldots

\newpage

Синяя плитка в душе контрастирует с желтой занавеской. Занавеска отсылает к 
стенам в коридоре. Там, где много людей~--- в подъезде, в коридоре общежития, в 
психушке~--- обязательно желтые стены. Желтый цвет~--- очередной пропущенный 
звоночек для пар, и облегчение для одиноких. Одинокие не видят трагедии в 
разлуке. Для них она в прошлом, для других~--- в будущем. И это не 
предположение~--- одинокие это знают.

Даже мессия, оставшись один, становится циником. Это не защита, это~--- 
естественная эволюция. То, что выглядит цинизмом в розовой дымке близорукости, 
в серой, скрашиваемой лишь желтизной стен реальности, становится истиной. В 
лучшем случае~--- пары расстаются. В худшем~--- долго и упорно делают вид, что счастье 
невозможно, а несчастье и скуку лучше переносить вдвоем с привычкой. Только 
привычка~--- это сигарета, которую ты куришь перед душем, или тридцать секунд 
ледяной струи в конце процедуры, никак не человек. Контрастный душ, кстати~--- 
лучшее средство от близорукости.

Выйти из душа и идти к комнате. По общаге носятся дети. Стоило создавать семью 
и хранить ее, чтобы ваши дети~--- почти наверняка лучшее, что вы создали в вашей, 
избегающей желтого, жизни~--- выросли здесь: где на пятьдесят человек шесть 
кабинок, четыре плитки, два душа и один коридор. Все ради иллюзии не-умирания в 
одиночку. В лучшем случае пары расстаются. В худшем~--- тоже: вы вместе лишь 
пока смерть не разлучит вас. И ничто не меняется если ты стремишься к одиночеству: 
на одного всего колыбель и могила. И только. Колыбель мы не помним, о могиле 
стараемся не вспоминать. Желтый цвет может нравиться, может пугать~--- он все 
равно остается лишь слоем желтой краски. Комната. В комнате светло-зеленные 
обои. Компромисс между глубиной синих плиток и обреченной желтизной стены. 
Всегда есть выбор. Между телевизором и интернетом. Между <<Кривым Зеркалом>> и 
ночной программой <<ТВ-1000>>. Между новым выпуском <<Нашей Раши>> и 
обновленными галереями с <<арбузика>>. Компромисс, выбор\ldots\ Альтернатива. Поход в клуб. 
Но ходить по клубам одному~--- не принято. Одному принято тупо ржать и упиваться 
осознанием тупости юмора и одному принято дрочить, параллельно отмечая 
уродливость моделей, помогающих утолить похоть. Ну и что. Альтернатива все 
равно есть. Спать. Сон снова дает выбор. Кошмары или поллюции. Чисто номинальный: и 
то, и другое от одиночества~--- но он все равно помогает дотянуть до утра. А 
утром на работу и красную клеточку на календаре можно перетянуть на следующее черное 
сочетание цифр. И, вернувшись, поставить <<Алису>>~--- почему нет? В конечном 
счете Кинчев ничем не отличается ни от Моцарта, ни от Кобейна, ни от Меладзе. 
Музыка~--- альтернатива тишине и компромисс с шумом. Всего лишь. 


Двадцать минут в троллейбусе~--- ежедневная вечность блаженства. Пауза между 
двумя бессмыслицами~--- жизнью и работой~--- и отсутствие одиночества: в этой паузе 
ты не один. А потом открываешь офис, заливаешь кофе, выходишь покурить. Над~--- жилая 
многоэтажка, под~--- заплеванный асфальт, впереди забитая монотонно-пестрыми 
авто стоянка, еще впереди~--- стройка, еще не проснувшаяся. И из громоздящихся над 
твоим офисом квартир многоэтажки спускается пара, целуется прямо напротив 
тебя~--- на фоне все также мертвой стройки~--- и расходятся к своим машинам: она~--- к 
серебристому пежо, он~--- к коричневому ауди. И еще три минуты, не смотря друг 
на друга соскребают ночной снег и утренний иней с лобовух. Мелочь. Но, прижав к 
асфальту оранжевый с черной окружностью по краю окурок, ты возвращаешься в офис 
и чувствуешь, что на твоих глазах только что произошло что-то важное. Возможно, 
всего лишь очередная разлука, возможно~--- очередной всплеск ослепительно 
белого цвета эту разлуку предвещающего, но важного: то ли для тех двоих, то ли для 
тебя. И после работы ты пойдешь гулять по полосатому, словно фиолетово-неоновая 
зебра городу, чтобы хоть как-то оттянуть синюю плитку душа и желтые стены 
коридора. И где-то впереди, за серой тенью всех этих зданий со светящимися 
вывесками~--- лето, грозы, радуги\ldots\ А пока~--- черная ночь. Интересно, 
какого цвета небо?..

\newpage

Как-то так. Составляя в уме этот набросок, я пропустил как мы оказались на 
улице. На улице было по-осеннему темное утро. Прямо перед подъездом~--- большая 
лужа. В ней~--- отражение неба, почему-то оно кажется черным, почти ночным. И 
краешек зеленой двери. Я оглянулся. Дверь подъезда выкрашена в коричневую 
краску. Я когда-нибудь это замечал? А дверь подвала? Железная. Просто железная. 
Я посмотрел на Мо.\\
--- Ты сейчас первый раз по коридору шел здесь?\\
--- Да. Если помнишь, мы вчера\ldots\\
--- Конечно--конечно. Какого цвета были стены?\\
Мо задумался. Потом неуверенно ответил.\\
--- Кажется, зеленые. Или желтые\ldots\ Желтые, пожалуй.


Я кивнул. Значит так, да\ldots\ Во двор въехала машина. Второй гольф, любимый 
транспорт шпаны и вышедших из нее бандитов низшего звена. Вроде, не местная, 
хотя не сказал бы, что я хорошо помню, кто из соседей на чем ездит. Не, думаю, 
не местная. Из машины вышли трое. Прилично одетые, но, в конце концов, это не 
девяностые. Зато взгляд у всех троих был правильный. Тупой, но цепкий. И три 
этих взгляда, перекрестным огнем пройдясь по двору, совместились на нас.

Значит, так, да? Так просто. Я пишу текст. Интересно, а зеленая дверь тогда 
откуда? Хотя, знаю. Что ж, логично. Мы переглянулись.\\
--- Manava,~--- Сказал Мо.~--- Сейчас и есть момент силы.\\
--- Не хочешь пройтись со мной по лужам?\\
--- Как скажешь,~--- Мо пожал плечами и зачем-то взял меня за руку,~--- Сказочник.


Мы провалились. Разумеется. Не то чтобы я не удивился, просто\ldots\ Кажется, 
Довлатов писал о том, что оказавшись в мире, полном абсурда, лучшее, что может 
сделать человек~--- принять его. Или не Довлатов. Не помню. Просто нам должен 
был открыться путь в Нижний Мир, он и открылся. Делов--то. Чего тут странного? 
Понятия не имею как отреагировали прибывшие по наши души, если они 
действительно за ними приехали, <<быки>>, видя как над нашими макушками смыкается вода в 
луже. Наверное, долго прыгали по ней, обдавая брызгами друг друга и возможных 
прохожих. Словно детсадовцы. На секунду я очень явственно представил: вот они с 
недоумением смотрят на лужу, бегут к ней, проверяют руками дно, пробуют 
наступить, сначала осторожно, потом в полную силу, наконец, прыгают в нее, все 
трое, скачут на месте, толкаясь, выражения лиц проходят полную гамму~--- 
растерянность, недоумения, ярость, заинтересованность и чистое детское 
неподдельное счастье\ldots\ Я так представил. Значит, наверное, так и было.


Итак, мы провалились. Это напоминало душ, причем тот редкий душ, когда ты не 
платишь за воду, или платишь за нее фиксированную сумму. Напор из открученного 
до предела крана. Сперва отцепился Мо, потом исчезло ощущение одежды, бойлер 
прогрелся и вода становилась теплее, обжигающе теплой, горячей. Я ощущал клубы 
пара вокруг меня, почти видел как пар заполняет\ldots\ Что? Отдернул белую 
душевую занавеску и уперся взглядом в массивный белый же (а каким ему еще быть?) 
унитаз. Никогда не любил совмещенные санузлы. Однако само помещение было просторным~--- 
ни в родительской, ни на съемных квартирах мне в таковых мыться не доводилось. На 
вбитом в стену крючке висело большое махровое полотенце и, рядом, массивный 
плотный синий халат в зеленую полоску. Любимый цвет, любимый размер. Мо нигде 
не было.

Я закрутил кран. Сел на край ванны. И что теперь? Напрашивалось банальное: 
просушиться, завернуться в халат и выйти. Дверь? Дверь есть, на этот раз вполне 
себе белая. Встал, взял полотенце, обтерся. Подошел к двери. Почему-то она мне 
не нравилась. Еще варианты? На стене за унитазом окно. Что за окном~--- не 
видно из-за пара. Мазанул по стеклу полотенцем. Ощущение было каким-то неправильным. 
Провел еще раз. Хм. Тщательно протер оконное стекло полотенцем, стер стекло. И 
что теперь, Боря? А ничего. За окном светило солнце и организованными рядами 
стояли деревья. Я еще раз посмотрел на дверь. Надел халат, впихнул ноги в 
стоявшие на полу банные тапочки и вылез в окно.

Там действительно был сад. Большой, насколько хватала взгляда. Что именно в нем 
растет~--- не скажу, я, конечно, однажды написал произведение о сборщиках 
фруктов, но писал я его исключительно по рассказам Саулюса. Да, еще один человек, с 
которым давно уже не общаюсь.

Сад. Ухоженные деревья, в какую сторону не посмотри. Широкая, тоже, видимо, 
присматриваемая тропинка в двух шагах от меня. За тропинкой, под деревом спиной 
ко мне сидит Мо. Наконец-то.

Я подошел к нему. Мо сосредоточенно смотрел на ствол дерева. Руки, ладонями 
вверх, вытянуты вдоль колен. Из центра ладоней пробиваются какие-то ростки.\\
--- Это что?\\
--- О!~--- Он повернул голову на сто восемьдесят градусов. Я отпрыгнул.~--- 
Здравствуй, мой молодой, но могущественный друг.\\
--- Это\ldots~--- Мо улыбался. Такое ощущение, что противоестественно вывернутая 
голова ни капли его не беспокоит.~--- У тебя шея не болит?\\
--- С чего бы?~--- Мо улыбнулся еще шире.~--- Ты все-таки сделал это. Мы в Нижнем 
Мире, а он живет по своим законам. Ну и по твоим\ldots\\
--- Ну--ну. Но давай я все--таки обойду тебя, ладно\ldots~--- Неуютное это 
ощущение~--- смотреть собеседнику в глаза, и при этом видеть его спину. Я встал\ldots\ В 
любой другой ситуации было бы правильным сказать <<я встал к нему лицом>>, но сейчас 
этот оборот возымел силу только после того, как Мо докрутил головой, совершив 
полный оборот. Улыбнулся мне:\\
--- Так лучше?\\
--- Привычнее\ldots~--- Я опустился на землю рядом с ним. Все это время, не 
считая вращений головой, Мо сидел неподвижно.~--- Так что ты делаешь?\\
--- А, это\ldots~--- Он скосил глазами на свои руки.~--- Здороваюсь с новым 
местом. 

Ростки вдруг напряглись. С заметным усилием синхронно вытянули из плоти свои 
корни, согнулись у основания ствола, взмахнули верхними листьями и спрыгнули с 
ладоней. Ловко семеня корнями, стремительно исчезли в глубине сада.\\
--- Поздоровался?\\
--- В некотором роде. Старая привычка~--- попадая в новый мир, дай знать ему, что 
ты~--- не враг.~--- Мо поднялся.~--- Что в доме? Я, оказавшись тут, осмотрелся. Тебя 
нигде нет. Уже думал искать тебя, а потом смотрю~--- прямо у меня на глазах дом 
вырастает. Значит, из него ты и выйдешь и нечего мне суетиться.\\
--- Не знаю\ldots~--- Я рассказал ему как покидал дом.~--- Ну очень мне та дверь 
не понравилась\ldots

Мо задумчиво покачал плечами. Мигнул, как бывает при не слишком качественном 
видеомонтаже, и оказался уже возле окна.\\
--- Я посмотрю. Ты пока по саду погуляй, ладно?~--- И полез вовнутрь.

По саду так по саду. Если уж мне в этом приключении полагается спутник, и 
спутник многоопытный, то глупо его не слушаться. Через несколько метров 
тропинка выворачивала на вполне широкую разъезженную дорогу. Дорога проходила мимо 
зеленного строительного вагончика и шла дальше~--- между натянутых пластиком 
рядами деревьев. Деревья под пластиком были поменьше тех, мимо которых я 
проходил сейчас и, кажется, я начинал узнавать это место. А я в свое время 
переживал, что не способен увидеть то, что сочиняю~--- никак тогда слова в 
образы в сознании не превращались.

Из глубины сада к тем другим деревьям подходило четверо. Один из них 
подсвечивался. Или даже нет, не подсвечивался~--- просто он выглядел более 
четким, более отчетливым чем другие. Так бывает в мультфильмах, особенно японских, 
когда у создателей нет времени или средств прорисовать сцену во всех подробностях, и 
они тщательно прорисовывают фигуру и движения главного персонажа, только 
намечая все остальные детали и всех остальных персонажей в кадре. Отчетливый передал 
двум другим, они казались моложе чем <<отчетливый>> и четвертый, две палки. Со 
своего места я не видел, но знал, что на концах этих палок прикреплены ножи, 
что эти палки служат для того чтобы срезать шнуры, на которые крепится пластик к 
натянутым с двух сторон от деревьев проволокам. Мало того, я знал, что будет 
дальше.\\
--- Интуиция у тебя работает. Хорошо, что не пошел~--- Я обернулся. Мо стоял у 
меня за спиной и казался сильно встревоженным.\\
--- Что там?\\
--- Ванная комната, дверь. За дверью~--- ванная комната, дверь. За дверью~--- 
ванная комната, дверь\ldots\ Продолжать?\\
Парни вдалеке встали с разных краев ряда. Подняли свои палки вверх. Срезали по 
первому шнурку.\\
--- И почему ты думаешь, что туда не стоило идти?\\
--- Потому что тех дверей, через которые я входил~--- не оставалось. Окна не 
разбивались. Потому что я прошел двадцать проходов и вернулся сюда, только 
воспользовавшись правом спутника\ldots\ Не знаю, может, если бы шел ты~--- 
картина была другая, но скорее всего, я бы продолжал ждать тебя у стены.~--- Ясно.~--- 
Я понимал, что как-то среагировать должен, но понятия не имел <<как?>>. Парни 
сближались.~--- Что такое <<право спутника>>?

Не то чтобы мне было интересно. Почему-то здесь я все воспринимал как данность. 
Наверное, когда вокруг тебя все становится слишком чудесным, то ты просто 
перестаешь удивляться.\\
--- Спутник может в любой момент ощутить странника и переместиться к нему. Это 
похоже на путешествие к известному тебе миру~--- если упрощенно, то ты мысленно 
находишь маяк и тянешься к нему. \\
--- Ясно,~--- Снова повторил я. Парни сближались. Сейчас к ним побежит Сержант, 
сейчас Аура попробует заколоть Артурчика\ldots\ Как давно это было.~--- Мо, ты 
знаешь, где мы сейчас?\\
--- К сожалению, нет. Когда отец проводил нас в физическое Милу, мы оказывались в 
местах, непохожих на это. Это было либо нечто, напоминающее наши джунгли, либо 
какие-то пещеры, выводящие\ldots\ Впрочем, выводить они могли куда угодно. Я 
думал, о том, что это мог бы быть Сад из Верхнего Мира, но\ldots\\
--- Это последние страницы одного из тех двух моих романов, о которых я тебе 
говорил.\\
--- Милое место,~--- Оценил Мо. Я усмехнулся:\\
--- Спасибо. 

Сержант побежал. Где-то на периферии возникла женская фигурка и заспешила в 
сторону разворачивающихся событий. Именно возникла~--- забавно наблюдать как 
визуализуются собственные ляпы. Я окончательно повернулся к Мо.\\
--- Есть идеи о том, что нам тут нужно?\\
Мо задумался.\\
--- Пока нет. Но я склонен считать то, что твой путь в Нижний Мир начинается с 
созданного тобой, добрым знаком.

Добрый знак. Такой и добрый. Я год рассылал этот роман всем известным мне и 
найденным с помощью интернета издательствам: от самых именитых до практически 
никому не известных. Ноль. Я давал ссылку на этот роман у себя в блоге~--- 
ноль, если не считать нескольких откровенно написанных по принципу <<поддержи френда 
своего>> комментариев в духе: <<Прочитать времени нет, но начало интересное>>. 
Некоторые даже обещали прочитать, и это обещание~--- было последним их 
комментарием в моем блоге вплоть до сегодняшнего дня. Читают, наверное.

После этого я перестал пытаться писать серьезно. Накатывающие иногда на меня 
волнами идеи коротких графоманских рассказиков, разливающиеся по берегу блога 
или просто остающиеся лужами вордовских файлов в компьютере~--- не в счет. 
Именно тогда я понял, что хотя чудес не бывает~--- это я понял чуть раньше, проблема 
не в этом. Проблема во мне и отсутствии у меня тех талантов, в которых я был с 
детства уверен. Пережить это было сложно, но, наверное, именно это называется 
взрослением.\\
--- Ну, раз добрый, значит хорошо\ldots~--- Обернуться, посмотреть, что там 
происходит? Зачем?~--- Знак получили, идем дальше. Если я правильно помню, где-то недалеко 
должна быть калитка.

Мо молча последовал за мной. Калитка была, а за калиткой мир менялся. Там где 
мы шли, вплоть до самого решетчатого забора, был теплый солнечный летний день, за 
забором сбоку от нас открывался шведский холмистый пейзаж, ну, или, точнее, то 
каким я представлял его себе когда-то. Впереди, там, где тропинка упиралась в 
калитку, за забором была ночь, в небе видны были звезды, калитка открывалась на 
широкую асфальтированную дорогу. Дорога вела через степь или пустыню~--- в 
темноте не поймешь. Как будто кто-то взял два совершенно разных рисунка и склеил между 
собой. Мы стояли перед калиткой.\\
--- Пойдем,~--- Позволив мне изучить открывающуюся картину, спросил Мо. На него 
самого, кажется, резкая смена пейзажа никакого впечатления не произвела. А мне 
все-таки сделалось жутко. Как-то в фильмах оно обычно не так выглядит. Ты идешь 
по какому-то месту, оно кажется однородным, ничто не предвещает резких перемен, 
пересекаешь какую-то черту и вдруг оказываешься в другом месте. Это\ldots\ Чуть 
не сказал <<логично>>, это привычно и, наверное, если бы с той стороны, где сейчас 
стоим мы, мы наблюдали продолжение <<шведского>> пейзажа, а, закрыв калитку за 
собой, оказались в другом месте~--- я воспринял это как должное. Но когда ты 
видишь черту за которой мир резко и ничем необоснованно меняется~--- это 
пугает. Я решил быть, или, во всяком случае, казаться рассудительным:\\
--- И все-таки~--- какую цель мы преследуем? Ну, кроме как спрятаться от 
любителей лотерейных выигрышей? И, если брать более конкретно, чем <<спасти мир>>? В 
смысле~--- именно тут нам что нужно сделать?\\
--- Дойти,~--- Мо посмотрел на меня, вздохнул и встал удобнее. Суток, которые я 
был знаком с ним, хватало чтобы понять: сейчас будет лекция. С другой стороны, я 
никуда не тороплюсь.\\
--- Я бы посоветовал тебе, в первую очередь, не думать о <<спасении мира>>. Не в 
этом твоя цель. Во всяком случае, не в этом твоя цель сейчас. Это, если хочешь, 
моя цель и в движении к ней я нашел тебя, потому что ты можешь помочь мне в ее 
достижении. Даже не так~--- ты можешь достичь ее для меня. Для меня.~--- Мо 
сделал ударение на последнем слове.~--- Конечно, в результате это пойдет на пользу 
всем живущим здесь, это пойдет на пользу самому миру. Конечно, ты сможешь гордиться 
своим деянием и гордиться по праву~--- это будет великий подвиг. И, конечно, ты 
его совершишь. Но этот подвиг станет твоей целью тогда, когда ты разберешься с 
собой. До тех пор, твоя цель~--- ты сам. Тебе надо понять кто ты, тебе надо 
принять себя. Принять, что нет ничего страшного в том, что ты не можешь хорошо 
делать некоторые вещи~--- это, если я прав, тебя гнетет очень сильно. Не 
можешь~--- 
значит, либо научишься, либо это не твое. У тебя другие таланты. Таланты, 
поверь 
мне, гораздо более значимые, я бы сказал~--- другого порядка. Заставь их 
раскрыться. Тебе нужно понять, что нет ничего плохого в том, что люди уходят из 
твоей жизни. Это нормально~--- у них есть своя. У тебя, кстати, тоже. И ты 
вполне 
можешь воспользоваться своей жизнью так, чтобы она стала жизнью и всех 
остальных. Поддержка других людей, помощь человечеству~--- тоже может быть 
целью, 
но и она может быть ею тогда, когда ты разобрался с собой. До этого ничего не 
получится. И не потому, что у тебя непременно не хватит сил. Может и хватит. 
Просто, для того чтобы достичь цели нужно действительно хотеть достичь ее. А 
пока существо не разберется со своими проблемами, пока существо не станет тем, 
кем оно в первоначальном плане должно быть~--- все остальные проблемы для него 
не 
будут настолько значимыми. Пока ты не разберешься с собой, ты не сможешь спасти 
чью-то чужую жизнь. В лучшем случае, тебе удастся отсрочить чью-то смерть, а 
это, согласись, не то\ldots


Мо замолк. <<Моук замолк>>~--- промелькнуло на краю сознания и я машинально 
улыбнулся. Мо заметил мою улыбку.\\
--- Извини за анализ. Для того чтобы понять переживания молодого члена разумного 
общества, запертого в остановившемся мирке, мне даже думать сильно не надо. Оно 
в любом случае упирается в <<Меня никто не любит>> и у <<Меня ничего не 
получается>>. Это нормально. Только у тебя времени нет через все это проходить 
так, как твои сородичи через это проходят. К тому же все указывает на 
неэффективность их метода. Так вот. Обычно в Нижний Мир спускаются для того 
чтобы пройти какой-то путь, справиться с какими-то препятствиями и получить 
какой-то предмет. Когда найдешь то, что тебе нужно~--- почувствуешь. Отец, 
правда, 
чаще всего знал куда идти, к кому и зачем, но и он, бывало, блуждал здесь, лишь 
смутно представляя цель. Полчаса назад я бы сказал, что по большому счету 
неважно в каком направлении идти. Но после твоей двери~--- я не уверен. Во 
всяком 
случае, даже случай в доме, указывает, что ты вполне способен находить 
направление сам. Кстати, сюда нас тоже ты вывел,~--- Он показал рукой на ночной 
пейзаж.\\
--- И еще тебе стоит помнить вот что. Этот мир~--- и неважно как в него 
попадать~--- в какой-то степени подчиняется страннику. В какой~--- всякий раз можно узнать 
лишь попробовав. Но тебе-то должен подчиняться даже внешний мир~--- тот, откуда ты 
пришел. Так что\ldots
Мо выразительно развел руками. Я медленно кивнул. Значит, подчиняется. 
Посмотрел 
на ноги~--- тапочки медленно, не слишком охотно расплылись и вокруг стоп 
неторопливо вырастали удобные фирменные кроссовки. На первом курсе о таких 
мечтал, не получилось. Посомневавшись, преобразовал халат в спортивный костюм. 
Костюмчик получился без ярлыка или эмблемы какого-то бренда, но удобный. 
Сойдет. 
Последний раз обернулся на сад. Людей отсюда видно не было, зато в отдалении, 
возле дома хозяев, привлекали глаз две явно лишние здесь и точно не мною 
придуманные раскидистые пальмы.\\
--- Твоя работа?\\
Мо проследил за моим взглядом. Улыбнулся:\\
--- Прижились\ldots\ Считай, мир принял наш поклон. Ну так что?\\
--- Ну так что?\ldots~--- Передразнил я.~--- Значит, нам нужно пойти туда, не 
знаем куда, найти то, не знаем что, но это что-то должно мне помочь понять как я могу стать 
величайшим волшебником нашего техногенного мира. Все ясно. Пошли. 

Я открыл калитку и шагнул в зону ночи. Мо медлил. Я обернулся~--- мой спутник, 
решившись, перепрыгнул через ограду, в прыжке превращаясь в птицу. Стройную 
птицу пестрого оперения с короткими широкими согнутыми крыльями.\\
--- Это кто?\\
--- Ястреб,~--- ответил ястреб и стремительно взмыл вверх. Вернулся.~--- Как мне 
всего этого не хватало, Сказочник, ты бы знал.\\
Он летел чуть выше моей головы, мы двигались вдоль дороги. Ночь дышала зноем.\\
--- Странно, я думал ты не заметил того времени, что провел расщепленным. \\
--- Я и не заметил.~--- Согласно кивнул клювом он.~--- Я говорю про дни после 
моего возвращения. Их было немного, конечно, но это страшно~--- понимать, что твои 
возможности взаимодействовать с миром на девяносто процентов заблокированы. Ты, 
конечно, понимаешь, что это не совсем так, что просто мир обездвижен и 
законсервирован, а твои способности никуда не делись. Но понимаешь ты это 
головой, а ощущения такие как были бы, допустим, у тебя если бы ты вдруг ослеп. 
И пусть тебе говорят, что это ненадолго, это эксперимент и на самом деле зрение 
осталось при тебе~--- легче тебе от этого не будет. –

Он вдруг перевернулся и словно лег в воздухе на спину. Расправил крылья и 
взорвался залпом разноцветных огней. Огни заиграли в небе новогодними салютами, 
взрываясь, распадаясь на новые огоньки других цветов, падая на дорогу. 
Последний из упавших на землю огней стал Мо.\\
--- Впечатляюще,~--- Сказал я ему.~--- Я тоже так могу?\\
--- Не знаю, попробуй\ldots~--- Он улыбнулся.~--- И, кстати, там через несколько 
метров дорога сворачивает и заканчивается.\\
--- Как заканчивается?\\
--- Обрывом.

Мы прошли эти несколько метров. Все было как сказал Мо~--- за поворотом дорога 
упиралась в обрыв. Только я не знаю, можно ли это называть обрывом: дорога, а 
вместе с ней земля просто кончались за поворотом, словно кто-то разрезал или, 
учитывая некоторую неровность края, разломал мир, в который мы попали и выкинул 
ненужный кусок. Мы стояли на краю бездны. Я перегнулся через край~--- внизу 
была 
темнота. Ни бликов света на воде, ни журчания реки, ни каких-нибудь костров, 
сигнализирующих о наличии дна. Ничего. Просто тьма. Впереди, насколько хватает 
глаз~--- ночной воздух и никакой твердой поверхности. Я сел на землю и свесил 
ноги 
за край. Когда еще удастся так посидеть, болтая ногами, на краю мира?\\
--- И что будешь делать, Сказочник?~--- Мо присел рядом.~--- Пойдем через 
пустыню? Я 
покачал головой. Впереди нас светила большая красноватая полная луна, над нами 
узором бисера в небе рассыпались звезды. Наверное, стоило бы сказать, что 
рисунок на небе был незнакомым, звезды были чужими и что-то еще в таком духе, 
но 
на самом деле из всех ночных созвездий я только Большую Медведицу в состоянии 
узнать. И то не наверняка. \\
--- Мне кажется, нам надо вперед.~--- Я ответил.~--- Может, пришло время мне 
научиться в птицу превращаться?\\
--- Попробуй,~--- равнодушно отозвался Мо. Я посмотрел на него:\\
--- Что не так?\\
--- Понимаешь, я верю, что ты можешь научиться оборачиваться птицей. Почему нет? 
Но ты не умеешь. А Нижний Мир, во всяком случае так всегда считалось, не служит 
для того чтобы научиться чему-то новому. Он помогает человеку научиться 
эффективно пользоваться тем, что человек и сам может. С другой 
стороны\ldots~--- Мо пожал плечами.~--- Попробуй.\\
--- И попробую.~--- Я встал.~--- Как ты это делаешь?\\
--- Просто меняю форму. 

Замечательно. Сначала меня подписывают совершать чудеса и подвиги, а потом 
обижаются, что я не понимаю правила этого их Нижнего мира и отказываются 
помогать. Я вздохнул.\\
--- Хорошо, с чего мне начать?\\
--- Не знаю\ldots\ Я же говорю, я просто становлюсь тем, кем мне надо. Можешь 
попробовать прыгнуть и помахать руками, не знаю.

Я так и сделал. Безрезультатно. Мо смотрел на меня с плохо скрытой усмешкой. Я 
улыбнулся в ответ~--- улыбка, по плану, должна была получиться широкой и 
открытой~--- и шагнул в пропасть.

Падал я долго или, во всяком случае, так мне казалось. Было темно, 
по-настоящему темно, и я в полной мере ощутил то, что пытался рассказать мне мой спутник. Я 
не видел ничего~--- ни того, что находится по сторонам, ни того, что ждет меня 
внизу, ни самого себя. Я ничего не видел и очень хотел стать птицей. Или полететь 
просто так. Или стать птицей. Внезапно подумал, что разрываясь между двумя 
желаниями, я уменьшаю интенсивность каждого из них. Нужно определиться. Ладно, 
началось все с птицы, пусть она и будет. Я очень хотел стать птицей. 
Почувствовать себя ею. Обратиться в нее. Превратиться. В птицу. Птица. Птица 
божия не знает\ldots\ Певчие птицы, так радостно-больно мне наблюдать\ldots\\
Птица, птица, птица.\\
Плюх.\\
Я приземлился прямо на то место, где сидел до того как начал свои попытки. 
Приземлился прямо на кобчик. Больно. Черт, больно. Больно-больно-больно.

Мо долго наблюдал как я, пытаясь унять боль, прыгаю вокруг него. Наконец, 
произнес:\\
--- Классный прикид.

Я остановился. Осмотрел себя. Мой миленький спортивный костюм превратился в 
нечто более подобающее вождю индейского племени, чем жителю столичного города, 
пусть и на отдыхе. На мне было оперение. Точнее, непонятного фасона накидка, 
обшитая перьями. Я потрогал голову~--- волосы, лоб, нос\ldots\\
--- Ну, хоть клюва нет.~--- Сел рядом с Мо. Помолчали.\\
--- Кстати, о птичках\ldots\ Просить тебя перенести меня через пропасть тоже было 
бы против правил?\\
--- Почему же?~--- Мо оживился.~--- Это уже идея. Спутник для того и нужен чтобы 
помогать тебе на пути. Только\ldots\ Только я, пожалуй, попробую слетать на 
разведку~--- ибо, я, конечно, практически всемогущ, но силам даже самого всемогущего 
ястреба есть предел. Не хотелось бы рухнуть с высоты птичьего полета в 
неизвестность.
Он перекинулся и полетел. Я наблюдал за его полетом. Пестрая птица стремительно 
удаляется вперед, превращаясь в темную точку на черном фоне, точку еле 
различимую, точку, неразличимую совсем. Я прищурился. Тьма, просеиваясь сквозь 
ресницы, распадалась на однородную ткань ночи и вплетенные в нее серебряные и 
золотые нити ночных светил. Интересно, в нормальном мире так же? Когда ты 
последний раз вдумчиво глядел на ночь, Боря? Давно\ldots\ Впрочем, даже в 
расслоившемся пейзаже я не смог найти птицу до тех пор, пока она не вынырнула 
из тьмы, возвращаясь. Ястреб очень забавно маневрировал между лучами света, словно 
избегая их. Интересно. Я раскрыл глаза: темная точка на черном фоне. Прикрыл: 
разрезающие тьму лезвия света и птица, пробирающаяся через препятствия в небе.

Мо вернулся. Птицей, как был, присел на край обрыва. Вполне по-человечески 
тяжело отдышался.\\
--- Ох, Сказочник. Сам решай. Я конца пропасти не нашел~--- возможно, его нет, 
возможно, он появится только тогда, когда ты до него дойдешь. Возможно. А еще 
может быть, что я не донесу тебя до этого благословенного момента и мы рухнем в 
пропасть. И, возможно, опять окажемся здесь, возможно, приземлимся там, где нам 
и надо, возможно\ldots\ Не знаю. Но попробовать можем.\\
--- Можем,~--- Кивнул я. Честно говоря, я не слишком внимательно его слушал. Игра 
с разными способами смотреть на мир увлекла меня. Я прищуривался и шевелил 
головой, сдвигая лучи света, мешая их между собой. В какой-то момент они 
сплелись в тугой толстый уходящий в небо канат. Параллельно канату с двух боков 
серебрилась натянутая леска~--- своеобразные перила. Я встал. Мо вспорхнул 
следом.\\
--- Что надумал?\\
--- Раз уж летать у меня не получается,~--- Я осторожно поставил ногу на канат. 
Нога почувствовала опору.~--- то придется нам с тобой ходить.

Однако, я не канатоходец. Стоило переставить и вторую ногу на мой воздушный 
путь, как я чуть не рухнул. Помогли <<перила>>, правда, при этом, ладони мне 
порезали до крови. Ладно, понемногу, мы никуда не торопимся. Ястреб Мо присел 
мне на плечо.\\
--- Не понимаю как и куда ты движешься,~--- Пояснил.~--- Боюсь потеряться.\\
--- А что ты видишь?~--- Я боялся, что мне придется постоянно поддерживать вид 
моей <<дорожки>> в сознании, но как только я ступил на нее, она стала существовать 
самостоятельно. Я мог отвлечься на разговор.\\
--- То же самое~--- небо, звезды, ночь, бескрайняя пустота\ldots\ Ты видишь 
что-то другое?\\
--- Ага,~--- Я кивнул, аккуратно переступая. Ястреб на плече не помогал 
сохранению 
равновесия.~--- Сейчас мы с тобой перебираемся по канату из света. В 
основном~--- 
лунного, но с вплетением и лучей звезд. Очень неудобно, надо сказать, нам с 
тобой по нему передвигаться\ldots~--- Я снова ухватился за боковую леску. 
Больно. 
Ястреб вспорхнул.\\
--- Да, неудобная поддержка.\\
Я замер:\\
--- Ты же не видишь, вроде?\\
--- Ты сказал~--- я увидел.~--- Теперь ястреб кружил вокруг меня, чуть выше 
уровня перил.~--- Если что, можешь за меня хвататься. Выдержу.\\
--- Спасибо\ldots~--- Значит, вот так. Я сказал~--- он увидел. 
Интересно\ldots~--- Интересно, а почему мир нижний, а идем мы вверх?\\
--- Нижний мир~--- только название, причем даже не оно само, а не самый точный 
перевод. Milu~--- это Milu, и раз уж ты оказался в нем, то в каком направлении 
ты идешь, уже не так важно. Ты уже в нем. А любой мир, по большому счету, является 
бесконечностью, если смотреть на него изнутри. Во всяком случае, если правильно 
смотреть.\\
--- Ну да?~--- Мы медленно, но верно продвигались вперед. Звезды, паче чаяния, 
ближе 
не становились, но край, с которого началось наше восхождение, удалялся вполне 
уверенно.\\
--- Ну да. Возьмем, к примеру, твой мир. Во-первых, есть физическая реальность, 
такая, как ее воспринимают обитатели мира. То есть, ваша планета с диаметром в 
районе тринадцати тысяч километров, Луна, на которой представители вашего вида 
бывали, планеты Солнечной Системы, вы их вроде как исследуете, само Солнце, в 
реальности которого никто не сомневается~--- и абстрактный для всех вас космос. 
Где заканчивается он?\\
--- А он относится к нашему миру? Я после твоего рассказа думал, что такие как ты 
как раз по планетам внутри космоса перемещаются\ldots\\
--- Восхитительно.~--- Ястреб присвистнул.~--- Согласно исследованиям ваших же 
ученых, 
для того чтобы достигнуть какого-нибудь не самого далеко тела за пределами 
вашей 
системы, двигаясь при этом с предельной для физических законов вашего мира 
скоростью, требуется несколько тысячелетий. Да, я столько прожил. Но ты 
действительно считаешь, что проводить столько времени в перелетах~--- это то 
времяпровождение, из-за которого мне моя жизнь нравится? Нет, звездное небо над 
вашими головами~--- это ваш мир. Просто вы его не используете. Раз 
бесконечность, 
согласен.\\
Я кивнул.\\
--- Только это слишком абстрактная бесконечность, она не убеждает. Не знаю, что 
там из своих заоблачных далей думаете вы, но я~--- землянин, и мой мир~--- 
маленькая голубая планета. Все остальное~--- так, красивое оформление.\\
--- Допустим.~--- В голосе Мо слышалось удовлетворение, он явно ожидал подобных 
слов от меня.~--- Допустим. Поэтому, космос ваш~--- это раз--бесконечность. 
Спускаемся на землю. Земля у нас~--- маленькая голубая планета, ты сказал. Откуда ты знаешь, 
что она голубая? С фотографий с орбиты? Ты сам-то там бывал? Для тебя Земля~--- это 
твой город и еще пара-тройка мест, при чем и они тебе досконально неизвестны. 
Что находится за их пределами~--- ты не знаешь. Знаешь названия, в голове у 
тебя есть смутные представления о том, что за этими названиями скрывается, но\ldots\ 
Можешь ли ты с уверенностью сказать, что Папуа--Новая Гвинея существует? Ты не 
знаешь этого, и Земля для тебя~--- бесконечна, ибо, скорее всего, за свою 
короткую жизнь ты так и не соберешься постичь ее. Это у нас~--- бесконечность--два.\\
--- Что, еще и три есть?\\
На мой скепсис ястреб не обратил внимания.\\
--- Конечно. И три, и четыре, и пять. До сих пор мы говорили только о мире, более 
менее, вашим видом познанном. Смотрим дальше. Даже на вашей планете остается 
океан, в который вы разве что заглянули, остается то, что происходит за корой 
Земли, ниже которой вы так и не опустились. Три-четыре. А теперь вспомни~--- 
это все реальность вашими глазами. А теперь представь вашу же реальность с точки 
зрения микроорганизма. С точки зрения микроба, клетки, атома, кварка, 
преона\ldots\ Ты способен это представить?\\
Я хмыкнул. Вспомнились старые размышления, времен работы на складе.\\
--- Работа склада напоминает работу системы пищеварения. Мы принимаем материал, 
сортируем его, расфасовываем, выкидываем попадающий к нам вместе с ним 
мусор~--- и распределяем все остальное среди дальнейших получателей. Почему бы не 
представить, что мы~--- живущие в чьем-то организмы микробы? Разве что они не 
идут, отпахав восемь часов, домой. Или идут?\\
--- Или идут,~--- Ястреб кивнул.~--- Просто с твоей точки зрения этого не 
возможно увидеть. А возможно, на том уровне жизнь вообще протекает иначе. Или, наоборот, 
кажется, что она протекает одним способом, но с какого-то другого ракурса это 
будет выглядеть иначе. Почему бы не представить, что все движение вашего мира, 
с точки зрения мира другого~--- это всего лишь полет слепленного пятилетним 
ребенком снежка? А может оказаться, что жизнь организма этого ребенка, это третий мир, в 
который, допустим ты и попал, прыгнув в своем мире в лужу\ldots\ Все это очень 
запутанно, но одно можно сказать наверняка: даже если не брать в расчет 
множество миров за пределами твоего мира, даже твой мир~--- является 
бесконечным множеством таких миров.


Я задумался. В принципе, в том, что говорил Мо не было ничего не слишком 
нового, не слишком оригинального. Чем, с другой стороны, не доказательство того, что в 
мире вообще не слишком много новых слов? В моем мире, во всяком случае. В мире, 
где не бывает чудес, а когда они все-таки свершаются~--- они воспринимаются как 
фон для еще одного бессмысленного разговора о вечном.\\
--- А еще есть бесконечный мир возможного, мир не случившегося, мир реализующегося 
и того, что остается нереализованным\ldots\ Пусть будет так. Ты скажи, тебе не 
кажется странным, что мы здесь, вместо того чтобы продираться сквозь опасности 
этого места, продираемся сквозь твои мудрствования?\\
Ястреб заклокотал.\\
--- Пройти по лунной тропе~--- это ерунда, это у нас за опасность не 
считается\ldots\ 
Если серьезно,~--- он успокоился,~--- то хороший вопрос. Возможно, это значит, 
что 
на своем пути ты замечаешь только мир, облеченный в слова, возможно, тебе стоит 
задуматься об этом. Возможно\ldots\\
--- Вниз посмотри.~--- Я уже понял, что когда Мо начинает играть в 
<<возможности>>, 
игра затягивается и не приносит пользы. Внизу, между тем, мерцали звезды и 
подсвечивала луна. Золотая тропа, каковой стал канат, уводила от звезд~--- 
куда-то 
выше, в темноту.~--- Что скажешь?\\
--- А что скажешь ты?~--- Вопросом на вопрос ответил ястреб, снова устраиваясь у 
меня на плече.~--- Куда мы сейчас движемся, вверх или вниз?\\
Я молча пошел вперед. Ястреб на плече нахохлился и пробурчал:\\
--- Какой ты умный, прям, страшно становится\ldots


Ничего не происходило. Ничего странного~--- пришло мне на ум и заставило меня 
улыбнуться. Да уж, действительно. Мы прыгнули в лужу, оказались на последних 
страницах произведения, которое я сам когда-то написал, оттуда попали в 
какую-то 
ночную пустыню, и теперь поднимаемся вниз или спускаемся вверх, короче, 
движемся 
по дороге из лунного света. Ничего странного, ни капельки. Ничего страшного~--- 
вот, что на самом деле казалось мне странным. Ничего страшного до сих пор не 
произошло, а я ожидал именно страшного. Возможно, это говорит что-то о 
особенностях моего восприятия, к этому, кажется, вел Мо. Итак, чего я боюсь? 
Высоты и потери ориентации в пространстве? Вроде бы, нет~--- с тех пор как наша 
дорога поменяла форму, став на самом деле чем-то вроде дороги, идти по ней не 
вызывает у меня никаких трудностей. Даже то, что небо под ногами и все 
удаляется 
от нас~--- любопытно, но не более. Бесконечности? Раз бесконечность, два 
бесконечность, три бесконечность, пять бесконечность\ldots\ А я четвертую 
сорвал. Не 
знаю. В том виде, в котором ее или их понимает мой спутник~--- нет, не боюсь. В 
том виде, в котором она нас сейчас окружает\ldots\ Я огляделся. Да, возможно 
это 
тоже бесконечность, бесконечность пустоты. Боюсь я ее? Нет, пожалуй, хотя сама 
ее идея мне не нравится. Итак, мне не нравится, но меня не пугает пустота. 
Пустота~--- это одиночество, так? Отсутствие чего-то важного, так? Почему она 
меня 
не пугает? Я оставался один, на самом деле, после того как я остался один 
несколько лет назад~--- ничего так и не изменилось. Новых людей в моей жизни не 
появилось, старые практически все остались на другой стороне. Было? Было. Я 
потерял веру в себя, пришел к выводу, что любой успех в моем случае~--- это 
чудо, 
и потерял веру в чудеса заодно. Все это было печально. Первые пару лет этого 
периода я действительно страдал так, как может страдать двадцатитрехлетний 
юноша. А он, кстати, может страдать так, как не сможет ни один тридцати или 
сорокалетний. У него обида на мир еще не вошла в привычку. Итак, я знаю, что 
такое пустота и все равно не боюсь ее. Почему? Что я делал, что я продолжаю 
делать? Работа и все связанное с необходимостью выжить~--- раз. Пиво~--- два. 
Графомания и всякая форумная активность~--- три. Этого хватает? На полноценную 
замену не тянет, наверное, но заполнить пустоту все-таки удается. Да, я не 
боюсь пустоты, потому что не верю в ее бесконечность и безграничие~--- ее всегда 
можно заполнить.\\
--- Да будет\ldots~--- Я запнулся. Кричать <<Да будет свет!>> как-то слишком 
претенциозно, правда?~--- Да будет что-нибудь!

Сначала появились звуки. Что среднее между гулом проводов с высоким напряжением 
и тихим наигрыванием какого-то джазового мотивчика на бас-гитаре. Мелодичный 
гул пустоты. Потом к нему присоединился саксофон ветра, потом появился и сам ветер. 
Потом ветер усилился. Потом он снес нас на несколько метров назад и нес бы 
дальше, если бы Мо не вцепился когтями мне в плечо, а клюв сразмаху не опустил 
бы в тропинку. В общем, мы развевались на ветру, удерживаемые лишь вбитым почти 
до основания в землю (или как это назвать?) клювом ястреба.\\
--- Замечательное что-нибудь,~--- Услышал я вдруг язвительный голос спутника. 
Выгнул шею~--- клюв по-прежнему торчал в поверхности, к клюву прилагалась голова с 
закрытыми глазами, от головы отходила широкая короткая шея. А вот в том месте, 
где шея срасталась с туловищем, из туловища выходила еще одна широкая мощная 
шея и венчала эту шею еще одна голова, клюв которой и издал сейчас неожиданный для 
меня комментарий. Наверное так выглядят говоруны. Или как-то похоже.~--- 
Полюбовался? Что теперь делать будем?\\
--- Идти против ветра, видимо\ldots\ Только не знаю каким именно способом.\\
--- Ну, думай. Только учти, что я не обещаю того, что смогу нас продержать так 
слишком долго, ладно?


Ну да, легко сказать: <<Думай>>. Подумаешь тут, когда в плечо тебе впились 
когти хищной птицы, а сам ты подобно тряпке трепыхаешься в полуметре от земли. В 
ногах, пожалуй, и все полтора будут. Впрочем, спутнику моему приходилось не 
слаще, понимая это, я промолчал и попытался думать. Ветер, ветер, ты 
могуч\ldots\ Что нам ветер да на это ответит\ldots\ Иди против ветра, на месте не 
стой\ldots\ До чего дошел прогресс\ldots\ А вот это мне нравится. Прогресс~--- это хорошо. 
Прогресс~--- это двигатель человечества. А нам надо двигаться вперед, шагать, смело шагать\ldots\
Шагай смелее, провинциалка\ldots

Тропинка поехала. Сама. Превратилась в что-то вроде движущейся дорожки, ведущей 
с первого этажа супермаркета на второй, и поехала. Ветер не давал двигаться нам 
самим~--- мы и не двигались. Волоклись, влекомые нашим транспортом, намертво 
прикрепившись к нему одним из клювов моего двухголового спутника. Второй клюв 
раскрылся в одобрительном клокоте:\\
--- Хорошо\ldots\\
--- А то,~--- Согласился. По бокам дорожки выросли поручни, ухватился за 
ближайший. \\
- Если хочешь, можешь меня отпустить.\\

Ястреб разжал когти. Только сейчас я почувствовал насколько мне больно. В ответ 
на мой крик ястреб криво улыбнулся.\\
--- Ерунда. Подозреваю, что в сравнении с тем, как Берс Тюрьму уничтожал~--- 
вообще все ерунда, а уж пара птичьих когтей в плечо\ldots\\
--- Тебя бы так\ldots~--- Буркнул я.~--- Что за Берс, что за Тюрьма?


Впервые за то время, что я его знал, Мо после предложения <<рассказать что-то>> 
задумался. Почесал клювом под крылом, вздохнул.\\
--- Сложно рассказывать о Берсе. Берс~--- это Берс. Среди всей бесконечности 
видов разумной жизни в бесконечном множестве миров, Берс, судя по всему, единственный 
по-настоящему уникальный. Сам себе вид. Возможно, среди первых странников и 
первых Хранителей сыщутся другие уникальные, те, кого просто породил Хаос, но 
это не гипотеза даже, легенда. Лично я ни с одним из древних не знаком. Знаком 
с Берсом. Является ли он прямым порождением Хаоса? Не знаю. И он сам не знает. Он 
знает о себе только то, что воспитывал его народ метеоров, с ними он 
путешествовал долгое время, они научили его путешествовать по мирам. Но он не 
метеор, метеоры просто подобрали его в своих странствиях. Об этом ему много 
позже сказал сам вечный вождь этого странного народа. Опять же, Берс~--- чуть 
ли не единственный из <<не--метеоров>>, кто по-настоящему может с ними 
договориться. 
Сам Берс обладает устоявшимся твердым физическим антропоморфным телом, однако 
так же может существовать в форме газа. Сам он затрудняется ответить какая 
форма для него первична, но, учитывая, что в моменты усталости и нехватки сил он 
часто становился газовой субстанцией или сочетал в своем облике плоть с газом~--- я 
склонен считать, что газообразная форма~--- первична. И еще одна, пожалуй, 
самая удивительная его особенность~--- его нельзя убивать.

Убить можно, убивать нельзя. В момент насильственной смерти, включая 
самоубийство, он просто уничтожает мир, в котором он в этот момент находится и 
всех оказавшихся в этом мире <<гостей>>. Это выяснил Орген, Хранитель мира Орс, 
когда отразил попытку Берса перехватить управление миром, Берс там структуру 
неба менял\ldots\ Не знаю каким способом Орген это выяснил, и сам Берс не знает~--- 
он вообще очень смутно помнит, что с ним было с того момента как его поймали и до 
того, как он очнулся в Тюрьме. Подозреваю, что Хранители просто чувствуют такие 
вещи, чувствуют то, что угрожает сохранности их мира. Во всяком случае, по 
смутным воспоминанием Берса, Орген советовался с другими Хранителями. 
Небывалый, как нам казалось, случай~--- Хранители, как и странники, предпочитают 
действовать поодиночке. Пожалуй, своеобразная неуязвимость Берса сыграла с ним злую 
шутку~--- уже тогда поймать Берса живьем было сложно, убить~--- легче. Но убивать его 
было нельзя. Поэтому его ловили, опять же~--- небывалый случай~--- силами нескольких 
Хранителей, поймали и поместили в Тюрьму. По словам Берса~--- лучше бы убили.

Тюрьма~--- мир-ссылка для пойманных Хранителями странников. Тайный мир~--- 
раньше его считали мифом, ибо никому из странников не удалось его найти и никому из 
странников не удалось с него выбраться. Никому кроме Берса, но и он провел там 
немало времени. По его рассказам, Тюрьма~--- это мир-планета, причем планета 
постоянно меняющаяся. Одним утром выходишь и видишь перед собой пустыню, другой 
день~--- леса, гуще и зеленее чем на моих родных островах. Иногда там годами 
царила зима, иногда каждый день~--- новое время года. И там нет никого кроме 
странников. Даже надзирателей нет~--- просто, просыпаясь, странники делают 
первый вздох и знают, что им нужно сегодня сделать. И что пока они этого не 
сделают~--- сегодня в Тюрьме не кончится.

В общем, Тюрьма~--- она и есть Тюрьма. Страшной и бесконечно жестокой делает ее 
не труд. Страшно для странников, в первую очередь~--- отсутствие свободы. 
Представь, что ты поймал дикого льва, повелителя джунглей и закрыл его на заднем дворе 
своего дома. А представь, что ты закрыл его там не одного, а вместе с многими 
ему подобными? Страшно. Но даже это еще не все. С Тюрьмы не просто нельзя было 
спрыгнуть в другой мир, в самой Тюрьме нельзя было пользоваться музыкой сфер 
мира. Хуже чем нельзя~--- способности странников блокировались почти полностью, 
им оставлялась возможность следить за своим здоровьем, за своим питанием и внешним 
видом. Представь, что ты птица, у которой отобрали крылья, разрешив взамен 
сколько угодно кататься в фуникулере. Вот это действительно страшно.

А еще там воняет. Когда Берс рассказывал мне, что вонь там доставляет 
практически такой же дискомфорт как невозможность использовать свои силы~--- я 
не поверил. Потом сам убедился~--- там действительно Воняет. С большой буквы.

Берс выбрался оттуда, став, в одночасье легендой. Выбрался и решил, что такого 
места быть не должно и что, создав такое место, Хранители преступили границы. 
Он объявил войну Хранителям и пообещал уничтожить Тюрьму. Вот только даже сбежав с 
нее на Кладру, он потом не мог найти обратной дороги. Мы долго блуждали по 
мирам, воевали с отдельными Хранителями и побеждали их, но к Тюрьме не 
приближались. Пока не вспомнили о еще одной легенде, легенде о том, что у 
Коридоров между мирами тоже есть свой Хранитель. Харон.

Мы медленно двигались вперед. Рассмотреть, что там, в этом впереди, было 
невозможно, от ветра слезились глаза. Помолчав, ястреб продолжил:\\
--- Вообще, наше путешествие достойно лучшего рассказчика. Я~--- хороший 
рассказчик, но я умею хорошо преподносить теорию и факты, а тут нужен сказитель. Мы не 
нашли Харона, зато мы нашли легендарный атрибут его вызова~--- навлоны. Они оказались 
не совсем тем, что мы ожидали найти. По легенде, по тому как нам удалось 
реконструировать ее, навлоны~--- специальные монеты, которые нужно положить на 
глаза и произнести слова призыва, тогда явится Харон и переведет призвавшего 
его в другой мир. А оказались они длиннющими гвоздями с вечно раскаленными ножками. 
Хотя шляпки их действительно напоминали монетки. Сам понимаешь, призывать 
Харона, пока оставались другие надежды, расхотелось. Навлоны Берс оставил при 
себе и до поры до времени о Хранителе Коридоров мы забыли.

Потом Берс нашел Тюрьму. Как~--- известно ему одному, во всяком случае, было 
известно ему одному тогда, когда я последний раз его видел. Тюрьма оказалась не 
совсем обычным миром и вели к нему не совсем обычные дороги. Берс их нашел, 
привел туда нас и там Хранители дали нам решающий бой. Бой, который мы~--- а 
нас тогда уже было немало, да и узники Тюрьмы с радостью встали под наши 
знамена~--- безнадежно проигрывали. Даже прорвав защиту Тюрьмы, научившись приходить и 
уходить оттуда, мы не смогли воспользоваться на ней нашими способностями. А без 
них\ldots\ Что такое человека против бога? И тогда Берс дал приказ уходить и 
уводить с собой узников. Когда странников, во всяком случае~--- живых странников, в мире 
не осталось~--- он призвал Харона. И удивительно~--- легенда сработала. Точнее 
так~--- он умер и легенда оказалась правдой. Берс умер, уничтожив своей смертью Тюрьму и 
всех собравшихся на ней Хранителей, и к Берсу пришел Харон и увел его в свои 
миры. Правда мы, долгое время, знали только первую часть этого. К моменту, 
когда оказалось, что Берс, погибнув, остался живым, мы успели оплакать его. А 
порадоваться его воскрешению~--- не успели, нужно было продолжать начатое. 
Продолжать превращать бесконечное множество миров, потерявших своих Хранителей, 
в упорядоченную вселенную. Успев лишь обняться с Берсом, я отправился сюда~--- 
и тут уже надолго выпал из событий. Все, что я успел услышать от него~--- это то, 
что, протыкая себе глаза навлонами, он чуть ли не впервые понял, что такое 
настоящая боль. А ты говоришь~--- когти\ldots\\
Мо замолчал.\\
--- По-моему, навлон древние греки клали в рот покойнику,~--- Через какое-то 
время сказал я.~--- А медяки на глаза~--- это вроде бы из похоронных обрядов славян. 
Но перевозчика душ усопших у древних греков тоже звали Харон.\\
--- Я в курсе. Вообще, память о Хароне сохранилась, в разных формах, конечно, как 
минимум в 247 мирах. И в 139 из них он фигурирует под схожими именами. Мы 
серьезно подошли к поиску.

Мы молчали. Мо, по-видимому, погрузился в воспоминания о своих приключениях, а 
я думал о том, что сравнение с мазохистским поступком этого Берса мне мало 
помогает. Плечо саднило.\\
--- Тебе не кажется, что ветер стихает?~--- Наконец, спросил Мо. Наша движущаяся 
дорожка вывезла нас на улицу. Такую\ldots\ Совершенно типичную улицу 
современного вечернего города~--- и поэтому странную улицу. Словно кто-то задался целью 
создать панораму абстрактного города, намешав в нее элементы разных городов. Советского 
типа пятиэтажки с тремя-пятью подъездами чередовались со сдвоенными 
двухэтажными домиками английского типа, но не было ни советского двора с забитой под 
завязкой автостоянкой и большим квадратом ни под что не используемой земли с желтым 
пятном песочницы посередине, ни узкой английской дороги с домиками-отражениями 
на другой стороне. Наша дорожка разрослась в широкую реку асфальта, по одной 
стороне которой стояли эклектичные постройки, по другую, параллельно, тянулась 
бетонированная спортивная площадка. Как полагается в американских фильмах, 
площадка была обнесена металлической решеткой, за решеткой подростки 
кучковались вокруг металлических урн и жгли мусор, в свете пламени на заднем фоне 
поблескивали толстые звенья заменявших баскетбольные сеточки цепочек. По другую 
сторону дороги молодежь мусор не жгла, но тоже присутствовала. Устроившись на 
крыльцах у подъездов, небольшие, человек по пять-семь компании, где на троих 
парней приходилась одна девушка пили из бутылок и банок пиво и слабоалкогольные 
коктейли, гоготали и, время от времени, косились на компании неподалеку. Мы с 
Мо стояли в начале улицы и изучали открывшийся нам вид.\\
--- Пошли?~--- Спросил Мо, но его вопрос прозвучал откуда-то издалека. Снаружи. Я 
же был внутри и смотрел на город, который тоже внутри меня, потому что я всю свою 
жизнь прожил внутри него. Прожил, стараясь не ходить вечерами по спальным 
районам, где у дверей подъездов сидят компании, одуревающие от безделья и 
осознания, что их мир~--- их двор давно изучен и приручен, тесен для них, но 
расширить его они не могут. Потому что нет бесконечности и за углом дома~--- 
следующий дом, точно такой же, следующий двор, точно такой же, вот только место 
у подъезда занято другими, точно такими же как они~--- точно так же изнывающими 
от скуки и ограниченности собственного мирка, точно такими же озлобленными на 
чужаков, тех, кто посмел вторгнуться в их мир, прошагать мимо их подъезда по их 
тротуару, когда они не смеют позволить себе шагнуть с него. И единственный 
способ расширить свой мир~--- глотать дрянные химические коктейли, пить пиво, 
из крепких\ldots\ Передавать по кругу косяк~--- этим, я знал, занимаются сейчас 
другие, те, что сгрудились у мусорок. Потому что любая другая бесконечность пугает, потому 
что в любой другой бесконечности~--- ты чужак. А ты знаешь, что чужаков никто 
не любит потому что ты сам не любишь их. Потому что в детстве нас всех успели 
обмануть рассказами о большой жизни, жизни насыщенной, полной свершений, 
приключений, чудес и смысла. А потом оказалось, что жизнь заканчивается 
крыльцом у подъезда, двором у подъезда и придатком двора в виде завода или стройки. И 
больше нет ничего, потому что за углом~--- другие подъезды с другими дворами и 
другими заводами. Другими, но такими же. Но в любом чужаке слышится скрытая 
насмешка, любой чужак вызывает подозрение~--- а вдруг у него иначе? Вот, он 
пересек черту и куда-то или откуда-то идет. Вдруг там~--- по-другому. Вдруг он 
счастливее?\\
--- Пацанчик, есть закурить?\\
--- Слушай, чувак, телефон разрядился, а мне сестре позвонить надо\ldots\\
--- Ты с какого района, слышь?\\
--- Брателло, на пиво не поможешь?\\
--- Ты вообще что за черт?

Нас заметили. Все и в один момент. Точнее, не нас~--- меня. Мо оставался 
снаружи, в бесконечности бесконечных миров. От тех, у кого исчезла вера в бесконечность, 
он был надежно защищен своим знанием о ней. А я? Я не успел ответить ни на один 
из заданных мне вопросов. Валяясь на асфальте, закрывая лицо от пинков, я 
усмехнулся про себя: во всяком случае, я объединил их. Признав во мне чужака, 
они все, хоть на время, стали одной компанией. Компанией несчастных людей, 
уверенных, что чудес не бывает и жизнь~--- это то и только то, что они видят со 
своих мест у крыльца. Всего лишь ночь назад я и сам был таким. Ну, почти таким. 
Таким же несчастным. А сейчас я уже знаю, что все не так, что чудеса бывают. 
Или\ldots\\
--- А ведь они почти правы, Мо.~--- Не отнимая рук от лица, заговорил я, ни мало 
не заботясь, услышит ли меня спутник. Впрочем, хотя это и не было важным, я был 
уверен~--- услышит.~--- Чудес действительно не бывает, потому что в мире, где 
возможно все~--- ничто нельзя с полным правом назвать чудом. Они просто не 
понимают, что все реально, но это же совсем другое дело, правда?

Возможно все~--- и мальчик-выпускник, всю жизнь учившийся на <<хорошо>> и 
<<отлично>> из-за любви к больной маме и не допущенный к экзаменам потому что сын 
директрисы при нем назвал его маму <<старой коровой>>, мальчик, который сейчас пинает 
меня, представляя, что снова пинает этого жирного ублюдка и его мамашу в придачу и, 
и, и~--- подаст жалобу, добьется пересмотра директорского решения, сдаст, в конце 
концов, экзамены на следующий год~--- и уедет учиться дальше, получится красный 
диплом, найдет хорошую работу, отправит маму лечиться в хорошую клинику. Потом 
откроет свою фирму и возьмет туда друзей с крыльца своего детства, тем, кому 
повезло меньше. Все возможно.

Возможно все~--- и парней, в промежутках между косяками, гоняющих мяч по 
площадке: один на один, три на три, пять на пять~--- заметят скауты. Предложат стипендии 
одним, контракты в клубе третьего-пятого дивизиона другим, напророчат всем 
блестящее свободное будущее, и даже в случае нескольких из них не ошибутся. Все 
возможно.

Возможно все~--- и отработавший первую неделю в своей жизни юнец, привыкший 
верховодить в своей дворовой компании, а тут попавший под пренебрежительные 
насмешки и не менее пренебрежительные приказы: <<Сгоняй за водкой>> старших 
<<опытных>> строителей, обнаружит со временем, что его лидерский талант никуда 
не делся даже среди <<чужих>> и получит к концу года предложение встать 
бригадиром. Все возможно.

Возможно все~--- и толстая девушка лет семнадцати, в прыщах и без макияжа, 
пинающая меня наравне с пацанами, вслух гордящаяся, что пацаны в компании 
принимают ее как настоящего <<своего>> чувака, а по ночам плачущая в подушку, 
вдруг, за полгода резко похудеет, нальется женственностью и красотой, отвергнет 
приставания <<своих>> пацанов и однажды на улице к ней, смущаясь и чуть 
заикаясь от смущения, подойдет по-настоящему ее парень\ldots\ Все возможно.

Возможно все~--- и кто-то напишет и даже издаст книгу.

Все возможно~--- и кто-то, стесняющийся сейчас говорить, что идет домой раньше 
потому что нужно готовиться к завтрашнему семинару, излечит рак.

Возможно все~--- и мелкий прилипала, получающий от старших жирные бычки и 
остатки пива за изображение учителей, соседок и, иногда, мелькающих по телевизору 
политиков, однажды будет стоять на сцене, сжимая в руке статуэтку Оскара.\\
--- Дядя, ты чего? Тебе плохо? Ты умрешь?\\
--- Все возможно,~--- Согласился я, пытаясь усесться на асфальте. Надо мной стоял 
восьмилетний мальчишка~--- белокурый и взволнованный. Взволнованный тем, что 
какому-то постороннему дяде плохо и он может умереть. Улица опустела и, 
почему-то, из вечерней превратилась в утреннюю.\\
--- Возможно все,~--- Повторил я, медленно вставая.~--- Но, думаю, умру я еще не 
сегодня. Спасибо, малыш.\\
Я медленно побрел вперед, к ближайшей скамейке на которой, прикрыв глаза, 
восседал в своем человеческом обличии Мо. Я присел рядом.\\
--- Что ты там говорил о функциях спутника? Помогать в преодолении пути, как-то 
так, да?\\
--- А что?~--- Мо, не открывая глаз, зевнул.~--- По-моему, ты и сам неплохо 
справился. Так что, пошли?\\
--- Погоди, дай очухаться\ldots~--- Побили меня знатно. Тело ломило.\\
Мо, наконец, открыл глаза и удивленно посмотрел на меня:\\
--- Ты дурак, Сказочник? Или до сих пор притворяешься?\\
А, ну да. Я же~--- Сказочник, а это, к тому же, Нижний Мир. Я закрыл глаза, 
глубоко вдохнул, дал кислороду разойтись по телу и выдохнул. Бодро встал:\\
--- Ну, пошли раз уж так.

И мы пошли. Улица оказалось не такой длинной как выглядела в вечернем 
освещении. Или, может быть, просто от этой улицы мы получили то, что должны были. В любом 
случае, уже через пару минут мы вышли на перекресток. Указателей не было, не 
было даже былинного камня с традиционной надписью. Было три дороги: левая 
уходила в туман, правая уходила в туман, центральная, как ни странно, уходила в 
туман. Мо ждал моего решения. Я плюнул.\\
--- А, какая разница!? Идем куда-нибудь!\\
Мо стоял и, улыбаясь, смотрел на меня. Я повторно плюнул.\\
--- Надоело уже, понимаешь? По-моему, сказка как-то затягивается, и, при этом, 
тянется слишком нудно, чтобы ее длину можно было оправдать. На мое 
<<творчество>> мы посмотрели, в бездну прыгали, по небу прошли, по ребрам получили. С 
устройством мира параллельно разобрались, спасибо. Что еще? Найти нам что-то 
надо? Так идем, найдем и хватит уже. Надоело.

Я шагнул вперед, но второго шага не сделал. Надо найти? Надо. Хорошо, 
примите-получите. Я обернулся назад. Давешний кучерявый мальчик одиноко возился 
в песочнице. Оставив перекресток за спиной, я прошел к мальчику.\\
--- Будь другом, одолжи лопатку.

Прокопав на три совка вглубь, я достал из ямки металлический ключ. 
Интересный~--- вместо обычных закруглений или квадратов ножка резьбы заканчивалась завитушкой, 
незавершенной восьмеркой уходящей назад. Ключ напоминал\ldots\ Мне 
потребовалось несколько мгновений~--- да, ключ был выполнен в форме ключа, только ключа 
скрипичного.\\
--- Доволен?~--- Вернув ребенку совок, я показал ключ Мо.~--- А теперь~--- мы 
возвращаемся.

Мо повертел ключ в руке, глубокомысленно хмыкнул и пошел к тому месту, куда 
вывела нас наша изначально лунная тропа. Я представил весь путь назад и 
решительно окрикнул спутника.\\
--- Э, нет, возвращаемся мы другим путем.

Я осмотрелся. У меня есть ключ, тут есть дома. Квест из банальных. Перешел 
дорогу и подошел к ближайшему из домиков английского типа. Ключ не подошел. Не 
подошел он и к следующему. И к следующему\ldots\ Через два часа, когда я 
перепробовал все английские дома и задумчиво рассматривал двери подъездов советским 
многоэтажек, Мо согласно кивнул:\\
--- Да, ты прав, Сказочник. Так проще\ldots\\
--- Тебя забыли спросить\ldots~--- Буркнул я и дернул ручку двери одного из 
подъездов. 

Дверь оказалась закрытой, ни замочной скважины, ни висячего замка на двери не 
было. То же самое и с другими подъездами. В каком-то смысле мне повезло~--- в 
каждом подъезде по пять этажей и, скорее всего, не меньше трех квартир на 
площадке. Да, легче просто двери подъездов подергать.\\
--- Возвращаемся?~--- Спросил Мо, когда я закончил свои манипуляции. 

Возвращаться~--- в смысле именно <<идти назад>>~--- не хотелось. И теперь уже 
даже не из-за качества и продолжительности дороги вперед, а просто~--- не хотелось 
соглашаться с мудрым, вечно правым Мо. Бодро пропев: <<Ни шагу назад, только 
вперед\ldots>> я пошел к перекрестку. Не останавливаясь на развилке, прошел 
прямо. Туман оказался густым и липким, словно не туман даже, а\ldots\\
--- Наверное так чувствует себя муха, оказавшись в цветке росянки. Или кролик 
внутри удава\ldots\\
За моей спиной хмыкнул Мо. Потом соизволил произнести:\\
--- Сказочник, никогда не делай сравнения, в которых ты не уверен. Быть 
разъедаемым заживо чьим-нибудь пищеварительным соком~--- это ни на что не 
похожее удовольствие. И уж меньше всего оно похоже на проход через жалкие газообразные 
сгустки какой-то хрени, как выразился бы наш Шаман\ldots\\
--- Доводилось?~--- Поинтересовался я. Вопрос прозвучал глухо~--- оказалось, что 
если не прикрывать рот рукой, то взамен слов в него залетают эти самые <<сгустки>>. 
Вкус <<хрени>> был рассчитан разве что на гурманов, то есть, проще говоря, был 
омерзительным. Мо закашлялся и, думаю, по схожим причинам, ответил так же 
глухо, коротко и, как следствие, неопределенно.\\
--- В жизни всегда есть место подвигу.

Через пару минут клейкой тишины туман выплюнул нас прямо к ограждающему 
яблоневый сад забору. Точнее даже не <<выплюнул>>: сама ограда находилась еще в 
тумане, и мы ее не нашли, а на нее наткнулись. Я наткнулся. В самом буквальном 
смысле, можно даже сказать~--- приложился. Но стоило нащупать задвижку, открыть 
калитку и переступить за нее~--- туман остался позади, а мы обнаружили, что 
находимся если не в точке начала нашего путешествия, то в месте весьма на нее 
похожем. Впрочем, нет~--- все-таки в том самом месте. Слева от нас, радостно 
здороваясь с нами, активно закачались две пальмы.~--- Прижились\ldots~--- 
Ласково улыбнулся им Мо.~--- Здравствуйте, мои хорошие.

По саду прошли без приключений. Я достаточно быстро сориентировался в созданном 
когда-то моею же фантазией лабиринте деревьев, а встретить нас, пожалуй, что 
никто и не мог~--- последняя происходившая в этом антураже сцена романа была 
отыграна на наших глазах, когда мы впервые попали сюда. Возле дома 
остановились. 
Не сговариваясь, обошли окно душа, завернули за угол, еще раз за угол~--- и 
встали возле входной двери. Переглянулись. Я дернул дверь. Она оказалась заперта и мы 
оба облегченно выдохнули.\\
--- Пробую?\\
--- Пробуй\ldots\\
Вставил ключ в скважину. Повернул. Не сразу, но замок поддался. Я обернулся и 
поймал ободряющую улыбку Мо.\\
--- Ни шагу назад?~--- Спросил он. Я кивнул.\\
--- Только вперед.\\
\\
Мы вошли.


\customsection{Путь Мауи}{melkij\_bes}{ALOHA~--- любить~--- значит быть счастливыми вместе}

На верхнем этаже открылась дверь центральной квартиры. Вверху мягкой обивки 
мигал золотом металлический номер <<42>>, но имел ли он какое-то отношение к 
действительности? Люди, вышедшие из-за двери, этого не знали. Вообще, если кто 
и мог ответить на этот вопрос, то разве что кто-то из куривших на лестничной 
клетке, на пол-пролета ниже ожившей квартиры. И то\ldots\ Наверняка ничего об 
этом не знал стоявший, прижавшись спиной к стене, самый молодой из курильщиков~--- ибо 
сам он, по виду сорокалетний чуть грузный мужчина с властным взглядом, обычно 
попадал в коридор подъезда с нижнего этажа. Сам он при этом подозревал, что ни 
номер, ни расположение квартиры значения не имеют, просто древняя зловредность 
некоторых заставляет любую дверь из которой не вышел бы он оказаться на 
несколько этажей ниже верхней площадки. Словом, хотя без сомнения, с первого 
взгляда на мужчину становилось ясно, что рассказать он может многое и ответить 
тоже на многое~--- за этим ответом обращаться стоило явно не к нему. Вряд ли 
можно было дождаться вразумительного ответа от сидящего на лестнице спиной к вышедшим 
лысого облаченного в халат коротышки. Он, скорее всего, буркнул бы что-нибудь в 
духе, что сам он тут в гостях, и вообще, когда развешивали таблички на квартиры 
– его очередной раз почему-то забыли спросить. Сами, конечно, ни раз уже об 
этом пожалели, но разве эти смешные гордецы признаются в этом\ldots\ А единственный 
из компетентных обитателей лестничной площадки, сидящий между окошком и видавшим 
виды, вынесенным в подъезд по причине того, что свой срок в качестве домашней 
утвари он уже явно отслужил, столиком на не менее потрепанном стуле крепкий 
старик с седой шевелюрой, но черной как смоль бородой вообще не любил отвечать 
на вопросы. Вместо ответа, впрочем, на так и не прозвучавший и, возможно, в 
голову никому не пришедший вопрос, он приветственно кивнул новоприбывшим:\\
--- Проходите, не стойте в дверях. Дайте посмотреть-то на вас уже\ldots

Словно по команде или, скорее, по многократно отрепетированному сценарию, к 
пришедшим обернулся сидящий на лестнице. Посмотрел, прищурившись, скривился, 
пожал плечами:\\
--- Нашел кем любоваться. Мало того, что три дня ими в окошко любуемся, а у 
мириневров, между прочим, финальная серия кубка, так еще сейчас вживую их 
интервьюировать будете. Лучше бы у меня интервью взяли\ldots\ Да что у меня~--- 
кто угодно вам больше интересного расскажет\ldots

А третий~--- как уже говорилось в меру молодой, в меру грузный мужчина молчал. 
Говорили его глаза, вступившие практически с первых секунд в непрерывный диалог 
со взглядом старшего из пришедших. Не отводя глаз, Мо сделал первый шаг вниз.\\
--- Что, Берс,~--- Одновременно с движением заговорил он.~--- Как на Кладре 
бедокурить, команду собрать, войнушку против всех устроить~--- так Моук 
поможет, Моук поддержит\ldots\ А как старина Мо в мире под расход списанном застрял~--- 
так мы вмешиваться не можем? У нас обязательства? Мы большие и важные? Что же 
вмешались-то?

Последний вопрос прозвучал, когда Мо оказался вплотную перед Берсом. Он прошел 
мимо сидящего Сееры, не заметив его, не повернулся на устроившегося в углу 
Харона и сейчас стоял рядом с Берсом. Ждал. Ответа? Ответа Берс не дал, вместо 
этого он раскрыл объятья:\\
--- Рад тебя видеть, старина. Живым и невредимым.\\
Подождав еще немного, Мо хмыкнул в ответ:\\
--- Взаимно, дружище!\\
Харон, полюбовавшись объятиями старых друзей, обернулся к продолжавшему стоять 
наверху Борису:\\
--- Спускайся и ты, Сказочник. Раз уж зашел\ldots\\
--- Да уж, сто лет, образно говоря, живем, а такое чудо впервые видим\ldots~--- 
Прокомментировал, так и не оборачиваясь, Сеера. \\
--- Воистину,~--- Кивнул Харон и обратился уже к Мо.~--- Ты, кстати, странник, 
неправ~--- мы и не вмешивались. Сказочник сам к нам зашел. А раз сам~--- то и мы чисты 
перед соглашением. К тому же, кто знает, может разговор с нами~--- тоже часть 
его пути по этому вашему Милу\ldots\\
--- А вы, собственно, кто?~--- Борис, спустившись к остальным, решился вступить в 
разговор. Получилось плохо и он сам понял это~--- слишком нарочито независимо 
прозвучал вопрос, слишком читались скрываемые под ним растерянность и 
неуверенность. На помощь пришел Мо.\\
--- Наш хозяин,~--- Он безошибочно указал на Харона,~--- по-видимому никто иной 
как Хранитель переходов Харон, я тебе о нем рассказывал. Этот,~--- Он ткнул локтем 
Берса в живот.~--- Когда-то мой друг и даже ученик, когда-то мой предводитель и 
командир, когда-то Черный Властелин, а ныне, по дошедшим до меня слухам, 
Хранитель Вселенной Берс. А кто вы, бесспорно уважаемый,~--- Мо слегка 
поклонился Сеере,~--- я могу только догадываться.\\
--- Харона он, видите ли, узнает, ни разу не виданного, а обо мне только 
догадывается\ldots~--- Пробурчал Сеера.~--- Нет, все-таки не ценят меня. Ну, 
потешь меня своими рассуждениями\ldots\\
--- Да, рассуждений на самом деле и нет,~--- Мо, как бы извиняясь, 
развел руками.~–-- 
Одна интуиция, по большей части. Когда-то давно мы решили допустить, что 
Хранитель Коридоров, он же~--- Хранитель Пространства реален. Нам было очень 
нужно, чтобы он оказался реальным и мы принялись искать его. Нашли~--- пусть не 
совсем его, но некоторые следы его существования и даже участия в жизни 
некоторых миров. И теперь я вижу его перед собой, что, на мой скромный взгляд, 
окончательно доказывает его реальность. А если реален он, то, соответственно, 
нет причин не допускать реальности и других таких же легендарных, таких же 
основополагающих Хранителей. А среди таких я знаю только одного\ldots~--- Мо 
задумался 
и поправился.~--- Впрочем, как знаю. Натыкался на упоминания о нем во время 
поисков Харона. И то, упоминания слишком малочисленные и мало отличимые просто 
от легенд и мифов народов. Но опять же~--- интуиция подсказывает мне, что вы, 
глубокоуважаемый~--- Хранитель Времени, известный в разных мирах и легендах как 
Бхага, Крон, Сеера\ldots\ Простите, о вас действительно мало известно и я 
понятия не имею~--- есть ли среди этих имен то, что соответствует действительности\ldots\\
--- Сеера\ldots~--- Сеера удовлетворенно кивнул.~--- А так да, не ценят меня, тут 
ты прав. И не ценили никогда.~--- Он повернулся к Харону.~--- Слышал, да~--- интуиция 
ему подсказывает. Интуиция!.. Так при мне собственную задницу никто не называл\ldots\\
Харон расхохотался. Поддержал его, по принципу <<сам пошутил~--- сам 
посмеялся>> и Сеера. Берс промолчал. Борис брезгливо, как человек, 
считающий юмор ниже пояса~--– юмором ниже плинтуса, пожал плечами. Мо почесал переносицу и задумчиво кивнул.\\
--- Спасибо. \\
--- Заметил, да?~--- Сеера неожиданно оборвал смех и посмотрел на Моука уже 
внимательнее. Кивнул в ответ.~--- Ну и молодец. Может и не зря тебя Берс ценит.\\
--- Не зря, не зря. Я вообще лапочка. Но все-таки\ldots\\
--- Все-таки, да. Мальчик прав. Пора разговоры разговаривать,~--- Харон потянулся 
на стуле и повернулся к Борису.~--- Расскажи, Сказочник, как добрался, что видел 
по дороге, что дальше делать думаешь?\\
Борис задумчиво сжал губы. Выдохнул. Поморщился. Мо, глядя на то как готовится 
к ответу спутник, покачал головой.\\
--- Эх вы, Хранители. Кто же так гостей встречает и кто же так разговоры ведет?\\
--- А как надо?~--- Харон скептически сдвинул брови. Мо оживился.\\
--- Таких гостей как мы надо встречать по высшему разряду. Лично я считаю, что 
встреча вообще должна происходить в парадной зале, сидеть мы должны на удобных 
креслах, обитых кожей кирубля, под ноги или на ноги нам должны быть положены 
шкуры красного тигрозверя\ldots\ Но, насколько я понимаю, звать нас в 
какой-нибудь <<внутрь>> хозяин не собирается?~--- Куда уж внутреннее\ldots~--- Улыбаясь, 
покачал головой Харон. Мо кивнул.\\
--- Так я и думал. Тогда, простите, экспромт-организация посиделок на выезде не 
наш конек, так что и уровень сервиса оставляет желать лучшего\ldots~--- Мо 
широко развел руками. Лестничный пролет преобразился. Площадка расширилась, превращая 
ведущие на верхний этаж ступеньки в почти отвесную лестницу, старенький столик 
разросся и вместе с размером приобрел солидность и вид крепкого дубового стола 
для компании <<6--8 человек>>. Сам стол оказался сервирован на пять персон, а 
вокруг него выросло пять стульев, явно из одного со столом гарнитура. Один из 
этих стульев вырос прямо под Хароном, заменив собой привычную хозяину мебель. 
Харон отпил из ближайшего к нему кубка.\\
--- Недурно\ldots\ Но невежливо вмешиваться в устройство чужого мира, тем более 
прямо на глазах его Хранителя.\\
Мо уселся за одно из свободных мест. Кивнул.\\
--- Вот и я говорю~--- невежливо вмешиваться в устройство чужого мира прямо на 
глазах его хозяина. А тем более невежливо брать на себя право решать имеет ли 
хозяин право на устройство своего мира, простите за тавтологию. \\
--- Ты к чему?~--- Борис не до конца, а точнее~--- совсем не понимал, что тут 
сейчас происходит, но в последних словах Мо крылся толстый намек на то, что происходит 
нечто не слишком приятное. А поскольку среди всех собравшихся знаком Борис был 
только с Мо, к тому же именно Мо втянул его во всю эту историю и сам навязался 
в спутники--союзники, казалось логичным считать, что и сейчас потомок забытого 
гавайского бога выступает на его, Бориса, стороне. Если, конечно, Борис 
правильно понял и стороны разделились. Мо поднял кубок и перед тем как 
пригубить наклонил в сторону Харона.\\
--- Пусть дяденьки расскажут. А мы пока перекусим. Не знаю как ты, а лично я за 
наше путешествие проголодался зверски.

Проголодался и Борис. После секундной нерешительности он обвел глазами тройку 
Хранителей, пожал плечами, и, стараясь сохранить независимый вид, сел за стол. 
Впрочем, сел рядом с Мо. За их спинами расхохотался Берс.\\
--- Мо, знал бы ты, как я соскучился по твоему хлебосольству! Давайте, отметим 
встречу!~--- Он грузно опустился на стул и поднял кубок. Добавил уже без 
смеха.~--- Не беспокойся. Мы посидим, поговорим и вы оба уйдете из этого подъезда. Слово.\\
--- Вопрос только в том~--- куда\ldots~--- Буркнул, занимая свое место за столом, 
Сеера.

Борис ждал пояснений или требующего пояснений вопроса от Мо. Мо молчал. То ли 
старшему спутнику все было ясно, то ли своим молчанием он давал понять 
младшему, что слово за ним.\\
--- В смысле <<куда>>?~--- Наконец спросил сам Борис.\\
--- Тут много дверей, ты, наверное, заметил. И раз уж ты оказался здесь~--- 
совсем не обязательно тебе выбирать дверь, ведущую домой. Давайте, я скажу 
тост\ldots~--- Харон поднял кубок, подержал его на весу и поставил на стол.~--- Не самый 
короткий тост, но под ситуацию\ldots\ Издревле все сущее~--- это хаос. Частицы хаоса~--- 
мелкие, мельчайшие частицы. Хаос~--- это вечное движение, вечная смена форм, вечная 
динамика~--- и время от времени в какой-нибудь из его областей частицы 
выстраиваются в нечто, наделенное собственной структурой. Иногда собственная 
структура очередной из форм хаоса допускает собственную эволюцию или хотя бы 
самоконтроль, и в таких случаях мы говорим о возникновении мира из Хаоса. Но на 
самом деле и этот мир~--- Хаос, просто Хаос, принявший на какое-то время вид 
упорядоченной системы. А любая система конечна~--- и в пространстве, и, 
спросите Сееру, во времени. И рано или поздно мозаика переразложится.
Так было\ldots\ И я не могу сказать <<есть и будет>>, ибо однажды~--- 
достаточно давно по некоторым меркам, но на самом деле совсем недавно~--- в миры хаоса пришел 
Черный Властелин. Пришел~--- и закрепил их, и связал из бывших элементов хаоса 
упорядоченную вселенной. Структуру, теперь уже в противовес Хаосу. Но Хаос 
продолжает жить и время от времени порождает новые миры~--- и часть из них 
присоединяются к упорядоченной вселенной. И вселенная расширяется. И, если она 
будет лишь расширяться, то когда-нибудь~--- не через десять веков и не через 
сто, но все здесь собравшиеся имеют шанс дожить до тех времен~--- когда-нибудь нам 
придется вспомнить, что любая система конечна. И в пространстве, и во времени.~--- 
Харон перевел дыхание и поднял кубок.~--- Так выпьем же за то, чтобы всего 
этого не произошло!

Харон щедро отпил. Двое хранителей последовали его примеру. Чуть помедлив, с 
выражением лица <<В чужой монастырь со своим уставом не ходят>>, сделал глоток 
Сказочник. Последним пригубил кубок Мо.\\
--- Это все красиво,~--- Сказал он.~--- Но при чем тут мы и наш мир? Вы приняли 
соглашение и соблюдаете его, наша борьба~--- наше дело.\\
На этот раз ответил Берс.\\
--- Понимаешь, Моук, это даже не соглашение~--- это скорее правило. Знаешь, как 
правила игры~--- на самом деле ничего не стоит их нарушить, но если их 
нарушать, то игра просто рассыплется. А так~--- да, мы приняли. И Кинич с Рафой тоже 
считают, что если мы не будем вмешиваться~--- формально мы соблюдем правила. Но 
мы не знаем, так ли это на самом деле, и как к нашем невмешательству отнесется 
Хаос. Просто не знаем~--- раньше мы не сталкивались с такой ситуацией. А 
рисковать\ldots\ Думай, что хочешь, но рисковать~--- страшно. Потому что сейчас 
мы рискуем не собственными жизнями как раньше, а жизнью каждого из обитателей 
каждого из миров и даже больше. Мы рискуем всем сущим. Есть у нас такое право?

Несколько долгих секунд Мо пристально смотрел на Берса. Потом грустно улыбнулся 
и принялся за еду. Промолчал.\\
--- И что вы предлагаете?~--- Глухо спросил Борис.\\
--- Вам не обязательно возвращаться. Раз вы самостоятельно покинули свой мир, 
так, что вам мешает войти в другие двери? Их тут много\ldots\ И поверь мне, Борис, 
странствие по мирам~--- наверное, самое захватывающее приключение во вселенной. 
А не веришь~--- спроси у своего наставника, не думаю, что он станет врать. Мы бы 
не предлагали вам этого, если бы вас действительно что-то держало в вашем мире. 
Но, Мо~--- ты называешь этот мир своим, а разве он имеет что-то общее с миром, 
сотворенным Мауи? Кто вообще помнит Мауи в том мире? А ты, Борис? Сказочник, ты 
прошел инициацию, ощутил свои силы и возможности. А теперь подумай~--- они 
нужны тебе в твоем мире? Зачем? Для кого? Разве есть в твоей жизни кто-то ради кого 
можно и нужно совершать чудеса, менять мир вопреки всему? А если нет~--- то не 
интереснее ли открыть неизведанное, встать на путь приключений и волшебства? На 
путь, не омраченный горечью обид, недовольства, осознания того, что чтобы ты ни 
делал~--- твой мир все равно неправильный? Просто потому что он такой 
есть\ldots\ Ты действительно хочешь продолжать разочаровываться в нем~--- день за днем, час за 
часом?\\
Борис покачал головой:\\
--- Мой мир~--- странный мир, тут вы правы. Хотя, пусть я так говорю и так 
считаю, но на самом деле мне не с чем сравнивать. Других миров я не видел. Впрочем, 
разрешите мне на правах гостя произнести ответный тост\ldots~--- Борис взял 
свой кубок, встал со стула, прошелся вдоль стола.~--- Это было давно. Для меня~--- давно. 
Вы же, наверное, могли бы сказать, что это было не далее как вчера. По вашим меркам. Я 
заканчивал обучение в университете, жил с девушкой, с которой мы были вместе с 
первого курса. Мы снимали квартиру вместе с семейной парой наших друзей, к нам 
часто заходили университетские друзья, или просто~--- друзья-приятели, которыми 
мы обросли за пять лет жизни и учебы в Вильнюсе. Жизнь была прекрасна и 
безоблачна. Мы, как я уже говорил, заканчивали учебу, снимали квартиру~--- и естественно, 
как любые молодые амбициозные студенты из провинции, не хотели зависеть от 
родителей. Девушка моя работала в ночную смену официанткой в одном вильнюсском 
казино, я работал на скользящем графике на крупном оптовом складе.
Работа моя заключалась в подготовке заказов. Выглядело это примерно так: я беру 
лист, на котором обозначено сколько чего нужно отобрать, хожу по складу, 
собираю все нужное в фирменные коробки, складываю коробки на поддон. Потом, когда заказ 
отобран, другой человек подъезжает с погрузчиком и увозит поддон уже в зону 
отправки. По правилам эти поддоны перед транспортировкой еще обматывать нужно, 
но обычно этого не делают~--- работать надо быстро, время тратить некогда, да и 
падать хорошо составленные коробки не должны. Не должны, но иногда падают. Один 
мною собранный поддон однажды упал, когда его погрузчиком подцепили. Может, я 
где коробки внизу плохо сложил, может, просто дернул его мужик слишком 
резко~--- не знаю. Смысл в том, что поддон поехал в одну сторону, коробки с него~--- в 
другую. И поехали коробки прямо на проходящую мимо работницу~--- тоже 
подрабатывающую студентку, к слову, к тому же беременную. Ее этими коробками 
кинуло прямо животом на стеллажи, и еще сверху накрыло. С девушкой составили 
акт о том, что произошедшее~--- несчастный случай и отвезли ее в больницу. А мы 
продолжили работать. Потом, когда я рассказал об этом дома~--- мне долго 
объясняли, что я-то точно ни в чем не виноват, да и вообще, все обойдется. Не 
обошлось. Через два дня узнали: у девушки выкидыш и детей она, скорее всего, 
иметь не сможет. И снова мне объясняли, что никто не виноват, а я не виноват 
тем более, Его Величество Случай\ldots\ А я их слушал~--- и понимал, что вот, в 
жизни другого, постороннего нам всем человека случилась необратимая трагедия, одним 
из косвенных виновников которой являюсь я~--- и ничего. Никого из моих друзей это 
не волнует, потому что я~--- вот он я, друг, сожитель, соарендатор, 
любимый\ldots\ Как можно признать, что я виновен в трагедии другого человека? Ведь, если признать, 
то нужно что-то менять в отношении ко мне, потому что признать, что твой 
близкий виноват в чем-то необратимом и продолжать считать его близким~--- это сложно, 
это почти чудо. Легче, особенно когда необратимое касается кого-то другого, просто 
не признавать. Они и не признавали. Это было забавно: они предавали меня, 
оправдывая. Потому что получалось, что никакой трагедии нет, что человек~--- не 
просто человек, я~--- ничего не может и все в руках случая, в руках, царящего в 
мире хаоса. Потому что получалось, что даже лучшие люди, а я считаю, что мои 
близкие~--- лучшие люди по умолчанию, выбирают равнодушие к миру для того чтобы 
сохранить душевный уют собственного мирка. И, что самое страшное~--- чудес не 
бывает: коробки упали, друзья оправдали меня, а я остался один со своим 
ощущением необратимости. Один в странном мире. Мире, где чужая смерть мало, что 
значит, и нужно согласиться с этим, если хочешь сохранить друзей. В мире, где 
человек не может никому по-настоящему помочь, а потому и чужая жизнь значит 
очень мало. Наверное, я тогда слишком бурно отреагировал на всю ситуацию, не 
знаю. Через какое-то время я отдалился от них и меня не очень-то и пытались 
вернуть: слишком я всех к тому моменту достал своими переживаниями и попытками 
объяснить, что так нельзя. Так вот, к тосту\ldots\ Мой мир~--- странный мир, 
это да. И человек, в принципе, не может спасти чужую жизнь, разве что~--- отсрочить чужую 
смерть. И то\ldots\ Но во время этой, как вы ее назвали, инициации~--- я понял 
одну вещь. У меня хватает сил чтобы жить с этим. И, пожалуй, у меня должно хватить 
сил и отыграть отведенную мне в этой игре роль: не спасти мир, нет, но хотя бы 
отсрочить его гибель. Эта роль мне даже нравится\ldots\ Я принимаю правила 
игры, почему нет?~--- Борис, наконец, обратил взгляд на присутствующих. Улыбнулся 
всем и поднял бокал:\\
--- За правила игры, которые все мы принимаем! За игру!~--- Залпом выпил остатки 
и поставил бокал на стол. Кивнул Мо:\\
--- Пойдем, нам пора возвращаться.\\
Улыбнулся остальным:\\
--- Счастливо оставаться, господа.

Не оглядываясь, поднялся к верхним квартирам, не доходя до двери с номером 
<<42>>, всунул свой ключ в замочную скважину ближайшей двери. Повернул, открыл дверь, 
вошел. Последовавший за ним Мо обернулся в дверях. Нашел глазами Берса:\\
--- Увидимся!\\
Дверь закрылась, оставляя повисшими в воздухе слова Хранителя Вселенной:\\
--- Удачи вам!\\
~\\

Мы вышли из дверей книжного рядом с моей работой. Логично. Если мне нужно 
откуда-то, но не из дома, выйти, то это должен был быть либо книжный, причем 
именно этот, либо один замечательный пивной бар у набережной. В него я 
последние годы частенько захаживал, и даже не только из-за того, что там разливают 
действительно вкусное пиво, но и потому что там я почему-то не чувствовал 
дискомфорта от того, что сижу в баре в одиночестве. Ну, сидит человек, пиво 
попивает, баскетбол смотрит\ldots\ Все нормально. В других местах так не 
получалось.

Мо сунул руку в карман, достал ее и пустую поднес к лицу. Долго рассматривал, 
потом хмыкнул:\\
--- Да, рефлексы все-таки страшная сила. Сказочник, что это было, кстати? То 
есть, Берс мне друг, но Кинич прав~--- он изменился и я не склонен считать, что наша 
дружба не позволила ему задержать нас. А у меня сложилось стойкое впечатление, 
что выпускать нас обратно в намерения Хранителей не входило ни в коем 
случае\ldots


Удивительно. Мо не понял. Хотя про пресловутое <<соглашение>>, заключенное 
силами Вселенной с Хаосом, рассказывал мне как раз он, в тот самый вечер, когда вместе 
с Региной и Шаманом завалился ко мне домой. Удивительно. Неужели я враз обрел 
мудрость, сопоставимую с вселенской? Мне, во всяком случае, кажется, что все 
очень просто.\\
--- Все очень просто. Они не могли нас задержать. Ну, в смысле~--- задержать 
насильно. И убить они нас, раз уж на то пошло, тоже не могли. Этим они нарушили 
бы соглашение о невмешательстве в дела нашего мира\ldots\\
Мо медленно кивнул и вдруг просветлел:\\
--- Что косвенно свидетельствует о том, что даже если мы справимся, мировой 
порядок не пострадает и Берсу сотоварищи не о чем беспокоиться. Как, Сказочник, 
мы справимся?

Ну надо же~--- это теперь Мо мое подтверждение нужно. Он что, действительно 
считает, что во мне что-то изменилось за наш поход?\\
--- Справимся\ldots\ Куда ж нам деваться с подводной лодки? Особенно, когда 
<<жить так хочется, ребята>>\ldots\\
--- И каковы наши действия?


Нате, пожалуйста. Теперь я руковожу группой и в подчинении у меня существо, 
которое сфинксов еще котятами помнить может. Замечательно. Ладно, 
действия\ldots\ Я задумался. Мир вокруг жил привычной жизнью, но за ее покровом нет-нет да 
мелькало что-то непривычное. Что? Я присмотрелся. Прохожие в пальто цвета 
мокрого асфальта привычно сливались с мокрым асфальтом. Покрашенные в 
практичные темные цвета автомобили обдавали прохожих брызгами. Ярким пятном выделялись 
строители, торопливо выходящие из дверей строй-магазина с материалами, 
необходимыми, но либо забытыми, либо закончившимися на объекте. Пожалуй, да~--- 
мир не изменился, изменился я. И даже не изменился, а стал свободнее. Позволил 
себе видеть то, что раньше запрещал замечать и вынужденно придумывал для своих 
<<графоманик>>. Тень покорно стоящего в пробке троллейбуса нетерпеливо 
подрагивала, стараясь подняться с асфальта. Дома в отдалении казались фоновым 
рисунком в духе аниме и если перевести взгляд с них на людей вблизи~--- то и те 
оказывались рисованными персонажами. Или даже слепленными. Один из таких~--- 
мужчина в стандартной темной куртке свернул с тротуара на тропинку, уходящую 
вглубь дворов, прошел мимо куста с двух сторон огибаемого лужей, вернулся, 
поднял лужу с земли и, нацепив ее на плечи ангельскими крыльями, полетел по 
своим делам.

Из торгового центра напротив вышли двое~--- казавшиеся удивительно яркими на 
общем фоне и изумительно реальными на фоне новой открывшейся мне прорисовки мира. 
Высокого молодого мужчину в красной дутой куртке, желтых штанах и желтых 
тяжелых ботинках венчала смешная норвежская туристическая кепка, а второй каким-то 
образом умудрялся казаться ярким несмотря даже на длинное стандартно-серое, 
разве что нестандартного фасона пальто. Его я совсем недавно видел в 
троллейбусе. И когда-то очень давно~--- посвятил их выходу из торгового центра 
небольшой эпизод своей <<нетленки>>.\\
--- Наскальной живописью идут заниматься\ldots\\
--- Кто?~--- Я, честно говоря, совсем забыл о Мо, увлеченный полнотой виденья 
мира.\\ 
Реплику вслух я произнес для себя. А Мо, оказывается, все это время терпеливо 
ждал, пока я разберусь со своими ощущениями.\\
--- Да так. Одни из обитателей нашего мира, практически родня Регины и Шамана. 
Кстати, по-моему нам пора воссоединиться с ними, не считаешь?\\
--- Ты банкуешь, тебе виднее\ldots~--- Подражая Шаману, улыбнулся Мо. И тут же 
спросил:\\
--- А как?\\
--- Сейчас что-нибудь нарисуется,~--- Пообещал я. Пора уже опробовать свои силы. 
Итак, нам нужно что-то быстрое, способное доставить нас куда-то в Россию\ldots\
Представлялось слабо. Я вдруг выяснил, что совершенно не ориентируюсь в 
географии своей номинальной родины и вообще не уверен, что мне кто-то сообщал 
название города из которого они в Вильнюс пожаловали. С другой стороны, я 
прекрасно представлял улицы их города, представлял Регину в ее ежедневных 
возвращениях с работы домой, представлял подворотни в которых глушит свою 
вечную жажду Шаман\ldots\ Я представлял то место, но оно в моем сознании никак не 
связывалось с картой Росси. Ладно. Допустим. Значит, нам нужно что-то способное быстро 
доставить нас из одного места в другое. Из нашей локации~--- в локацию Шамана с 
Региной, доставить~--- без таможенных и паспортных проблем, без долгих 
изнурительных сидений на неудобных пассажирских местах, с ветерком\ldots\
Пронзительный скрежет тормозов вывел меня из задумчивости. Прямо перед нами 
остановилась шикарная машина из дорогих. Моих неглубоких познаний в автомобилях 
хватило только на то чтобы понять, что напоминает остановившийся перед нами 
зверь Ламборджини. Но только напоминает и, при этом, гораздо больше он 
напоминал мне что-то другое, хорошо знакомое. Из машины вышел коротко бритый щетинистый 
мужик в потрепанной кожаной куртке и тренингах. Подошел ко мне:\\
--- Берешь машину, братан? Ураган, а не тачка?


Говорил он со странным акцентом. С таким акцентом в американских фильмах 
говорят по-русски изображающие русских или условно-русских бандитов актеры. До меня 
начало доходить.\\
--- Ураган, говоришь? А тебя самого как звать-то?\\
--- Нико.~--- Мужчина протянул мне руку.\\
--- А я~--- Сказочник,~--- Я машинально представился своим виртуальным ником и 
вдруг понял, что это уже не только виртуальный ник. Забавно.~--- Спасибо, брат. Мы 
одолжим машину.

Сели. Подумав, я направил машину к ближайшему мосту. Издалека мост поблескивал 
квадратиками текстур.\\
--- Ни на моем, ни на компах Игоря или Арвидаса так и не удалось запустить эту 
игрушку на высочайшей графике\ldots\\
Мо помолчал и потом, решив не вдаваться в подробности, спросил:\\
--- Так мы едем к Королеве с Шаманом?\\
--- Ага,~--- Кивнул я, въезжая на знакомый по ежедневным проездам на троллейбусе 
мимо него, но теперь необычно красочный и странно пустой для середины 
дня мост.~--- С ветерком.\\
Мо помолчал, глядя в окно. Потом вздохнул.\\
--- А я по-прежнему не могу даже информацию нормально считать\ldots\ Интересно, 
как там у них.\\
--- Нормально,~--- Уверенно ответил я, поняв, что действительно знаю ответ на 
этот вопрос.~--- С боями, но прорвутся.\\
\\
Я вдавил педаль газа.

\newpage

Регина почувствовала пол под ногами и в тот же момент поняла, что сейчас на 
него завалится. Шаман, в плечо которого она вцепилась перед прыжком, падал. Первым 
рефлекторным желанием было отпустить его, но почему-то вместо этого она 
вцепилась в Миху второй рукой, потянула на себя и удержала. Высокий, но худой и 
ссохшийся Шаман на самом деле не обладал внушительным весом, так что больших 
проблем это не составило. Без особого труда Регина дотащила его до кровати. 
Отпустила~--- он рухнул как подкошенный~--- и присела на корточки рядом.\\
--- Миша, что такое? Плохо?

Не шевелясь и не открывая глаз, Шаман ответил утвердительно, но нецензурно. 
Потом все-таки повернулся к Регине.\\
--- А постояльцу твоему надо будет литру поставить за науку. Без нее~--- фиг бы я 
тебя сейчас дотащил. Да и сам вряд ли дотащился\ldots~--- Шаман снова закрыл 
глаза.\\ 
Регина вздохнула и непроизвольно погладила его по голове.\\
--- Кофе?~--- Миха только кивнул. Но за кофе пришел вслед за Региной на кухню. 
Покачиваясь и морщась, как после страшнейшей ломки, но пришел. На взгляд Регины 
коротко ответил:\\
--- В гробу отлежусь. В любом случае, сначала~--- дела.\\
Сели с кофе. Не обращая внимания, что напиток только что с огня, Шаман залпом 
выпил полкружки. Регина чуть отпила:\\
--- И как мы их делать будем?\\
Миха пожал плечами:\\
--- Быстро и без лишнего шума, хозяйка. Мы можем просто приехать туда, оформить 
ксивы и что там еще полагается?\\
Пришла очередь Регине пожимать плечами:\\
--- Думаешь, я каждый день крупные выигрыши оформляю? Думаю, можем, но сначала 
все равно нужно позвонить и договориться\ldots\\
--- И засветиться\ldots~--- Вздохнул Миха.~--- А работают пацаны шустро.\\
Он достал из кармана мятую пачку, вытянул сигарету. Покрутил ее в пальцах и 
положил на стол.\\
--- Хозяйка, дай я звоночек сделаю\ldots

Регина принесла ему телефон. Шаман взял трубку в правую руку, снова вздохнул, и 
несколько раз согнул--разогнул ладонь левой руки. На ладони проступили синие, 
напоминающие вязь тюремного искусства татуировки, цифры. Набрал их.\\
--- Не болей, Серенький, не кашляй\ldots\ Шаман это\ldots\ Богатым буду, 
говоришь? Конечно, буду, поэтому и звоню\ldots\ Просьба у меня к тебе есть, Серенький, не 
откажешь? Врач мне нужен проверенный\ldots\ Да здоров я, ну, как обычно\ldots\ Адвокат, в 
смысле\ldots\ Ну почему сразу <<уголовный>>, ты за кого меня держишь? Выигрыш надо оформить, вот 
и хочу специалиста с собой взять чтоб проследил\ldots\ Большой выигрыш, 
да\ldots\ Не я, человек один хороший\ldots\ Срочно, да\ldots\ Спасибо, дорогой, гадом буду~--- 
не забуду. Записывай или запоминай\ldots~--- Шаман продиктовал адрес, покосился на Регину: 
правильно ли?~--- и дождавшись от нее кивка распрощался с собеседником. Допил 
кофе и, наконец, закурил:\\
- Ну а теперь~--- ждем\ldots

Докурив, Шаман с разрешения Регины вернулся в кровать. Регина помыла кружки, 
турки, протерла плиту. Протерла стол, осмотрелась и протерла шкафчики. Подмела 
пол на кухне и в коридоре. Бездумно: дом за несколько дней ее отсутствия 
запылился и в нем следовало прибраться. Не думать было легко и приятно. С 
влажной тряпкой в руках Регина прошла в комнату и протерла мебель там. 
Собралась включить пылесос, но посмотрела в сторону Шамана. Шаман спал. Отказавшись в 
пользу гостя от идеи, присела на краешек кровати. Даже во сне лицо Михаила 
казалось измученным и напряженным. Очередной раз подумала: какой же он 
несчастный\ldots\ Это невинное наблюдение будто пробило выстроенную Региной 
плотину и поток впечатлений вдруг захлестнул ее. Возникло страстное желание разобраться, 
проанализировать и разложить все по полочкам. Самая нижняя полочка 
своеобразного стеллажа сознания, самая приземленная: за ней гонятся бандиты. Это почти 
нормально. Это нормально для этого мира, это вполне возможно при ее работе, 
хотя до сих пор бог миловал. В это можно верить. Хорошо. Что им от нее надо? А это 
уже вторая полочка, повыше. Она выиграла кучу денег. В это уже верится с 
трудом, но~--- билет у нее. Ладно, иногда бывает и такое. Выиграла она его по подсказке 
странного человека, забравшего тело у ее случайного любовника. До этого 
странный человек путешествовал по другим мирам, а за время их знакомства совершил 
множество всяких мелких чудес и привел в ее дом и ее жизнь еще одного, спящего 
сейчас у нее на кровати, волшебника. Это~--- бред, как бред и то, что в 
компании этих волшебников она недавно без всяких автобусов, поездов или самолетов 
побывала в Литве и так же моментально вернулась назад. Как то, что в Литве они 
нашли молоденького паренька, придумавшего ее и Шамана, как то, что этот чем-то 
обиженный на весь мир мальчик~--- единственный, кто может спасти этот мир. И 
все это бред. И все это было и есть. И еще за ней гонятся бандиты.

Регина вздохнула. Разобралась. Разложила по полочкам. Стремление 
проанализировать происходящее сменилось простым желанием спрятаться от него. 
Укрыться за пазухой у кого-то сильного, надежного, своего. Кто защитит и 
погладит по голове. <<Тссс, королева, все хорошо. Все это~--- просто дурной 
сон. Я отгоню его от тебя. Спи\ldots>> Она снова взглянула на спящего. Вряд ли от него 
можно было услышать что-то подобное. Вряд ли его искренне можно назвать надежным. Но 
он сильный. И, странное дело, она знает его без году неделю, и ей всегда 
претили блатари, но\ldots\ Но, кажется, что он~--- кто-то очень родной и близкий. Если 
вспомнить весь тот бред, который и вспоминать-то не надо, он и так в мыслях, то в 
каком-то смысле Шаман и вправду родной. Родственничек. Захотелось заплакать. Еще недавно 
у нее была жизнь. Может быть, не самая лучшая, может быть, пустая и скучная, 
может быть~--- одинокая и не слишком счастливая, но ее. Была. Недавно. А потом 
раз~--- и оказалось, что не было никогда этой жизни, не было ее переживаний, ее 
опыта, ее ошибок и работы над ними. Все это кто-то придумал. Причем, придумал так, 
походя, сочиняя историю, занявшую всего неделю из тридцати с чем-то лет ее 
жизни. Все эти годы~--- всего лишь смутный условный фон, для одной любовной 
интрижки с легким ароматом магии. А ему? Она провела пальцами по ладони Шамана. 
Ему еще хуже. И в то же время, возможно, он наоборот где-то в глубине души рад, 
что вся его жизнь~--- чья-то глупая фантазия. Это не он~--- урод, это~--- 
Сказочник. А он, Миша, Миха Шаман~--- он и не урод вовсе. Просто очень несчастный, 
измученный и битый, пусть и придуманной, но жизнью человек. Ему еще хуже. Регина всхлипнула. 
Очень хотелось человеческого тепла. Очень хотелось почувствовать, что все не 
так странно. Очень хотелось. Она прилегла сбоку от Шамана и уткнулась лицом ему в 
плечо. Шаман дернулся, но не проснулся. Вскоре уснула и Регина.

Разбудил их звонок. Точнее, услышав звонок, вскочил Шаман, помчался к коридору, 
запнулся, и вместо того чтобы подойти к входной двери забежал в туалет. Там, 
глубоко вздохнув, осторожно прошел сквозь стену. На лестничной площадке 
терпеливо, сфокусировав взгляд на дверном номере, ждал лысый полный низенький 
очкарик. Появление Шамана он заметил только когда тот подошел вплотную к нему.\\
--- Кто?~--- Коротко и хрипло со сна спросил Шаман.\\
--- Райкин Евгений Натанович, юрист,~--- Солидно ответил ждущий, по-видимому, 
ничуть не удивившись внезапному появлению спрашивающего.~--- По просьбе Сергея Коврова.\\
--- Проходи,~--- удовлетворенно кивнул Шаман, и тут же мысленно чертыхнулся. 
Сосредоточившись~--- и все-таки слишком много для него мелких и крупных чудес, 
слишком много~--- нажал на ручку двери, усилием воли заставляя провернуться 
механизм дверного замка. Дверь открылась.\\
--- Нехорошо это~--- двери отпертыми держать,~--- Переступив порог, 
прокомментировал адвокат.~--- Небезопасно это в наше-то время\ldots

Именно к этому моменту удалось наконец-то выпутаться из сна и подняться с 
кровати Регине. Увидев в прихожей мужчин, она ойкнула и, избегая смотреть на 
Шамана, поздоровалась с гостем:\\
--- Здравствуйте. Хотите чаю или кофе?\\
--- От кофе не откажусь,~--- Последовал степенный ответ и Регина юркнула на 
кухню. 

За кофе Шаман ввел адвоката в суть дела. Коротко, не вдаваясь в ненужные 
детали. Нужно оформить выигрыш, а поскольку уже завтра Регина улетает в деловую 
командировку, которая может затянуться надолго~--- сделать это нужно быстро. 
Никаких бандитов, никакой мистики, никакого риска. Все, что от Райкина 
требуется~--- помочь своим присутствием надавить на неспешную бюрократическую машину.\\
--- Н--да\ldots~--- После инструктажа протянул Евгений Натанович.~--- Я, честно 
говоря, не слишком понимаю чем смогу быть вам полезен, но, как говорится, любой каприз за 
ваши деньги. Тем более, в данном случае это деньги Сергея Витальевича\ldots\ 
Мне нужны ваши паспортные данные и телефонный номер фирмы, если он у вас есть.\\
--- Из тачки позвонишь,~--- Отмахнулся Шаман.~--- У тебя, кстати, колеса есть или 
такси вызывать будем?\\
--- Мне вас еще и подвезти?~--- Адвокат выразительно вздохнул.~--- Хорошо. В 
таком случае жду вас на улице.\\
Когда за Райкиным закрылась дверь, Шаман вздохнул.\\
--- Не нравится мне этот <<юрист>>\ldots\ Но выбирать не приходится, да и 
Серенький вроде подляну подкинуть не должен\ldots\ Ну что, хозяйка, понеслась?\\
--- Страшно-то как\ldots~--- Регина неуверенно улыбнулась и впервые с пробуждения 
посмотрела на Миху.~--- Ты хоть немножко отдохнул?\\
--- Ага, спасибо\ldots~--- Михо вдруг смущенно кашлянул.~--- Прости, что я 
внаглую твою койку прихватизировал\ldots\ Как-то совсем вырубило\ldots\\
--- Да ладно, кровать большая, двоим там никогда тесно не было\ldots~--- Резко 
рассмеявшись, Регина выбежала из кухни.~--- Через пять минут буду готова!

Офис оказался ветхим двухэтажным зданием в тупичке на окраине бизнес-квартала. 
Особого доверия его внешний вид не внушал. Но перед входом висела большая 
уверенная таблица <<Викинг-лото: главный офис>> и дежурил молодой симпатичный 
сотрудник мальчик в деловом костюме. Мальчик, по всей видимости, и был тем 
самым ответственным специалистом, с которым, в результате долгих переговоров с 
приемным отделом и потом~--- кем-то из менеджеров, в конце концов соединили 
Райкина. Нацепивший на ухо хэндс-фри Райкин, ни на миг не отвлекаясь от дороги, 
провел разговор так, что даже Шаман с заднего сиденья одобрительно хлопнул 
адвоката по плечу: <<Не зря хлеб ешь!>>. Их согласились принять, сразу же все 
оформить и пообещали, что даже в приемной ждать не придется~--- их сразу 
встретят. И вот сейчас встречающий мальчик подбежал к машине и открыл перед Региной дверь.\\
--- Здравствуйте, вы, наверное, Регина? Меня зовут Андрей Старухов, я буду 
заниматься оформлением вашего выигрыша. Кстати, прежде всего~--- поздравляю 
вас! Как от себя, так и от всей нашей компании. Обычно мы встречаем победителей 
букетом и коллекционным шампанским, и сегодня мы приносим самые искренние 
извинения за то, что отошли от обычного протокола. Поверьте, это никак не 
связано с неуважением к вам, но поймите~--- мы оказались несколько не 
подготовлены к столь скорому вашему появлению. Но, повторяю, в этом нет даже намека на 
какое-то пренебрежение к вам. Да и как вами можно пренебрегать~--- вы выйдете 
из этого офиса одним из самых богатых жителей нашего города. Я вам даже немножко 
завидую\ldots\ Но, пойдемте. Ваши спутники, если изъявят желание, могут 
дождаться вас в нашей комнате отдыха. Она шикарнейше оборудована, мы очень заботимся о 
комфорте своих посетителей, да и сами сотрудники, к чему лукавить, иногда не 
прочь отдохнуть. Так что, спешу заверить, они с огромным удовольствием проведут 
то время, которое мы с вами посвятим заполнению скучных бумажек,~--- Сотрудник 
безудержно тараторил, при этом улыбался и подводил вышедших из машины вперед, 
неуклонно оттесняя Регину от спутников. Тонущую в потоке его речи Регину 
очередной раз спас Шаман:\\
--- Успокойся, оратор. Мы с ней.\\
--- Конечно, я понимаю ваше желание. Но, к прискорбию, есть такая вещь, как 
правила\ldots~--- Старухов повернулся к Шаману и, столкнувшись со стальным 
взглядом того, стушевался.~--- Впрочем, дело ваше.\\
Погрустневший Старухов молча провел троицу в просторный кабинет. Дождавшись, 
пока посетители устроятся за широким столом, скучно поинтересовался:\\
--- Чай, кофе, сок?

После чего так же лаконично, без прошлой болтливости, перешел к делам. Взял у 
Регины анкетные данные, сверил их с ее документами, проверил по базе данных 
серийный номер билета, спросил о нем: где приобрела, когда, в какое время 
суток. Потом попросил у Регины номер банковского счета и, на какое-то время, вообще 
замолчал, погрузившись в компьютер. Потом зажужжал принтер.\\
--- Ну вот,~--- Улыбнулся Старухов, кладя перед Региной, кипу бумаг.~--- Изучите 
договор, подпишите и на этом формальности будут окончены. В течение трех 
рабочих дней деньги, за вычетом налогов, будут переведены на указанный вами счет.

В полнейшей тишине Регина внимательно изучила бумаги. Договор был составлен 
корректно. Передала бумаги Шаману, тот, не глядя, подвинул их по столу 
дальше~--- Райкину. Адвокат так же пристально как до этого Регина просмотрел договор. 
Кивнул:\\
--- Все хорошо, Регина. Проверьте еще раз~--- правильно ли указаны ваши данные и 
можете подписывать.\\
Получив подписи и свой экземпляр бумаг, Старухов расцвел в улыбке:\\
--- Вот и все, Регина! Поздравляю вас~--- с помощью нашей компании и вашей удачи 
вы вступаете в новую жизнь! Уверен, эта жизнь будет для вас счастливой и 
безоблачной!..\\
Не скрывая своего пренебрежения к словоблудию клерка, Шаман поднялся из-за 
стола. Сотрудник запнулся.\\
--- Еще один момент. С вами хотели бы пообщаться представители наших партнеров. Я 
надеюсь, вы окажете им такую любезность?\\
Адвокат удивленно склонил голову на бок:\\
--- А в чем, собственно, дело?\\
Отвечать ему Старухов не торопился, ожидая ответа от Регины. Регина смотрела на 
Шамана. Шаман кусал губы. Наконец, побарабанив пальцами по столу, сел.\\
--- А у нас есть выбор?\\
Не отвечая напрямую Шаману, Старухов улыбнулся Регине:\\
--- Уверяю вас, это не займет много времени\ldots\\

Словно по команде в кабинет вошли трое мужчин. Шаман оценивающе рассматривал 
их. Двое~--- типичные быки, третий, точнее центральный~--- посолиднее. У него, вон, 
портфельчик в руках. Кожаный. Сейчас так, с понтом, положит портфель на стол и 
заговорит.\\
--- Здравствуйте, Регина. Доброго дня и вам, я так понимаю~--- Борис, да? Ба, 
Женечка, а ты что тут делаешь?~--- Адвокат, к которому и было обращено это 
<<Женечка>>, встрепенулся. Посмотрел на Регину, причмокнул губами и поднялся со 
стула.~--- Да, в общем, уже ухожу, на самом деле. Все, что должен был~--- 
сделал. Передавай от меня поклон \ldots~--- Не закончив, адвокат прошел к дверям. 
Обернулся на пороге:\\
--- Удачи вам, Регина, и вам, Михаил. Профессиональный совет на прощание: 
подумайте над предложением этих людей. Не стоит от него отказываться\ldots

Шаман усмехнулся. Спекся, адвокатишко. Интересно, а Серенький его решение 
одобрит? Ладно, позже выяснится. Пока о Регине думать надо. Посмотрел на 
женщину: она сосредоточенно рассматривала лидера пришедших <<партнеров>>.\\
--- А вас не Аркадий зовут?~--- Наконец, спросила она.~--- Вы, кажется, водителем 
у Шамиля Вагантовича работаете?

Тот, кого Регина назвала Аркадием, поморщился, но уже через секунду рассмеялся. 
Смех был громким, раскатистым, беспечным. Услышав этот смех, Шаман понял, что 
живыми их отсюда не выпустят. Посмотрел на ушедшего в тень Старухова, обнаружил 
в глазах того панику и обреченность. Точно не выпустят. Аркадий, наконец, 
отсмеялся.\\
--- Узнали, Регина, узнали. Помню, вас Шамиль Вагантович несколько месяцев назад 
домой подвозил. Он, кстати, очень высокого мнения о вас. Говорит, что вы лучший 
из известных ему финансовых аналитиков. Собственно, я тогда и не буду резину 
тянуть. Мы представляем компанию, занимающуюся инвестициями, у вас появились 
свободные финансы, мы предлагаем вам наше посредничество в их инвестировании. 
Вот договор,~--- Он открыл портфель и вытащил из него голубенькую папку.~--- 
Можете почитать, если хотите, но я гарантирую~--- это очень выгодное для вас 
предложение, так что можете сэкономить ваше и наше время и сразу подписать.\\
Почувствовав на себе растерянный взгляд Регины, Шаман вздохнул.\\
--- Мы все--таки посмотрим.\\
--- Ну--ну,~--- Аркадий пожал плечами.~--- Дело ваше. Только, пожалуйста, не 
затягивайте. А вы, все--таки, не Борис, да? Жаль, очень хотелось с ним 
познакомиться\ldots\ Ладно, думайте, а мы пока кофе попьем\ldots~--- Аркадий 
кивнул своим и троица бандитов, а вместе с ними и сникший Старухов, покинули кабинет.\\
--- Козлы надутые,~--- Дождавшись, пока закроется дверь, прокомментировал 
Шаман.~--- И ведь совсем за лохов держат, даже не заперли\ldots\ У тебя труба связь ловит?\\
Регина посмотрела на телефон.\\
--- Нет.\\
--- Я так и думал\ldots\ Ладно, хорошо, что они такие самодовольные, суки. 
Фраеров обламывать~--- святое дело\ldots~--- После этих слов Шаман закрыл глаза и 
откинулся на спинку стула. Через минуту Регина решилась спросить:\\
--- Миш, а может подписать~--- и черт с ними?\\
Шаман тяжело вздохнул и открыл глаза.\\
--- Хозяйка, не мешай мне. Их и так для меня сейчас многовато. 
А насчет подписать~--- смысл? Как только у тебя отожмут деньги~--- ты исчезнешь. Подарит тебе 
компания в дополнение к выигрышу какой-нибудь круиз в экзотические страны на полгода, 
шефу твоему на ящик придет письмо от тебя об отставке, он подпишет\ldots\ А 
через полгода ты просто не вернешься, вот и все. Самое смешное тут, что даже если они 
эти деньги не отожмут~--- ты исчезнешь. Выигрыш ты оформила, а остальные 
формальности они и без тебя спорядкуют, не переживай.\\
--- А ты?~--- Регину неприятно удивило, что Шаман сейчас говорил исключительно о 
ней. Шаман посмотрел на нее:\\
--- А что я? Со мной вообще никаких проблем. Жил--жил алкаш и нарик Миха, да и 
перестал. Ничего необычного\ldots\\
--- И что нам делать?\\
--- Мне~--- бить первым. Тебе~--- постараться мне не мешать\ldots~--- Шаман снова 
глубоко вздохнул.~--- Посиди тихо, ладно\ldots\\

Но сосредоточиться он не успел. Открылась дверь и в кабинет вошла давнишняя 
четверка, причем Старухова подталкивал один из быков.\\
--- Ну что,~--- Аркадий сухо смотрел на Регину.~--- Надумали? А, кстати, 
Михаил\ldots~--- Он перевел взгляд на поднявшегося Шамана.~--- Там еще Женечка не успел уехать, так 
он согласился тебя подбросить, если что. Можешь идти, тебе-то это все до фонаря 
должно быть\ldots\ Ковер говорит, ты толковый мужик, фигли.\\

Шаман молчал и с каждой секундой его молчания Регине становилось все страшнее. 
Наконец, Миха широко улыбнулся.\\
--- Значит, Ковер говорит, да? Ссучился Серенький окончательно, а жаль\ldots\ Не, 
ты, Аркаш, прости, но я вместе с Региной уйду. А контрактом своим подтереться 
можешь. Или там за Барой подтереть, сам решай.\\
--- Быкуешь, отброс, хамишь\ldots~--- Аркадий равнодушно пожал плечами.~--- Ну и 
хрен с тобой\ldots

Один из спутников Аркаши достал пистолет. Словно в замедленной съемке Шаман 
смотрел как по дуге движется рука бойца, как его плечо, дуло и голова Шамана 
выстраиваются медленно на одной прямой.\\
--- Пачкай поменьше,~--- Брезгливый голос Аркадия. Пистолет опускается. 
Неторопливо, наверное, с черепашьей скоростью выползает пуля, плывет по воздуху, 
приближается\ldots

События последних дней вымотали Шамана. По-хорошему, только после переброса Мо 
с Региной в Вильнюс ему следовало пару недель крепко пить, вмазаться пару 
раз\ldots\ Не получилось. Наука Мо, точнее тот единственный урок, который преподал Шаману 
древний попаданец, слегка придавала сил, но сейчас и она была бесполезна. 
Времени расслабиться, подготовиться, принять мир в себя и на прочую лабуду 
просто не было. Пуля приближалась. Пуля~--- это на самом деле не плохо. 
Пуля~--- это лучше чем постоянные ломки, это лучше чем пробуждение в собственном дерьме и 
блевоте после приходов, это лучше чем память без единого светлого воспоминания 
и будущее без единой надежды на что-нибудь. Черт бы с ней, с пулей. Пусть себе. 
Плохо то, что уже следующая будет принадлежать Регине~--- этой маленькой 
растерянной женщине, в чью маленькую придуманную жизнь недавно ворвалась совсем 
неожиданная реальность, захлестнула ее, поволокла за собой и как за 
спасательный круг она схватилась за него, за Миху Шамана, поверив ему. Он врывался в ее 
квартиру, блевал на ее полу, представал перед ней после жестких отходняков~--- 
а она ему поверила. Приняла. Жалась к нему, искала от него тепла и поддержки, 
спала, уткнувшись в него\ldots\ Следующая пуля~--- ей? Нет!

А Регина просто перестала воспринимать происходящее. Точнее~--- перестала 
верить, что все это происходит. Так не бывает. Ей даже не было страшно. Она просто не 
верила. Раз~--- и бандиты входят в кабинет. Раз~--- и Миша отказывается уйти. 
Раз~--- и в Мишу стреляют. Раз~--- и он\ldots\ Он не падает. Пошатнувшись от удара 
вошедшей в грудь пули, Шаман продолжает стоять. Он смотрит на стрелявшего, все остальные 
смотрят на него. Наконец, он кашляет. Так, словно подавился чем-то. Сплевывает 
на ладонь оливковую косточку:\\
--- Маслиной, значит, меня угостить решили\ldots

Так не бывает. Регина смотрит на Шамана и не узнает его. Перед ней~--- не тот 
измученный жизнью несчастный мужчина, которого она привыкла видеть. Перед ней 
даже не тот страшный агрессивный наркоман, которого она однажды застала на 
пороге своего дома. Напротив бандитов стоял сильный, беспощадный, жестокий 
зверь, уверенный в обреченности своих жертв. Зверь улыбается. Зверь смотрит на 
возомнившую себя охотником дичь. На пистолет в его опущенной правой руке. 
Поднимает свою левую руку, приставляет указательный палец к виску. Бандит 
зеркально повторят его движения. Разница в том, что в голову бандита упирается 
не палец~--- ствол. Миша нажимает средним пальцем на воображаемый курок. Звучит 
выстрел. Один бандит падает. Двое других инстинктивно оттирают забрызганные 
чужими мозгами лица. Зверь поворачивается к Аркадию.\\
--- Прости, мы все-таки напачкали.

Аркадий тянется за пистолетом, но замирает. Глаза Шамана поймали его взгляд. 
Аркадий не может пошевелиться. Он не может даже вздохнуть. У него набухают 
вены, синеет кожа, вываливается изо рта распухший язык. Он падает. Регина зачем-то 
отмечает, что это не заняло даже десяти секунд.

Так не бывает. Зверь находит следующую добычу. Ей он почти улыбается. Она почти 
ни в чем не виновата. Она просто пришла со всем стадом.\\
--- Умри,~--- Ласково говорит он ей и она слушается.\\

Так не бывает.\\

Зверь медленно превращается в Мишу, но в Шамане остается что-то звериное. 
Неуловимая текучесть движений, плавность звучания голоса, выражение лица.\\
--- Ты,~--- Говорит он, подходя к сползшему по стене Старухову.~--- Значит так, 
давай-ка в срочном порядке получим бабло сегодня.~--- После небольшой паузы 
уточняет.~--- Сейчас.

Старухов медленно поднимается, хотя Регине кажется, что это Шаман взглядом 
поднимает его.\\
--- Вы понимаете, что вы оба~--- трупы? Вы понимаете, кто эти 
люди\ldots\ Были\ldots~--- Кажется, Андрей только после последнего слова 
окончательно понял, что произошло. Его начинает бить истерика.~--- Вас убьют. 
Сегодня. И даже, если вас арестуют раньше~--- 
все равно убьют. И меня. Заодно. Что вы\ldots\-
Регине искренне интересно: они~--- что? Но Шаман не дает Старухову закончить 
фразу. Он бьет его ладонью по щеке. Со стороны кажется~--- мягко, но клерку 
хватило. Он умолк. \\
--- Бабло.~--- Повторяет Шаман.~--- Сейчас.\\
--- Я не уверен, что это вообще\ldots~--- Начинает Старухов, но осекается.~--- Я 
попробую.

Шаман кивает. Клерк подходит к столу, нажимает кнопку внутренней связи, меняет 
решение и садится за компьютер. Шаман неуклонно следует за ним и теперь стоит у 
него за спиной. Через какое-то время снова жужжит принтер. Старухов встает и 
подносит Регине взятый из принтера лист. \\
--- Распишитесь в получении,~--- Говорит ей. Голос его бесцветен.~--- Хотите 
зайти с моего компьютера, проверить?\\
--- Мы тебе верим,~--- Успокаивает его Шаман.~--- А теперь поспи полчаса. Здесь 
ничего не было, мы пришли, оформили и получили выигрыш, ушли. Никаких партнеров, 
никаких Аркадиев. Да?\\
--- Да,~--- Соглашается Старухов, садится в кресло и засыпает. Шаман тоже 
садится, но в отличие от Старухова, продолжает бодрствовать. Черты его лица 
расплываются, его лицо искажает гримаса боли и перенапряжения. Одежда на груди набухает 
кровью.\\
--- Ты как?~--- Спрашивает Регина. Ее голос дрожит.~--- Жить будешь?\\
--- Не знаю.~--- Невесело усмехается Шаман.~--- Вроде не должен.\\
Регина наконец-то верит. Да, Шаман не знает. Да, вроде не должен.\\
--- Не умирай,~--- просит она.~--- Пожалуйста.\\
Шаман поворачивается к ней и неожиданно легко соглашается.\\
--- Да, пока не буду. Рано.\\
Вздыхает. Его лицо снова становится той страшной маской, с которой он только 
что ходил по кабинету.\\
--- Я сейчас на 15 минут отойду. Если не вернусь~--- иди домой, ладно.

Не дожидаясь ответа, встает и проходит сквозь ближайшую стену. Регина долго 
смотрит ему вслед, потом возвращается взглядом к креслу, где он сидел. Он в 
нем. Без движения, без сознания, умиротворенный. Регину бьет дрожь.

\newpage

Бара отдыхал. Те, кто завидуют <<хозяевам жизни>> просто не знают, сколько сил 
и нервов тратиться на такое <<хозяйство>>. Много. Очень много. И хорошо, что 
можно на пару часов раньше объявить для себя конец дня,.. очередную 
ассистентку, благосклонно отпустить ее домой, закрыться в роскошном, со вкусом 
обставленном кабинете, достать из бара бутылку коньяка, налить бокал, включить 
негромко музыку и расслабиться перед вечером. Последнее время Бара предпочитал 
Отард и Моцарта. Кому-то его вкусы могли казаться слишком простецкими, но в 
этом городе Бара мог позволить себе не заморачиваться о мнении других. Уже мог. 
Повезло.

Первый бокал Бара всегда выпивал за везение. Он никогда не страдал от излишней 
скромности, знал, что в меру умен, хитер, рассудительно жесток и обладает 
хорошими лидерскими качествами. Знал и умел всем этим пользоваться. Но всегда 
посвящал первый глоток своему везению, зная, что именно ему обязан всем, чего 
достиг. Везению и чему-то необъяснимому.

Бара родился слишком поздно. Слишком поздно родился, слишком поздно вырос, 
слишком поздно отправился на поиски лучшей доли в Россию. К этому времени все 
было поделено, а те способы передела, которые Бара любил, понимал и 
приветствовал постепенно сходили на нет. Он был нагл, он беспредельничал, он не 
боялся идти против авторитетов города, который выбрал своим, но понимал при 
этом: не те времена. Как только он всерьез перейдет дорогу кому-либо из сильных 
– с ним будет покончено. Без стрел, без подкрепленных автоматами <<братьев>> 
встреч на ночной дороге, просто~--- покончено. Но он рисковал. До тех пор, пока 
однажды не привлек на себя внимание Туза~--- не просто одного из хозяев города, 
а хозяина всех хозяев. Ему повезло. Вышедший из тех самых времен, о которых так 
сожалел Бара, Туз не стал сразу отдавать приказ о устранении молодого наглеца. 
Он захотел посмотреть на него. Бару привезли в огромный загородный дом, 
служивший Тузу штабом, крепостью, офисом и всем остальным. И тут ему повезло 
второй раз. На середине разговора, когда передние зубы Бары уже валялись на 
полу, а почки напоминали отбивные, в доме появилось нечто. Нечто было страшным, 
двигалось шатаясь и воняло гнилью и перегаром. Но самое удивительное, что это 
нечто прошло по всему дому, оставляя (как потом уже выяснил Шамиль) за собой 
трупы, вошло в подвал, где проводил воспитательную беседу Туз, посмотрело на 
Туза и его помощников~--- и все они сдохли. Потом жуткий мужик долго смотрел на 
привязанного к стулу Бару, сплюнул и исчез. А Бара, придя в себя, понял, что 
ему представился шанс.

В заваленном трупами доме Туза Бара провел три дня, большую часть времени~--- в 
компании <<своих>> юристов и компьютерных спецов. Еще день он провел в компании 
<<своих>> нотариусов. А на пятый день он оказался официальным и законным 
хозяином всего законного хозяйства покойного, обладающим, к тому же, знаниями о всех или 
почти всех менее легальных контактах и делах Туза. С таким раскладом можно было 
играть. Бара сыграл. И выиграл. Через несколько лет и несколько десятков 
галлонов крови город признал его лидерство. Не стал Бара, поверив чутью, 
трогать только бизнес Сергея Коврова, наоборот, наладил с ним официальные партнерские 
отношения. И почему-то был уверен, что сделал правильно. Смущало все эти годы 
только одно: почему у олицетворения его везения облик такой страшный, что 
заставляет его, совсем непугливого мужчину, до сих пор время от времени 
просыпаться в холодном поту?\\
--- Привет, Шамиль.


Бара вздрогнул и отвел глаза от бокала. Напротив него, по-хозяйски 
расположившись в кресле для посетителей, сидело то самое нечто. Правда, сейчас 
мужчина выглядел не так отвратительно. Зато в его облике появилось что-то 
совсем далекое от человеческого.\\
--- А я тебя помню,~--- Равнодушно сообщил мужчина. Действительно, равнодушно. 
Внимание его при этом было сконцентрировано на бутылке коньяка. Он открыл ее, 
понюхал, облизнулся, но пить не стал. С явным сожалением отставил.~--- Да, 
помню.\\
--- Я тебя тоже,~--- Бара рискнул вступить в беседу.~--- Ты~--- мой страшный 
ангел.\\
--- Да ну?~--- Мужчина посмотрел на Бару с чем-то, что могло сойти за интерес. 
Улыбнулся почти добродушно. Необратимость. Слово давно ставшего родным русского 
языка наконец-то всплыло в голове Бары. Необратимость. Вот, что сквозило во 
всем облике человека, вот, что придавало ему нечеловечность. Необратимость. А 
нечеловечный человек тем временем доверительно сообщил Баре:\\
--- А я тебя убить решил.\\
--- Зачем?~--- Бара даже не думал спорить. Спорить было бессмысленно.\\
--- Так надо,~--- Мужчина пожал плечами и снова покосился на бутылку. Потом 
непринужденно поинтересовался.~--- Или ты жить хочешь?\\
Бара не стал спорить.\\
--- Хочу.

Мужчина кивнул. Чего, мол, удивительного~--- хочет человек жить. А кто не 
хочет? Снова улыбнулся Баре. Бара осознал, что начинает понимать бывшее для него 
раньше исключительно абстрактным выражением <<мурашки по коже>>.\\
--- Тут женщина в лотерею выиграла.~--- Сообщил <<страшный ангел>>.~--- Так это 
ее выигрыш, да?\\
И разжевал как ребенку:\\
--- Ее. Не твой. Ее. Понял?\\
Бара понял. И даже не сильно расстроился. Кивнул.\\
--- Вот и славно,~--- Кивнул в ответ мужчина.~--- Живи тогда.

\newpage

Появление двойника Миши Регина заметила сразу. Даже не заметила~--- 
почувствовала. 

Прошло даже меньше обещанных пятнадцати минут~--- минут пять, не больше. Все 
это время Регина сидела, забравшись с ногами на стул, обняв коленки и неотрывно 
смотря на тело Шамана. Он не должен умереть. Он. Не. Должен. Умереть. Повторяла 
она, покачиваясь в такт своим словам. А потом появился второй. Посмотрел 
издалека на Регину, улыбнулся ей, подошел к своему телу и внимательно посмотрел 
на него. Развернулся и подошел к ближайшему трупу. Труп начал медленно таять, 
растекаясь по полу, пока не превратился в лужу странного цвета жидкости. Регине 
стало плохо. Она закрыла глаза. Когда открыла~--- на полу оказалось три лужи, а 
на стуле рядом с ней сидел Миша. Один. Живой. Привычный.\\
--- Пойдем, хозяйка.~--- Улыбнулся он ей.~--- Здесь нам больше делать нечего.\\
--- И что теперь?~--- Они беспрепятственно покинули главный офис <<Викинг-лото>> 
и теперь медленно брели по вечернему городу. Словно ничего и не произошло.\\
--- Теперь?~--- Шаман пожал плечами.~--- Ты банкуешь, хозяйка. Я бы на твоем 
месте отметил. Все--таки приличный куш сорвала.\\
--- Кстати, да!~--- Регина остановилась и повертела головой. Банкомат обнаружился 
в двух шагах от них.~--- Надо же проверить!

Подошла, вставила карточку, ввела пин, выбрала <<Баланс>>~--- и повалилась в 
руки подхватившего ее Шамана.\\
--- Ой, мамочки\ldots~--- Вздохнула через некоторое время.~--- Ой\ldots\\
В руках Шамана было уютно и спокойно. Нехотя высвободилась. Встряхнула головой.\\
--- Ой\ldots\ Насчет отметить~--- не знаю, но нервы успокоить точно надо. Зайдем 
куда-нибудь?\\
--- Ставишь?~--- Спросил Шаман, но поймав укоризненный взгляд Регины, 
смутился.~--- Шучу, хозяйка\ldots\\
--- Глупый вопрос, Миша,~--- Только и ответила она.~--- Пошли.

Регина заказала бокал вина, Шаман, усмехнувшись чему-то своему, попросил бокал 
коньяка и чашку кофе. Кофе стал неожиданностью для него самого, но вдруг 
захотелось.\\
--- Ну, что~--- Шаман поднял бокал.~--- Как сказал этот болтун в офисе~--- ты 
начинаешь новую жизнь! Пусть она будет счастливой и радостной!\\
--- Мы начинаем.~--- Тихо, но твердо поправила Регина поднимая свой.~--- Мы. \\
Они смущенно выпили и замолчали\ldots\ Регина сменила тему.\\
--- А вообще~--- поверить не могу, что все это произошло. Что все это происходит. 
Миш, это все реально? На самом деле?\\
--- Реально\ldots~--- Шаман смущенно покрутил чашку на блюдце.~--- Регин, это не 
ко мне вопрос. Ты, если за реальность хочешь узнать, у Мо спроси~--- он тебе 
расскажет. А я таких умностей не думаю. У меня вообще думалка плохо работает, 
знаешь\ldots\ Просто жизнь мне такая выпала~--- неважно, по случайности, по божьему умыслу, по воле 
Борьки нашего\ldots\ Просто выпала и я ее просто живу. А насколько она 
реальна\ldots\ Не, это к Мо. Регина помолчала. Потом улыбнулась.\\
--- Значит, будем просто жить, почему нет? А столько, сколько нужно чтобы на 
самом деле Мо понять~--- я, боюсь, не выпью\ldots\\
Шаман посмотрел на Регину.\\
--- Знаешь, думаю, что столько даже я не выпью\ldots

Они рассмеялись. Громко, свободно, в голос. Возможно, шутка и не стоила такого 
смеха, но зато смех развеял возникающее напряжение, смех ответил на не 
прозвучавшие вопросы, смех хоть немного смыл осадок прошедшего дня. За окном 
эхом вторила их смеху пара прохожих~--- мужчина и женщина. На мгновение пары 
увидели друг друга, но тут же отвели глаза и вернулись каждый~--- к своему. У 
каждого~--- своя история.\\
--- Интересно, как там они?~--- Спросила, отсмеявшись, Регина.~--- Как ты думаешь?\\
--- Нормально,~--- Ответил Шаман, не подозревая, что отвечает практически моими 
словами.~--- Прорвутся.\\
--- Хорошо бы\ldots~--- Регина тряхнула головой и улыбнулась Мише.~--- А мы как?\\
--- Мы?~--- Шаман ответил улыбкой.~--- Мы~--- лучше всех, мы уже прорвались.\\
--- Да,~--- Кивнула Регина.~--- Мы прорвались. У нас все хорошо.

Ее вдруг переполнило счастье. Может, вино начало действовать. Может, реакция 
после стресса. Может\ldots\ Не важно, как сказал Шаман~--- просто такая жизнь 
выпала. Она просто стала счастлива. У нас все хорошо. Мы~--- лучше всех.\\
--- У нас все хорошо,~--- Повторила она.~--- У нас все хорошо\ldots\ Знаешь, не 
хочу возвращаться.\\
--- Домой?\\
--- Никуда. И домой тоже. Ты что-то о круизе говорил? Поехали? Куда-нибудь, куда 
угодно, всюду?\\
Миха внимательно смотрел на нее. Регина вдруг заметила, что в его лице 
появилось что-то новое. В его глазах, помимо боли, помимо нечеловеческой усталости, 
подавляемой нечеловеческой волей, помимо злости и озлобленности, появилось 
что-то еще. Что-то светлое. Что-то робкое. Что-то спокойное.\\
--- Вдвоем?~--- Наконец, спросил он.\\
Регина покачала головой.\\
--- Нет. Вместе.\\
Они снова рассмеялись. Захлебываясь собственным счастьем, Регина продолжила:\\
--- Паспорт у меня с собой, значит, домой можно не возвращаться. С деньгами 
заграничные нам оформят быстро, а пока можем снять что-то. Домик, квартиру, 
номер в гостинице\ldots\ Нет! В пансионате! Никогда не жила в пансионате. А 
потом поедем. Везде! Всюду! Куда угодно! Вместе! Вдвоем!\\
Миха улыбался. Такой теплой улыбки у него она еще не видела. Да он и сам, 
наверное, никогда раньше не улыбался так тепло.\\
--- А они?\\
--- Они\ldots~--- Регина задумалась.~--- А им мы записку оставим. У них же все 
хорошо, да? Значит, прочитают!\\
--- Так все равно возвращаться\ldots~--- Улыбаясь, напомнил Шаман. Регина 
очередной раз тряхнула головой:\\
--- Никаких возвращений! Они же великие волшебники, сказочники и так далее! 
Прочитают!~--- Подмигнув Мише, она начала водить пальцем по окну.


\customsection{Путь Мауи}{melkij\_bes}{MANA~--- вся сила идет изнутри}

Мы въезжали в то ли угаданный, то ли придуманный мной город, когда на лобовом 
стекле нашего <<Урагана>> появились узоры, сложившиеся в слова. <<У нас все 
хорошо! Едем в отпуск! Понадобимся~--- свяжетесь! Квартирой пользуйтесь! Удачи вам, 
мальчики!>> И с небольшой задержкой: <<P.S. Борь, не пиши больше грустных 
историй. Не надо! Мо, спасибо за сказку!!!>> <<И вам~--- удачи!>>,~--- мысленно пожелал 
я и включил дворники. Надпись исчезла. Я покосился на спутника:\\
--- Я же говорил~--- прорвутся. Только, мне интересно: почему Сказочник~--- я, а 
за сказку благодарят тебя?\\
--- Это просто, Сказочник.~--- Мо зевнул.~--- Потому что своей сказкой ты навязал 
им жизнь. А я\ldots~--- Он дотронулся пальцем до стекла.~--- Я подарил. Не думай 
об этом, нельзя жить своей жизнью, не задевая жизни других. А тебе~--- особенно нельзя. 
Скажи лучше, что теперь делать будем?\\
--- Готовиться,~--- Я пожал плечами.~--- Не знаю пока как, но готовиться. Что нам 
остается? Нам оставалось чуть больше трех недель.


В квартиру Регины нас впустил Мо. Странно, но я почувствовал себя как дома. 
Впрочем, не странно. После путешествия в нижний мир и автомобильного 
путешествия из Литвы в Россию через квазиамериканские мосты, радужные спирали и огромные 
канализационные туннели свыкнуться с мыслью, что эта квартира~--- создана мною 
и мною же пропитана, хоть я сам здесь впервые, было легче легкого. Двуспальный 
разложенный диван-кровать, занимающий большую часть комнаты~--- такой, правда, 
по моим воспоминаниям, в сильно худшем состоянии был у моих родителей. Смятый плед 
с дельфинами~--- мой аналогичный достался кому-то в наследство, когда я покидал 
студенческую общагу. Компьютерный столик\ldots\ Его, вообще, интересно где, по 
ее версии, Регина нашла. Нам такой в <<Senukai>> завозили~--- эксклюзивная партия, 
мало и очень дорого. Но очень симпатичные~--- я долго облизывался. Жаль, даже с 
учетом полагающейся работникам скидки, позволить себе этот столик я не мог ни при 
каких раскладах. Зато у Регины стоит. Шкаф, книжная полка, комод\ldots\ Узкий 
короткий коридор с тупичом, упирающимся в входную дверь и рукавом, ведущим на кухню, 
мимо дверей в туалет и ванную. На внутренней стороне двери в ванную на крючке висит 
махровый халат.

Пока я знакомился с квартирой, Мо колдовал на кухне. Причем, судя по результату 
его действий, <<колдовал>>~--- в самом прямом смысле этого слова. Когда я 
добрался до кухни, меня ждали заставленный незнакомыми мне блюдами стол, наполненный 
пурпурной жидкостью графин и чрезвычайно довольный Моук.\\
--- Все-таки как восхитительно хорошо пользоваться тем, что у тебя есть!~--- 
Возвестил он, после того как мы утолили голод.~--- Ты представить себе не 
можешь, Сказочник, насколько не по себе мне, когда я простенький ужин не могу 
приготовить, не говоря уже о чем-то большем\ldots\\
--- Ты знаешь, я, наверное, даже представляю,~--- Я смотрел поверх Мо. Из 
вентиляционной трубы в углу стены, отодвинув решетку, выбрался напоминающий 
персонажа мультфильма барабашка. Вылез, осмотрелся, снова нырнул в трубу и, 
вооружившись ведерком и кисточкой, побежал по стене к расползающейся от потолка 
трещине.\\
--- На что ты там смотришь?~--- Мо проследил за моим взглядом и снова повернулся 
ко мне. Тщательно закрасив стену, барабашка отошел на пару шагов назад, рассмотрел 
результат своей работы, удовлетворенно кивнул. Неторопливо вернулся в 
вентиляцию, затянул на место решетку. Я улыбнулся.\\
--- На многообразие реальности, видимо. Тоже~--- что у меня всегда было, но чем я 
никогда не пользовался.\\
--- Многообразие~--- это хорошо.~--- Философски согласился Мо и придвинул мне 
блюдо с\ldots\ С чем-то.~--- Попробуй, кстати, кракозяблов в подливке утреннего света. 
Идеально, в таких условиях я их, конечно, приготовить не мог, но вполне терпимо 
получилось.\\
А! Это же кракозяблы! В подливке утреннего света! Как же я сразу не узнал! Я 
попробовал. Действительно, вкусно.\\
--- Знай наших! Один приятель на меня обижается, считая, что я пускаю на корм 
представителей ближайшего к нему вида, но при этом сам же с удовольствием их 
ест\ldots~--- Улыбнулся Мо.~--- Сказочник, как ты думаешь, ты можешь меня опять 
из этого мира выпустить? \\
--- Не знаю. Я пока очень смутно представляю~--- что и как я могу. Все, что 
получилось~--- получилось интуитивно и с твоей помощью\ldots~--- Я отодвинул 
тарелку и посмотрел Мо в глаза.~--- А ты, я так понимаю, хочешь меня своей помощи лишить, 
да?\\
После долгого обмена взглядами, Мо вздохнул:\\
--- Как тебе сказать\ldots\ Нет, я не хочу лишать тебя своей помощи, хотя я не 
думаю, что она тебе так уж понадобится. Просто, проблему этого мира я уже решил, найдя 
тебя. Теперь я хочу разобраться со своими проблемами. Имею право?\\
Мы снова помолчали.\\
--- Имеешь,~--- Наконец, согласился я.~--- Я подумаю, как это сделать.\\
Мо кивнул.\\
--- Спасибо. Кстати, я же тебе рассказывал, как с помощью Шамана на связь со 
своими выходил? Помнишь?\\
--- Угу. А что?\\
Мо поднялся из-за стола: \\
--- Ничего. Просто Шамана сейчас рядом нет, и мне интересно~--- хватит ли того, 
что ты просто знаешь, что так можно и хочешь, чтобы у меня получилось. Ты же хочешь?

Я пожал плечами вслед уходящему с кухни Мо. Выложил из кармана ключ, 
выполненный в форме музыкального, и покрутил его в пальцах. У меня есть ключ. К одной двери 
он уже подошел, но уверен~--- та дверь не единственная. Еще у меня есть ожившие 
вдруг персонажи. Еще у меня есть мир, наполненный отсылками к моему опыту. И 
Мо, который хочет уйти. И я, который должен со всем этим разобраться. И Мо, который 
уходит. То есть, на самом деле, у меня есть только я и этот я должен каким-то 
образом разобраться с собой, а разобравшись с собой~--- разобраться с некоей 
изначальной силой, когда-то давным-давно породившей мой мир и теперь заявляющей 
свои права на него. На мой мир со всем его многообразием, моими ожившими 
персонажами и отсылками к моему опыту. Вполне логично, что Мо уходит. Он лишний 
в этой цепочке. Я вздохнул. Поднялся из-за стола, взял с него одну из тарелок, 
наложил туда понемногу всего, наполнил бокал~--- и поставил бокал с тарелкой на 
холодильник в углу кухни.\\
--- Спасибо,~--- буркнуло мне вентиляционное отверстие в стене. Я прошел в 
комнату. \\
Мо сидел спиной к монитору за компьютерным столиком, откинувшись на спинку 
стула, с чувством глубокой удовлетворенности на лице. Увидев меня, улыбнулся:\\
--- А знаешь~--- сработало!\\
--- Рад за тебя,~--- Я улыбнулся в ответ. Вполне искренне улыбнулся, хочу 
заметить. \\
--- Так ты уходишь? Если решил, то давай уже не тянуть. У меня, знаешь ли, дела, 
а я все еще не знаю с какой стороны за них браться\ldots\\
--- Решил,~--- Мо поднялся со стула.~--- К тому же у меня тоже не слишком много 
времени и вряд ли Сеера решит мне помочь еще раз. А ты\ldots\ По-моему, ты 
знаешь все, что тебе нужно знать.\\
--- Посмотрим.~--- Я не стал спорить. Вместо этого осмотрел комнату в поисках 
подходящей замочной скважины. Дверь на балкон, секции компьютерного столика, 
бар\ldots\ Как их много, оказывается. Интересно, они все были изначально, или 
это я их захотел? Неважно. Шкаф. Шкаф подойдет. Под взглядом Моука прошел по комнате к 
шкафу, вставил ключ, повернул и, не глядя, открыл дверцу:\\
--- Добро пожаловать в Нарнию!~--- Не удержался. Мо, уже одной ногой в шкафу, 
остановился:\\
--- А это цитата? Она мне не знакома\ldots\ Все-таки не то тело я выбрал. Удачи, 
Сказочник!~--- И исчез за парой платьев Регины и длинным черным пальто в 
оранжевую клетку. Я закрыл дверцу. Платья, пальто, деловой комплект пиджак-юбка. Мо в 
шкафу не было. Я вернул ключ в карман и сел за компьютер.

\newpage 

А Моук, пройдя по темному, пахнущему сухим деревом туннелю, выбрался на 
солнечную зеленую лужайку. Жужжали насекомые, щебетали птицы, ветер шевелил 
листву и где-то вдалеке журчал ручей. Все это сливалось в незнакомую, но 
неуловимо привычную мелодию.\\
--- Здравствуй, мир!~--- Сказал Мо и рассмеялся.~--- Здравствуй, мой хороший!

В небе припекало летнее солнце. Мо дотянулся до него, почесал ему бок, 
взлохматил лучи и, дождавшись, пока оно блаженно заурчит, накрыл облаком. Стало 
прохладнее. Мо снова расхохотался~--- и исчез. Впрочем, нет, не исчез~--- он 
оставался и оставался в этом мире. Просто его не было в конкретном месте здесь, 
он был везде, он слился с этим миром, он был им. Невероятное ощущение. И один 
Сеера может сказать сколько и в каком исчислении Мо его не испытывал. Он знал, 
что соскучился по этому чувству, знал, что ему не хватает его~--- но до этого 
момента не представлял насколько. Все сравнения, все объяснения, все 
описания~--- на самом деле ничего не объясняли и не описывали. Нельзя передать неописуемое. 
Просто нельзя.

Забыв о возможном риске или, точнее, откинув мысли о нем, немного пошалил. 
Поменял форму паре островов, поигрался с генами местных жителей, превратив 
рыжих в блондинов, сровнял с землей пару гор и возвел пару новых. Мир податливо млел, 
ласкаемый волей Мо.

Времени действительно было мало. Мо напоследок приласкал мир и попытался найти 
его координаты. Странно, получалось, что этот мир находится где-то вне 
известного Моуку множества миров. Действительно, что ли?.. Решив, что сейчас не 
время для научных загадок, Мо попробовал нащупать путь к Кладре без координат 
отправной точки. В конце концов, Берсу однажды удалось что-то подобное. 
Некоторое время превратившееся в стрелку компаса сознание Мо металось между 
разными притягивающими направлениями, но все-таки удалось нащупать то, откуда 
тянулась золотая дымка, пахло домом и правильностью. Отдался ему.

Этот трактир стоял на отшибе Золотого Города, каковой и был всем миром Кладры, 
веками. Уже многие поколения местных жителей считали, что этот трактир стоял 
тут всегда, но те из гостей города, что сами себя называют <<странниками>>, знали, 
что это не так. Уже многие поколения местных жителей уверенно сказали бы, что даже 
в памяти самого древнего из их предков этот трактир был <<заброшенным трактиром>> 
и вряд ли вообще когда-то был открыт, но и в этом они ошибались. Двери трактира 
за всю историю его существования распахивались неоднократно, правда, вряд ли 
кто-нибудь смог бы высчитать твердый график работы этого заведения. Бывало~--- 
и трактир оставался открытым десятилетиями и всегда за стойкой стоял улыбчивый, 
похожий на всех своих предшественников, если они у него были, хозяин, и всегда 
находилась свободная комната, отвечавшая запросам любого постояльца, и всегда 
готовился ужин, отвечавший толщине любого кошелька. Бывало~--- и гас очаг в 
кухне трактира и не пощелкивали ярким огнем факелы на стенах общего зала, и не 
поднималась затворка на воротах, хоть и не было кроме нее никакого замка, и 
проходили местные жители по своим делам почти не замечая это заброшенное 
здание, а странники, изучив ворота, кивали понимающе и исчезали по своим странным 
делам. И так было, и так бывало. А теперь было иначе. Не было запирающей ворота балки, 
но и сами ворота не распахнулись широко, оставшись чуть приоткрытыми. Не 
дымилась весело труба, но тонкой, чуть заметной в золотом отсвете воздуха, 
струей стелился над трактиром дымок. Не горели факелы, но чуть теплились 
угольки в камине. И самое удивительное~--- хотя столы были заставлены яствами и 
напитками, самим трактирщиком даже не пахло.

Скрипнули ворота, вторила скрипу входная дверь и в полумрак общей залы вошел 
первый посетитель. Был он высок, костляв и человекоподобен. Из-за спины у него 
торчала рукоять меча, кончик которого постоянно бился ему при ходьбе в лодыжку. 
Но он привык. Вошел, осмотрелся, сел за один из столов. Вздохнул.\\
--- Ибитуте,~--- Вслух произнес он.~--- Сколько же лет прошло\ldots\\
--- Считать ты, Чесно, надо понимать, так и не научился. Зато с антиквариатом 
своим до сих пор расстаться не можешь,~--- Язвительно ответила ему, вообще 
ответа не ждавшему, пустота. Вгляделся в нее. Красноватый от тусклого мерцания углей 
полумрак у одного из столов наливался синевой, словно чернильницу кто-то разлил 
на пол, а содержимое ее на жаре испаряться начало. Тот, кого назвали Чесно, 
беззлобно усмехнулся:\\
--- Раз такой умный, Инди, сосчитай~--- сколько раз моя железяка твою жизнь 
спасала?\\
--- Индиго, с твоего позволения.~--- Пятно под столом сжалось и по ножке 
переползло на спинку стула.~--- А вообще, рад тебя видеть, Чесно!

Собравшемуся ответить что-то пятну по имени Индиго Чесно помешал грохот 
распахнувшейся двери. Следующий посетитель не церемонился. Дверь открыл широко, 
сам во внутрь вошел грузно и шумно. И закричал громко:\\
--- Мо! Старина! Наконец-то!~--- Но, обведя зал глазами, обескураженно сбавил 
тон.~--- А где он?\\
--- Не знаю\ldots\ Но раз позвал~--- появится,~--- Первым отозвался Чесно, а 
Индиго лишь колыхнулся, соглашаясь с его словами.~--- Садись, Рафа, в ногах правды нет.

Пришедший хмыкнул, но спорить не стал. Выбрал стол, устроился за ним, 
придирчиво оглядел его содержимое и подтянул к себе одно из блюд.\\
--- За встречу, дорогие!


Зал наполнялся. Пришел Ноябрь, обдав собравшихся осенней прохладой, и устроился 
рядом с камином, подставив теплу свой вечно промокший плащ. Разогнал мрак 
святящийся МоняЛи. В обнимку с двумя феями~--- Мелиренией и Дажоли~--- появился 
симпатичнейший выходец мира Кичирику Обаятельный. Смешно дергая коленями, 
мелкими прыжками в залу попал напоминающий гигантского кузнечика Кракозябл, а 
вслед за ним~--- пара элементалей Каподастр и Шокепестр, Кап и Шок. Последним 
прибыл Кинич-Ахау. Удивленно взмахнул хвостом:\\
--- Надоу же~--- весь оурден в сборе. А самоуго виноуника торжества не видно? Я 
почему-то так и думал, что он последним явится\ldots


Старые знакомые здоровались, обнимались, шутили, делились новостями, и трактир, 
прислушиваясь к гулу внутри себя, напряженно дрожал. Ему было страшно. 
Многовековое, не одно тысячелетие пережившее существо, много раз смотревшее в 
глаза смерти, не понаслышке знакомое с небытием дрожало от страха. Проведшее 
весь, пусть и бесконечно краткий в масштабе остальной жизни, период своего 
человеческого существования в компании брата и почти не вспоминавшее о нем в 
дальнейшем, утверждавшее и, что самое главное, всегда верившее в то, что 
странники~--- одиночки по своей сути существо сейчас до одури боялось, что те, 
кого оно бездумно привыкло называть друзьями, откажут ему. Страх обмануться в 
дружбе~--- неожиданный и неприятный страх~--- кольнул его после встречи с 
Берсом, заставил бросить Сказочника ради этого собрания--проверки. И сейчас существо, 
никогда не переживавшее о правильности уже принятых решений, тщательно гнало от 
себя мысль о том, что без этой встречи старых друзей можно было обойтись. 
Справились бы и вдвоем со Сказочником. Сказочник и один справится. Это его мир. 
Это его жизнь. Так, может, действительно~--- плюнуть, явиться наконец перед 
ними, поднять бокал за встречу, поделиться открытиями и отправиться дальше? 
Приключений во вселенной всегда хватит\ldots

Моук перестал быть трактиром. За стойкой появился трактирщик Мо. В этот же миг 
на дальней стене общего зала затанцевали разноцветные тени, а сам зал 
наполнился негромкой разухабистой музыкой. Разговор собравшихся прервался на полуслове, 
все повернулись к Мо. Только Кинич мяукнул себе под нос:\\
--- Пижон\ldots\\
Мо улыбнулся.\\
--- Здравствуйте, дорогие!..

\newpage


Следующие дни я потратил на подготовку. Со стороны, конечно, могло казаться, 
что я просто бездельничаю, но~--- кто и с какой стороны мог компетентно критиковать 
меня? Разве что Мо и его всемогущие знакомцы, но Мо не было рядом, а Берс и 
прочие хозяева Вселенной, если даже и следили за мной, делиться своим мнением 
не спешили. Да и права не имели. Меня такая ситуация устраивала чуть больше чем 
полностью. Я собирался спать, гулять, читать и, конечно, писать.

Сон, впрочем, помогал слабо~--- никаких откровений мне во сне не явилось, да и 
вообще~--- ничего слишком осмысленного. Разве что~--- музыка. Повторяющаяся из 
ночи в ночь мелодия~--- я уверен, не слышанная мною ранее. Чтение особой пользы тоже 
не принесло. Оно, конечно, крайне полезно для общего развития, но все то, что я в 
эти дни читал про теорию хаоса, оставалось у меня в голове только той общей 
информацией, что я и так владел. <<И грянул гром>>, на мой взгляд, вполне 
достаточный источник для изучения <<эффекта бабочки>>, а применение теории 
хаоса для изучения эпилепсии вряд ли могло мне чем-то помочь. Впрочем, того, что я 
смог понять своими далекими от математики с физикой мозгами, хватило чтобы 
прийти вот к какой мысли. Если верить всем этим ученым~--- Хаосу совсем 
необязательно забирать наш мир из лап Берса и его упорядоченной вселенной. Он и 
так проник в структуру нашего мира и, с некоторыми натяжками, большинство 
систем так или иначе опирается на него. Интересно~--- как хаос может быть системой? И 
правы ли люди, разделяющие хаотические процессы и случайные? И вообще~--- 
относится ли все это к тому Хаосу, который сейчас интересен мне?

Еще я пытался разобраться с историей этого города. Нет, все было красиво, но 
оставляло поле для вопросов. Город основан в 1567 году. Нормально, старше, 
между прочим, многих литовских городов. Но, если считать первую единичку данью 
реалистичности~--- не может ведь какой-то провинциальный город, почти не 
влияющий на большинство процессов государственного значения~--- не слишком ли надуманный 
дальнейший числовой ряд? Если верить Википедии, то в этом 1567 году ничего 
слишком интересного не происходило, да и из всех городов, кроме этого, основан 
был только чилийский Кастро. При этом~--- все как полагается: в этом городе 
останавливались наполеоновские войска. Здесь, опять же, до конца двадцатых 
годов двадцатого века скрывался от красных один из генералов царской армии граф 
Алексей Николаевич Пореченков. Пореченков? Алексей Николаевич? Осужден по 
одному из бесчисленных шпионских процессов? Тоже совпадение? Возможно, у этого 
дворянского рода, как оказалось, долгая и славная история еще со времен Иоанна 
IV. Актер к этому семейству отношения не имеет. Допустим. В окрестностях этого 
города проходили бои во время Второй Мировой Войны, они даже упоминаются в 
некоторых редакциях российских учебников по истории. Упоминаются, проверял. И 
среди именитых выходцев этого города я тоже обнаружил пару знакомых мне 
фамилий. И еще~--- здесь, оказывается, какое-то время жил Бродский. Дом, в котором он 
жил~--- снабженный табличкой, все как полагается, оказался очень похожим на <<дом, где 
жил Бродский>> в Вильнюсе~--- но ведь и это ничего не значит? И вообще~--- этот 
город мелькал иногда в новостях, о нем когда-то упоминало Дельфи, я знал о 
существовании этого города задолго до того как в моей жизни появились Мо, Шаман 
и Регина. Хорошо, двое последних вообще появились по моей воле, пусть так. Но 
не мог же только из-за того, что им нужно было где-то жить, появиться целый город? 
А если мог, то, значит, здесь же должны обитать и другие обитатели моих 
фантазий. Например, герой одного из преследующих меня с юности сюжетов, Дима 
Огнев~--- он, вроде, когда последний раз мои мысли направлялись в его сторону, 
покидал Литву и попадал на должность учителя литературы в какой-то русский 
интернат. Есть здесь интернаты? Интернаты были, но следов Димы Огнева я не 
обнаружил.

Зато, гуляя по городу, я находил многое другое, не менее интересное. Как-то на 
светофоре рядом со мной стоял снежный человек в длинном пальто, надетом поверх 
делового костюма. Из лужи, на обочине дороги, однажды вынырнула пиранья. Среди 
пяти- и двенадцатиэтажек в новостроящемся районе возводили башню, вид которой 
вызывал в памяти сказки и мультфильмы о заточенных злобными колдуньями 
принцессах. Вообще, оказалось, что мир одновременно выполнен и как мультик, и 
как компьютерная игра, даже в виде раскадровки в духе графических новелл. 
Просто обычно мы этого не видим. Как-то я попытался увидеть мир в <<стиле Нео>>~--- не 
получилось. Наверное, потому что я ничего не смыслю в программировании. Вместо 
этого, я увидел его сплошным потоком текста, точнее~--- обрывочных текстов. 
Забавно. Была идея стереть пару случайных предложений и посмотреть, что из 
этого выйдет, но не стал. Зато когда присыпанный первым снегом городской парк 
предстал мне масштабным, выполненным красками, полотном, я все-таки поработал ластиком. 
За первым слоем краски скрывался пейзаж того же парка, но летом. Яркие 
насыщенные зеленые краски, грубые коричневые мазки деревьев\ldots\ Я стер и 
этот слой.
 
За ним\ldots\ Наверное, это был все тот же парк, но уже в духе 
абстракционистов. Во всяком случае, я так понял. Стерев и это, выяснил, что изначальным полотном для 
картины послужила грифельная доска. На ней до сих пор сохранились белые узлы 
букв. Я даже смог разобрать: <<Тема сочинения: Мой лучший выходной>>. Остаток 
ластика в руке закрошился кусочками мела. Я нарисовал на доске веселую рожицу и 
пошел дальше. Когда в следующий раз я проходил мимо этого парка~--- дети в нем 
активно лепили полчища снеговиков.

Где-то к началу второй недели заглянул, проходя мимо, в маленький антикварный 
магазинчик. Какие-то <<пережившие революцию>> брошки и ожерелья, несколько 
икон, картины, пара полок с книгами, посуда и декоративные игрушки. Постоял возле 
книг, ни нашел среди них ничего для себя интересного и уже собирался уходить, 
как вдруг меня остановила знакомая мелодия. Та самая, которую, остановившись в 
квартире Регины, я постоянно слышал во сне. Я огляделся. В паре метров от 
книжного раздела, если так можно назвать те две полки, девушка, такой же 
случайный посетитель, видимо, поставила на место музыкальную шкатулку и 
направилась к выходу. Моя очередь.

Шкатулка мне показалась странной. Не думаю, что пальмы, море и песок~--- 
стандартные сюжеты для разрисовывания вещей подобного рода. Она была украшена 
именно в таком духе. Открывалась она просто поднятием крышки, при открытии~--- 
выдавала ту самую привлекшую меня мелодию. При этом зачем-то на верху крышки 
было отверстие для ключа. Самого ключа рядом со шкатулкой я не нашел. Спросил у 
продавца.\\
--- А вы знаете,~--- Ответил мне он.~--- Его и не было. И вообще, насколько я 
понял, эта дырка тут~--- чистой воды декорация. Сначала я думал, что раньше в ней был 
врезан замок, но посмотрите~--- никаких следов\ldots

Как называется мелодия шкатулки, он тоже не знал. Шкатулку я купил. Поставил ее 
дома на компьютерный столик и даже вставил свой <<скрипичный>> ключ~--- он 
вполне успешно вписался в отверстие. Поворачивать, правда, не стал: чувствовал, что 
время еще не пришло.

Возвращаясь домой, я обычно доставал из кухонного шкафчика бокал <<Varniukai>> 
и шел на балкон. К таким мелочам, как возможность доставать из шкафчика любимое 
разливное пиво я привык очень быстро. Местное мне не понравилось и такие 
приятные бонусы позволяли мне не скучать по литовскому пиву. Впрочем, если бы я 
и скучал по чему-то, то только и исключительно по нему. Сидя на балконе, я 
размышлял о том, что чувствую себя здесь и сейчас очень комфортно. Конечно, 
появлялись мысли связаться с Арвидасом и Игорем, сообщить о себе на работу, 
но\ldots

Я знал, что, когда в первый день после выходных не вышел, меня сразу же 
уволили. 

Знал, что у Арвидаса очередной пик его отношений и обо мне за это время он ни 
разу не вспомнил, знал, что и Игорю не до меня~--- сдвинулся с мертвой точки 
его давний проект собрать команду и написать игру. Только в дела Регины с Шаманом я 
не вдавался, признав правоту слов Мо о <<навязанной жизни>>. Раз так 
получилось, то пусть теперь поживут без моего присутствия в их судьбах. Пока могут. Но даже 
не концентрируясь на них, я чувствовал~--- и у них все хорошо. Такой мир мне 
нравился. Пусть все дорогие мне люди были далеко, но у них все было в порядке. 
А мне было хорошо здесь~--- на балконе, с бокалом пива, рассматривая ночь. 
Интересное дело, город окончательно погрузился в раннюю зиму, а открывавшиеся 
мне со своего места ночные пейзажи каждый раз менялись. Причем менялось не 
только время года~--- менялся рисунок созвездий, очертания близлежащих домов, 
которые иногда вообще переставали быть домами, количество лун. Мир был 
многообразен, и это было хорошо. Мир был~--- и это было прекрасно. И я должен 
был сохранить его. Во всем его многообразии. Не зная, наверняка, как это 
сделать~--- я стремился хотя бы записать его. Я возвращался в комнату.

Комната тоже дарила сюрпризы. Правда, в этих сюрпризах я находил отголоски 
рефлексий Пруста, но даже если так~--- какая разница. Мучительное стремление 
написать что-то не эпигонское, не подражательное, что-то свое~--- оставило 
меня. 

Какая разница? Я всю жизнь мучился поиском себя, я писал и графоманил потому 
что это казалось ближе всего к тому направлению, где ждет меня смысл и 
самореализация, но теперь я знал. Я не писатель, я~--- Сказочник. Сказки не 
должны быть оригинальными. Они просто должны быть. Должны быть в мире, давая самому 
миру силы быть. Многообразие миров, многообразие мира, многообразие 
комнаты\ldots\

Хаос, дарующий систему. Хаос, спасающий от хаоса. Все это не нужно создавать. 
Оно есть. Все это нужно сохранить. Хотя бы записать. Хотя бы мне.

Двумя маленькими лунами безграничного мирка, отражая преломляющийся в атмосфере 
окна солнечный свет уличных фонарей, горят глаза-кнопки плюшевого мишки. Мишка 
висит на шкафу. Высоко. Далеко.  Днем это было далеким глубоким космосом, а 
самого Мишку звали Тэдди и пришел он из другого мира. Мира, окно в который 
открывается лишь на два часа в день, но зато потом волшебным образом тот мир 
становится частью этого: своего, ограниченного, но бескрайнего мирка.  Мирка, 
где сейчас мишка, лишенный всех своих имен и ипостасей, лишенный темнотой 
своего облика, висит на шкафу, почти не существуя, оставаясь лишь за счет 
глаз-кнопочек, светящихся двумя лунами.

Две луны совершенно уместны в этом мирке, пространство которого, обрамленное в 
чужих глазах четырьмя стенами, способно выливаться из рамок, следуя за 
фантазией своего обитателя и и свертываться до пределов одной лишь кровати вопреки его 
воле. Сужаться, приближая все чужое, неизведанное, страшное; подпуская все то, 
что должны сдерживать четыре стены, такие незыблемо-бетонные и такие зыбкие.

Шум ветра, напряженная дробь веток о подоконник, смазанные, непонятные голоса 
чужих людей~--- людей ли?~--- все это протекло в форточку, скопилось, 
столпилось совсем рядом. Стоит за спиной темным, уклонившимся от света всех возможных 
здесь лун, темнее углов комнаты, сгустком враждебного. В его очертаниях можно найти 
силуэт одежного шкафа, но чтобы разглядеть это нужно перевернуться, отречься от 
защитного одеяла  мирка, отказавшегося уже и от большей  части кровати, 
забившегося в угол между сбившейся в ком  подушкой и единственным оставшимся 
своим участком стены. Маленький вырванный у тьмы фрагмент обоев~--- таких 
обычно незаметных и таких привычно знакомых. Отвернуться от них, отказаться от него 
ради призрачной возможности взглядом изгнать застывших за спиной призраков в 
шкаф, чтобы навеки~--- до утра запечатать его семью, а лучше одиннадцатью 
волшебными печатями~--- нет, решиться на это невозможно. Риск слишком велик, а 
возможность успешного исхода слишком мала. Пугающе мала.

Есть другой способ. Более действенный и более опасный. Встать и включить 
солнце. 

Надежное, проверенное, загорающееся по первому требованию трехламповое солнце 
разгонит тьму. Оно укрепит своим питательным светом источившиеся стены, вернет 
мирку целостность. Но чтобы зажечь его нужно встать, пройти в другой конец 
комнаты, прокрасться вдоль линии фронта, проскакать по  полям, прорубиться 
через джунгли, проплыть океан. И все это время за спиной\ldots

За спиной морской рябью морщинится ковер. Кровать протекает, держится на нем из 
последних сил. Корабль терпит крушение и единственный выход~--- встать с него. 
Встать и пойти по волнам.Стать богом. Спасти свой мир как подобает богу. 
Вечному. Вечному сыну. Вечному ребенку.

Да будет свет!

В последний день перед пенсией электрик Леонид Васильевич Римский, копаясь в 
распределительном щитке второго подъезда дома номер 57, упал со стула. Падая, 
ухватился за панель с предохранителями и оставил без электричества весь 
подъезд. Сам он утверждал, что его напугал спускающийся с верхних этажей мужчина, 
неожиданно его окликнувший. Случайность.

Оставшись без интернета, работник call-центра крупной телефонной компании 
Максим Свиридов, взявший отгул по причине похмелья после вчерашней пьянки, впрочем, 
представленная начальству версия гласила, что причина~--- пищевое отравление, 
вызванное съеденной вчера в обед шаурмой, решил сходить за <<лекарством>>. 
Купив в ближайшем супермаркете четыре пол-литровых бутылки <<Балтики>>, вышел из 
магазина. На улице с просьбой <<прикурить>> к нему сразу же обратился случайный прохожий. 
Протягивая ему зажигалку, Макс с удивлением заметил, что мужчина курит тонкие 
сигареты. <<Слимка>> напомнила ему о Марине. Марину вчера вечером привел к 
Максу Славик, она была одноклассницей последнего и сейчас они случайно встретились в 
одном вагоне метро. <<Сейчас>> было часа за три до их появления на пороге Макса 
и этого времени одноклассникам вполне хватило для того чтобы поделиться 
рассказами о годах после школы, согреться, зацепиться и прийти к мысли о продолжении 
банкета. Почему местом <<продолжения>> Славик выбрал квартиру Макса, Макс не 
знал. 

Но курившая тонкие сигареты Марина оказалась женщиной остроумной, 
привлекательной и свободной. К тому же жила она в том же районе. И, если Макс 
правильно запомнил вчерашние детали, находилась сейчас в отпуске и приглашала 
заходить на огонек. Проводив взглядом мужчину, Макс направился в 
противоположенную дому сторону.

Приняв клиента, таксист дядя Петя двинулся в направлении указанного адреса. По 
радио <<Шансон>> пел Кричевский и, выезжая со двора, дядя Петя вполголоса 
подпевал <<Киевлянке-киявлянке-киевляночке>>. Разрешив таксисту допеть дуэтом с 
исполнителем, пассажир попросил сменить станцию. Пожав плечами, дядя Петя нажал 
кнопку поиска. Короткая тишина и в машину ворвалось: <<\ldotsСоколовой день 
рожденья, ей сегодня тридцать лет!>> Дядя Петя очень давно не слышал этой песни, а 
когда-то, пригласив под нее на танец, Натусика он сделал ей предложение\ldots\
Покосившись через зеркальце крайнего вида на пассажира, таксист потянулся к 
тумблеру громкости. Подняв через секунду глаза обратно на дорогу, с чертыханием 
дернул руль. Молодой мужчина с мешком в одной руке и открытой бутылкой пива в 
другой, решивший вдруг перейти дорогу в неположенном месте, уловил движение и 
отпрыгнул в сторону. Проезжая мимо него, дядя Петя усмехнулся, глядя как Макс 
смешно размахивает руками, стараясь на льду сохранить равновесие. Так ему и 
надо.

На побережье Калифорнии обрушился никем не предсказанный и никем не ожидаемый 
торнадо. В ответ на претензии синоптики только разводили руками.

Мелодия в моем сне начала распадаться. Я проснулся. Не торопясь умылся, выпил 
на кухне чашечку чая, заел ее бутербродом и, вернувшись в комнату, повернул ключ. 
Пора. Оставалось открыть шкатулку.\\
\\
<<Сказочник, что происходит?>>~--- Неожиданно раздался у меня в голове голос 
Шамана. Я был рад любой задержке.\\
<<Началось, по-моему. Ты тоже почувствовал?>>\\
<<Да\ldots>>~--- Шаман помолчал.~--- <<Но ведь сегодня только 
19-ое\ldots\ Вроде>>.\\
Я пожал плечами:\\
<<Не стоит, по-моему, требовать чрезмерной точности от будильника, заведенного 
столько лет назад. И без того\ldots>>\\
<<Обосрались фраера, короче>>~--- Шаман не был склонен прощать майя ошибку.~--- 
<<Сможешь меня вытянуть или мне самому силы тратить>>?\\
<<Смогу\ldots>>~--- Я задумался.~--- <<Но зачем? Если Мо был прав и я должен 
защитить наш мир, то я должен и сам справиться. Если нет\ldots\ Наслаждайтесь последними 
днями\ldots>>\\
<<Сам?>>~--- Переспросил Шаман.~--- <<А Мо где?>>\\
<<Ушел>>,~--- Я не стал вдаваться в подробности.\\
<<Слился, значит\ldots\ Странно, он казался надежным\ldots >>~--- Я отчетливо 
услышал вздох Шамана.~--- <<Ладно, Сказочник, не ссы~--- прорвемся. Давай, дергай меня к 
себе, а то у меня от этих разговоров уже башка трещит>>\ldots\\

Это сюда из Вильнюса мы ехали на машине. Да и тогда я подозревал, что на самом 
деле все может быть проще. Сейчас-то я знал, что все действительно проще. 
Всего-то делов~--- совместить в пространстве квартиру Регины и номер в каком-то 
Карибском отеле, выключить из возникшего помещения Регину и пришедшую к ним с 
завтраком горничную, взять Шамана и вернуть квартиру на место. И все это нужно 
проделать только в голове. Мир таков, каким я его вижу. \\
--- Привет, Миха. Хорошо выглядишь.\\
--- Здравствуй, Сказочник.

Мы открыли шкатулку. Мелодия в этот раз звучала дребезжаще, как будто ее 
исполняли по-прежнему старательно, но на расстроенных инструментах.\\
--- Что это?\\
--- Если я что-то понимаю~--- место, которое Моук называет сферами мира. А 
звук~--- это, соответственно, музыка сфер.

Квартира исчезла. Исчезло, в общем-то, все~--- место, где мы находились, не 
было похожим ни на что мне известное. Голубая пустота, пронизанная множеством 
искрящихся струн. И больше ничего. Не было даже нас, хотя мы-то точно были тут. 
Зато, не чувствуя себя, я чувствовал каждую струну в этом странном инструменте, 
мог коснуться каждой из них, сыграть на них, натянуть или, наоборот, ослабить. 
И что самое главное~--- я знал как это делать. И сделал.


Звучание наладилось, а место, и вместе с ним~--- мы, приобрело форму. Мы стояли 
на сцене в маленьком концертном зале. Такие бывают в некоторых подвальных 
кабаках~--- тех, которые позиционируют себя как культурные или андеграундные места и в них 
время от времени дают концерты начинающие или просто малоизвестные группы. На 
сцене были расставлены инструменты, вокруг~--- стояли стулья, а у 
противоположенной стены располагался бар. Но людей нигде не было. Только мы.\\
--- Так лучше,~--- Осмотревшись, решил я.~--- Устраивайся.\\
Сам я сел на высокую деревянную табуретку. Шаман пододвинул к себе такую же.\\
--- И что теперь?\\
--- Первый бой мы выиграли,~--- Я пожал плечами.~--- Ждем. У тебя за спиной по 
синтезатору таракан ползет. Будь добр, прихлопни его.
Шаман так и сделал, успев убить насекомое до того как оно оказалось на самих 
клавишах. Музыка продолжала звучать ровно. Развернувшись на стуле, поднялся и 
пошел от барной стойки через ряды зрительских стульев к нам человек. Мужчина, 
но я не берусь подробнее описать его внешность. Он выглядел по-разному. Попробуйте 
представить себе, что кто-то взял и наложил друг на друга вырезанные из фото- 
или кинопленки все найденные им там фигурки мужчин. А потом проявил, или 
загрузил получившееся в прожектор. И вы смотрите на эту фигуру и видите в ней 
одновременно всех.

Он шел на нас. Он улыбался нам, во всяком случае~--- по большей части. 
Остановившись у самой сцены, он заговорил. Голос был под стать облику. То есть, 
это не был голос~--- это был нестройный хор голосов, каждый из которых выражал 
мысль, которую хотел озвучить этот мужчина, по-своему.\\
--- Здравствуй, Сказочник,~--- Сказало он, и по залу эхом разнеслось: <<Привет, 
Боря>>, <<Как дела, дорогой?>>, <<Ну, ты как вообще?>>, <<А, это ты\ldots\ А с 
тобой кто?>>, <<Ну, ты влип парень>>~--- и многое другое, чего я уже просто не разобрал.\\
--- Здравствуй,~--- Ответил я, удержав зашатавшуюся вдруг ударную установку.\\

Эмиссар хаоса, давайте называть его так, долго смотрел на меня. Потом 
расхохотался.\\
--- Ты готов к бою? Ты думаешь, я буду с тобой драться? Мне это не нужно, 
Сказочник. Я просто возьму твой мир, пока мы с тобой будем тут мило беседовать. 
А потом уйду. И уйду. И я уйду. А ты сможешь оставаться, изображая и дальше из 
себя человека--оркестр\ldots

Я его понял. А поняв~--- постарался сдержать радость. Всем, наверное, знакомо 
это чувство, когда тебе известно, что некто собирается сделать тебе гадость, и ты 
гадаешь~--- когда и какую, решаешь для себя, что, скорее всего, он сделает то 
или это, готовишься к ответу, оттачиваешь фразы и позы, с которыми ты его 
обломаешь, а потом~--- бац, и оказывается, что он сделал совсем другое. И ты стоишь в этой 
гадости, раскрываешь рот, но сказать ничего не можешь, от неожиданности 
растеряв все слова. Гадкое чувство. А вот когда оказывается, что ты угадал удар и готов 
к нему\ldots\ Я старательно сдерживал радость. Медленно-медленно поднял на него 
глаза, стараясь, чтобы мой взгляд выглядел испуганным и смирившимся.\\
--- Забирай.

Дернулся Шаман, но не успел ничего ни сказать, ни сделать. Эмиссар мигнул.

Где-то далеко огромная взбудораженная цунами волна накрыла японские острова. 
Где-то далеко жители Японии продолжали заниматься своими делами, ехали на 
работу, встречали знакомых, провожали уплывающих в круизы родственников. Над 
ними светило солнце.

Из общей мелодии выбился мотив одной флейты, но, спохватившись, невидимый 
флейтист вернулся в общий строй. Эмиссар, подрагивая, словно разрываясь в 
нескольких направлениях, внимательно смотрел на меня. Ничего не случилось. 
Только труба на этот раз дала петуха, но тут же исправилась. Я улыбнулся. 
Медленно--медленно, стараясь, чтобы моя улыбка выглядела доброй. Это 
замечательное чувство~--- оказаться подготовленным к удару. Восхитительное.\\
--- Не получается? И не получится. Этот мир слишком многообразен. Он состоит из 
множества миров. Эти миры перетекают один в другой, накладываются друг на 
друга, смешиваются и сливаются. И все они~--- этот мир. И все они~--- я. Ты не сможешь 
просто забрать его, извини. Потому что, пока есть я~--- в какой-то из комнат я 
обязательно зажгу свет, разгоняя ночные кошмары. Пока есть я, в каком-то из 
городов я обязательно подниму глаза к небу и увижу цвет звезд. Извини, но не 
получится.\\
--- Хорошо,~--- Эмиссар кивнул и по залу раскатилось: <<Убедил>>, 
<<Договорились>>, <<Как скажешь>>, <<Ты уверен?>>\ldots~--- Раз ты так хочешь, то я убью тебя и возьму 
этот мир.

Восхитительное чувство, да. Только не понятно чему я радовался. Все хорошо, он 
сначала убьет меня, а потом захватит мой мир. И как мне ему противостоять? Что, 
пулять файерболами? Я попробовал. В этот же миг дернулся, делая что-то свое, 
Шаман. Миг спустя Шаман отлетел со сцены и впечатался спиной в стену, я 
оказался примотанным к стоящей тут же виолончели, причем примотанным ее же собственными 
струнами, а эмиссар поднимался на сцену. Поднялся. Один шаг. Другой. 
Поскользнулся, поставив ногу на какую-то синюю лужу и, не удержав равновесия, 
упал. В этот же миг, опутавшие меня струны разрубил чей-то меч. В этот же миг с 
пола зала Шамана поднимал, приговаривая: <<Нормально, Миха, тебе еще повезло>> 
Мо.

Я огляделся. Спасший меня меч держал в руках высокий худой мужчина, в котором 
только при очень тщательном рассмотрении можно было узнать представителя 
нечеловеческого, хоть и какого-то очень близкого этому, рода. Гигантский 
кузнечик в это же время старательно и сноровисто перевязывал струны виолончели. 
Женщина в длинном белом платье встала напротив поднимающегося эмиссара и, сняв 
висящий над головой нимб, направила круг на него. Из круга, замедляя движения 
эмиссара, ударил свет. Еще одна женщина, вообще не озабоченная вопросом одежды, 
начала наигрывать на синтезаторе, попадая при этом в общую, до сих пор звучащую 
мелодию. Я чувствовал, что ее игра каким-то образом помогает ее приятельнице.

Убедившись, что Шаман в порядке, ко мне подошел Мо:\\
--- Прости, Сказочник. Я действительно не знал согласятся ли мои друзья нам 
помочь, а давать пустые надежды\ldots\\
--- Они все~--- твои друзья?~--- Эмиссару почти удалось встать, но вновь 
метнувшееся ему под ноги синее пятно, вернула ситуацию в исходное положение. Впрочем, 
насколько я понимал, всем им удавалось только на какое-то время затруднить 
движения эмиссара, но не нанести ему хоть какой-то ощутимый вред. Вряд ли это 
будет длиться долго.\\
--- Не все.~--- Мо покачал головой.~--- Кто-то связан обязательствами перед 
Берсом и поэтому не может участвовать в приключении. Элементали всегда предпочитали 
действовать на материальном уровне, а не уровне сфер, так что их тут нет. И 
Моня с Ноябрем сейчас ведут бои на своих полях. Думаю, если бы не их усилия~--- Хаос 
уже давно раскидал нас.

Словно в подтверждение его слов, эмиссар пошевелился и вдруг распался. Он 
перестал быть множественным~--- его стало множество. И пока один из этого 
множества поднимался с пола, остальные рьяно кинулись в бой. Часть их, будто 
решив преподать мне урок, поливало моих союзников пламенем, часть окружало 
синее пятно, с помощью каких-то странных штыков мешая тому двигаться. Часть вступило 
в бой на мечах со спасшим меня почти человеком. Часть бросилась на сцену, но была 
перехвачена Мо и оставившей клавиши обнаженной девицей. Вокруг меня кипел бой, 
а я, оставшись за его пределами, понимал, что мне отпущено не так много времени. 
Срочно нужно было найти выход. Эмиссар непобедим, потому что он хаос. Эмиссар 
плевать хотел на любые законы физики, логики, внутреннего построения сюжета, на 
причинно-следственные связи и пространственно-временные ограничения. Потому что 
он хаос. Он разрушает любую систему, потому что он хаос. Хаос не может быть 
элементом системы. Бой вокруг меня никак не сливался в одну общую картину, его 
можно было увидеть лишь отдельными кадрами. Словно комикс.

Комиксом я его и увидел. Точнее, финальной подготовкой к его сбору. У меня за 
спиной висела увеличенная в разы страница, начинающаяся с фрейма, в котором мы 
с Шаманом оказываемся на сцене. Дальше~--- Шаман убивает таракана. Дальше~--- к 
нам подходит эмиссар. Мы разговариваем. Появляются странники. Начинается бой. 
Эмиссар становится полчищем эмиссаров. На странице оставалось место для еще 
одного фрейма. Передо мной стоял выбор~--- какой из кадров поместить туда. Я не 
мог включить их все, даже если отдать им новую страницу. Они бы тогда ломали 
всю структуру повествования, ритмику, убивали напрочь сюжет. Нужен один, но такой, 
каким можно страницу закрыть. Я выбрал тот, на который попали все мои союзника, 
а эмиссар был только один~--- тот, который все еще поднимался с пола. Поместил 
его на свободное место. Перевернул страницу.

Я стоял на кафедре. Напротив меня сидела комиссия. Комиссия была ко мне 
расположено дружественно, я знал это. На самом деле, с защитой не должно было 
быть никаких проблем, если бы не одно <<но>>: председатель комиссии, по 
совместительству~--- мой оппонент, не разделял их позиций. Он искренне хотел 
меня завалить. Сейчас он~--- вполне человечный, но крайне неприятный тип, встал со 
своего место и прошелся по аудитории.\\
--- Прекрасно, Борис, прекрасно! Как вижу, вы неплохо подготовились. И у меня к 
вам остался один единственный вопрос: что дальше? Я не уйду просто так, оставив 
ваш мир вам, а вы, вы можете разве что убить меня здесь. Вы правда хотите убить 
меня в вашем мире, Борис? Лишить ваш мир случайностей, неожиданностей, оставив 
в нем только лишь предопределенность и неизбежность? Вам нужна абсолютно 
выверенная система, Борис? Система, элементы которой лишены свободной воли? 
Система, единственной созидательной силой которой являетесь вы, Борис. Убейте 
меня, у вас получится~--- и вы получите такую систему.

Это был удар. Я видел Берса, взвалившего на себя всю ответственность за 
вселенную. Я создал, а потом увидел Шамана, ставшего жертвой 
предопределенности. Я знал людей, существование которых на одной планете со мной мне искренне 
претило. Я сталкивался с идеями и убеждениями, людей, исповедующих которые, я 
бы с удовольствием расстреливал. Но с тем же Игорем мы подружились после пары, на 
грани мордобоя, юношеских мировоззренческих споров. И тот же Арвидас до сих пор 
искренне не переносит Игоря, всячески избегая ситуаций, в которых из-за меня 
может оказаться в одной с ним компании. При этом не вмешивается в мои с ним 
отношения.

Мне нравится смотреть, как зимой дети лепят снеговики. Мне нравится, когда 
летом жарко. Но первый рассказик я написал после того как однажды в середине июля 
выпал снег и утром я обнаружил мечущегося над снегом в поисках привычных цветов 
ошалевшего, потерявшегося шмеля. И хотя я более менее понимаю всю важность для 
нашего мира закона притяжения, я просто люблю песню <<Зимовья>> о <<летайте 
самолетами и сами по себе>>. Он прав. Я не хочу мир, где все предопределено. 
Мало того, я уверен, что мой мир не хочет быть предопределенным.\\
--- Не убью,~--- Кивнул я.~--- Вы правы.

И теперь уже мелодия не выдержала. Сначала задрожала, задребежала, потом начали 
по одному выходить из общего звука инструменты~--- пока, наконец, не стихла. В 
тишине эмиссар Хаоса набирал мощь. Я чувствовал, как там~--- <<на материальном 
уровне>>~--- мир понемногу начинает разрушаться. Я чувствовал, как здесь~--- в 
бессмысленной уже и перестающей быть отдельным уровнем <<сфере мира>> пытаются 
вступить в бой с эмиссаром мои союзники, но не могут, обездвиженные его волей и 
моим нежеланием победы. Я чувствовал как я умираю и не знал~--- есть ли иной 
выход.

В этот момент тишину нарушил звук смс--сообщения. Почему-то от этого звука 
давление эмиссара спало. Шаман без всяких проблем достал телефон. Прочитал 
сообщение. Посмотрел на меня. И, словно перед ним действительно стоял простой 
университетский профессор, пошел на эмиссара.\\
--- Паря, ты можешь быть крут как вареные яйца. Мне насрать. У меня будет 
ребенок. И этот ребенок вырастет. И этот гребанный мир будет местом достойным моего 
ребенка. Ты понял?~--- Последний вопрос Шаман почти прошипел, сжимая в руке 
горло эмиссара. Эмиссара, казалось, рука Михи не волновала вообще.\\
--- Ребенок\ldots~--- Задумчиво протянул он. В этот момент я почувствовал, что 
мне легче, а разрушение мира остановилось.~--- Реальный ребенок созданий Сказочника\ldots\
Вы\ldots~--- Легко высвободившись из рук Шамана, эмиссар повернулся к странникам.~--- У 
этого мира появился свой независимый Хранитель. Этот мир не входит в вселенную Берса. 
Меня это устраивает. Передайте ему. Сказочник,~--- Эмиссар повернулся ко мне и 
насмешливо поклонился.~--- Было приятно познакомиться.

Эмиссар исчез, зато вместо него вернулась музыка. Впрочем, вернулась не совсем 
то слово. Мелодия изменилась. В ней появились новые, делающие ее полнее, 
мотивы. \\
Я подошел к Мо.\\
--- Вот так, да?~--- Спросил я у него.~--- Значит, я готовлюсь к борьбе за свой 
мир, планы придумываю, жертвовать собой героически собираюсь. У меня за спиной, 
оказывается, отряды союзников собираются, появляющиеся в самый подходящий 
момент\ldots\ А в результате~--- все заканчивается зачатием очередного 
Спасителя? Очередной <<ребенок, который спасет мир>>, да?\\
Мо рассмеялся.\\
--- Вся сила идет изнутри, Боря. Вся сила идет изнутри\ldots\\
--- Изнутри кого?\\
--- В данном случае?~--- Мо задумался.~--- Регины, в первую очередь. Шамана, 
опять же в первую. Ну и тебя, конечно.\\
--- И на том спасибо,~--- Я нарочито ворчливо завершил эту часть разговора.~--- И 
что дальше?\\
--- Не знаю\ldots~--- Мо улыбнулся и стукнул меня по плечу.~--- Это твой мир, 
Сказочник. Твой~--- и ребенка твоих персонажей. Вам самим решать~--- как ему быть и как 
вам в нем жить. А мне приятно, что я поучаствовал в спасении родного мира, но я лечу 
дальше. Во вселенной полно приключений, знаешь ли\ldots\ Будет желание~--- 
залетай, мой трактир на Кладре все знают.\\
Мы с Шаманом снова оказались в голубой, пронизанной струнами, пустоте.\\
--- Поздравляю,~--- Сказал я.\\
--- Спасибо\ldots~--- Чуть помедлив, ответил Шаман.~--- Борис.\\
--- Тебя назад закинуть?\\
--- Конечно.~--- Шаман вздохнул.~--- Может, и сам заглянешь? Регина будет 
рада\ldots\\
--- Потом как-нибудь\ldots~--- Вздохнул в ответ я и закрыл шкатулку.\\
~\\
Я стоял посреди комнаты в квартире Регины. Мир был спасен, я~--- всемогущ. Я 
включил компьютер и набрал адрес своего блога.

\newpage

\customsection{Путь Мауи}{melkij\_bes}{PONO~--- эффективность является 
измерением правды}

\noindent --- Здравствуйте, дорогие!~--- Сказал, насладившись эффектом своего появления и, 
одновременно, собравшись с силами, Моук.~--- Надеюсь, никто не плакал на моих 
поминках? Это было бы преждевременно~--- лучшие вечера в этом трактире еще не 
прошли. И, конечно, проходить они будут с вашим участием, но сегодня я вас 
собрал не для этого. Я хочу предложить вам приключение! Приключение, стоящее 
наравне с теми, в которых нам с вами довелось побывать до того как я познакомил 
вас с Берсом. Приключение, не уступающее тем, которые мы пережили под его 
предводительством. Приключение, исход которого, в конце концов, очень важен для 
меня\ldots~--- Мо перевел дыхание. Собравшиеся молчали, ожидая продолжения. 
Только слышно было как старательно разгрызает кость мирмидского буйвола Рафа. Мо 
продолжил. Он рассказал им о том, как попал на мир X-579, как впервые 
столкнулся с пришедшим туда на инспекцию потенциальных владений эмиссаром хаоса, как 
побывал в небытие. Как выбрался, сорвав в этой части рассказа искренние 
аплодисменты публики, как выяснил, что мир <<заморожен>>, как познакомился с 
Шаманом и нашел Сказочника. Рассказал про встречу с Берсом и древними 
Хранителями.\\
--- Сказочник помог мне выбраться,~--- Закончил свой рассказ Моук.~--- Он считал, 
что я сбегаю, а я не стал говорить ему, что надеюсь вернуться. Вернуться с 
подкреплением. Друзья мои! Будучи там, я узнал, что мой отец закончил свою 
жизнь, пытаясь трахнуть смерть. Ему не повезло, но он пытался. Я предлагаю вам 
всем бросить вызов чему-то большему. Бросить вызов самому Хаосу. Пусть Берс не 
может вмешиваться~--- он Властелин Вселенной и, наверное, такие вещи как дружба 
и взаимопомощь для него теперь должны значить меньше. Но мы с вами~--- вольные 
странники. Герои легенд и мифов многих миров, члены и главные боги многих 
пантеонов\ldots\ Нам проще, правда?\\


Они согласились. Не сразу, не все, но почти. Отказались только Кинич, Рафа и 
Обаятельный~--- все они были на службе у Берса, все они опасались, что их 
вмешательство может быть воспринято Хаосом как нарушение договора. Все они 
остались до конца вечера, пируя и делясь историями наравне со всеми.\\
~\\


Мо протирал бокалы. Приключения, которые, как он хвалился перед Сказочником, 
его ждут, могли ждать и дальше. В первую очередь ему хотелось насладиться 
любимейшим делом. С момента его возвращения трактир не закрывался вообще. Правда, основной 
приток посетителей шел ближе к вечеру, а днем он занимался уборкой, заготовкой 
новых блюд и просто радовался жизни. Как сейчас.

В зал медленно, словно с трудом принимая это решение, вошел мужчина. Роста выше 
среднего, телосложения крупного, не толстый, хотя живот и начал предательски 
округляться, но грузный. Так выглядят профессиональные военные, отошедшие от 
активных действий, но оставшиеся в армии на руководящих должностях. Мужчина 
медленно прошел к стойке.\\
--- Нальешь чего-нибудь, Мо?\\
Моук мельком взглянул на вошедшего:\\
--- Как обычно?\\
---- Давай,~--- Тот кивнул. Принял из рук трактирщика бокал, отпил.~--- А ты 
изменился с нашей последней встречи\ldots\\
--- Обещал одному человеку, что верну то тело. К тому же, оно мне теперь без 
надобности\ldots\\
--- И как хозяин, выжил?~--- С неподдельным интересом спросил посетитель.~--- 
Все-таки столько времени\ldots\\
--- Не знаю\ldots~--- Мо равнодушно пожал плечами.~--- Наверное, выжил. Валяется 
теперь скорее всего в какой-нибудь психушке\ldots\ Какая мне разница?\\
Помолчали.\\
--- А меня ты не позвал\ldots~--- Сказал вдруг посетитель.\\
--- Не позвал,~--- Мо не стал спорить с очевидным.~--- С тобой мы до этого все 
выяснили.\\
Посетитель поморщился как от пощечины и запил ответ Моука. Помолчав, Мо спросил:\\
--- И как? Для вас обошлось без последствий?\\
--- Без. Хаос решил, что мы не имеем отношения к сопротивлению, к тому же его 
вполне устраивает автономное существования данного мира. Все хорошо, продолжаем 
жить дальше.\\
--- Ну и славно,~--- Мо кивнул.~--- Рад за вас.\\
Посетитель поморщился. Вздохнул.\\
--- Мо, я не буду ничего объяснять. Прости, но я правда не мог поступить 
иначе\ldots\\
--- А я не спорю, Берс\ldots~--- Мо вернулся к бокалам.~--- Просто неприятно.\\
Посетитель допил свой. Поднялся.\\
--- Пошли. Харон хочет тебе кое-что показать.\\
Не поворачиваясь, Мо пожал плечами. Берс усмехнулся.\\
--- Да пошли уже, хватит дуться. Где твое любопытство? Что мне~--- силой тебя 
тащить\ldots\\
Они вышли из дверей трактира и оказались в обычном подъезде обычной пятиэтажки. \\
--- Поднимайтесь,~--- Раздалось сверху.~--- Мы вас уже заждались\ldots\\
--- Вот всегда так,~--- Прокомментировал Берс.~--- Как кто-то посторонний, так 
прямо перед Хароном выходит. А как я~--- так еще топать черт знает сколько 
этажей\ldots\\

Харон сидел у окна. Рядом с ним стоял повидавшие много столик, на столики стоял 
граненный стакан. На ступеньках сидел Сеера. Оба кивнули Мо как доброму 
приятелю.\\
--- Смотри,~--- Харон поманил его к окну.~--- Тебе будет 
интересно\ldots

В этот раз окно Харона никуда не выходило. Наоборот, оно входило в знакомую Мо 
комнату. Прямо напротив, спиной к нему за компьютерным столом сидел Сказочник. 
Что-то печатал, постоянно отрываясь от текста и откидываясь на спинку стула. 
Потом вообще, резко встал и вышел на балкон.\\
--- Не знаю, какой он из себя Сказочник,~--- Прокомментировал Харон,~--- Но, на 
мой взгляд, для писательского поприща твоему другу выдержки и самоконтроля сильно 
не хватает. Ладно, смотри сюда\ldots

Харон высунул руки в окно и, ухватив воздух, потянул изображение на себя. 
Теперь окно выходило прямо на монитор компьютера Сказочника. \\
--- Прошу!\\
Мо вчитался в текст.

\begin{quote}
Привыкаешь ко всему кроме вони. \\

Я, кажется, встречал тех, кто утверждал, что привыкнуть можно вообще ко всему. 
Да и сам я, уверен, раньше считал, что вонь далеко не самое страшное из 
возможных мучений.\\

Они не правы. Я был не прав.\\

Вонь~--- это ужасно. Я привык делить неудобный барак с пятидесятью собратьями 
по несчастью. Я привык каждое утро просыпаться в месте, отличающемся от того, где 
я засыпал. Я привык к физическому труду и почти полной невозможностью 
пользоваться естественными для меня способностями. Я привык. Но воняет тут страшно. Каждый 
день~--- по-новому, может в этом все дело.\\

Поняв, что окончательно проснулся, я вышел во двор. В этот раз нам приготовили 
зиму. Прекрасно, снегом и умоюсь. Двор был заметен сугробами и я понял~--- нам 
надо расчистить его. Прекрасно. Умывшись, призвал в руку лопату и принялся за 
работу. Когда работаешь~--- не так чувствуешь вонь. К тому же, тут все равно 
больше нечего делать, а день не закончится пока мы не выполним запланированное. 
Вскоре ко мне присоединился Игнат. Некоторое время мы работали молча, но, 
кажется, Игнату молчание доставляет физическое неудобство.\\
--- Берс, расскажи как ты сюда попал?\\
--- Зачем?~--- Коротко ответил я. Ничего рассказывать ему я не собирался. 
Во-первых, несмотря на то, что я торчу здесь уже кучу времени, воспоминания о том, что 
было <<до срока>> по-прежнему весьма обрывочны. Во-вторых, разговаривать не 
хотелось\ldots\\
--- А хочешь тогда я расскажу, за что я тут?\\

Тем более, нет. В попадании в Тюрьму Игната не было никакой тайны. По всем 
своим повадкам, он~--- типичный представитель техники <<богоборья>>. <<Богоборцы>>, 
приходя в новый мир, почти ничем не выдают свое присутствие, сразу же сливаясь с 
представителями разумной жизни. Тратят годы и века, накладывая <<паутину 
внимания>> чтобы убедить разумную жизнь, что Хранителя, в любом его виде, в их 
мире нет. Зато потом, если паутина сплетена достаточно удачно, хватает 
правильно и вовремя дернуть ее для того чтобы воля мира просто изъяла Хранителя из своей 
системы. Дальнейшее~--- дело техники. Игнат, видимо, либо сплел, либо дернул 
паутину недостаточно удачно.\\

\noindent --- Нет,~--- Ответил я.~--- Не хочу. Работай\ldots>>
\end{quote}

\noindent Моук внимательно прочитал текст. Повернулся к Берсу.\\
--- Так и было?\\
--- Мне кажется, да,~--- Кивнул Берс.~--- Вскоре после этого разговора с Игнатом 
я и сбежал.\\
Теперь Моук повернулся к Харону.\\
--- Это значит, что?..\\
Харон неторопливо перелистывал пейзажи за окном. Наконец, остановился на виде 
на красноватую пустыню. \\
--- А ты сам что думаешь?\\
Мо задумался. Посмотрел поочередно на Берса, Сееру, Харона. Улыбнулся.\\
--- А какая, в конечном счете, разница?  
