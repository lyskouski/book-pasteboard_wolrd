\newpage
\thispagestyle{empty}
~\\
\newpage

\custompart{Скелет}{Ktirff}{http://www.citadel-liga.info}
~ 
\vspace{3mm}
\customsection{Скелет}{Ktirff}{День совершеннолетия}
 
\noindent --- Вот и настал День твоего совершеннолетия, сынок! В этот праздник 
мы с мамой вручим тебе подарок, который будет сопровождать тебя всю дальнейшую 
жизнь.

Рецепторы Марка поморщились от яркого света. На дворе стоял солнечный день, но 
родители решили, что для торжественной обстановки его маловато. Он протянул из 
своего тела щупальце со зрительным нервом и удлиняя его, осмотрел со всех 
сторон странный коричневый цилиндр.\\
--- Ты наверное и понятия не имеешь, что это стоит перед тобой,~--- поверхность 
тела отца содрогнулась от мелкой ряби, означающей смех,~--- но скоро ты очень 
привяжешься к нему.\\
--- Не томи ребенка, Мартин~--- вставила свое слово мать. Она подползла к 
подарку и нежно прикоснулась к нему. Кажется, она относилась к нему, столь 
бережно, как к своему сыну.

Марк был заинтригован. Используя все возможные анализаторные рецепторы, он 
попытался догадаться, что перед ним. Предмет слабо пахнул чем-то  похожим на 
лак, которым была покрыта некоторая мамина бюжитерия. На нем, как у 
обыкновенной двери были две ручки, но Марк не решился их дернуть, пока папа не 
сказал свою речь.\\
--- Дорогая, главный подарок в жизни следует дарить неспешно и ответственно, 
только в самых бедных семьях могут забыть о ритуале.
Итак, слушай меня внимательно, Марк. Мы, люди не всегда были так гормонично и 
удобно устроены. Человек, живший десять тысяч лет назад, посчитал бы нашу 
внешность ужасной. Конечно, в курсе истории вам подробно описывали отличия, но 
я хочу сконцентрировать твое внимание на нескольких основных моментах.

Сын Мартина принял форму пирамиды идеальной формы, отлично подходящей для 
вдумчивого слушания.\\
--- Во-первых, общество. Люди отличались друг от друга не только эластичностью, 
но и цветом, физическими способностями. Самое глупое было в том, что 
неполноценное люди, их называли калеками и инвалидами почти точно навсегда 
становились изгоями в жизни. Их сторонились, жалели и все же очень немногие 
соглашались принять их с дефектами, равными себе. Стариков обычно изживали или 
забывали о них. Лозунгами эпохи были комфорт и красота, какое значение не 
вкладывали бы в эти слова.

Марк приобрел багровый цвет, показывая свое негодование. Пирамида его тела 
увеличила площадь своего основания, он приготовился к долгой лекции. 
Разумеется, ему хотелось поскорее узнать, что за подарок ему презентуют, но еще 
предстояло получить разрешение на ночное гуляние с друзьями, так что торопиться 
не следовало.\\
--- Я не буду рассказывать о войнах, которые то и тело вспыхивали из-за 
различий в цвете тела, сейчас нам кажется это смешно. Ты знаешь об этом и без 
меня, я горжусь, что мой сын --- отличник.\\
Мартин неспешно кружил вокруг предмета с видом ученого профессора.\\
--- Касси, расскажи Марку о втором фундаментальном различии, я просинтезирую 
себе немного воды, от волнения захотелось пить.

Пирамидальный Марк повернулся к матери ребром, слушая вполуха. Кассандра 
никогда не отличалась особым красноречием. Вместо этого он сосредоточился на 
внутренней уборке своего организма.\\
--- Раньше люди думали по-другому. Ты можешь представить себе~--- 1,5 
килограмма думающего вещества, которое находилось в маленькой коробочке 
управляли телом в 50--80 раз большим!\\
--- Мам, я знаю, мы это проходили.\\
--- Слушай дальше, не перебивай маму, это очень важно для тебя.

Марк заелозил на месте, пирамида стала более округлой, свидетельствуя о его 
невнимательности.\\
--- Это сейчас мы привыкли и не можем представить себе без полного ощущения 
своего тела. Но наши предки могли десятилетиями мучаться и не подозревать, что 
в них не так. Самые просвещенные из них учились слушать свое тело, однако и это 
был мизер по сравнению с тем, что умеет сейчас даже новорожденный.

Марк хорошо знал эту историю. Несколько тысяч лет назад произошла так 
называемая <<Великая революция Тела>>, которая разом изменила все представление 
о жизни. Впрочем, ни в одном учебнике истории не было толком объяснено, что же 
такое произошло, а взрослые с загадочным видом говорили <<когда придет время, 
вы все узнаете>>.

И вот теперь Марк чувствовал, что за дверьми этого странного предмета скрыт 
ответ на самый главный вопрос прошлого. Ему не терпелось узнать правду, а 
родители казалось, томили его в нетерпении, рассказывая хорошо известные, 
очевидные факты.

Мартин подполз к жене и приникнув к ней значительной частью рецепторов 
организма важно сказал:\\
--- Отлично, Кассандра. Я думаю, наш сын хорошо учил уроки и теперь пора его 
удивить. Знаешь ли ты, Марк, о третьем фундаментальном отличии?
<<Нет, они определенно издеваются>>~--- подумалось молодому человеку. Вслух 
он сказал:\\
--- На всех занятиях нам рассказывали только о двух постулатах и Великой 
Революции.\\
--- Ну а в чем она заключалась, Марк?~--- отец хитро вытаращил зрительные 
нервы прямо напротив своего сына.\\
--- В том, что\ldots\ Хм, мир человека изменился настолько, что он никогда не 
смог бы жить как прежде,~--- Марк недовольно сжал свои клетки, признавая свое 
поражение в этом вопросе.\\
--- Хорошо, слушай и запоминай. Мы навеки потеряли свой\ldots\ Скелет.
Марк потрясенно не удержал форму и стал похож на лужицу. Скелет! Низшая форма 
жизни, которую он всегда считал прерогативой животных, оказывается был и у 
человека!\\
--- Да--да, сынок, он самый. До Великой революции Тела, люди считали его 
неотъемлемой формой развитой жизни и были скованы изнутри этим панцирем, 
лишающим всякой эластичности. Клетки из которых состоят наши современные 
организмы были узкоспециализированными и обычно выполняли только одну функцию. 
Сложно представить, как трудно жилось нашим предкам и в память о их великом 
подвиге, у каждого совершеннолетнего гражданина есть вот это. Мы с мамой рады 
посвятить тебя в священную тайну.

Марк вытянулся в тончайшую высокую башню, что говорило о его сильнейшем 
напряжении.\\
--- Перед тобой стоит твой шкаф~--- одна из вещей, которая прежде была 
необходима людям. Открой его и ты увидишь свой подарок.

Мартин с Кассандрой наслаждались произведенным эффектом и наблюдали как Марк 
протягивал свои щупальца к ручкам шкафа и подергивающимися движениями обхватил 
их. Тот уже догадывался, что предстанет перед ним.

Дверцы податливо распахнулись перед Марком, обнажая перед ним его собственный 
скелет. Он ошеломленно разглядывал этот белесо-желтоватый каркас, который 
когда-то давно у его далекого прадедушки поддерживал всю жизнь.

Марк был умным учеником и хорошо знал функции скелета у животных, поэтому без 
труда определил, что из его элементов чему служило.\\
--- Я\ldots\ я потрясен,~--- пробормотал Марк, рассматривая такое родное чудо.\\
--- Мы вручаем тебе его на хранение, следи за ним, большую часть контроля за 
его целостностью будет проводить сам шкаф. Обдумай услышанное и завтра ты 
узнаешь о предназначении скелета в твоей жизни и поверь, оно значительно. С 
Днем рождения, сынок! И кстати, этой ночью мы останемся отдыхать у своих 
родителей, так что квартира в полном твоем распоряжении.

\customsection{Скелет}{Ktirff}{Предназначение скелета}

Ночь темным коктейлем разлилась по небу, огибая островки сияющих звезд. Город 
еще немного сопротивлялся ее влиянию, но вскоре только редкие огоньки городских 
фонарей и одиноких окон нарушали идиллию. Однако, не только они. В квартире 
Марка бушевал задорный костер веселья.

Праздник удался, но только отчасти. Хозяин торжества, погруженный в раздумья, 
мало обращал внимание на сверстников. Они радовались за него, но не понимали 
его состояние, ведь их совершеннолетие еще не наступило.

Хотя, среди них и был еще один человек, посвященный в эту тайну, разговор с ним 
у Марка не вышел. На вопрос, о его мыслях про третий постулат и какие чувства 
охватывали его, когда он получил свой скелет в шкафу, тот ответил:\\
--- Нашел о чем париться! Ну, была у нас когда-то, эта штуковина, ну и что? Я 
так тебе скажу, поступить в институт, устроиться на хорошую работу и найти себе 
девушку, он мне никак не помог, а стало быть и мне особо не надо о нем 
беспокоиться. Какие-то умники решили, что он нужен и флаг им в руки~--- я нашел 
этому шкафу лучшее применение. Он как холодильник, какую еду не запихнешь, 
ничего не испортится. Со скелетом~--- в тесноте, да ни в обиде.\\
--- Возможно, ты прав,~--- Марк слушал собеседника и наблюдал, как тот 
развлекался, отщепляя от себя небольшие кусочки тела и жонглируя ими с 
потрясающей ловкостью.\\
--- Одно мне только не нравится в этой штуковине. Твои деды тебе рассказали, 
что нам в ДНК вписали код, что пока его не разблокировать, мы почти ни с кем не 
можем поговорить о нем. Вот захочешь~--- напряжешь все силы, а сказать о 
скелете ничего не сможешь. Я признаться удивлен, что ты спросил меня. Может ты 
какой-то особенный?~--- зрительный нерв принял форму несоразмерного телу 
гигантского глаза, подмигнул и собеседник, рассмеявшись, пополз в заждавшуюся 
компанию. Марк поплелся за ним.

Его не оставляла в покое мысль о предназначении скелета. Он понимал, что 
концепция шкафа--холодильника не очень похожа на правду. Зачем ученые встроили 
в ДНК человека запрет на разглашение? Почему великий подвиг прошлого так 
засекречен? Следующим днем, родители должны были дать ответы на эти вопросы.
Всю ночь громкая музыка, синтезированный алкоголь в пределах допустимого для их 
возраста, не давал спать соседям юноши. Его друзья имитировали строение 
звукового аппарата разных животных и веселились на всю катушку. Близилось утро 
и только когда гости разошлись, Марк заснул.

\newpage

В полдень, он проснулся и подготовился к встрече с родителями. Те невозмутимо 
провели Марка в комнату со шкафом и расположились рядом с ним. Мартин тут же 
предложил сыну открыть его.\\
--- Теперь ты узнаешь о значении своего скелета. Готов поспорить, ночью тебе не 
давали покоя мысли о нем. Расскажу тебе для начала, как он был 
изготовлен,~--- отец сегодня был спокойным и царящая атмосфера походила на 
безмятежную.\\
--- На специальном заводе~--- исследовательском центре, учитывая особенности 
твоей личности, его сконструировали специально для тебя. Он уникален, можешь 
быть уверенным в том, что такой есть только у тебя. Его рост~--- отношение к 
пропорциям плазмы твоего тела и скелет, как и ты, будет изменяться с возрастом.
Марк осматривал его и думал, насколько важное событие в жизни он сейчас 
переживал.\\
--- Ты думаешь, что шкаф поддерживает обеспечение необходимым сырьем для 
скелета и прав. Но это происходит без твоего непосредственного контроля, однако 
не совершай глупейшую ошибку, располагая внутри шкафа еще что-то! Он 
предназначен только для твоего скелета,~--- отец наставительно принял вид 
шара гармонии.\\
--- Скелет имеет значение, которые ты определишь для него сам. Все свои 
секреты, победы, радости и огорчения без колебания доверяй ему. Он станет твоим 
самым преданным молчаливым другом и никто, кроме тебя без разрешения не сможет 
открыть двери шкафа и вообще, что-то сделать с ним. Попробуй, положи в него 
свои переживания о узнанной тайне.


<<Как\ldots\ как мне доверится этому ящику?>>~--- думал Марк и начал отделять 
от себя субстанцию--комок чувств. Он, как и любой человек, мог делиться с 
другими людьми своими эмоциями, давая им небольшие комки чувств, но они всегда 
возвращались к хозяину. Теперь он вручал их своему скелету.

Сфера получилась довольно большой и переливалась разными цветами неоднозначных 
эмоций. Протягивая свое щупальце--руку, Марк поднес ее к шкафу.\\
--- Опускай прямо на скелет, не бойся,~--- сказала Кассандра доверительно 
пульсируя зеленым. Марк отпустил сферу и замер в ожидании.

Та, тут же приклеилась к скелету и начала изменять форму. Цвета 
перераспределились по ней и она начала делится в области ребер. Маленькие 
разноцветные шарики покатились в разные стороны, располагаясь кто где. Марк с 
замирающим дыханием смотрел на происходящее.\\
Мартин, прижавшись к жене, внимательно наблюдал за превращениями.\\
--- Видишь, они как живые, Марк. И будут бороться за свою жизнь, обрати 
внимание.

Действительно, маленькие копошащиеся мыслеэмоциональные существа принялись 
кидаться друг в друга маленькими кусочками себя. Они приобрели удобную форму в 
виде небольших прототипов животных. Кто-то стал пурпурным слоном, швыряющим 
массивные заряды своим хоботом, кто-то закидывал оппонентов крошечными 
камешками, словно белка на ветке дерева (на самом деле, сидя на носовой кости). 
Другие стали птицами, закидывая противников сверху, а один даже~--- маленькой 
юркой ящерицей подползал к соперникам и, откусив от себя кусочек, закрепив на 
враге, быстро шнырял в сторону.\\
--- Они получили название <<Мыслевики>>. Те формы, которые они приобретают, 
целиком зависит от твоих знаний и предпочтений. Иногда про них говорят~--- 
если ты хочешь разузнать о человеке, посмотри на его мыслевиков.\\
--- Почему они дерутся друг с другом?~--- спросил Марк, не отводя 
зачарованного взгляда.\\
--- Так это же твои переживания! Или ты думаешь, они однородны? Каждый мыслевик 
отстаивает свою идею и борется за жизнь. Раньше они воевали друг с другом 
внутри твоего организма и были скованы правилами поведения, усталостью, 
ленью\ldots\ тем, что происходило внутри и вокруг тебя. Теперь же они свободны 
и 
продолжают свою войну, даже когда ты спишь.\\
Мартин кольцом опоясал шкаф сына.\\
--- Но было бы ошибкой считать их независимыми. Вы находитесь в глубочайшем 
соподчинении и тебе не требуется открывать шкаф, чтобы войти с ними в контакт. 
Они, на другом уровне, но по-прежнему в тебе, а ты косвенно, а иногда прямо 
воздействуешь на них. Хочу сразу предупредить, что убрать живущих мыслевиков 
нельзя, можно только подселить новых на скелет. Обычно, это происходит 
произвольно, но если ты хочешь проследить развитие какой-то важной мысли~--- 
скатертью дорога к своему шкафу.

Марк подбросил внутрь еще несколько сфер, наблюдая за их превращениями. Каждое 
успешное попадание мыслевика сопровождалась небольшой вспышкой на теле 
соперника; промах приводил к тому, что маленький комок медленно полз к своему 
родителю.\\
--- Обрати внимание, они могут убивать друг друга. Собственно, с этой целью и 
воюют.

Мыслевик--слон недолго думая, хотел придавить оказавшуюся рядом ящерицу своей 
ногой. Но та юрко изворачиваясь, скользила под его телом, тут и там поражая 
громадину. Слон весь мерцал и все более медленно реагировал на ее действия. 
Наконец, ящерица заползла на его голову и словно танцуя, продолжила атаки.
Ноги слона подкосились и он грузно свалился на бок. По реберной кости прошла 
мелкая, едва видимая трещинка. Шкаф мгновенно отреагировал на событие, с 
боковых стенок легкий пар, направляемый мини--вентиляторами достиг места 
разлома 
и конденсировал до состояния жидкости.

Еще пара секунд и кость стала как новая. Тем временем с мыслевиком-ящерицей 
проходили удивительные превращения. Оранжевый цвет от яркого света изнутри 
приобрел золотистый оттенок. Ящерица лежала на поверженном сопернике и словно 
впитывала в себя его останки. Постепенно увеличиваясь в размерах, она потеряла 
свою ловкость. Теперь этот мыслевик прикрепился к ребру и стрелял во врагов 
быстрыми, неожиданными прикосновениями языка. Потеряв подвижность, мыслевик 
обрел невиданную защиту.\\
--- Интересная ситуация,~--- сказала мама, тоже наблюдавшая за боем. Она писала 
книги и, несомненно, почерпнула из столкновения новый сюжет: я часто смотрю на 
свой скелет и что только на нем не замечаю.\\
--- Я могу как-то узнать, что конкретно представляет этот мыслевик?~--- спросил 
Марк.\\
--- Почти что. Коснись нервом к нему и ты почувствую основную эмоцию~--- гнев, 
радость, огорчение, ностальгию. Но мысленное содержание нам недоступно. Наши 
предки называли мыслевиков подсознанием.

Продемонстрировав хорошие знания, Кассандра ободряюще подтолкнула сына к 
действию, пустив слабенький электрический заряд.

Марку было очень интересно, что за процессы проходят за стеной познания. Какие 
возможности открывались теперь! Он чувствовал сильную связь с золотым 
мыслевиком и осторожно дотронулся его.

Мощная волна восторга прошла через мысли Марка. Хотелось рассмеяться и 
безудержно что-то делать. Марк быстро отдернул рецептор, чувствуя достаточность 
ощущений. Попробовав еще несколько мыслевиков, он осознал разные эмоции, после 
чего с удовольствием вернулся к ящерице. Та стала еще крупнее. Вдруг Марк 
понял, чем хочет заниматься в жизни. В этот момент мыслевик удачным выстрелом 
языка поборол еще одного соперника.\\
--- Пап, а есть какая-то организация, занимающаяся исследованиями скелета?\\
Отец слегка удивился, но ответил: \\
--- Да, такая существует при правительстве. Она 
засекречена и про нее по правде крайне мало известно. Ведь в нашем обществе 
табу на вопросы о скелете в шкафу, не забывай.\\
--- Ага, теперь я знаю, что мне туда лежит дорога.\\
Мартин хотел что-то сказать, но Кассандра опередила его.\\
--- Сына, не ошибись в выборе. В любом случае, мы всегда поддержим тебя.

С благодарностью взглянув на родителей, Марк закрыл шкаф, и семья отправилась 
полдничать.

\chaptermark{<<Скелет>>, автор: Ktirff}{http://www.citadel-liga.info}
\customsection{Скелет}{Ktirff}{Секрет}

Его мания изучить свой скелет была неописуема. В школе он наедине расспрашивал 
учителей, а дома был готов до бесконечности любоваться скелетом и делать личные 
заметки.

Вскоре Марк узнал о небольшом секрете, который его крайне заинтересовал. 
Оказывается, по смерти человека, в течение нескольких часов его шкаф можно было 
открыть ближайшим родственникам. Ему не терпелось воспользоваться этой 
возможностью, хотя Марк прекрасно понимал постыдность своих мыслей.
Он не стал никому рассказывать, да и не смог бы признаться в таком желании. 
Продолжая учиться уже в университете, Марк как всегда был на высоте в оценках. 
Ему пророчили судьбу ученого, а сам он делал все, чтобы устроиться на работу в 
государственное предприятие. В нем Марк надеялся найти способ добраться до 
сакрального исследовательского центра.

Золотистая ящерица оккупировала голову скелета и чувствовала там себя весьма 
комфортно. Иногда для развлечения, Марк подсаживал к ней в гости сторонних 
мыслевиков и наблюдал за победой своего любимца.

Спустя несколько лет бабушка Марка серьезно заболела и Мартин с Кассандрой не 
находили себе место от беспокойства. Сын настойчиво следовал за ними, понимая, 
что это его возможно единственный шанс в обозримом будущем. И хотелось верить, 
что так оно и будет, все же Марк не был бессердечным негодяем, преследующим 
только личные цели. Ему было очень жаль бабушку, и он делал все, чтобы она 
поправилась, наперекор своему тайному желанию.

Однажды ночью, когда сон все никак не хотел сковывать усталый разум, Марк 
услышал, как бабушка направляется куда-то наверх своего дома. То был чердак, 
захламленный всякой всячиной. Ему стало ясно, где хранился ее скелет в шкафу.
Растекаясь вдоль стен, Марк тенью следовал за всеми перемещениями бабушки, 
охватываемый противоречивыми чувствами. Шпионить, конечно, нехорошо, но вдруг с 
ней что-то случится? Успокаивая тем самым совесть, ему все равно не удавалось 
заглянуть ей за спину, слишком очевидно было бы его присутствие.

Все попытки упросить родителей показать их скелеты оказывались неудачны. Кому 
охота делится своими потаенными секретами! Мартин объяснил сыну, что только раз 
в жизни человек просто обязан открыть свой шкаф другому. Это происходило во 
время женитьбы, непосредственно перед свадьбой. Молодожены открывали друг другу 
свои тайны и клялись в вечной любви.

Кстати, именно поэтому количество разводов свелось к минимуму. Мало кто после 
своей настоящей любви мог открыться снова. А если и делал это, то крайне редко 
что-то получалось вновь. Из этого Марк сделал вывод, что любить надо до 
последней капли и не сомневаться в принятом решении.

Бабушка чувствовала себя все хуже с каждым днем и пасмурным осенним вечером 
скончалась. Все произошло быстро и вот уже по дому суетились врачи, а 
многочисленные дети и внуки оказались вовремя, чтобы она успела попрощаться.
Скрывшись в общей неразберихе, Марк судорожно пытался совладеть с замком на 
чердачной двери. Он раздобыл дубликат ключа, но запирающий механизм был 
неисправен. Проникнув щупальцем внутрь замка, он нашел нужный паззл и открыл 
чердак. Скрип двери казался потрясающе громким даже на фоне шума внизу.
Марк скользнул к шкафу и открыл тайное. Шкаф словно сопротивлялся непрошенному 
гостю, но не мог ничего с ним поделать.

Перед Марком предстал бабушкин скелет. Вернее его небольшие части скрытые под 
телом массивного черного мыслевика. Он пульсировал и был похож на одежду 
скелета. Это удивительное зрелище сопровождалось негромкими всхлипывающими 
звуками. Их издавали маленькие прорешины повсюду возникающие на громадном 
мыслевике. Они тут же медленно затягивались, словно болото. Маленькие 
разноцветные мыслевики воевали друг с другом на теле покровителя, даже не 
пытаясь бороться с ним. Периодически мыслевик-гигант захватывал какую-то жертву 
и неспешно заглатывал ее в свою утробу.

Марк не верил увиденному. Что-то парализовало все мысли бабушки и он решил 
узнать, каким образом. Протянув нерв он аккуратно коснулся черного чудовища.
Смерть! Разбежалась по телу Марка и охладила все его клетки. Забегая внутрь 
всей 
его сущности угрожала всякой жизни на пути. Марк в панике отдернулся и 
стремительно закрыл шкаф.

Так вот что снедало бабушку! Ему поплохело и он осел лужицей посреди 
заброшенного чердака. Марк почувствовал как прямо сейчас на его скелете родился 
маленький поистине неубиваемый мыслевик. Его рост был только вопросом времени.
Многое предстояло обдумать и пересмотреть. Он узрел то, чего не следовало бы 
видеть никогда. Шкаф\ldots\ Какие тайны хранит у каждого из нас? Марк зарекся 
никогда больше без надобности не заглядывать на скелет в чужом шкафу.

\emph{Несколькими часами позже он скользил по тихим вечерним улицам. Лужа, 
по которой прополз молодой ученый, волной окатила обочину, спасая от неминуемой 
гибели дождевого червя на прицеле у крупного жука. Страх не прошел, но 
спрятался 
в облике черного мыслевика. Может быть для него существует соперник? Если и 
нет, Марк собирался его придумать.}

\newpage
\thispagestyle{empty}
~